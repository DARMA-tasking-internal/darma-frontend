%!TEX root = sandReportSpec.tex

\chapter{Evolution of the Specification}
\label{chap:evolution}
\section{Specification History}
\label{sec:past}
Version 0.1 of the specification existed in \gls{API} form only, and
the documention of that version of the spec differs substantially
enough from the current one that it is not included in this work.
In this version of the specification:
\begin{compactenum}
\item all input and output dependencies had to be explicitly enumerated by the application
developer,
\item data was passed to all tasks (even inline tasks) via function parameters,
\item all inputs and outputs to each task were declared using \gls{coordination semantics},
\item explicit versioning of inputs/outputs was required to keep data
logically distinct, and
\item sequential ordering of statements within DARMA had no significance for task ordering.
\end{compactenum}

 Application developer concerns regarding version 0.1 of the specification
 centered around the 1) verbosity of the approach, 2) the difficulty of
 reasoning about correct program order of tasks, and 3) the fact that 
 \inlinecode{create_work} functioned poorly in the contexts of hierarchical data
 structures and dependencies, like classes with members that were also
 classes.  The first two of these issues are addressed in version \specVersion\ of the
 specification, and the third concern will be addressed in later releases of the specification.
   
\todo{Do we want to discuss 0.2 at all?}


\section{New Features in \specVersion}
\label{sec:current}
In version \specVersion\ of the specification we:
  \begin{compactenum}
  % DSH: C++ doesn't really have ``introspection'', so I removed this
\item leverage the \CC\ \gls{capture} mechanism to minimize verbosity of the \gls{front end}
    \gls{API},
\item provide \gls{sequential semantics} within a \gls{rank} to
    facilitate reasoning about program order,
\item introduce the use of
    \gls{handle} variables to access data in the \gls{key-value store} to limit number of key-value operations for often-used data,
\item require explicit publication of all data (potentially) required off-node, and
\item implement implicit versioning within a task/scope based on sequential program order.
    \end{compactenum}


\section{Planned Features in Future Releases}
\label{sec:future}
% DSH: Removed this TODO item.  This list will always be incomplete and/or in
% flux.
% \todo[inline]{Make sure list is accurate - change releases if necessary,
% add missing things}

As part of the \gls{co-design} process, this specification will evolve quickly.
  Based on feedback thus far, there are already
many additional features planned for future incarnations of the specification that
will be released this calendar year (2016).  These are summarized below:

\todo[inline]{Move 0.3.0-alpha stuff from 0.3 bullet below to previous section and update version label below to say 0.3.1 or whatever is appropriate}

\begin{compactdesc}
\item[0.3:]
\begin{compactitem}
\item support for \inlinecode{create_work()} within functors and/or
function-like interface (i.e., not with lambdas)
\item hierarchical \glspl{dependency} (e.g., classes that have dependencies as
member variables) and containment and aliasing management
\item task creation within class member functions
\item support for automatic data partitions and collectives
\end{compactitem}
\item[0.4:]
\begin{compactitem}
\item Full support for classes in an actor-model mode 
% DSH: I removed this TODO.  We haven't elaborated on it here because we haven't
% worked out the details yet.  It seems silly to put a todo in a list of what
% are essentially todos anyway.
% \todo[inline]{I don't know what this means - elaborate on this, add details regarding migratability tradeoffs}
\end{compactitem}
\item[0.5:]
\begin{compactitem}
\item Include support for expression of \inlinecode{execution space} and
\inlinecode{memory space} and assignment of work among these abstract machine
model concepts
\item Custom data models supporting arbitrary data slicing/interference tests
\item Data staging hooks to accompany custom slicing
\end{compactitem}
\item[0.6:]
\begin{compactitem}
\item Load balancing hooks and hints to expose existing backend load balancing algorithms and hints to the user
\end{compactitem}
\end{compactdesc}
