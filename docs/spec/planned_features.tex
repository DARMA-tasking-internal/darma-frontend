%!TEX root = sandReportSpec.tex

\chapter{Evolution of the Specification}
\label{chap:evolution}
\section{Specification History}
\label{sec:past}
Version 0.1 of the specification existed in \gls{API} form only, and was not
formally documented.  In this version of the specification:
\begin{compactenum}
\item all input and output dependencies had to be listed by the application
developer;
\item all communication off node was
performed directly using \gls{coordination semantics} directly;
\item  the use of \inlinecode{version} was required to keep data
logically distinct;
\item \gls{sequential semantics} within a DARMA rank were not enforced.
\end{compactenum}

 Application developer concerns regarding version 0.1 of the specification
 centered around the 1) verbosity of the approach, 2) the difficulty of
 reasoning regarding the program order, and 3) the fact that 
 \inlinecode{create_work} cannot be called
 within the scope of class instance member functions.  The first two of these issues are
   addressed in version 0.2 of the specification, and the third concern will be
   addressed in later releases of the specification.
   


\section{New Features in 0.2}
\label{sec:current}
In version 0.2 of the specification we:
  \begin{compactenum}
\item leverage the \CC\ \gls{capture} and
    \gls{introspection} mechanisms to minimize verbosity of the \gls{front end}
    \gls{API};  
\item enforce \gls{sequential semantics} within a \gls{rank} to
    facilitate reasoning about program order; 
\item introduce the use of
    \glspl{handle} to access data in the \gls{key-value store}, and no longer
    require data to be tagged with a \inlinecode{version} in all
    settings.
    \end{compactenum}
\todo[inline]{should we list subscriptions in 0.3 or 0.2?}
\todo[inline]{is this list complete?}


\section{Planned Features in Future Releases}
\label{sec:future}
\todo[inline]{Make sure list is accurate - change releases if necessary, add missing
  things}

As part of the \gls{co-design} process, this specification will evolve quickly.
  Based on feedback thus far, there are already
many additional features planned for future incarnations of the specification that
will be released this calendar year (2016).  These are summarized below:

\begin{compactdesc}
\item[0.3:]
\begin{compactitem}
\item support for create works within functors 
\item hierarchical \glspl{dependency} (e.g., classes that have dependencies as
    member variables)
\item members functions without automatic capture can be called asynchronously
when provided list of the member variables and member functions you will use
within. \todo[inline]{elaborate on this -- add details regarding migratability issues}
\item support for collectives
\end{compactitem}
\item[0.4:]
\begin{compactitem}
\item Full support for classes in an actor-model mode \todo[inline]{elaborate on this,
add details regarding migratability tradeoffs}
\end{compactitem}
\item[0.5:]
\begin{compactitem}
\item Include support for expression of \inlinecode{execution space} and
  \inlinecode{memory space}.
\end{compactitem}
\end{compactdesc}
