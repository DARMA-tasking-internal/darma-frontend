%!TEX root = sandReportSpec.tex

\chapter{Evolution of the Specification}
\label{chap:evolution}
\section{Specification History}
\label{sec:past}
Version 0.1 of the specification existed in \gls{API} form only, and
the documention of that version of the specification differs substantially
enough from the current one that it is not included in this work.
In version 0.1 of the specification:
\begin{compactenum}
\item all input and output dependencies had to be explicitly enumerated by the application
developer,
\item data was passed to all \glspl{task} (even inline \glspl{task}) via function parameters,
\item all inputs and outputs to each \gls{task} were declared using \gls{coordination semantics},
\item explicit versioning of inputs/outputs was required to keep data
logically distinct, and
\item sequential ordering of statements within \gls{DARMA} had no significance for task ordering.
\end{compactenum}

 Application developer concerns regarding version 0.1 of the specification
 centered around the 1) verbosity of the approach, 2) the difficulty of
 reasoning about correct program order of \glspl{task}, and 3) the fact that 
 \codelink{create_work} functioned poorly in the contexts of hierarchical data
 structures and dependencies, like classes with members that were also
 classes.  The first two of these issues are addressed in version \specVersion\ of the
 specification, and the third concern will be addressed in later releases of the specification.

\section{New Features in \specVersion}
\label{sec:current}
In version \specVersion\ of the specification we:
  \begin{compactenum}
  % DSH: C++ doesn't really have ``introspection'', so I removed this
\item leverage the \CC\ \gls{capture} mechanism to minimize verbosity of the \gls{front end}
    \gls{API},
  \item introduce a functor interface that is more feature rich than the
    \gls{lambda} interface,
\item provide \gls{sequential semantics} within an \gls{execution stream} to
    facilitate reasoning about program order,
\item introduce the use of
    \gls{handle} variables to access data in the \gls{key-value store} to limit number of key-value operations for often-used data,
\item require explicit publication of all data to the \gls{key-value store} for
  data shared between \glspl{execution stream}.
\end{compactenum}



\section{Planned Features in Future Releases}
\label{sec:future}

As part of the \gls{co-design} process, this specification will evolve quickly.
  Based on feedback thus far, there are already
many additional features planned for future incarnations of the specification that
will be released this calendar year (2016).  These are summarized below:

%\todo[inline]{Jeremy/David: When is modification of fetched data going to be possible?}
%\todo[inline]{Jeremy/David: When is support for functors with templated call operators planned?}
\begin{compactdesc}
\item[0.3.1:]
\begin{compactitem}
\item Hierarchical \glspl{dependency} (e.g., classes that have dependencies as
member variables) and containment and aliasing management
\item Task creation within class member functions
\item Support for collectives
\end{compactitem}
\item[0.4:]
\begin{compactitem}
\item Schedule-only \glspl{handle} for "branch" \glspl{task} that create many other
  \glspl{task}, but do not read data
\item Include support for expression of \inlinecode{execution space} and
\inlinecode{memory space} and assignment of work among these abstract machine
model concepts
\item Custom data models supporting arbitrary data slicing/interference tests
\item Data staging hooks to accompany custom slicing
\end{compactitem}
\item[0.5:]
\begin{compactitem}
\item \Gls{leaf task} optimizations for \glspl{task} that create no
  \glspl{subtask}
\item Load balancing hooks and hints to expose existing backend load balancing algorithms and hints to the user
\item Serialization of polymorphic classes
\end{compactitem}
\item[0.6:]
\begin{compactitem}
\item Distributed containers (vectors and maps distributed across execution streams)
\item Serialization of polymorphic classes
\item \inlinecode{read_write_access} \gls{fetch}ing of published data
\end{compactitem}
\end{compactdesc}
