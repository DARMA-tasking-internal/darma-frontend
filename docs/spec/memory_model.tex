\section{Memory Model}
\label{sec:mem_model}

\todo[inline]{Janine says: memory model and data model are a bit muddled in the text
  currently, these need to be more clearly differentiated/articulated}.

\todo[inline]{a number of key points  are listed here - they need to be fleshed
  out and strung together more coherently}

\todo[inline]{add gls info in this section}

The DHARMA memory model is a \gls{PGAS}-like memory model that uses
\gls{coordination semantics} in place of direct communication.


Data that acts as any kind of dependency to deferred work within a rank, or
that needs to be communicated between ranks should be tracked within a
\gls{key-value store}, in which every data \inlinecode{value} is associated
with a \inlinecode{key}.
The \inlinecode{key} in DHARMA is not a simple
string, rather it can be an arbitrary \gls{tuple} of values (see Chapter~\ref{chap:front_end}) for details.
This makes it very easy for the applicaiton developer to create an expressive
and descriptive \inlinecode{key} for a given \inlinecode{value} in the store.


Data \glspl{handle} into the \gls{key-value store} are used to manage
complexities associated with parallelism and communication.  
The \inlinecode{handle} data structure maintains a smart pointer into the data
store.  The application developer \inlinecode{publish}es data associated with a
handle that should be visible
outside of its \gls{rank}.  When publishing,  the user must specify the
number of \glspl{rank} that will read the data (for garbage
    collection puprposes).  Note, this may sound restrictive, but is in fact
less information than is required by \gls{MPI}, where both the number of ranks
and the actual IDs of the ranks must be specified by the application developer
for communication purposes.

In addition to facilitating communication, \inlinecode{handle} data structures track additional information required 
to enforce \gls{sequential semantics} in the presence of asynchronously defined
work (see Chapter~\ref{chap:middle_end} for details on how this is supported).

\todo[inline]{Intoduce notions of access type (initial access , read access, write access). The runtime can leverage this permissions information to introduce additional parallelism when possible.}

\todo[inline]{what do we want to say about versioning here?, introduce lamport
clock and point to details in middle end?}

\todo[inline]{is fetch depracated in 0.2?}

\todo[inline]{add discussion of subscribe}

\todo[inline]{add discussion of locality - while we are introducing a key value
  store, this doesn't mean things will be placed randomly throught the system}


Although not supported in the current version specification, we introduce the
following key features of the DHARMA memory model that will be supported in future versions
of the specificaiton:

\paragraph{Memory spaces} This
  construct will be used to provide the application developer and the runtime
  system finer-grained control and additional flexibility to communicate
  policies regarding data locality and data movement. 
\todo[inline]{add discussion of data staging between memory spaces}






