%!TEX root = sandReportSpec.tex

\section{Compatible Memory Models}
\label{sec:mem_model}

The memory model for DARMA encompasses how variables are accessed (pointer, iterator, accessor) and when updates become visible to parallel threads (concurrency).  
DARMA provides a global memory space in which data are identified by unique tuple identifiers.  
Any objects published into the tuple space can be read/written by any thread/process. 
This is similar to UPC which provides a global address space, but data is always accessed by pointer address rather than tuple. 
This further contrasts with MPI 2-sided in which data are accessed by address, but there is no global memory space.  
Data within in an MPI process is ?private? and can only be exchanged via messages.  
DARMA , like UPC, further contrasts with PGAS (partitioned global address spaces) in that the global memory space is ?flat? and is not explicitly partitioned across processes.
The tuple space vs address space is only relevant for creating and managing coarse-grained tasks. 
Within a task, tuple identifiers are resolved to a specific local address and the standard C++ memory model applies.

Also critical is concurrency in the memory model and when/how updates data are made visible to parallel threads.  
Again, within tasks, the C++ memory model applies.  
At the task level (coarse-grained), DARMA ensures atomicity of all tasks. 
The DARMA scheduler enforces the C++ sequential consistency model at the level of tasks in the same way that C++ ensures sequential consistency at the level of instructions. 
DARMA understands read/write usages of tasks and ensures that writes are always visible to subsequent reads - and reads always complete before subsequent writes.  
This happens automatically within a process (shared memory), but requires an extra annotation by the app developer for objects exchanged between processes (distributed memory) to ensure data consistency.

The details of the global memory model are managed through a data \glspl{handle} object.  
The application developer \inlinecode{publish}es data associated with a handle that should be globally visible
outside of its \gls{rank}.  
When publishing, the user must specify an \gls{access group} for that data.  
Declaring an access group informs the runtime who currently needs or will need the data,  
allowing garbage collection and \gls{anti-dependency} resolution.
In most cases, the access group will be declared as the number of readers (1, in the case of simple point-to-point send).
Once all read handles are released (go out of scope in C++ terms), garbage collection or anti-dependency resolution can occur.




