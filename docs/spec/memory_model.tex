\section{Compatible Memory Models}
\label{sec:mem_model}

For every piece of data that is communicated between \glspl{rank} in DARMA, the
user is required to provide an associated \inlinecode{key}.  
A \inlinecode{key} in DARMA is not a simple
string, rather it can be an arbitrary \gls{tuple} of values (see Chapter~\ref{chap:front_end}) for details.
This makes it very easy for the applicaiton developer to create an expressive
and descriptive \inlinecode{key} for each piece of data.
The data is stored with its associated \inlinecode{key} in a \gls{key-value
store}.  \todo[inline]{add discussion of locality - 
  this doesn't mean things will be placed randomly throught the system}


Data \glspl{handle} into the \gls{key-value store} are used to manage
complexities associated with parallelism and communication.  
The \inlinecode{handle} data structure maintains a smart pointer into the data
store.  The application developer \inlinecode{publish}es data associated with a
handle that should be visible
outside of its \gls{rank}.  When publishing,  the user must specify the
number of additional handles that will be created to the data (for garbage
collection purposes and for \gls{anti-dependency} management).
Note, this may sound restrictive, but is in fact
less information than is required by \gls{MPI}, where both the number of
\glspl{rank} and the actual \gls{rank} IDs must be specified by the application developer
for communication purposes.

In addition to facilitating communication, \inlinecode{handle} data structures track additional information required 
to enforce \gls{sequential semantics} in the presence of asynchronously defined
work (see Chapter~\ref{chap:translation_layer} for details on how this is supported).

\todo[inline]{Intoduce notions of access type (initial access , read access, write access). The runtime can leverage this permissions information to introduce additional parallelism when possible.}

\todo[inline]{what do we want to say about versioning here?, introduce lamport
clock and point to details in tranlation layer?}

\todo[inline]{add discussion of subscribe}

\todo[inline]{refer back to memory spaces (already mentioned in programming
    model section). discuss how this feature impacts memory model}



