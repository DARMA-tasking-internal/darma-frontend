
\section{Memory Model}
\label{sec:mem_model}
The memory model for \gls{DARMA} encompasses how variables are accessed %(e.g., pointer, iterator, accessor) 
  and when updates become visible to parallel threads (concurrency).  
Within a \gls{DARMA} \gls{execution stream}, memory is local or private, and the
standard \CC{} memory model applies. 
To share data between \glspl{execution stream}, \gls{DARMA} uses a flat global
\gls{memory space} in which data is identified by unique \gls{tuple} identifiers, i.e. a \gls{key-value
  store} in which keys exist in a \gls{tuple space}.
Any object published into the \gls{key-value store} can be read/written by any thread/process. 

In \gls{DARMA} a data \gls{handle} is conceptually a \gls{reference counted pointer} into the
\gls{key-value store}.  Data \glspl{handle} are used to manage the
complexities associated with \gls{task parallelism} and inter-\gls{rank} communication.  
When data needs to be made accessible off-\codelink{rank}, the application developer 
\codelink{publish}es the \gls{handle}.  Each \gls{handle} has a globally unique handle ID
(i.e, a \codelink{key} that is an arbitrary \gls{tuple} of values into the
 \gls{key-value store}).  
Before a \codelink{task} can begin, \gls{handle} identifiers are resolved by
the \gls{runtime system} to a specific local address. Within the task, the standard \CC{} memory model applies.

When publishing, the user must specify an \gls{access group} for that data.  
Declaring an \gls{access group} informs the \gls{runtime system} that other
\glspl{rank} currently need or will need the data,  
allowing the runtime to manage \gls{garbage collection} and \gls{anti-dependency} resolution.
In most cases, the \gls{access group} will be declared as the number of readers (1, in the case of simple point-to-point send).
Once all read \glspl{handle} are released (go out of scope in \CC{} terms),
\gls{garbage collection} or \gls{anti-dependency} resolution can occur.


In addition to facilitating coordination between \glspl{rank}, \gls{handle} data
structures support \gls{sequential semantics} (see
Chapter~\ref{chap:translation_layer} for details).
Here concurrency is critical to the \gls{memory model} and when/how updates data
are made visible to parallel threads.
Again, within \glspl{task}, the \CC{} \gls{memory model} applies.
At the \gls{task}-level (coarse-grained), \gls{DARMA} ensures atomicity of all
\glspl{task}.
The \gls{DARMA} translation layer enforces the \CC{} sequential consistency
model at the level of \gls{task} in the same way that \CC{} ensures sequential
consistency at the level of instructions.
\gls{DARMA} understands read/write usages of \glspl{task} and ensures that
writes are always visible to subsequent reads - and reads always complete before
subsequent writes.
The use of \glspl{handle} enables this to happen automatically within an
\gls{execution stream}.




