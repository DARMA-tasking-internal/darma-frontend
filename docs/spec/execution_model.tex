%!TEX root = sandReportSpec.tex

\section{Compatible Execution Models}
\label{sec:exec_model}

As was mentioned earlier, DARMA is intended to support a range of execution
models.  An initial \gls{spmd} launch of an algorithm must be supported, and if
no imbalances are present, the runtime implementation should maintain this
optimal initial decomposition. In addition to the coarse-grained distributed
parallelism specified by the \gls{spmd} launch, additional task- and
pipeline-parallelism is introduced via the \CC-embedded task annotations.

Although not yet supported in version 0.2 of the specification, several
important features will play a role in the DARMA execution model:
\begin{compactdesc}
\item{\bf Fine-grained \gls{fork-join} parallelism}
Syntax to express both task-parallel and data-parallel code will be provided.
Furthermore, the user will be able to express the \gls{execution space} on
which it would like the code to be run, along with the fraction of the resources within
that space it would like to consume (e.g., number of threads on a CPU).
\todo[inline]{add detail re type of fork-join parallelism supported here: Is
  this terminally strict?, fully strict?}
%JJW 02/15/2016 I don't think we'll ever support the actor model
%The question is to what degree we will support C++ member functions as tasks
\item{\bf Field slicing in classes and class member functions as tasks:}\todo[inline]{elaborate on
  this}
\end{compactdesc}

\todo[inline]{refer back to execution spaces (already mentioned in programming
    model section). discuss how this feature impacts execution model}
