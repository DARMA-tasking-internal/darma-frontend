\section{Execution Model}
\label{sec:exec_model}
DHARMA supports \gls{spmd} parallelism as the dominant form of parallelism in
distributed application codes. Similar to \gls{MPI}, the application developer
is in charge of the initial problem decomposition into a set of
independently operating \glspl{rank}. 
However, DHARMA has a \gls{deferred} \gls{execution model}, that differs from the
\gls{imperative} model of \gls{MPI}.
In DHARMA, the application developer describes the
work they would like done, providing the runtime the flexibility to optimize performance and
exploit additional parallelism when possible.    
Specifically, the application developer uses \gls{coordination semantics}
(see~\ref{sec:mem_model} for details) in place of direct communication (e.g.,
    \inlinecode{send/recv}).  Rather than explicitly (imperatively)
perform work in program order and block on data requests, work is enqueued to
be performed, and is free to run once all of it's data \glspl{dependency} are met.


DHARMA enables the expression of additional \gls{task parallelism} and
\gls{pipeline parallelism} via permissions qualifiers on data 
and through \CC-embedded task annotations (described in
    Chapter~\ref{chap:front_end}). Task annotations can be nested.
\todo{elaborate on this point}
These annotations are translated by
the DHARMA \gls{middle end} which uses standard \CC\ constructs (e.g., lambdas,
  reference counted pointers) and \gls{template
metaprogramming} to manage the parallelism.  The \gls{middle end} requires
\CC14 standard features (details provided in Chapter~\ref{chap:middle_end}), 
  however the \gls{front end} \gls{API} does \emph{not}
require knowledge of \CC14 to use. Furthermore, the \gls{back end} is a
simple set of abstract \CC\ classes whose functionality must be implemented
according to the functionality specfied in Chapter~\ref{chap:back_end}.

DHARMA enforces \gls{sequential semantics}, meaning that within a \gls{rank}, you can reason about the code as
though it were being deployed sequentially.   

\todo[inline]{talk about over-decomposition - add details regarding
  initial mapping and overdecomposition}

Although not yet supported in version 0.2 of the specification, several
important features will play a role in the DHARMA execution model:

\paragraph{Expressive Underlying Abstract Machine Model}
Notions of \glspl{execution space} and \glspl{memory space} will be introduced
formally in later
versions of the specification.  These abstractions (or similar ones) appear in other runtime
solutions~\cite{kokkos, others} \todo{add relevant citations here} to
address deficiencies in the abstract machine model used by runtimes that
support \gls{spmd} parallelism (i.e., uniform compute elements, flat memory
    spaces).  Using such abstractions
facilitates performance portable application development across 
a variety of execution spaces.


\paragraph{Dynamically-determined fork-join parallelism}
In future versions of the specification, additional fork join parallelism will
be supported.  The user will be able to express the \gls{execution space} on
which it would like to be run, along with the fraction of the resources within
that space it would like to consume (e.g., number of threads on a CPU).
\todo[inline]{add detail re type of fork-join parallelism supported here: Is
  this terminally strict?, fully strict?}

\paragraph{Collectives}
In future versions of the specification, commonly used collectives
will be supported natively.
\todo[inline]{Jeremy: provide details here}

