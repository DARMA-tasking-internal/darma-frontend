%!TEX root = sandReportSpec.tex

In this document we provide the specification for DARMA,
a co-design research vehicle for \gls{AMT} programming models and runtime
systems.  This specification serves to facilitate three primary activities: 
1) gathering and communicating application requirements, 2) exploring AMT
design space tradeoffs, and 3) communication between the AMT community and
other computer science research communities.
The DARMA code base comprises three main layers: 
an API for writing application code that expresses the algorithm,  
a translation layer implicitly invoked by application-level code that maps application-level constructs
to an abstract backend runtime API.
The application-facing user-level \gls{API} has the feel of an \gls{EDSL} in \CC,  
inheriting the generic language constructs of \CC and adding \gls{semantics} that facilitate
distributed parallel programming.  
Though the \gls{EDSL} uses modern \CC constructs unfamiliar to many programmers to implement these semantics, 
it is nonetheless fully embedded in the \CC language and as such will compile with any
\CC compiler that supports the full \CC11 standard ands subset of \CC14.
There advanced C++ features are widely supported (\inlinecode{gcc >= 4.8, clang >= 3.5, icc >= 16}).  
The \gls{translation layer} leverages \CC\ \gls{template
metaprogramming} to map the user's code onto the \gls{back end} runtime \gls{API}.
The \gls{back end} \gls{API} is a set of abstract classes and function
signatures that runtime system developers must implement in accordance with the
specification requirements in order to interface with application code written
to the DARMA front end.  
Each of these three layers is described herein. We also
inlcude a list of application requirements driving the specification (along
with a list of the applications contributing to the requirements to date), and
conclude this document with a summary of the planned changes in upcoming
versions of the specification.

