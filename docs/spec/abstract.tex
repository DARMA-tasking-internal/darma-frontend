%!TEX root = sandReportSpec.tex

In this document, we provide the specification for \gls{DARMA}, 
a co-design research vehicle for \gls{AMT} \glspl{programming model} and
\glspl{runtime system}.  \gls{DARMA} serves to:
1) insulate applications from \gls{runtime system} and hardware idiosyncrasies,
2) improve \gls{AMT} runtime programmability by co-designing an \gls{API} directly with
  application developers, 3) synthesize application \gls{co-design} activities into
  meaningful requirements for runtimes, and 4) 
facilitate \gls{AMT} design space characterization and definition, accelerating the
development of \gls{AMT} best practices.

The \gls{DARMA} code base comprises two main layers: 
an \gls{API} for writing application code that expresses the algorithm and  
a translation layer implicitly invoked by application-level code that maps application-level constructs
to an abstract \gls{back end} \gls{API}.
Executable \gls{DARMA} applications link to a third layer implementing the
abstract \gls{back end} runtime \gls{API}.
The implementations may be external, but default implementations are provided
in the \gls{DARMA} code distribution.
The application-facing user-level \gls{front end} \gls{API} 
has the feel of an \gls{EDSL} in \CC,  inheriting the generic
language constructs of \CC and adding \gls{semantics} that facilitate
distributed parallel programming. Though the \gls{EDSL} uses
\CC constructs unfamiliar to many programmers to implement these semantics, 
it is nonetheless fully embedded in the \CC language and
requires a widely supported subset of \CC{}14 functionality \compilerReqs.
The \gls{translation layer} leverages \CC\ \gls{template
metaprogramming} to map the user's code onto the \gls{back end} runtime \gls{API}.
The \gls{back end} \gls{API} is a set of abstract classes and function
signatures that \gls{runtime system} developers must implement in accordance with the
specification requirements in order to interface with application code written
to the \gls{DARMA} front end.  

Each of these three layers is described herein. We also
include a list of application requirements driving the specification (along
with a list of the applications contributing to the requirements to date), some basic rules 
and examples, and conclude with a brief history of changes between previous
versions of the specification, and summary of the planned changes in upcoming
versions of the specification.  Appendices walk the user through a more detailed set
of examples of applications written in the \gls{DARMA} \gls{API} and provide
additional technical details for those who are interested.

