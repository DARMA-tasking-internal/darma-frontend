%!TEX root = sandReportSpec.tex

In this document, we provide the specification for DARMA 
\todo{acronym DARMA not defined anywhere, this is first time it appears},
a co-design research vehicle for \gls{AMT} programming models and runtime
systems.  This specification serves to facilitate three primary activities: 
1) gathering and communicating application requirements, 2) exploring AMT
design space tradeoffs, and 3) communication between the AMT community and
other computer science research communities.

The DARMA code base comprises three main layers: the \gls{front end} \gls{API},
\gls{translation layer}, and \gls{back end} \gls{API}.  
The \gls{front end} is the user-level \gls{API}. 
Its use has the feel of an \gls{EDSL} in \CC,  inheriting the generic
language constructs of \CC and adding \gls{semantics} that facilitate
distributed parallel programming. Though the \gls{front end} \gls{EDSL} uses
\CC constructs in non-traditional ways to implement these semantics, it is
nonetheless fully embedded in the \CC language. As such, it will compile with any
\CC compiler supporting the \CC11 standard with a small subset of \CC14
required for advanced features (\inlinecode{gcc >= 4.8, clang >= 3.5})\todo{FR:these have changed to 5.1 or so, right?}.
The \gls{translation layer} leverages \CC\ \gls{template
metaprogramming} to map the user's code onto the \gls{back end} \gls{API}.
The \gls{back end} \gls{API} is a set of abstract classes and function
signatures that runtime system developers must implement in accordance with the
specification requirements in order to interface with application code written
to the DARMA front end.  
Each of these three layers is described herein. We also
include a list of application requirements driving the specification (along
with a list of the applications contributing to the requirements to date), some basic rules 
and examples, and conclude with a brief history of changes between previous
versions of the specification, and summary of the planned changes in upcoming
versions of the specification.
