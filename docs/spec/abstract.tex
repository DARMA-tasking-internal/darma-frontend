In this document we provide the specification for DARMA,
a co-design research vehicle for \gls{AMT} programming models and runtime
systems.  This specification serves to facilitate three primary activities: 
1) gathering and communicating application requirements, 2) exploring AMT
design space tradeoffs, and 3) communication between the AMT community and
other computer science research communities.
The DARMA code base comprises three main layers: the \gls{front end} \gls{API},
\gls{translation layer}, and
\gls{back end} \gls{API}.  The \gls{front end} is the user-level \gls{API}.  It
is implemented as an \gls{EDSL} in \CC,  inheriting the generic language
constructs of \CC, and adding \gls{semantics} that facilitate distributed
parallel programming.
The \gls{translation layer} is a library that heavily leverages \CC\ \gls{template metaprogramming} 
to map the user's code onto interactions with a \gls{back end} \gls{API}.  
The DARMA \gls{translation layer} requires \CC\ features supported by the \CC14
standard, and will work with the following compilers: \inlinecode{gcc >= 4.8, clang >= 3.5}.
The \gls{back end} \gls{API} is a set of abstract classes, that runtime system
developers must implement in accordance with the specification requirements in
order to be compliant with the specification.
Each of these three layers is described herein. We also inlcude a list of
application requirements driving the specification (along with a list of the
applications contributing to the requirements to date), and
conclude this document with
a brief history of changes between
previous versions of the specificaiton, and a summary of the planned changes 
in upcoming versions of the specification.

