%!TEX root = sandReportSpec.tex

\chapter{DARMA Front End API}
\label{chap:front_end}
This chapter describes the DARMA \gls{front end} \gls{API}. In this chapter we
introduce the following constructs supported by the \gls{front end} \gls{API}:
\begin{compactitem}
\item Keyword arguments
\item \gls{spmd} support
\item Data Access Handles
\item Task Creation
\item Serialization Manager
\end{compactitem}

\todo[inline]{DSH: The following sections need some significant reordering
IMHO.}

\section{Deferred Work Creation}
\label{sec:deferred}
In DARMA, like other AMT runtime systems, the user creates blocks of work that
are executed when the proper permissions on the data they use are available. 
Deferred work is communicated to the runtime system via the
\inlinecode{create_work()} function, which utilizes the C++ lambda mechanism to
yield the following syntax:
% DSH merged [=]{ into one line for a couple of reasons; chief reason is we may
% want to macro this at some point so that it can conditionally be compiled into
% [=] __cuda_inline__ {  or something like that for GPU use.
\begin{CppCode}
  create_work([=]{
    // work to do
  });
\end{CppCode}

While this syntax leverages \CC11 lambdas, the user does not need to understand
\CC11 standard features to use \inlinecode{create_work()} (this
complexity is managed by DARMA's translation layer, as summarized in
Chapter~\ref{chap:translation_layer}).  All work specified within a
\inlinecode{create_work} is queued for deferred execution, and can be run when
all of its \glspl{dependency} and \glspl{anti-dependency} are met.  
These \glspl{dependency} are determined from the use of 
\inlinecode{AccessHandle<T>} objects in the deferred context. 
\inlinecode{AccessHandle<T>} objects are described in more detail below, in
Section \ref{sec:handles}.  In a basic sense, though, handles help maintain
\gls{sequential semantics} in the programming model.  For instance:
\begin{CppCode}
AccessHandle<int> my_handle = initial_access("some_data_key");
create_work([=]{
  my_handle.set_value(24);
});
// TODO finish this!!!

\end{CppCode}

\todo[inline]{DSH:  HERE is where I am right now.}

% which manage the complexity necessary (within DARMA's
%\gls{translation layer}) to 1) preserve \gls{sequential semantics} within a \gls{rank}, while performing work asynchronously, and 2) \inlinecode{publish} data visibly outside a \gls{rank}.
%\Glspl{dependency} and \glspl{anti-dependency} for a 
%\inlinecode{create_work} are inferred by the runtime via
%\gls{introspection} (again, within the \gls{translation layer}). 
%The following simple example illustrates how \inlinecode{handles} and
%\inlinecode{create_work} can be used to gether in a simple ``hello world''
%example:


\begin{CppCode}
#include <darma.h>
using namespace darma_runtime;
using namespace darma_runtime::keyword_arguments_for_publication;

int main(int argc, char** argv) {

  darma_init(argc, argv);

  size_t me = darma_spmd_rank();
  size_t n_ranks = darma_spmd_size();

  auto greetingMessage = initial_access<std::string>("myName", me);

  create_work([=]
  {
    greetingMessage.set_value("hello world!");
  });

  create_work([=]
  {
    std::cout << "DARMA rank " << me << " says " 
              << greetingMessage.get_value() << std::endl;
  });

  darma_finalize();

}
\end{CppCode}

In this example, 1) the DARMA environment is initialized, 2) each rank creates a
task to store a greeting message into a string, and 3) each rank then creates a
task to print the message and its rank to standard output.
Note that, in this example, \inlinecode{publish} is not required as the
dependency \inlinecode{greeting_message} is createed for each rank individually
(i.e., none of the dependencies must be visible outside of its rank).
In order to guarantee \gls{sequential semantics} (i.e., to make sure the value
of \inlinecode{greetingMessage} is set before it is printed),
\inlinecode{greetingMessage} must be a \inlinecode{handle}.


\section{Keyword arguments}
\label{sec:keyword}


Similar to higher-level languages like Python, the DARMA C++ interface allows the user
to specify arguments to many of the API functions and constructs using either positional
or keyword arguments.  In addition, many optional arguments may {\em only} be specified using
keyword arguments.  The syntax for specifying a keyword argument is identical to that
of Python: \inlinecode{keyword=value}.  For instance, if there is a function \inlinecode{some_function} in the
DARMA API that accepts positional or keyword arguments \inlinecode{arg_a}, \inlinecode{count}, and \inlinecode{flag}, that 
function can be invoked equivalently in any of the following ways:
\begin{CppCode}
/* some_function signature:
 *  void some_function(std::string arg_a, int count, bool flag);
 */
// All of the following are equivalent:
some_function("hello", 42, true);
some_function(arg_a="hello", count=42, flag=true);
some_function(count=42, flag=true, arg_a="hello");
some_function("hello", flag=true, count=42);
\end{CppCode}
Note that positional arguments may {\em not} be specified after the first keyword argument,
and an argument cannot be specified more than once, even as a positional and keyword
argument.  Both of these lead to compile-time errors. Omitting a required argument is 
also a compile-time error, as is giving an argument of the incorrect type: 
\begin{CppCode}
// Error: arg_a specified more than once
some_function("hello", 42, true, arg_a="whoops!");

// Error: missing required argument flag
some_function("hello", count=42);

// Error: cannot convert bool to std::string
some_function(arg_a=false, flag=true, count=42);
some_function(false, 42, true);

// Error: positional argument given after first keyword argument
some_function(arg_a="hello", 42, flag=true);
\end{CppCode}
The illusion of Python-like keyword arguments is accomplished using \inlinecode{constexpr} instances
of a class with the assignment operator overloaded.  The arguments are passed to the 
callable using perfect forwarding, and thus the overhead from the keyword argument 
trick is entirely at compile-time.  (More implementation details are given in
Section \ref{sec:kwargs}).  These instances are defined in very descriptive namespaces,
and frequently used keywords are aliased into more general namespaces.  For instance, the keywords for \inlinecode{some_function}, above, would likely be
defined in a namespace named using the following convention:
\inlinecode{darma::keyword_arguments_for_some_function}.
If \inlinecode{some_function} was a 
widely used construct, its more important keywords would also be defined in the namespace
\inlinecode{darma::keyword_arguments}.  If \inlinecode{some_function} belonged to a broader category of 
constructs, certain of its keywords may also be aliased into a categorical namespace
as well.  For instance, if \inlinecode{some_function} implemented some memory management functionality,
its keywords may be aliased into \inlinecode{darma::keyword_arguments_for_memory_management} or 
something.  All of this is done so that the user can minimize verbosity without introducing
naming conflicts.  In certain contexts, it may be expedient for the user to simply
put \inlinecode{using darma::keyword_arguments} at the beginning of the calling context.  In other,
more complicated contexts, importing all DARMA keywords into the local context may
lead to naming conflicts with local variables.  You don't have to guess at which namespaces
a keyword is provided in or aliased into; every time a keyword argument or a callable
accepting that argument is introduced in the documentation, the namespace to which 
it belongs and all namespaces it is aliased into are given.



\section{SPMD support}
\label{sec:spmd}
\gls{spmd} being the dominant model of parallelism in DARMA, \todo{are we
comfortable saying this like this?  Is it even really true?} the spin-up of the
runtime in each \gls{spmd} unit is kept analogous to MPI. This should make it a little easier for existing MPI based application codes to transition 
to DARMA. The initialization and termination of the runtime in each unit is via the calls \inlinecode{darma_init()}
and \inlinecode{darma_finalize()}. The total number of \gls{spmd} units are queried with the call \inlinecode{darma_spmd_size()},
and the rank is queried with \inlinecode{darma_spmd_rank()}. A typical user written main program will look as follows:
\begin{CppCode}
int main(int argc, char**argv){

	darma_init(argc, argv);
	size_t n_ranks = darma_spmd_size();
	size_t me = darma_spmd_rank();

	//lots of code

	darma_finalize();
}
\end{CppCode}

The rank is a very useful concept to orchestrate dependencies in a \gls{spmd}
model since data pertaining to a rank can be associated with a key that utilize
the rank for uniqueness. Similarly, the unique key used by a different rank
from which data needs to be fetched can be constructed from the knowledge of
the rank creating th edata.  The example below illustrates this concept, where
the rank is integral to the key associated with data originating on that rank
(the topic of data access handles will be discussed shortly in Section~\ref{sec:handles}):
\begin{CppCode}
size_t me = darma_spmd_rank();
auto data_handle = initial_access<double>("data_key", me);
\end{CppCode}  

\section{Data Access Handles}
\label{sec:handles}

\todo[inline]{This needs to somehow go after \inlinecode{create_work()} is
introduced, the code examples don't actually work as is}

\todo[inline]{JB Question to David/Jeremy: do we need to introduce versions, or
is this no longer needed? DSH: Versions go with publish(), but some of it needs
to be here because e.g., \inlinecode{read_access<T>()} needs to reference them.}

Transactions with the \gls{key-value store} are conducted using data handles that are of different access types which differentiate the 
various transactions. At the most basic level user code can declare 1) a handle to data that does not yet exist in the system
but needs to be created, 2) a handle to data that already exists and needs to
be read, or 3) a handle to data that it wants to overwrite. 

The first of these types is denoted as \inlinecode{initial_access} in DARMA, which informs the runtime that the data with the 
specified key does not yet exist, and the user intends to create this data and publish it shortly. Hence, an \inlinecode{initial_access}
data handle is usually followed by a memory allocation, a value assignment and finally a publish operation, as illustrated below: 
\todo[inline]{revise this.  It breaks a bunch of rules}
\begin{CppCode}
auto float_handle = initial_access<float>("float_key");
float_handle.set_value(3.14);
\end{CppCode}


The second type of transaction, requesting read-only access to a piece of data, is via the access handle \inlinecode{read_access}. The runtime
guarantees that any task that uses this handle will be launched once the data is available anywhere in the system, even on a remote node.
It will be illegal to overwrite and publish the data that is accessed via this handle. Conversely, it is illegal to request a \inlinecode{read_access}
handle to data that will never be created or published by any other task.
\todo[inline]{revise this.  It breaks a bunch of rules}
\begin{CppCode}
auto float_handle = read_access<float>("another_float_key");
float val = float_handle.get_value();
std::cout << "Value read with key another_float_key is " << val;
\end{CppCode}

And the last type, which informs the runtime that the data with the specified key will be read and overwritten, is with the \inlinecode{read_write_access}
handle, illustrated below
\todo[inline]{revise this.  It breaks a bunch of rules}
\begin{CppCode}
auto float_handle = read_write<float>("yet_another_float_key");
std::cout << "Value read with key yet_another_float_key is " 
          << float_handle.get_value();
float_handle.set_value(3.14*2.0);
\end{CppCode}

\todo[inline]{include discussion and syntax for ``demotion'' of access from
read\_write to read access only}

\subsection{Publish}
\label{subsec:publish}
The application developer must \inlinecode{publish}data associated with a
handle that should be visible
outside of its \gls{rank}.  When publishing,  the user must specify the
number of additional handles that will be created to the data (for garbage
collection purposes and for \gls{anti-dependency} management).
Note, this may sound restrictive, but is in fact
less information than is required by \gls{MPI}, where the number of readers and
their \gls{rank} identifiers must be specified.

\begin{CppCode}
auto float_handle = initial_access<float>("float_key");
float_handle.set_value(3.14);
float_handle.publish(n_readers=1);
\end{CppCode}
The \inlinecode{n_readers} specification in the publish call informs the
runtime that the data (associated with \inlinecode{float_key}) will
only ever be read once, and hence can be safely garbage collected soon after.

\todo[inline]{H: Need to explain what happens if initial access is not followed by
a publish. Illegal, or unsafe programming. JB: Is this true? I thought you only
had to publish things that will be visible outside of a rank.  Need to update
this text if that is not the case.  DSH:  Janine is correct here.}

\subsection{Subscribe}
\label{subsec:subscribe}
\todo[inline]{add subscribe (0.2.1)}

\subsection{Keys}
\label{subsec:keys}
In the examples in this Section, the \inlinecode{key} to the accessor has always been a
single string.  Note, this need not be the case as was alluded to in
Section~\ref{sec:spmd_support}. A
\inlinecode{key} in DARMA 
can be an arbitrary \gls{tuple} of values.  This 
makes it very easy for the applicaiton developer to create an expressive
and descriptive \inlinecode{key} for each piece of data.  Tuples can comprise
different bit-wise copiable data types.  The example at the end of
Section~\ref{sec:spmd} illustrates the use of the \gls{rank} within the handle
\inlinecode{key}.  The follwoing example shows the use of an
aribitrary \gls{tuple} as a \inlinecode{key}:
\begin{CppCode}
  int neighbor_id
  double other_identifier;

  // some code that sets neighborID and other_identifier
  
  auto float_handle = initial_access<float>("float_key", 
                                            neighbor_id, 
                                            other_identifier);
\end{CppCode}



\subsection{Decorators}
\label{subsec:decorators}
\todo[inline]{add information here regarding decorators to create\_work: right
now we have waits(), reads(), writes(), and reads\_writes()}


\section{Serialization API}
\todo[inline]{flesh out this section for 0.2.1 or 0.3 (whenever it gets more
settled in the backend API)}
