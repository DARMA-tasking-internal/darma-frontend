%!TEX root = sandReportSpec.tex

\chapter{Versioned Asynchronous Smart Pointer (VASP) Semantics}

\todo[inline]{this appendix has been superceded by a paper that DSH and JJL are
working on for CLUSTER16.  We need to reference that and or copy pertinent
parts there}

\section{Introduction}

The Versioned Asynchronous Smart Pointer (VASP) abstraction is a mechanism for
translating sequential semantics on a piece of data into logically idempotent,
strongly ordered uses of that data.  It extends the smart pointer abstraction,
which abstracts a view of a place in memory, to also include a view of a point
in time (usually in the future) for that memory.  The strong ordering of
``uses''\footnote{``uses'' will be more rigorously defined later} of the VASP's
data can be used to construct a directed acyclic task graph (task DAG), on which
scheduling algorithms may optimize the ordering of evaluations of these ``uses''
without deviating from the sequential intent of the program.  A hierarchical
versioning system allows intervening ``uses'' of a VASP, arising from an
evaluated ``use,'' to be inserted into the existing strong ordering without
losing the sequential intent of the program and without requiring the parent
``use'' to wait for all child ``uses'' to finish evaluating (as is the case,
for instance, in a strict fork-join tasking model).

\section{Nomenclature and Taxonomy}

For the purposes of discussion in this context, we will refer to a block of
deferred work (a deferred ``use'') of one or more VASPs as a {\it task}.
In our taxonomy, tasks {\it capture} the VASPs they will use. The block of code
(the {\it context}) that makes up a task that uses a VASP is known as that
VASP's {\it capturing context}, and a given deferred use of a particular VASP is
known as a {\it capture} of that VASP. The context immediately following a
capture of a VASP is known as its {\it continuing context}.  Each continuing
context must correspond to exactly one capturing context, and vice versa.  VASPs
can also have non-deferred ``uses,'' which we will call an {\it immediate
operation}.\footnote{The use of ``immediate'' here does not always imply
non-blocking or that the work associated with the operation may be carried out
as soon as the call is made (though this is the case for the allowed operations
on a simple VASP as described in see Section \ref{sec:simplevasp}).  It does
imply that this work will be completed before the operation returns, thus
resulting in only one output.} All immediate operations on a VASP are {\it
non-capturing}; that is, they do not result in an output of both a captured VASP
and a continuing VASP, only a continuing one.  Conversely, all deferred
operations on a VASP are {\it capturing} and thus always result in both a
captured VASP and a continuing VASP as outputs.  

The user expresses intent by writing a {\it program} containing a sequence of
operations on VASPs that refer to the user's data.  The order in which the
operations are expressed in the program\footnote{More formally, the order in
which they would be evaluated in a strictly work-first traversal of the
program's task DAG.} is called the {\it program order}.  A key feature of VASP
semantics is that the results and side effects of executing the program must be
deterministically identical to that of execution in program order.  Any of the
possible orders in which the execution of the program actually proceeds is
called an {\it execution order}, which need not be deterministic but must
maintain results and side effects of the program order.

\section{VASP States}

\subsection{A motivating example}
\label{subsec:motiv}

In order to better sequential semantics on VASPs, not all operations can be
allowed on a given VASP at a given point in the program.  Consider the following
(illegal) motivating pseudocode example using the read-only capture ({\em
ro-capture}), modify capture ({\em mod-capture}), {\tt get\_value()}, and {\tt
set\_value()} operations (described in more detail below):
\begin{vaspPseudo}
VASP<int> a; (* \label{ln:ill-cap1} *)
mod-capture(a){  (* \label{ln:ill-cap2} *)
  a.set_value(1);  (* \label{ln:ill-cap2-1} *)
}; 
ro-capture(a){ (* \label{ln:ill-cap3} *)
  mod-capture(a){ (* \label{ln:ill-cap4} *)
    a.set_value(2); (* \label{ln:ill-cap4-1} *)
  }; 
};
ro-capture(a){ (* \label{ln:ill-cap5} *)
  print a.get_value(); (* \label{ln:ill-cap5-1} *)
}; 
\end{vaspPseudo}
The above example, if legal, would prevent the scheduler from properly defining
a strong ordering of operations on {\tt a} without evaluating the {\it
ro-capture} on line \ref{ln:ill-cap3}.  If the execution order that the
scheduler chooses executes lines \ref{ln:ill-cap1}, \ref{ln:ill-cap2},
\ref{ln:ill-cap3}, \ref{ln:ill-cap5}, \ref{ln:ill-cap2-1}, \ref{ln:ill-cap5-1},
then \ref{ln:ill-cap4} and \ref{ln:ill-cap4-1}, the program would print a
different value than the sequential semantics would imply.  This is, obviously,
because the capture in line \ref{ln:ill-cap3} is lying about what it actually
does --- it claims to only read {\tt a}, so the scheduler deduces that it is
safe to run line \ref{ln:ill-cap5-1} before, during, or after the execution of
line \ref{ln:ill-cap4}, leading to the errant outcome.  Thus, we quickly come to
the rule that the captured VASP output of a {\it ro-capture} operation cannot
be an argument to a {\it mod-capture} operation.  A number of other rules can be
arrived at by similar logic, the results of which are presented below.

\subsection{A Two-part State Naming System}
\label{subsec:twopart}

To facilitate the definition of rules similar to the one described above, it is
useful to define a set of states to which a given VASP can be unambiguously
assigned at a given point in a program order (i.e., independent of execution
order).  From these states we can construct a set of allowable operations on a
given VASP at a given point in the program order as well as the states of the
corresponding outputs of operations on that VASP.

In the system for naming VASP states that we have devised, each name has two
components.  The first letter represents the VASPs scheduling permissions ---
what sort of deferred operations are allowed.  The second letter represents the
VASP's immediate permissions --- the allowed immediate operations.  Each of
these permissions components has three possible levels: None (N), Read (R), and
Modify (M).  This leads to nine possible states, six of which are meaningful in
the simple semantics described in Section \ref{sec:simplevasp}.  

By way of example, the rule derived from the discussion in Section
\ref{subsec:motiv} can be expressed as follows:  any VASP in a state without $M$
scheduling permissions --- for instance, $NN$, $RN$, $RR$, $RM$, etc. --- cannot
be an argument to a {\em mod-capture} operation.  The full analysis is as
follows.  Line \ref{ln:ill-cap1} initilizes a VASP in the state
$MN$.\footnote{An expanded semantic, described below, includes a write-only
state for VASPs with no data to read.  For simplicity, this discussion is
omitted here.}  The {\em mod-capture} operation in line \ref{ln:ill-cap2} yields
two outputs:  a captured VASP in the state $MM$ (which is thus the state of $a$
at line \ref{ln:ill-cap2-1}) and a continuing VASP in the state $MN$.  (The
specifics of the state transitions induced by various operations are discussed
below).  The {\em ro-capture} in line \ref{ln:ill-cap3}, operating on this
continuing VASP, yields a captured VASP in the state $RR$ and a continuing VASP
in the state $MN$.  The {\em mod-capture} in line \ref{ln:ill-cap4} is then
revealed to be illegal, since it operates on a VASP in the state $RR$.

\subsection{VASP Versioning}

The states described in Section \ref{subsec:twopart} yield a set of rules for
allowed operations, but do not help in establishing a strong ordering in which
those operations must be evaluated to conform to the program order results.  To
establish this ordering, each VASP must have an unambiguously discernable
version number associated with a given VASP at a given point in the program
order.  Moreover, this version should always return true for the less than
operation with the version associated with the same VASP at a different point in
the program which must happen after operations at the former point in order to
ensure correct (program-ordered) results.

In order to establish this ordering in situations where program order and
execution order may differ wildly, and thus where intervening operations on
VASPs may be discovered upon evaluating deferred operations, VASP versions are
hierarchical --- for instance, $3.7.14.0.2$.  Operations on VASPs may
trigger a {\tt push()} of a subversion (e.g., $2.4 \rightarrow
2.4.0$), an increment (e.g., $2.4 \rightarrow 2.5$, always incrementing
the part at the greatest depth), or a {\tt pop()} and increment (e.g., $2.4
\rightarrow 3$).  Versions maintain a logical less than relationship across
differing depths: $2.3.1 < 2.4 < 3 < 3.0.1$, though a {\tt push()} of a zero
results in an equality: $2.4.0 == 2.4 == 2.4.0.0$.  This hierarchical versioning
allows ``undiscovered'' intervening uses of VASPs to be arbitrarily inserted
into the ordering of the VASP uses that the scheduler must abide by.

\todo[inline]{explain how this versioning system exposes logical idempotency
within a version.}

\section{Simple VASP Semantics}
\label{sec:simplevasp}

We first present a simplified subset of the full VASP semantics that may be
sufficient for many use cases.  In this simplified subset, only states with
scheduling permissions greater than or equal to immediate permissions are
allowed.  Table \ref{tab:immsimp} shows the state transitions induced by
immediate operations in simple VASP semantics, and \ref{tab:capsimp} shows the
same information for deferred operations.

\todo[inline]{finish this section by describing the operations in the table}

\begin{table}
\begin{center}
{\small
\begin{tabular}{l|cc|cc|cc|cc}
 & \multicolumn{2}{c|}{\inlinecode{get_value()}} 
 & \multicolumn{2}{c|}{\inlinecode{set_value()}} 
 & \multicolumn{2}{c|}{\inlinecode{mark_read_only()}} 
 & \multicolumn{2}{c}{\inlinecode{release()}} \\
 \hline
 State & { Allowed? } & { Continuing }
 & { {\footnotesize Allowed? } } & { {\footnotesize Continuing } }
 & { {\footnotesize Allowed? } } & { {\footnotesize Continuing } }
 & { {\footnotesize Allowed? } } & { {\footnotesize Continuing } } \\
 \hline
 $NN$
 & No & -
 & No & -
 & No & -
 & Yes${}^*$ & \{ {\em NN}, \inlinecode{v} \} \\
 $RN$
 & No & -
 & No & -
 & Yes${}^*$ & \{ {\em RN}, \inlinecode{v} \} 
 & Yes & \{ {\em NN}, \inlinecode{v} \} \\
 $RR$
 & Yes & \{ {\em RR}, \inlinecode{v} \}
 & No & -
 & Yes${}^*$ & \{ {\em RR}, \inlinecode{v} \} 
 & Yes & \{ {\em NN}, \inlinecode{v} \} \\
 $MN$
 & No & -
 & No & -
 & Yes & \{ {\em RN}, \inlinecode{v.pop()++} \}
 & Yes & \{ {\em NN}, \inlinecode{v} \} \\
 $MR$
 & Yes & \{ {\em MR}, \inlinecode{v} \}
 & No & -
 & Yes & \{ {\em RR}, \inlinecode{v.pop()++} \}
 & Yes & \{ {\em NN}, \inlinecode{v} \} \\
 $MM$
 & Yes & \{ {\em MM}, \inlinecode{v} \}
 & Yes & \{ {\em MM}, \inlinecode{v} \}
 & Yes & \{ {\em RR}, \inlinecode{v.pop()++} \} 
 & Yes & \{ {\em NN}, \inlinecode{v} \} \\
\end{tabular}
}
\caption{Immediate operations on the various states of a VASP under the
simplified semantics.  State transitions marked with an asterisk (*) effectively
represent no-ops and could generate warnings at the implementer's discretion.}
\label{tab:immsimp}
\end{center}
\end{table}
% % 


\begin{table}[b]
\begin{center}
{\small
\begin{tabular}{l|ccc|ccc}
 & \multicolumn{3}{c|}{\em{ro-capture}} 
 & \multicolumn{3}{c}{\em{mod-capture}}  \\
 \hline
 State
 & {\footnotesize Allowed? } & {\footnotesize Captured } 
 & {\footnotesize Continuing } 
 & {\footnotesize Allowed? } & {\footnotesize Captured } 
 & {\footnotesize Continuing } \\
 \hline
 NN
 & No
 & -
 & - 
 & No
 & -
 & - \\
 RN
 & Yes 
 & \{ {\em RR}, \inlinecode{v} \} 
 & \{ {\em RN}, \inlinecode{v} \} 
 & No
 & -
 & - \\
 RR
 & Yes 
 & \{ {\em RR}, \inlinecode{v} \} 
 & \{ {\em RR}, \inlinecode{v} \} 
 & No
 & -
 & - \\
 MN
 & Yes 
 & \{ {\em RR}, \inlinecode{v} \} 
 & \{ {\em MN}, \inlinecode{v} \} 
 & Yes
 & \{ {\em MM}, \inlinecode{v.push(0)} \} 
 & \{ {\em MN}, \inlinecode{v++} \} \\
 MR
 & Yes 
 & \{ {\em RR}, \inlinecode{v} \} 
 & \{ {\em MR}, \inlinecode{v} \} 
 & Yes
 & \{ {\em MM}, \inlinecode{v.push(0)} \} 
 & \{ {\em MN}, \inlinecode{v++} \} \\
 MM
 & Yes 
 & \{ {\em RR}, \inlinecode{v++} \} 
 & \{ {\em MR}, \inlinecode{v++} \} 
 & Yes
 & \{ {\em MM}, \inlinecode{v.push(0)++} \} 
 & \{ {\em MN}, \inlinecode{v++} \} \\
\end{tabular}
}
\caption{Deferred (capturing) operations on the various states of a VASP under
the simplified semantics.}
\label{tab:capsimp}
\end{center}
\end{table}

\subsection{Versioning Example}

This example illustrates how the hierarchical versioning maintains program
ordering regardless the order in which the DAG is built.

\todo[inline]{elaborate on this example}

\begin{vaspPseudo}
VASP<int> a, b;
// a, b both at version 0
mod-capture(a){
    // a at verion 0.0
    a.set_value(5);
};
// a at version 1
mod-capture(a, b){
    // a at version 1.0, b at version 0.0
    b.set_value(a.get_value()*2);
    a.set_value(a.get_value() + 1);
};
// a at version 2, b at verion 1
capture(ro(a), mod(b)){
    // a still at version 2, b at version 1.0
    mod-capture(b){ /* b at version 1.0.1 */ };
    // b at verion 1.1
    ro-capture(b){ /* b still at version 1.1 */ };
    // b still at version 1.1
    mod-capture(b){
        // b at version 1.1.0
        mod-capture(b){ /* b at version 1.1.0.1 */ };
        // b at version 1.1.1
    };
    // b at version 1.2
};
// a still at version 2, b at version 2
\end{vaspPseudo}


\section{Full VASP Semantics}

The full extension of the simple semantics described in Section
\ref{sec:simplevasp} extends the states to include all nine possibilities and adds some operations
necessitating this extension.

\todo[inline]{finish this section, add tables}

%* A possible future expansion is to introduce a new state that
%\inlinecode{mark_read_only()} transistions {\em Empty} VASPs into.  This state
%would have identical semantics to {\em Empty} except {\em mod-capture} would not
%be allowed.

%\begin{sidewaystable}
%{\small\centering
%\begin{tabular}{l|cc|cc|ccc|ccc}
% \hline
% Operations:
% & \multicolumn{2}{c}{\inlinecode{get_value()}} 
% & \multicolumn{2}{c}{\inlinecode{set_value()}} 
% & \multicolumn{3}{c}{\em{ro-capture}} 
% & \multicolumn{3}{c}{\em{mod-capture}}  \\
% \hline
% State & {\footnotesize Allowed? } & {\footnotesize Continuing }
% & {\footnotesize Allowed? } & {\footnotesize Continuing }
% & {\footnotesize Allowed? } & {\footnotesize Captured } 
% & {\footnotesize Continuing } 
% & {\footnotesize Allowed? } & {\footnotesize Captured } 
% & {\footnotesize Continuing } \\
% \hline
% Empty ({\em E}) 
% & No & - 
% & No & - 
% & Yes 
% & \{ {\em S}, \inlinecode{v} \} 
% & \{ {\em E}, \inlinecode{v} \} 
% & Yes 
% & \{ {\em M}, \inlinecode{v.push(0)} \} 
% & \{ {\em E}, \inlinecode{v++} \} \\
% Satisfied ({\em S}) 
% & Yes & \{{\em S}, \inlinecode{v}\} 
% & No & - 
% & Yes 
% & \{ {\em S}, \inlinecode{v} \} 
% & \{ {\em S}, \inlinecode{v} \} 
% & No
% & -
% & - \\
% Modifiable ({\em M}) 
% & Yes & \{{\em M}, \inlinecode{v}\} 
% & Yes & \{{\em M}, \inlinecode{v}\} 
% & Yes 
% & \{ {\em S}, \inlinecode{v++} \} 
% & \{ {\em E}, \inlinecode{(v++)++} \} 
% & Yes
% & \{ {\em M}, \inlinecode{(v.push(0))++} \} 
% & \{ {\em E}, \inlinecode{v++} \} 
%\end{tabular}
%}
%\end{sidewaystable}


%{\small\centering
%\begin{tabular}{l|cc|cc}
% \hline
% & \multicolumn{2}{c|}{\inlinecode{wait_modifiable()}}
% \hline
% State & { Allowed? } & { Continuing }
% & { Allowed? } & { Continuing } \\
% \hline
% RN
% & No & -
% & Yes & \{ {\em RN}, \inlinecode{v} \} \\
% RR
% & No & -
% & Yes & \{ {\em RR}, \inlinecode{v} \} \\
% MN
% & Yes & \{ {\em MM}, \inlinecode{v} \}
% & Yes & \{ {\em RN}, \inlinecode{v.pop()++} \} \\
% MR
% & Yes & \{ {\em MM}, \inlinecode{v} \}
% & Yes & \{ {\em RR}, \inlinecode{v.pop()++} \} \\
% MM
% & Yes & \{ {\em MM}, \inlinecode{v} \}
% & Yes & \{ {\em RR}, \inlinecode{v.pop()++} \} \\
%\end{tabular}
%}
%{\small\centering
%\begin{tabular}{l|cc|cc|cc}
% \hline
% & \multicolumn{2}{c|}{\inlinecode{get_value()}} 
% & \multicolumn{2}{c|}{\inlinecode{set_value()}} 
% & \multicolumn{2}{c}{\inlinecode{wait_readable()}} \\
% \hline
% State & { Allowed? } & { Continuing }
% & { Allowed? } & { Continuing }
% & { Allowed? } & { Continuing } \\
% \hline
% RN
% & No & -
% & No & -
% & Yes & \{ {\em RR}, \inlinecode{v} \} \\
% RR
% & Yes & \{ {\em RR}, \inlinecode{v} \}
% & No & -
% & Yes & \{ {\em RR}, \inlinecode{v} \} \\
% MN
% & No & -
% & No & -
% & Yes & \{ {\em MR}, \inlinecode{v} \} \\
% MR
% & Yes & \{ {\em MR}, \inlinecode{v} \}
% & No & -
% & Yes & \{ {\em MR}, \inlinecode{v} \} \\
% MM
% & Yes & \{ {\em MM}, \inlinecode{v} \}
% & Yes & \{ {\em MM}, \inlinecode{v} \}
% & No & -
%\end{tabular}
%}
%% 
%
%
%\vspace{2em}
%
%{\small\centering
%\begin{tabular}{l|cc|cc}
% \hline
% & \multicolumn{2}{c|}{\inlinecode{wait_modifiable()}}
% & \multicolumn{2}{c}{\inlinecode{mark_read_only()}} \\
% \hline
% State & { Allowed? } & { Continuing }
% & { Allowed? } & { Continuing } \\
% \hline
% RN
% & No & -
% & Yes & \{ {\em RN}, \inlinecode{v} \} \\
% RR
% & No & -
% & Yes & \{ {\em RR}, \inlinecode{v} \} \\
% MN
% & Yes & \{ {\em MM}, \inlinecode{v} \}
% & Yes & \{ {\em RN}, \inlinecode{v.pop()++} \} \\
% MR
% & Yes & \{ {\em MM}, \inlinecode{v} \}
% & Yes & \{ {\em RR}, \inlinecode{v.pop()++} \} \\
% MM
% & Yes & \{ {\em MM}, \inlinecode{v} \}
% & Yes & \{ {\em RR}, \inlinecode{v.pop()++} \} \\
%\end{tabular}
%}