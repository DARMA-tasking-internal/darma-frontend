\newglossaryentry{runtime systems}
{
  type=glossary,
  name={runtime systems},
  description={}
}
\newglossaryentry{API}
{
  type=glossary,
  name={API},
  plural={APIs},
  first={application programmer interface (API)},
  description={An application programmer interface (API) is set of functions and tools provided by a library developer to allow an application programmer to interact with a specific piece of software or allow a developer to utilize prebuilt functionality}
}

\newglossaryentry{concept}
{
  type=glossary,
  name={concept},
  description={\todo{fill out concept}}
}
\newglossaryentry{concurrency}
{
  type=glossary,
  name={concurrency},
  description={A condition of a system in which multiple tasks are logically
    active at one time}
}

\newglossaryentry{parallelism}
{
  type=glossary,
  name={parallelism},
  description={A condition of a system in which tasks are actually active at
    one time}
}

\newglossaryentry{co-design}
{
  type=glossary,
  name=co-design,
  description={Co-design refers to a computer system design process where
    scientific problem requirements influence architecture design and
      technology and constraints inform formulation and design of algorithms
      and software.  Co-design methodology requires the combined expertise of
      vendors, hardware architects, system software developers, domain
      scientists, computer scientists, and applied mathematicians working
      together to make informed decisions about features and tradeoffs in the
      design of the hardware, software and underlying
      algorithms~\cite{co-design}}
}

\newglossaryentry{memory model}
{
  type=glossary,
  name={memory model},
  description={Describes the interactions of processing entities (e.g.,
      threads) with memory, including how they store and retrieve data}
}

\newglossaryentry{distributed memory model}
{
  type=glossary,
  name={distributed memory model},
  description={ Each processor has its own private memory. 
     Computational tasks can only operate on their local data. When remote data
     is required, it is communicated between the remote and local tasks}
}

\newglossaryentry{distributed shared memory model}
{
  type=glossary,
  name={distributed shared memory model},
  description={ Physically distinct memory that can be accessed as one
    logically shared address space}
}

\newglossaryentry{data model}
{
  type=glossary,
  name={data model},
  description={A model capturing assumptions or restrictions on the structure of data}
}


\newglossaryentry{reference counted pointer}
{
  type=glossary,
  name={reference counted pointer},
  plural={reference counted pointers},
  description={An abstract data type that stores a traditional pointer, along
    with the number of shared references to that pointers
    memory location.  Objects referenced by the contained raw pointer are only
     destroyed when all copies of the reference counted pointer are destroyed} 
}

\newglossaryentry{lambda}
{
  type=glossary,
  name={lambda},
  description={ In \CC{} a lambda is a mechanism for 
    defining an unnamed function object at the location 
    where it is invoked. Lambdas are capable of capturing (see \gls{capture}) variables in
    scope. See~\cite{lambda} for more detail},
}

\newglossaryentry{closure}
{
  type={glossary},
  name={closure},
  plural={closures},
  description={A closure is a record storing a function, together with an
    environment.  See~\cite{closure} for more detail},
}

\newglossaryentry{capture}
{
  type=glossary,
  name=capture,
  description={In \CC{} the capture list specifies which variables defined
    outside the lambda are available for use within the lambda. Variables may
    be captured by value or reference.  See~\cite{lambda} for more detail},
}

\newglossaryentry{introspection}
{
  type=glossary,
  name=introspection,
  description={The ability of a program to examine properties of an object at
    runtime},
}

\newglossaryentry{overdecomposition}
{
  type=glossary,
  name=overdecomposition,
  description={\todo{define}},
}

\newglossaryentry{execution stream}
{
  type=glossary,
  name={execution stream},
  description={\todo{define}},
}

\newglossaryentry{process}
{
  type=glossary,
  name={process},
  plural={processes},
  description={\todo{define}},
}

\newglossaryentry{deferred execution}
{
  type=glossary,
  name={deferred execution},
  description={Execution of work is not performed until all dependencies are
    met.}
  }

\newglossaryentry{deferred work}
{
  type=glossary,
  name={deferred work},
  description={Work
performed by code inside the capturing lambda passed to the
\codelink{create_work} construct (as well as other deferred constructs which
may be added to future versions of the specification)},
}

\newglossaryentry{deferred task}
{
  type=glossary,
  name={deferred work},
  description={\todo{define}}
}

\newglossaryentry{captured work}
{
  type=glossary,
  name={captured work},
  description={See \gls{deferred work}},
}

\newglossaryentry{captured context}
{
  type=glossary,
  name={captured context},
  description={See \gls{deferred work}},
}

\newglossaryentry{continuing context}
{
  type=glossary,
  name={continuing context},
  description={the code after a \codelink{create_work}},
}

\newglossaryentry{deferred context}
{
  type=glossary,
  name={deferred context},
  description={See \gls{deferred work}},
}

\newglossaryentry{slicing}
{
  type={glossary},
  name={slicing},
  description={\todo{yeah, something about data subsets}},
}

\newglossaryentry{interference test}
{
  type={glossary},
  name={interference test},
  plural={interference tests},
  description={\todo{yeah, something about data subsets}},
}


\newglossaryentry{access group}
{
  type={glossary},
  name={access group},
  description={\todo{yeah, something about send/recv, broadcast, scopes}},
}

\newglossaryentry{keyword argument}
{
  type={glossary},
  name={keyword argument},
  plural={keyword arguments},
  description={An argument that is passed to a function as a keyword=value.},
}

\newglossaryentry{positional argument}
{
  type={glossary},
  name={positional argument},
  plural={positional arguments},
  description={An argument passed to a function, whose corresponding parameter
    is inferred by the argument's position within the function call.},
}

\newglossaryentry{perfect forwarding}
{
  type={glossary},
  name={perfect forwarding},
  description={A mechanism for forwarding arguments of one function to another
    in \CC{} that avoids copying and maintains lvalue/rvalue nature of the
      arguments. See~\cite{perfect-forwarding} for more detail},
}

\newglossaryentry{rank}
{
  type=glossary,
  name=rank,
  plural=ranks,
  description={\todo{define} },
}

\newglossaryentry{operation}
{
  type=glossary,
  name=operation,
  description={\todo{define} },
}

\newglossaryentry{PGAS}
{
  type=\acronymtype,
  name={PGAS},
  first={Partitioned Global Address Space (PGAS)},
  description={A programming model assuming a global memory address space that is logically
    partitioned such that a portion of it is local to each process or thread~\cite{PGAS}},
}


\newglossaryentry{dependency}
{
  type=glossary,
  name=dependency,
  plural=dependencies,
  description={See \gls{Read-After-Write}},
}

\newglossaryentry{associative array}
{
  type=glossary,
  name={associative array},
  plural={associative arrays},
  description={An abstract data type composed of a 
    collection of key-value pairs, such that each possible key appears just
      once in the collection~\cite[associative-array]. Data is retrieved from an associative array
  via its key, rather than its address in the array},
}

\newglossaryentry{key-value store}
{
  type=glossary,
  name={key-value store},
  plural={key-value stores},
  description={A database that has an \gls{associative array} as its underlying
    data model.  In DARMA, a key-value store},
}

\newglossaryentry{abstract machine model}
{
  type=glossary,
  name={abstract machine model},
  plural={abstract machine models},
  description={A model of a computer system that is designed to allow
    application developers to focus on the aspects of the machine that are
      important or relevant to performance and code
      structure~\cite{AbstractMachine}},
}

\newglossaryentry{execution space}
{
  type=glossary,
  name={execution space},
  plural={execution spaces},
  description={A \gls{abstract machine model} abstraction used to describe where work is executed},
}

\newglossaryentry{memory space}
{
  type=glossary,
  name={memory space},
  plural={memory spaces},
  description={An \gls{abstract machine model} abstraction used to describe
    where data resides},
}

\newglossaryentry{version}
{
  type=glossary,
  name=version,
  plural=versions,
  description={},
}
\newglossaryentry{handle}
{
  type=glossary,
  name=handle,
  plural=handles,
  description={},
}

\newglossaryentry{migratable}
{
  type=glossary,
  name=migratable,
  description={},
}

\newglossaryentry{serialization}
{
  type=glossary,
  name=serialization,
  description={The process of converting a \CC{} object into a sequence of bytes that
    can be transmitted over the network or stored}, 
}



\newglossaryentry{Trilinos}
{
  type=glossary,
  name=Trilinos,
  description={The Trilinos Project is an effort to develop algorithms and
    enabling technologies within an object-oriented software framework for the
      solution of large-scale, complex multi-physics engineering and scientific
      problems~\cite{trilinos}}
}

\newglossaryentry{tuple}
{
  type={glossary},
  name={tuple},
  plural={tuples},
  description={A tuple is a finite ordered list of elements.  See~\cite{tuple}
  for more detail},
}

\newglossaryentry{tuple space}
{
  type={glossary},
  name={tuple space},
  plural={tuple spaces},
  description={A repository of \glspl{tuple} that can be
  accessed concurrently, used to relate input to output patterns. A tuple
  space served as the underpinning to \gls{Linda} \gls{programming
  language}. Tuple spaces can be considered a generalization of a
  \glspl{key-value store}. Implementations of tuple spaces have 
  been developed for a number of other
  programming models including Java and Python.  See~\cite{tupleSpace} for
  more detail} 
}

\newglossaryentry{Linda}
{
  type={glossary},
  name={Linda},
  description={ Linda is a model of coordination and communication among several parallel
      processes operating upon objects stored in and retrieved from shared,
    virtual, associative memory~\cite{Linda}}
}



\newglossaryentry{EDSL}
{
  type=glossary,
  name={EDSL},
  plural={EDSLs},
  first={embedded domain specific language (EDSL)},
  description={A \gls{DSL} that is defined as a library for a
    generic host programming language. The embedded domain specific language
      inherits the generic language constructs of its host language -
      sequencing, conditionals, iteration, functions, etc. - and adds
      domain-specific primitives that allow programmers to work at a much
      higher level of abstraction},
  see={DSL}
}
\newglossaryentry{ATDM}
{
  type=\acronymtype,
  name={ATDM},
  first={Advanced Technology Development and Mitigation (ATDM)},
  description={This \gls{ASC} program includes laboratory code and computer engineering and science projects that pursue long-term simulation and computing goals relevant to the broad national security missions of the National Nuclear Security Administration}
}
\newglossaryentry{programming language} %linked, 
{
  type=glossary,
  name={programming language},
  plural={programming languages},
  description={A programming language is a syntax and code constructs for
    implementing one or more \glspl{programming model}.
  For example, the \CC{} programming language supports both \gls{functional} and
    \gls{procedural} \gls{imperative} \glspl{programming model}}
}

\newglossaryentry{programming model} %linked, 
{
  type=glossary,
  name={programming model},
  plural={programming models},
  description={A parallel programming model is an abstract view of a machine
    and set of first-class constructs for expressing algorithms. The programming
      model focuses on how problems are decomposed and expressed.  In
      \gls{MPI}, programs are decomposed based on \gls{MPI} ranks that coordinate via
      messages. This programming model can be termed \gls{spmd}, decomposing
      the problem into disjoint (non-conflicting) data regions.  \Charm{}
    decomposes problems via migratable objects called \glspl{chare} that
      coordinate via remote procedure calls (entry methods). Legion decomposes
      problems in a data-centric way with logical regions.  All parallel
      coordination is implicitly expressed via data dependencies.  The parallel
      programming model covers how an application \emph{expresses}
    \gls{concurrency}.
In many cases, the \gls{execution model} and programming model are closely tied
and therefore not distinguished. In these cases the non-specific term parallel
model can be applied
    \todo{Double check to see if we are using parallel vs concurrency correctly
      in this description} }
}

\newglossaryentry{front end}
{
  type=glossary,
  name = {front end},
  description = {A software stack may comprise many layers, separating the 
    user from the hardware.  Each layer comprises a front end
      and a \gls{back end}.  The front end provides a set of abstractions and the
      user interface for the functionality implemented by the back end}
}

\newglossaryentry{translation layer}
{
  type=glossary,
  name = {translation layer},
  description = {The \CC \gls{template metaprogramming} layer between the
    DARMA \gls{front end} and the set of abstract classes that must be implemented
    by an implementation of the \gls{back end}}
}
\newglossaryentry{back end}
{
  type=glossary,
  name = {back end},
  description = {A software stack may comprise many layers, separating the 
    user from the hardware.  Each layer comprises a \gls{front end}
      and a back end.  The front end provides a set of abstractions and the
      user interface for the functionality implemented by the back end}
}

\newglossaryentry{semantics}
{
  type=glossary,
    name={semantics},
    description={A mathematical model representing the intended computational behavior of program}
}

\newglossaryentry{sequential semantics}
{
  type=glossary,
  name={sequential semantics},
  description={Computational behavior of code is equivalent to running it
    sequentially, in program order}
}

\newglossaryentry{coordination semantics}
{
  type=glossary,
  name={coordination semantics},
  description={The operations to support communication between different
    computation activities. Independent parallel workers never directly
      communicate, rather they ``coordinate'' indirectly via a \gls{key-value
        store} or \gls{tuple space}. \gls{Linda} is a notable \gls{programming
        language} with coordination semantics
  },
}


\newglossaryentry{template metaprogramming}
{
  type=glossary,
  name={template metaprogramming},
  description={In template metaprogramming templates are used by a compiler to
    generate additional source code, (e.g., compile-time constants, data
        structures, funcitons), which is merged by the compiler with the rest
      of the user-provided source code prior to compilation} 
}

\newglossaryentry{spmd} %linked, defined
{
  type=glossary,
  name={SPMD},
  first={single-program multiple-data (SPMD)},
  description={The term single-program multiple-data (SPMD) refers to a parallel \gls{programming model} where the same tasks are carried out by multiple processing units but operate on different sets of input data. This is the most common form of parallelization and often involves multithreading on a single compute node and/or distributed computing using \gls{MPI} communication}
}




\newglossaryentry{bulk synchronous}  
{
  type=glossary,
  name={bulk synchronous},
  description={The bulk synchronous model of parallel computation (BSP) is
  defined as the combination of three attributes: 1)~A number of
  components, each performing processing and/or memory functions; 2)~A
  router that delivers messages point to point between pairs of
  components; and 3)~Facilities for synchronizing all or a subset of
  the components at regular intervals of $L$ time units where $L$ is
  the periodicity parameter.  A computation consists of a sequence of
  supersteps. In each superstep, each component is allocated a task
  consisting of some combination of local computation steps, message
  transmissions and (implicitly) message arrivals from other
  components. After each period of $L$ time units, a global check is
  made to determine whether the superstep has been completed by all
  the components. If it has, the machine proceeds to the next
  superstep. Otherwise, the next period of $L$ units is allocated to
  the unfinished superstep.  See Reference~\cite{BSP}
  and~\cite{wikiBSP} for more details} 
}

\newglossaryentry{CSP}
{
  type=glossary,
  name={CSP},
  first={communicating sequential processes (CSP)},
  description={CSP (communicating sequential processes) is the most popular
    concurrency model for science and engineering applications, often being
      synonymous with \gls{spmd}. CSP covers execution models where a usually fixed number of independent workers operate in parallel,
      occasionally synchronizing and exchanging data through inter-process communication. Workers are \emph{disjoint processes},
      operating in separate address spaces.  This also makes it generally synonymous with message-passing in which data exchanges between parallel workers are copy-on-read, creating disjoint data parallelism.  The term sequential is historical and CSP is generally applied even to cases in which each ``sequential process'' is composed of multiple parallel workers (usually threads)}
}

\newglossaryentry{event-based}
{
  type=glossary,
  name={event-based},
  description={The term event-based covers both \glspl{programming model} and
    \glspl{execution model} in which an application is expressed and managed as a set of events with precedence constraints, often taking the form of a directed graph of event dependencies}
}

\newglossaryentry{execution model} %linked,
{
  type=glossary,
  name={execution model},
  plural={execution models},
  description={A parallel execution model specifies how an application creates
    and manages \gls{concurrency}. This includes, e.g., \gls{CSP} (communicating
        sequential processes), strict \gls{fork-join}, or \gls{event-based}
      execution.  These classifications distinguish whether many parallel
      workers begin simultaneously (e.g., \gls{CSP}) and synchronize to reduce
      \gls{concurrency}
      or if a single top-level worker forks new tasks to increase \gls{concurrency}.
      These classifications also distinguish how parallel hazards (\gls{WAR},
          \gls{RAW}, \gls{WAW}) are managed. Execution models fall into the
      follwing broad categories: \gls{conservative execution}, \gls{phased
        execution}, \gls{copy-on-write data-flow execution}, and
        \gls{speculative execution}.
    In many cases, the \gls{programming model} and execution model are closely tied and
      therefore not distinguished. In other cases, the way execution is managed is decoupled from
      the \gls{programming model} in \glspl{runtime system} with
      \gls{declarative} \glspl{programming model} like Legion or Uintah. The execution model is implemented in the
      \gls{runtime system}
  }
}

\newglossaryentry{DEP}
{
  type=glossary,
  name={DEP},
  plural={DEPs},
  first={\gls{DARMA} Enhancement Plan (DEP)},
  description={The method by which changes are made to the \gls{DARMA}
    specification}
}


\newglossaryentry{conservative execution}
{
  type=glossary,
  name={conservative execution},
  description={The \gls{runtime system} only spawns tasks in parallel that are guaranteed not to conflict.  
The application exposes \gls{RAW}/\gls{WAR} conflicts, allowing the
  \gls{runtime system} to decide which tasks can safely 
run in parallel.  Independent threads do not need to explicitly synchronize.
Execution begins with zero \gls{concurrency}
and grows conservatively to the maximum allowed \gls{concurrency}}
}

\newglossaryentry{barrier}
{
  type=glossary,
  name={barrier},
  description={\todo{add desc}}
}

\newglossaryentry{phase barrier}
{
  type=glossary,
  name={phase barrier},
  plural={phase barriers},
  description={Where
    \gls{RAW} or \gls{WAR} conflicts may 
  exist, a \gls{phase barrier} is executed to guarantee safe execution. The
  term \gls{phase barrier} has previously 
  been used in Legion~\cite{LegionThesis} and X10~\cite{X10Phasers}. Barriers may be local operations 
  or global collectives }
}

\newglossaryentry{phased execution}
{
  type=glossary,
  name={phased execution},
  description={The \gls{runtime system} spawns many tasks in parallel.  Where
    \gls{RAW} or \gls{WAR} conflicts may 
  exist, a \gls{phase barrier} is executed to guarantee safe execution. The
  term \gls{phase barrier} has previously 
  been used in Legion~\cite{LegionThesis} and X10~\cite{X10Phasers}. Barriers may be local operations 
  or global collectives.  Execution begins with maximum parallelism and
  \gls{concurrency} decreases when 
  necessary to satisfy synchronization constraints}
}

\newglossaryentry{copy-on-write data-flow execution}
{
  type=glossary,
  name={copy-on-write data-flow execution},
  description={This is an intermediate between \gls{conservative execution} and
    \gls{phased execution}, with the additional constraint that the application guarantees no 
    \gls{WAR} conflicts.  Tasks are written to follow a
    write-once, read-many policy when necessary to avoid \glspl{anti-dependency}.
    The only synchronizations required are \gls{RAW}, ensuring that a
    value exists before a task can run.  Similar to \gls{conservative
      execution},
    tasks spawn once all their \gls{RAW} dependencies are met, forking new
      \gls{concurrency}.  
    Once running, tasks do not synchronize because there are no \gls{WAR} conflicts
    to avoid.  This approach often has higher memory requirements, and the
    necessary \gls{garbage collection} adds complications}
}

\newglossaryentry{garbage collection} %linked,
{
  type=glossary,
  name={garbage collection},
  description={}
}
\newglossaryentry{runtime system} %linked,
{
  type=glossary,
  name={runtime system},
  plural={runtime systems},
  description={A parallel runtime system primarily implements portions of an
  \gls{execution model}, managing how and where concurrency is managed and created. 
 Runtime systems therefore control the order in which parallel work (decomposed
     and expressed via the \gls{programming model}) is actually performed and executed.  Runtime systems can range greatly in complexity. A runtime could only provide point-to-point message-passing, for which the runtime only manages message order and tag matching. A full \gls{MPI} implementation automatically manages collectives and global synchronization mechanisms.
  Legion handles not only data movement but task placement and out-of-order task execution, handling almost all aspects of execution in the runtime.  Generally, parallel execution requires managing task placement, data placement, concurrency creation, concurrency managed, task ordering, and data movement. A runtime comprises all aspects of parallel execution that are not explicitly managed by the application}
}

\newglossaryentry{HLR}
{
  type=glossary,
  name={HLR},
  first={high-level runtime (HLR)},
  description={A high-level runtime is generally any aspect of the runtime system that implicitly creates concurrency via higher-level logic based on what is expressed via the application programming model.  High-level runtimes generally involve data, task, and machine models expressed in a \gls{declarative} fashion through which the runtime reasons about application concurrency. This implicit creation of concurrency differs from the low-level runtime (\gls{LLR}), which only executes operations explicitly specified.  Legion and Uintah both implement extensive HLRs while Charm++ has very little implicit behavior}
}

\newglossaryentry{LLR}
{
  type=glossary,
  name={LLR},
  first={low-level runtime (LLR)},
  description={A low-level runtime is generally any aspect of a \gls{runtime system} that manages explicitly specified data movement and task scheduling operations. There is very little implicit behavior. The runtime is only responsible for ensuring that events and operations satisfy explicit precedence constraints. This contrasts with a high-level runtime (\gls{HLR}) that implicitly creates parallelism from a declarative program, converting higher-level program logic into explicit operations in the LLR}
}

\newglossaryentry{DSL} %linked, defined
{
  type=glossary,
  name={DSL},
  first={domain specific language (DSL)},
  description={Domain specific languages (DSL) are a subset of
    \glspl{programming language} that have been specialized to a particular
      application domain. Typically, DSL code focuses on what a programmer
      wants to happen with respect to their application and leaves the
      \gls{runtime system} to determine how the application is executed}
}

\newglossaryentry{simd} %linked, defined
{
  type=glossary,
  name={SIMD},
  first={single-instruction, multiple-data (SIMD)},
  description={The term single-instruction multiple-data (SIMD) refers to a type of instruction level parallelism where an individual instruction is synchronously executed on different segments of data. This type of \gls{data parallelism} is best illustrated by \gls{vector processing}}
}
\newglossaryentry{mimd} %linked, defined
{
  type=glossary,
  name={MIMD},
  first={multiple-instruction, multiple-data (MIMD)},
  description={The term multiple-instruction multiple data (MIMD) refers to a parallel \gls{programming model} where a set of processing units, operating asynchronously, execute different instructions on different sets of data}
}
\newglossaryentry{mpmd} %linked, defined
{
  type=glossary,
  name={MPMD},
  first={multiple-program multiple-data (MPMD)},
  description={The term multiple-program multiple-data (MPMD) refers to a
    parallel programming model where tasks operate on disjoint data like \gls{spmd}, but are not constrained to perform the same tasks}
}

\newglossaryentry{actor model} %linked, defined
{
  type=glossary,
  name={actor model},
  description={An actor model covers both aspects of programming and
    \glspl{execution model}.  In the actor model,
  applications are decomposed across objects called actors rather than processes or threads (\gls{MPI} ranks). The actor model shares similarities with active messages.  Actors send messages to other actors, but beyond simply exchanging data they can invoke remote procedure calls to create remote work or even spawn new actors.  The actor model mixes aspects of \gls{spmd} in that many actors are usually created for a data-parallel decomposition.  It also mixes aspects of \gls{fork-join} in that actor messages can ``fork'' new parallel work; the forks and joins, however, do not conform to any strict parent-child structure since usually any actor can send messages to any other actor}  
}

\newglossaryentry{fork-join} %linked, defined
{
  type=glossary,
  name={fork-join},
  description={A model of concurrent execution in which child tasks are forked
    off a parent task.  When child tasks complete, they synchronize with join
      partners to signal execution is complete.  \Gls{fully strict} execution
      requires join edges be from parent to child while \gls{terminally strict} requires child tasks to join with grandparent or other ancestor tasks. This style of execution contrasts with \gls{spmd} in which there are many parallel sibling tasks running, but they did not fork from a common parent and do not join with ancestor tasks}
}

\newglossaryentry{fully strict} %linked, defined
{
  type=glossary,
  name={fully strict},
  description={ Fully strict \gls{fork-join} execution requires join edges between parent and
    child tasks} 
}

\newglossaryentry{terminally strict} %linked, defined
{
  type=glossary,
  name={terminally strict},
  description={Terminally strict \gls{fork-join} execution requires child tasks to join with
    grandparent or other ancestor tasks}
}

\newglossaryentry{data parallelism} %linked, defined,
{
  type=glossary,
  name={data parallelism},
  description={A type of parallelism that involves carrying out a single task
    and/or instruction on different segments of data across many computational
      units. Data parallelism is best illustrated by \gls{vector processing} or \gls{simd} operations on
      \glspl{CPU} and \glspl{MIC} or typical \gls{bulk synchronous} parallel applications}
}
\newglossaryentry{vector processing}  
{
  type=glossary,
  name={vector processing},
  description={A vector processing is performed by a central
  processing unit (\gls{CPU}) that implements an instruction set containing
  instructions that operate on one-dimensional arrays of data called
  vectors, compared to scalar processors, whose instructions operate
  on single data items. Vector processing can greatly improve
  performance on certain workloads, notably numerical simulation and
  similar tasks. Vector machines appeared in the early 1970s and
  dominated supercomputer design through the 1970s into the 1990s,
  notably the various Cray platforms. As of 2015 most commodity \glspl{CPU}
  implement architectures that feature instructions for a form of
  vector processing on multiple (vectorized) data sets, typically
  known as \gls{simd}. Common examples include MMX, \gls{SSE},
  AltiVec and \gls{AVX}}
}

\newglossaryentry{thread pool} %linked, defined
{
  type=glossary,
  name={thread pool},
  description={\todo[inline]{add entry for thread pool}}
}

\newglossaryentry{task} %linked, defined
{
  type=glossary,
  name={task},
  description={\todo[inline]{add entry for task}}
}

\newglossaryentry{task parallelism} %linked, defined
{
  type=glossary,
  name={task parallelism},
  description={A type of parallelism that focuses on completing multiple tasks simultaneously over different computational units. These tasks may operate on the same segment of data or many different datasets}
}
\newglossaryentry{pipeline parallelism} %linked
{
  type=glossary,
  name={pipeline parallelism},
  %description={Pipeline parallelism involves breaking up a task into a sequence of individual tasks and then overlapping the execution of the steps from multiple tasks that would otherwise have had to be done sequentially}
  description={Pipeline parallelism is achieved by breaking up a task into a sequence of
individual sub-tasks, each of which represents a stage whose execution can be
  overlapped}
}
\newglossaryentry{task-DAG} %linked
{
  name={task-DAG},
  description={A use of a directed acyclic graph (\gls{dag}) that represents
    tasks as nodes. Edges between two nodes capture task/task precedent
      constraints}
}
\newglossaryentry{dag}  %linked
{
  type=\acronymtype,
  name={DAG},
  first={directed acyclic graph (DAG)},
  description={A directed acyclic graph (DAG) is a directed graph with no cycles. This type of data representation is common form for representing dependencies}
}

\newglossaryentry{CDAG}  %linked
{
  type=\acronymtype,
  name={CDAG},
  first={computational directed acyclic graph (CDAG)},
  description={ A CDAG is a data-flow task graph that has two
types of vertices - tasks $T$ and data $D$ that compose the complete set of vertices $V$.
Edges never directly connect two tasks and instead edges are only ever described between a task vertex, $t$, and a data vertex, $d$
indicating that (depending on direction of the edge) data is either consumed or produced by a task.
}
}

\newglossaryentry{asynchronous}
{
  type=glossary,
  name=asynchronous,
  description={}
}

\newglossaryentry{data prefetching}
{
  type=glossary,
  name={data prefetching},
  description={Requesting data movement operations before they are needed to
    increase \gls{concurrency}}
}

\newglossaryentry{speculative execution}
{
  type=glossary,
  name={speculative execution},
  description={Potential data hazards are ignored and, in some cases, work is performed prior to
    whether or not it is known whether it will be required.  By performing the
      work speculatively, the delay associated with waiting to know whether or not the work
      was in fact required are avoided. Conflicts that are detected after the
      fact lead to rollback or recovery}
}

\newglossaryentry{AMT}
{
  type=glossary,
  name={AMT},
  first={asynchronous many-task (AMT)},
  description={See \gls{AMT model}}
}

\newglossaryentry{AMT model}
{
  type=glossary,
  name={AMT model},
  plural={AMT models},
  description={Asynchronous many-task (AMT) is a categorization of programming
    and \glspl{execution model} that break from the dominant \gls{CSP} or \gls{spmd} models.
      Different \gls{AMT RTS} implementations can share a common AMT model.
	An AMT \gls{programming model} decomposes applications into small,
  \gls{migratable} units of work (many tasks) with associated inputs (dependencies or data blocks) rather than simply decomposing at the process level (\gls{MPI} ranks).
	An AMT \gls{execution model} can be viewed as the coarse-grained, distributed memory analog of instruction-level parallelism, extending the concepts of data prefetching,
	out-of-order task execution based on dependency analysis, and even branch
    prediction (\gls{speculative execution}). 
	Rather than executing in a well-defined order, tasks execute when inputs become available.
	An AMT model aims to leverage all available \gls{task parallelism} and
  \gls{pipeline parallelism},
  rather than rely solely on \gls{data parallelism} for \gls{concurrency}.
	The term asynchronous encompasses the idea that 1) processes (threads) can
  diverge to different tasks, rather than executing in the same order; and 2)
  \gls{concurrency} is maximized (minimum synchronization) by leveraging multiple forms of parallelism.
	The term {\em many-task} encompasses the idea that the application is
  decomposed into many \gls{migratable} units of work, to enable the overlap of
  communication and computation as well as \gls{asynchronous} load balancing strategies}
}



\newglossaryentry{AMT RTS} % amt rts vs amt runtime
{
  type=glossary,
  name={AMT RTS},
  first={asynchronous many-task runtime system (AMT RTS)},
  description={A runtime system based on \gls{AMT} concepts. An AMT RTS provides a specific implementation of an \gls{AMT model}}
}

\newglossaryentry{imperative}  %linked
{
  type=glossary,
  name={imperative},
  description={A style of programming where statements change the state of a program to produce a specific result. This contrasts to declarative programming that focuses on defining the desired result without specifying how the result is to be accomplished}
}
\newglossaryentry{declarative} %linked
{
  type=glossary,
  name={declarative},
  description={A style of programming that focuses on using statements to define what a program should accomplish rather than how it should accomplish the desired result}
}
\newglossaryentry{procedural} 
{
  type=glossary,
  name={procedural},
  description={A style of programming where developers define step by step instructions to complete a given function/task. A procedural program has a clearly defined structure with statements ordered specifically to define program behavior}
}
\newglossaryentry{functional} 
{
  type=glossary,
  name={functional},
  description={A style of programming that treats computation as the
    evaluation of mathematical functions and avoids changing-state and mutable
      data}
}


\newglossaryentry{in-situ} 
{
  type=glossary,
  name={in-situ},
  description={\textit{In-situ} analysis involves analyzing data \emph{on site} or \emph{in place} where it was generated, in contrast to \gls{in-transit} which first migrates data to another physical location}
 %This method does however have side-effects with regards to application performance and scalability for some computational and memory intensive applications
}

\newglossaryentry{in-transit}
{
  type=glossary,
  name={in-transit},
  description={In-transit analysis is a method for performing analysis on an applications raw computational data while the application is running by offloading the simulation data to a set of processing units allocated for data analytics. Typically, this method involves more network communication and requires a balance between the compute hardware running the application and analysis but allows an application to resume its computations faster. This contrasts with \textit{\gls{in-situ}} analysis that operates on data in-place}
}

\newglossaryentry{Read-After-Write} 
{
  type=glossary,
  name={Read-After-Write},
  description={Read after write (\gls{RAW}) is a standard data dependency (or potential hazard) where one instruction or task requires, as an input, a data value that is computed by some other instruction or task}
}

\newglossaryentry{Write-After-Read}
{
  type=glossary,
  name={Write-After-Read},
  description={Write after read (\gls{WAR}), also known as an anti-dependency, is a potential data hazard where a task or instruction has required input(s) that are later changed. An anti-dependency can be removed at instruction-level through register renaming or a task-level through copy-on-read or copy-on-write}
}

\newglossaryentry{Write-After-Write}
{
  type=glossary,
  name={Write-After-Write},
  description={Write after write (\gls{WAW}), also known as an output
  dependency, is a potential data hazard where data dependence is only
  written (not read) by two or more tasks. In a sequential execution, the value
  of the data will be well defined, but in a parallel execution, the
  value is determined by the execution order of the tasks writing the
  value}
}

\newglossaryentry{data-flow dependency}
{
  type=glossary,
  name={data-flow dependency},
  plural={data-flow dependencies},
  description={A data dependency where a set of tasks or instructions require a certain sequence to complete without causing race conditions. Data-flow dependency types include \gls{Write-After-Read}, \gls{Read-After-Write} and \gls{Write-After-Write}}
}

\newglossaryentry{anti-dependency}
{
  type=glossary,
  name={anti-dependency},
  plural={anti-dependencies},
  see={[Glossary:]{Write-After-Read}},
  description={See \gls{Write-After-Read}}
}

\newglossaryentry{chare}
{
  type=glossary,
  name={chare},
  plural={chares},
  description={The basic unit of computational work within the \Charm{} framework. Chares are essentially \protect\CC{} objects that contain methods that carry out computations on an objects data asynchronously from the method's invocation}
}
\newglossaryentry{RPC} 
{
  type=glossary,
  name={RPC},
  plural={RPCs},
  first={remote procedure calls (RPC)},
  description={Remote Procedure Call (RPC) is a protocol that one
  program can use to request a service from a program located in
  another computer in a network without having to understand network
  details. RPC uses the client/server model}
}

\newglossaryentry{DARMA} 
{
  type=glossary,
  name={DARMA},
  first={DARMA (Distributed Asynchronous Resilient Models and Applications)},
  description={ DARMA is an AMT portability layer serving as a vehicle for
    community-based co-design activities.  The layer aims to 1) insulate applications from runtime system and hardware idiosyncrasies,
2) improve AMT runtime programmability by co-designing an API directly with
  application developers, 3) synthesize application co-design activities into
  meaningful requirements for runtimes, and 4) 
facilitate AMT design space characterization and definition, accelerating the development of AMT best practices}
}

\newglossaryentry{remote procedure invocation} 
{
  type=glossary,
  name={remote procedure invocation},
  plural={remote procedure invocations},
  see={[Glossary:]{RPC}},
  description={See RPC}
}

\newglossaryentry{active message passing}
{
  type=glossary,
  name={active message passing},
  description={An Active message is a messaging object
  capable of performing processing on its own. It is a lightweight
  messaging protocol used to optimize network communications with an
  emphasis on reducing latency by removing software overheads
  associated with buffering and providing applications with direct
  user-level access to the network hardware. This contrasts with
  traditional computer-based messaging systems in which messages are
  passive entities with no processing power}
}

\newglossaryentry{zero-copy} %linked
{
  type={glossary},
  name={zero-copy},
  description={Zero-copy transfers are data transfers that occur directly from send to receive location without any additional buffering. Data is put immediately on the wire on the sender side and stored immediately in the final receive buffer off the wire on the receiver side. This usually leverages \gls{RDMA} operations on pinned memory}
}

\newglossaryentry{load balancing} %couldn't find any references
{
  type={glossary},
  name={load balancing},
  description={Load balancing distributes workloads across multiple
  computing resources. Load balancing aims to optimize resource use,
  maximize throughput, minimize response time, and avoid overload of
  any single resource. Using multiple components with load balancing
  instead of a single component may increase reliability and
  availability through redundancy}  
}

\newglossaryentry{work stealing} %linked
{
  type={glossary},
  name={work stealing},
  description={The act of one computational unit (thread/process), which has completed it's workload, taking some task/job from another computational unit. This is a basic method of distributed load balancing}
}
\newglossaryentry{task stealing} %linked
{
  type={glossary},
  name={task stealing},
  see={[Glossary:]{work stealing}},
  description={See \gls{work stealing}}
}
\newglossaryentry{logical region} %linked
{
  type={glossary},
  name={logical region},
  plural={logical regions},
  description={A collection of objects operated on by a task. Various parameters define the behavior of logical region when operated on by a given task including privilege, coherence and behavior}
}
\newglossaryentry{privilege} %linked
{
  type={glossary},
  name={privilege},
  plural={privileges},
  description={A Legion parameter that defines the side-effects a task will have on a given logical region. These side-effects could include read, write or reduction permissions}
}
\newglossaryentry{coherence} %linked
{
  type={glossary},
  name={coherence},
  description={An input parameter within the Legion runtime that determines the types of manipulations one function can do to another function's logical region}
}
\newglossaryentry{patch} %linked
{
  type={glossary},
  name={patch},
  plural={patches},
  description={A unit of data within a structured mesh involved with discretizing workloads into data-parallel segments. Data segments takes the form of cells that can contain particles and/or member data. As the basic unit of parallel work, Uintah uses computations in the form of tasks over a single patch to express parallelism}
}

% Added by Greg Sjaardema
\newglossaryentry{MPI+X}
{
  type={glossary},
  name={MPI+X},
  description={A hybrid \gls{programming model} combining \gls{MPI} and
  another parallel \gls{programming model} in the same application. The
  combination may be mixed in the same source or combinations of
  components or routines, each of which is written in a single parallel
  \gls{programming model}. \gls{MPI}+Threads or \gls{MPI}+OpenMP are the most
  common hybrid models involving \gls{MPI}. \gls{MPI} describes the
  parallelism between processes (with separate memory address spaces) and
  the ``X'' typically provides parallelism within a process (typically
  with a shared \gls{memory model})}
}

% Added by Greg Sjaardema
\newglossaryentry{PSAAP-II}
{
  type={glossary},
  name={PSAAP-II},
  first={Predictive Science Academic Alliance Program II (PSAAP-II)},
  description={The primary goal of the \gls{NNSA}'s Predictive Science Academic
  Alliance Program (PSAAP) is to establish validated, large-scale,
  multidisciplinary, simulation-based ``Predictive Science'' as a
  major academic and applied research program. The Program Statement
  lays out the goals for a multiyear program as follow-on to the
  present ASC Alliance program. This ``Predictive Science'' is the
  application of verified and validated computational simulations to
  predict properties and dynamics of complex systems. This process is
  potentially applicable to a variety of applications, from nuclear
  weapons effects to efficient manufacturing, global economics, to a
  basic understanding of the universe. Each of these simulations
  requires the integration of a diverse set of disciplines; each
  discipline in its own right is an important component of many
  applications. Success requires both software and algorithmic
  frameworks for integrating models and code from multiple disciplines
  into a single application and significant disciplinary strength and
  depth to make that integration effective}
}

% Added by Greg Sjaardema
\newglossaryentry{ASC}
{
  type={glossary},
  name={ASC},
  first={Advanced Simulation and Computing (ASC)},
  description={The Advanced Simulation and Computing (ASC) Program
  supports the Department of Energy's National Nuclear Security
  Administration (NNSA) Defense Programs' shift in emphasis from
  test-based confidence to simulation-based confidence. Under ASC,
  computer simulation capabilities are developed to analyze and
  predict the performance, safety, and reliability of nuclear weapons
  and to certify their functionality. ASC integrates the work of three
  Defense programs laboratories (Los Alamos National Laboratory,
  Lawrence Livermore National Laboratory, and Sandia National
  Laboratories) and university researchers nationally into a
  coordinated program administered by NNSA}
}

% Added by Greg Sjaardema
\newglossaryentry{PIM}
{
  type={glossary},
  name={processing in memory},
  description={Processing in memory (PIM) is the concept of placing
  computation capabilities directly in memory. The PIM approach can
  reduce the latency and energy consumption associated with moving
  data back-and-forth through the cache and memory hierarchy, as well
  as greatly increasing memory bandwidth by sidestepping the
  conventional memory-package pin-count limitations}
}

% Added by Greg Sjaardema
\newglossaryentry{scratchpad}
{
  type={glossary},
  name={scratchpad},
  description={Scratchpad memory, also known as scratchpad, scratchpad
  RAM or local store in computer terminology, is a high-speed internal
  memory used for temporary storage of calculations, data, and other
  work in progress. In reference to a microprocessor, scratchpad
  refers to a special high-speed memory circuit used to hold small
  items of data for rapid retrieval. It can be considered similar to
  the L1 cache in that it is the next closest memory to the \gls{ALU} after
  the internal registers, with explicit instructions to move data to
  and from main memory, often using \gls{DMA}-based data transfer. In
  contrast to a system that uses caches, a system with scratchpads is
  a system with \gls{NUMA} latencies, because the memory access
  latencies to the different scratchpads and the main memory
  vary. Another difference from a system that employs caches is that a
  scratchpad commonly does not contain a copy of data that is also
  stored in the main memory}
}

% Added by Greg Sjaardema
\newglossaryentry{SoC}
{
  type={glossary},
  name={SoC},
  first={system-on-chip},
  description={A system on a chip or system on chip (SoC or SOC) is an
  integrated circuit (IC) that integrates all components of a computer
  or other electronic system into a single chip. It may contain
  digital, analog, mixed-signal, and often radio-frequency
  functions--all on a single chip substrate. The System on Chip
  approach enables HPC chip designers to include features they need,
  and exclude features that are not required in a manner that is not
  feasible with today's commodity board-level computing system
  design. SoC integration is able to further reduce power, increase
  integration density, and improve reliability. It also enables
  designers to minimize off-chip I/O by integrating peripheral
  functions, such as network interfaces and memory controllers by
  integrating the components onto a single chip}
}

\newglossaryentry{POD}
{
  type={glossary},
  name={POD},
  first={plain old data},
  description={In \CC{}, POD stands for Plain Old Data---that is, a
  class or struct without constructors, destructors and virtual
  members functions and all data members of the class are also POD}
}

\newglossaryentry{multi-level memory}
{
  type={glossary},
  name={multi-level memory},
  plural={multi-level memories},
  description={A hybrid memory system that integrates multiple types
  of memory components with different sizes, bandwidths, and access
  methods. There may be two or more levels with each level composed of
  a different memory technology, such as NVRAM, DRAM, 3D Stacked, or
  other memory technologies. This is an extension of the L1, L2, and
  L3 cache memory systems of current \gls{CPU} architectures. As a result,
  future application analysis must account for complexities created by
  these multi-level memory systems with or without coherency. Despite
  the increased complexity, the performance benefits of such a system
  should greatly outweigh the additional burden in programming brought
  by multi-level memory. For instance, the amount of data movement
  will be reduced both for cache memory and scratch space resulting in
  reduced energy consumption and greater performance~\cite{AbstractMachine}}  
}

\newglossaryentry{exascale}
{
  type={glossary},
  name={exascale},
  description={Exascale computing refers to computing systems capable
  of at least one exaFLOPS, or a billion billion ($10^{18}$)
  calculations per second. Such capacity represents a thousandfold
  increase over the first petascale computer that came into operation
  in~2008. The \gls{DOE} is planning to develop and deliver capable
  exascale computing systems by 2023-24. These systems are expected to
  have a one-hundred to one-thousand-fold increase in sustained
  performance over today's computing capabilities, capabilities
  critical to enabling the next-generation computing for national
  security, science, engineering, and large-scale data analytics.
  Leadership in \gls{HPC} and large-scale data analytics will advance
  national competitiveness in a wide array of strategic sectors. An
  integrated government-industry-academia approach to the development
  of hardware, system software, and applications software, will be
  required to overcome the barriers of power efficiency, massive
  parallelism, and programmability to attain maximum benefit from
  exascale computers}
}

\newglossaryentry{MIC}
{
  type={glossary},
  name={MIC},
  plural={MICs},
  first={Many Integrated Core Architecture (MIC)},
  description={Intel Many Integrated Core Architecture or Intel MIC
  is a coprocessor computer architecture developed by Intel
  incorporating earlier work on the Larrabee many core architecture,
  the Teraflops Research Chip multicore chip research project, and the
  Intel Single-chip Cloud Computer multicore microprocessor. Prototype
  products codenamed Knights Ferry were announced and released to
  developers in~2010. The Knights Corner product was announced in~2011
  and uses a 22~nm process. A second generation product codenamed
  Knights Landing using a 14~nm process was announced in June~2013.
  Xeon Phi is the brand name used for all products based on the
  Many Integrated Core architecture}  
}

\newglossaryentry{RDMA}
{
  type={glossary},
  name={RDMA},
  first={remote direct memory access (RDMA)},
  description={Remote direct memory access (RDMA) is a direct memory
  access from the memory of one computer into that of another without
  involving either one's operating system. This permits
  high-throughput, low-latency networking, which is especially useful
  in massively parallel computing}
}

\newglossaryentry{Concept}
{
  type={glossary},
  name={concept},
  plural={concepts},
  description={A concept is a description of the supported operations on a type
  to be used in generic programming.  In C++, there is no language level support
  for concepts (yet), but the idea can still be applied to C++ templates and
  deduced types in the context of API specification.  DARMA performs most of
  its concept checking using the \inlinecode{void_t} detection idiom [TODO
  CITATION HERE].}
}


\newacronym{ACES}{ACES}{Advanced Computing at Extreme Scale}
\newacronym{AMR}{AMR}{adaptive mesh refinement}
\newacronym{CFD}{CFD}{computational fluid dynamics}
\newacronym{CPU}{CPU}{central processing unit}
\newacronym{CLE}{CLE}{Cray Linux Environment}
\newacronym{CUDA}{CUDA}{Compute Unified Device Architecture}
\newacronym{DOE}{DOE}{U.~S.~Department of Energy}
\newacronym{GPU}{GPU}{graphics processor unit}
\newacronym{HAAP}{HAAP}{Heterogeneous Advanced Architecture Platform}
\newacronym{HPC}{HPC}{high-performance computing}
\newacronym{LANL}{LANL}{Los Alamos National Laboratory}
\newacronym{LLNL}{LLNL}{Lawrence Livermore National Laboratory}
\newacronym{MPI}{MPI}{Message Passing Interface}
\newacronym{NERSC}{NERSC}{National Energy Research Scientific Computing Center}
\newacronym{NNSA}{NNSA}{National Nuclear Security Administration}
\newacronym{NUMA}{NUMA}{non-uniform memory access}
\newacronym{RMCRT}{RMCRT}{Reverse Monte Carlo Ray Tracing}
\newacronym{SNL}{SNL}{Sandia National Laboratories}
\newacronym{NTV}{NTV}{near-threshold voltage}
\newacronym{RAW}{RAW}{\gls{Read-After-Write}}
\newacronym{WAR}{WAR}{\gls{Write-After-Read}}
\newacronym{WAW}{WAW}{\gls{Write-After-Write}}
\newacronym{PDE}{PDE}{partial differential equation}
\newacronym{DMA}{DMA}{direct memory access}
\newacronym{ALU}{ALU}{arithmetic logic unit}
\newacronym{AVX}{AVX}{Advanced Vector Extensions}
\newacronym{SSE}{SSE}{Streaming SIMD Extensions}

