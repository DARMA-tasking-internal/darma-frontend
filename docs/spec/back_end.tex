%!TEX root = sandReportSpec.tex

\chapter{Backend}
\label{chap:back_end}

\lstMakeShortInline[style=CppCodeInlineStyle]{\|}

The \gls{back end} \gls{API} is organized into two namespaces:
\begin{compactitem}
\item {\ttfamily darma\_runtime::abstract::frontend} 
\item {\ttfamily darma\_runtime::abstract::backend}  
\end{compactitem}

The first contains
abstraction base classes of entities that are implemented in the \gls{translation
layer} and are the only constructs in that layer that the \gls{back end}
\gls{runtime system} is
allowed to interact with.  The second contains abstract base classes that must
be concretely implemented in the \gls{back end} \gls{runtime system} and
are the only \gls{back end} abstractions the \gls{translation layer} is allowed to interact
with.  Below is a summary of the requirements to implement these abstractions,
the documentation for which is taken from the Doxygen-style comments in
the source code itself.  As such, the source code may be a better resource for
those interested in this part of the document, but we have included it here for
completeness. 

Before diving into the API iteslf, though, we'll give a brief overview of the abstractions and
semantics that DARMA uses to express the dependency structure of the user's program to the
backend.  It should be clear by now that a primary goal of DARMA is to translate from semantics and
abstractions that are most convenient for application developers into semantics and abstractions
that are most convenient for runtime system developers.  Chapter~\ref{chap:front_end} details the
abstractions we've devised to address the needs of application developers, and this chapter details
those devised for interaction with runtime system developers.  As such, it is critical that we
recieve feedback on the usefulness (or awkwardness) of these abstractions when they are used to 
interpret the user's execution needs for an existing AMT \gls{runtime system}.

\section{{\tt Use}--{\tt Flow} Semantics}

We are calling the general semantics that we use to express the graph of application tasks and
dependencies {\it Use-Flow semantics}.  Roughly speaking, a |Flow| is a data node on the data-task
directed acyclic graph (DAG) and a |Use| (or, more accurately, a set of |Use|s) is a task node.  In
both cases, however, the analogy is only a rough one, and because of the intricacies and
difficulties involved in representing a DAG that is simultaneously being executed and constructed
(or ``discovered''), the exact semantics on |Use| and |Flow| objects are significatly more nuanced.

\subsection{{\tt Flow} Objects, In Brief}

A |Flow| is an object, issued by the backend and completely opaque to the translation layer, that
the translation layer uses to represent the relationships between |Use| objects.  It is exposed to
the translation layer via the |typedef| |flow_t|, which must support pointer-like copy semantics
(i.e., |flow_t a = b| should not make a copy of |b|; it should just make |a| point to the same
thing |b| points to).  Otherwise, there are no specific requirements on the implementation of
|flow_t|; its form is entirely up to the backend implementer.

The life cycle of a |Flow| object begins with a call by the translation layer to one of the
|make_*_flow| methods of |abstract::backend::Runtime| (a pointer to a valid instance of which must
be returned when called from the free function |abstract::backend::get_backend_runtime()| any
context after the frontend returns from the backend's call to |frontend::darma_top_level_setup()|
until the end of the program).  In the current specification, these
include |make_initial_flow()|, |make_fetching_flow()|, |make_null_flow()|, |make_next_flow()|,
and |make_forwarding_flow()|.  Its life cycle ends with a call by the translation layer to
the |release_flow()| method of |Runtime|, the invocation of which must be preceded by the 
end of the lifetime of any |Use|s that use that |Flow| and any (zero or one) {\it aliases}
established on that |Flow| (both of these will be explained in more detail below, after the
introduction of |Use|s).

\subsection{{\tt Use} Objects, In Brief}

A |Use| is an object created and managed by the translation layer, the abstract interface to which
is exposed in |abstract::frontend::Use|.  


%\section{Important Backend Concepts}
%Although some of this terminology was given in the introduction, we repeat definitions here.
%Some of the terms here have \CC{} classes that directly represent them.
%Other terms are only concepts, useful in illustrating the use of other \CC{} classes.
%\begin{compactdesc}
%\item [Task:] The work unit instantiated directly by the application developer. 
%Tasks are guaranteed to make forward progress, but are interruptible.
%\item [Execution stream:] An \gls{execution stream} will consist of a sequence of
%  many \glspl{task}, and, like \glspl{task}, is guaranteed to make forward progress.
%  All \glspl{execution stream} are \glspl{task}, but \glspl{execution stream} specifically
%  have no parent \gls{task} and are the root of an independent \gls{task-DAG}.
%  There is no class corresponding uniquely to an \gls{execution stream} since all
%streams are \glspl{task}.
%\item [Operation:] This is a unit of execution that is guaranteed to be non-interruptible. 
%  An \gls{operation} is not equivalent to a \gls{task} since \glspl{task} are interruptible.  
%  \Glspl{operation} are the smallest, schedulable units of work.  
%A \gls{task} consists of a sequence of \glspl{operation}.
%While \glspl{task} are explicitly instantiated by the application developer,
%\glspl{operation}
%(individual portions of \gls{task}) can be implicitly instantiated by the
%\gls{runtime system}.
%There is no class provided corresponding directly to an \gls{operation}.
%Since only one component \gls{operation} of a \gls{task} may be active at any given
%time, a \gls{task} always corresponds uniquely to an \gls{operation}.
%\item [Handle:] The \gls{DARMA} generalization of a variable. Handle encapsulates both a unique immutable name (key) and an immutable type. 
%\item [Logical Time:] An abstract notion of time progressing as
%  \glspl{operation} are performed on the values encapsulated by a Handle.
%There is no class corresponding to logical time. 
%The progression of logical time for Use objects is encapsulated in the input and output Flows (see below).
%\item [Use:] A Use represents intended access to a Handle at a particular moment in logical time. 
%Uses carry particular immediate and scheduling permissions and therefore have some intent of Read, Write, or Modify (and/or some intent to schedule tasks that require these).
%  Uses are always unique to an \gls{operation}.
%Operations cannot add or remove Use instances from its context;
%however, tasks, being interruptible, can add and remove (i.e., create and release) Use instances.
%\item [Flow:] A Flow encapsulates a data-task relationship. 
%An input Flow indicates that a Use requires a particular value before its
%corresponding \gls{operation} begins.
%An output Flow indicates that a Use produces a particular value after being
%released at the end of its corresponding \gls{operation}.
%All Use objects have exactly two Flow objects --- one input and one output ---
%and each Flow is associated with exactly one Use.
%A Modify Use will have an input Flow indicating the value consumed and an output Flow indicating the value produced.
%A Read Use will also have an output Flow even though it produces no data since the ``output'' indicates the release of data and clearing of an anti-dependence.
%\item [Dependency:] Although Dependency is not a class in \gls{DARMA}, a
%  \gls{task} will always have an initial set of Uses whose input Flows must become
%  available for the \gls{task} to begin.
%  This initial set of uses are the ``dependencies'' of a \gls{task}.
%\end{compactdesc} 

\lstDeleteShortInline{\|}

\subimport{./doxygen-backend/latex/}{refman}
