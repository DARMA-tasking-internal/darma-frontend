%!TEX root = sandReportSpec.tex

\chapter{Backend}
\label{chap:back_end}

The back end API is organized into two namespaces:
\begin{compactenum}
\item {\ttfamily darma\_runtime::abstract::frontend} 
\item {\ttfamily darma\_runtime::abstract::backend}  
\end{compactenum}

The first contains
abstraction base classes of entities that are implemented in the translation
layer and are the only constructs in that layer that the backend is
allowed to interact with.  The second contains abstract base classes that must be concretely implemented in the backend and
are the only backend abstractions the translation layer is allowed to interact
with.  Below is a summary of the requirements to implement these abstractions,
the documentation for which is taken from the Doxygen-style comments in
the source code itself.  As such, the source code may be a better resource for
those interested in this part of the document, but we have included it here for
completeness. \todo{DAVID: add coments here about key and version going away} 

\section{Important Backend Concepts}
Although some of this terminology was given in the introduction, we repeat definitions here.
Some of the terms here have C++ classes that directly represent them.
Other terms are only concepts, useful in understanding the use of other C++ classes.
\begin{compactdesc}
\item [Task]: The work unit instantiated directly by the application developer. 
Tasks are guaranteed to make forward progress, but are interruptible.
\item [Execution stream]: An execution stream will consist of a sequence of many tasks, and, like tasks, is guaranteed to make forward progress.
All execution streams are tasks, but execution streams specifically have no parent task and are the root of an independent task-DAG.
There is no class corresponding uniquely to an execution stream since all streams are tasks.
\item [Operation]: This is a unit of execution that is guaranteed to be non-interruptible. 
An operation is not equivalent to a task since tasks are interruptible.  
Operations are the smallest, schedulable units of work.  
A task consists of a sequence of operations.
While tasks are explicitly instantiated by the app developer, operations (individual portions of task) can be implicitly instantiated by the runtime.
There is no class provided corresponding directly to an operation.
Since only one component operation of a task may be active at any given time, a task always corresponds uniquely to an operation.
\item [Handle]: The DARMA generalization of a variable. Handle encapsulates both a unique immutable name (key) and an immutable type. 
\item [Logical Time]: An abstract notion of time progressing as operations are performed on the values encapsulated by a Handle.
There is no class corresponding to logical time. 
The progression of logical time for Use objects is encapsulated in the input and output Flows (see below).
\item [Use]: A Use corresponds to a Handle at a particular moment in logical time. 
Uses are always unique to an operation.
Operations cannot add or remove uses from its context.
Tasks, being interruptible, can add and remove Use instances.
Uses carry particular permissions and therefore have some intent of Read, Write, or Modify.
\item [Flow]: A Flow encapsulates a data-task relationship. 
An input Flow indicates that a Use requires a particular value before its corresponding operation begins.
An output Flow indicates that a Use produces a particular value after being released at the end of its corresponding operation.
A Modify Use will have an input Flow indicating the value consumed and an output Flow indicating the value produced.
A Read Use is only required to have an input Flow while a Write Use will have an an output Flow.
\item [Dependency]: Although Dependency is not a class in DARMA, a task will always have an initial set of Uses that must become
available for the task to begin.
This initial set of uses are the ``dependencies'' of a task.
\end{compactdesc} 

\subimport{./doxygen-backend/latex/}{refman}
