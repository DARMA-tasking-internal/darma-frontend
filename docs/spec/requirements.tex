%!TEX root = sandReportSpec.tex

\chapter{Requirements} 
\label{chap:requirements} 
\section{High-level Philosophy}
The \gls{front end} \gls{API} requirements are informed by a few high-level design principles:
\begin{compactitem}
\item Keep simple things simple
\item Keep tractable things tractable
\item Make difficult things tractable
\item New \glspl{programming model} should not complicate reasoning about code correctness
\item New \glspl{programming model} should simplify application-specific performance optimizations
\item Pareto rule: 80\% of the compute benefit from modest human effort preferred over 100\% of compute benefit from massive human effort
\end{compactitem}

Essentially, code written in the \gls{DARMA} \gls{programming model} should be
not be more difficult than existing \glspl{programming model}.
Additionally, the \gls{DARMA} \gls{programming model} should not pass off 100\% 
of the responsibility for high-performance to the runtime/compilers.
Rather, \gls{DARMA} should enable application developers to express performance improvements in ways not previously possible.

Our approach is informed by what we see as the ``axiomatic'' challenges facing high-performance computing:
\begin{compactitem}
\item \gls{spmd} (\gls{data parallelism}) will remain the dominant parallelism
  and primary structure of application codes
\item New architectures will have too much compute capacity for basic \gls{data
  parallelism} to fill
\item \Gls{task parallelism} and \gls{pipeline parallelism} will help ``fill'' the compute capacity on machines
\item The traditional \gls{abstract machine model} (flat memory spaces, uniform compute elements) will get further from 
  actual system architecture as accelerators and deep memory hierarchies become more commonplace
\item Applications with dynamic load balance or dynamic sparsity will require composable, migratable chunks of work
\end{compactitem}

\section{Application Requirements for the front end API}
Based on \gls{co-design} efforts with application and \gls{runtime system}
development teams, the following \gls{DARMA} \gls{front end} \gls{API}
requirments have been identified:
\begin{compactitem}
\item The \gls{DARMA} \gls{front end} \gls{API} must enable the development and
  deployment of \gls{spmd} algorithms in an intuitive and simple way.
\item The \gls{DARMA} \gls{front end} \gls{API} must not limit the ability of
  hte application developer to use their own data structures.
\item The \gls{DARMA} \gls{front end} \gls{API} must support collective
  communication operations.
\item The \gls{DARMA} \gls{front end} \gls{API} must not limit the ability of
      the application developer to express and control the initial problem
      decomposition.
\item The \gls{DARMA} \gls{front end} \gls{API} must not limit the application
  developer's ability to mix and express all forms of parallelism.
\end{compactitem}

\section{Back end runtime system requirements}
Althouth a primary purpose of the \gls{DARMA} specification is to provide a
\gls{back end} \gls{runtime system} specification that is relatively execution
model agnostic, we will synthesize our application and runtime-system co-design
activities into a list of \gls{back end} \gls{runtime system} requirements.  To
date, the following requirements have been identified:
\begin{compactitem}
\item A \gls{DARMA}-compliant \gls{runtime system} must support an efficient
  \gls{spmd} launch of an application code.
\item A \gls{DARMA}-compliant \gls{runtime system} must not limit the ability
  of the application developer to use their own data structures.
\item A \gls{DARMA}-compliant \gls{runtime system} must efficiently implement
  distributed, key-value-style coordination between multiple streams of
  execution.
%\item A \gls{DARMA}-compliant \gls{runtime system} must support the allocation
%  and arbitration of resources to support \glspl{elastic task}.
\end{compactitem}


\section{Co-design contributors}
In addition to the authors listed on this document,
the \gls{API} is being \gls{co-design}ed and vetted with application developers
and computer scientists whose knowledge spans the entire runtime software stack.

\paragraph{Applications affecting the design and requirements:}
\begin{compactitem}
\item Sandia \gls{ASC} \gls{ATDM} electromagnetic plasma code (POC: Matt Bettencourt)
\item Sandia \gls{ASC} \gls{ATDM} reentry code (POCs: Micah Howard, Steve Bova)
\item \gls{Trilinos} Phalanx package for finite element matrix assembly (POC: Roger Pawlowski)
\item Uncertainty quantification driver (POCs: Eric Phipps, Francesco Rizzi)
\item Domain decomposition preconditioners for linear solvers (POCs: Ray Tuminaro, Clark Dohrman)
\end{compactitem}
 
\paragraph{Computer Science Research Efforts}
\begin{compactitem}
\item Kokkos (POCs: Carter Edwards, Christian Trott)
\item Data Management (POCs: Craig Ulmer, Gary Templet)
\item Low-level operating systems requirements (POCs: Stephen Olivier, Ron
    Brightwell)
\end{compactitem}



