%!TEX root = sandReportSpec.tex

\appendix
\newcommand{\specialcell}[2][c]{%
  \begin{tabular}[#1]{@{}c@{}}#2\end{tabular}}


\chapter{Handles Usage Rules}

\todo[inline]{FR: Placeholder 
just to write text for explaining 
rules on using handles}


\section{Overview}
Handles are assigned states, and these states change 
based on the operations applied to them. In other words, 
handles' states transition. However, not all states 
are allowed at all times. The ``permissions'' on 
what it is allowed or not changes based on the 
context. 
Permissions can refer to: 
\begin{itemize}
\item Scheduling: permissions on a handle 
within a \inlinecode{create_work}.
\item Immediate: permissions that apply immediately.
\end{itemize}

For spec-0.2, there are two main types of handle:
\begin{itemize}
\item \inlinecode{initial_access<>}: when a handle of 
this type is first created, it is assigned 
``Modify/None'' permissions.
%
\item \inlinecode{read_access<>}: when a handle of 
this type is first created, it is assigned 
``Read/None'' permissions.
\end{itemize}


Example: 
\hspace{-0.75cm}
\begin{minipage}[t]{0.45\linewidth}%
\centering
\begin{vaspPseudo}
initial_access<int> a;
//a is in Modify/None
a.set_value(1); //NOT allowed
a.get_value();  //NOT allowed
create_work([=]
{
    a.set_value(1); //allowed
});
\end{vaspPseudo}
\end{minipage}
\hspace{0.55cm}
\begin{minipage}[t]{0.45\linewidth}
\centering
\begin{vaspPseudo}
read_access<int> b;
b.get_value(); //NOT allowed
create_work([=]
{
    b.set_value(1); //allowed
});
\end{vaspPseudo}
\end{minipage}


\begin{table}[!h]
\begin{center}
{\small
\begin{tabular}{cc|cc|cc|cc|cc}
 \multicolumn{2}{c|}{\inlinecode{}}
 & \multicolumn{2}{c|}{\inlinecode{get_value()}} 
 & \multicolumn{2}{c|}{\inlinecode{set_value()}} 
 & \multicolumn{2}{c|}{\inlinecode{mark_read_only()}} 
 & \multicolumn{2}{c}{\inlinecode{release()}} \\
 \hline
 \specialcell{Scheduling\\ permissions} 
 & \specialcell{Immediate\\ permissions}  
 & { {\footnotesize Allowed? } } \hspace{-0.75cm} & { {\footnotesize After } }
 & { {\footnotesize Allowed? } } \hspace{-0.75cm} & { {\footnotesize After } }
 & { {\footnotesize Allowed? } } \hspace{-0.5cm} & { {\footnotesize After } }
 & { {\footnotesize Allowed? } } \hspace{-0.5cm} & { {\footnotesize After } }\\
 \hline
 None & None
 & No & -
 & No & -
 & No & -
 & Yes${}^*$ & {\em None/None} \\
 Read & None
 & No & -
 & No & -
 & Yes${}^*$ & {\em Read/None}
 & Yes & {\em None/None} \\
 Read & Read
 & Yes & {\em Read/Read}
 & No & -
 & Yes${}^*$ & {\em Read/Read}   
 & Yes & {\em None/None}   \\
 Modify & None
 & No & -
 & No & -
 & Yes & {\em Read/None} 
 & Yes & {\em None/None}   \\
 Modify & Read
 & Yes & {\em Modify/Read}  
 & No & -
 & Yes & {\em Read/Read} 
 & Yes & {\em None/None}   \\
 Modify & Modify
 & Yes & {\em Modify/Modify}  
 & Yes & {\em Modify/Modify}  
 & Yes & {\em Read/Read}  
 & Yes & {\em None/None}   \\
\end{tabular}
}
\caption{Immediate operations on the various states. 
State transitions marked with an asterisk (*) effectively
represent no-ops and could generate warnings.}
\label{tab:immsimp}
\end{center}
\end{table}
%
