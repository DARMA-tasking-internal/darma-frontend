%!TEX root = sandReportSpec.tex

% \appendix
\newcommand{\specialcell}[2][c]{%
  \begin{tabular}[#1]{@{}c@{}}#2\end{tabular}}
\newcommand{\yes}{\text{\ding{51}}}
\newcommand{\no}{\text{\ding{55}}}



\chapter{Handles Usage Rules}

\todo[inline]{FR: Placeholder 
just to write text for explaining 
rules on using handles, can be moved some other place if more useful}


\section{Overview}
Handles are assigned states, and these states change 
based on the operations applied to them. In other words, 
handles' states transition. However, not all states 
are allowed at all times. The ``permissions'' on 
what it is allowed or not changes based on the 
context. 
Permissions can refer to: 
\begin{itemize}
\item Scheduling: permissions on a handle 
within a \inlinecode{create_work} (more generally within a deferred work).
\item Immediate: permissions that apply immediately.
\end{itemize}

For spec-\specVersion, there are two main types of handle:
\begin{itemize}
\item \inlinecode{initial_access<>}: when a handle of 
this type is first created, it is assigned 
``Modify/None'' permissions.
%
\item \inlinecode{read_access<>}: when a handle of 
this type is first created, it is assigned 
``Read/None'' permissions.
\end{itemize}


Example (1): 
\hspace{-0.75cm}
\begin{minipage}[t]{0.45\linewidth}%
\centering
WRONG
\begin{vaspPseudo}
initial_access<int> a
//a is in Modify/None
a.set_value(1) $\no$ 
a.get_value()  $\no$
\end{vaspPseudo}
\end{minipage}
\hspace{0.55cm}
\begin{minipage}[t]{0.45\linewidth}
\centering
CORRECT
\begin{vaspPseudo}
initial_access<int> a
//a is in Modify/None
create_work([=]{ //modify capture
    a.emplace_value(1)  $\yes$
    a.set_value(1)      $\yes$
    a.get_reference()=1 $\yes$
});
\end{vaspPseudo}
\end{minipage}



Example (2): 
\hspace{-0.75cm}
\begin{minipage}[t]{0.45\linewidth}%
\centering
WRONG
\begin{vaspPseudo}
read_access<int> b
//b is in Read/None
b.get_value()   $\no$
b.set_value(1)  $\no$
create_work([=]{ //modify capture
  b.set_value(1) $\no$
});
\end{vaspPseudo}
\end{minipage}
\hspace{0.55cm}
\begin{minipage}[t]{0.45\linewidth}
\centering
CORRECT
\begin{vaspPseudo}
read_access<int> b
//b is in Read/None
create_work([=]{ //modify capture
  b.get_value()  $\yes$
});
\end{vaspPseudo}
\end{minipage}





\begin{table}[!h]
\begin{center}
{\small
\begin{tabular}{cc|cc|cc|cc}
 \hline
 \multicolumn{2}{c|}{\inlinecode{}}
 & \multicolumn{2}{c|}{\inlinecode{get_value()}} 
 & \multicolumn{2}{c|}
 {
    \specialcell{ \inlinecode{emplace_value()} \\ 
                  \inlinecode{set_value()}\\
                  \inlinecode{get_reference()}\\
                  } 
 } 
 & \multicolumn{2}{c}{\inlinecode{release()}} \\
 \hline
 \specialcell{Scheduling\\ permissions} 
 & \specialcell{Immediate\\ permissions}  
 & { {\footnotesize Allowed? } } \hspace{-0.cm} & { {\footnotesize Continuing as}}
 & { {\footnotesize Allowed? } } \hspace{-0.cm} & { {\footnotesize Continuing as}}
 & { {\footnotesize Allowed? } } \hspace{-0cm} & { {\footnotesize Continuing as }}\\
 \hline
 None & None
 & No & -
 & No & -
 & Yes${}^*$ & {\em None/None} \\
 %
 Read & None
 & No & -
 & No & -
 & Yes & {\em None/None} \\
 %
 Read & Read
 & Yes & {\em Read/Read}
 & No & -
 & Yes & {\em None/None}   \\
 %
 Modify & None
 & No & -
 & No & -
 & Yes & {\em None/None}   \\
 %
 Modify & Read
 & Yes & {\em Modify/Read}  
 & No & -
 & Yes & {\em None/None}   \\
 %
 Modify & Modify
 & Yes & {\em Modify/Modify}  
 & Yes & {\em Modify/Modify}  
 & Yes & {\em None/None}   \\
\hline
\end{tabular}
}
\caption{Operations on the various states. 
Transitions marked with an asterisk (*) effectively
represent no-ops and could generate warnings.}
\label{tab:immsimp}
\end{center}
\end{table}
%




\begin{table}[b]
\begin{center}
{\small
\begin{tabular}{cc|ccc|ccc}
 \hline
 \multicolumn{2}{c|}{} 
 & \multicolumn{3}{c|}{\em{read-only capture}} 
 & \multicolumn{3}{c}{\em{modify capture}}  \\
 \hline
 \specialcell{Scheduling\\ permissions} 
 & \specialcell{Immediate\\ permissions}  
 & {\footnotesize Allowed? } & {\footnotesize Captured } 
 & {\footnotesize Continuing as} 
 & {\footnotesize Allowed? } & {\footnotesize Captured } 
 & {\footnotesize Continuing as} \\
 \hline
 None & None & No & - & - & No & - & - \\
 %
 Read & None 
 & Yes 
 & {\em Read/Read}
 & {\em Read/None}
 & No
 & -
 & - \\
 %
 Read & Read
 & Yes 
 & {\em Read/Read}
 & {\em Read/Read}
 & No
 & -
 & - \\
 %
 Modify & None
 & Yes 
 & {\em Read/Read}
 & {\em Modify/None}
 & Yes
 & {\em Modify/Modify} 
 & {\em Modify/None} \\
 %
 Modify & Read
 & Yes 
 & {\em Read/Read}
 & {\em Modify/Read}
 & Yes
 & {\em Modify/Modify} 
 & {\em Modify/None} \\
 %
 Modify & Modify
 & Yes 
 & {\em Read/Read} 
 & {\em Modify/Read} 
 & Yes
 & {\em Modify/Modify}
 & {\em Modify/None} \\
\end{tabular}
}
\caption{Deferred (capturing) operations on the various states.}
\label{tab:capsimp}
\end{center}
\end{table}


