%!TEX root = sandReportSpec.tex

\section{Programming Model}
\label{sec:programming_model}
DARMA is a mixed \gls{imperative}/\gls{declarative} \gls{programming model}.
As much as possible, sequential imperative semantics are used to produce intuitive, maintainable code.
However, the ``procedural imperative'' function calls and code blocks do not execute immediately.
Rather than explicitly perform all work in program order and block on data requests,
 DARMA provides \CC-embedded task annotations allowed work to be
 deferred and performed asynchronously.
Thus DARMA is also declarative, describing the work to be done but leaving a runtime free to schedule and run the when its data \glspl{dependency} and
\glspl{anti-dependency} are met.  
Task parallelism is primarily achieved through permissions/access qualifiers
on data that enable that enable a runtime to more easily reason about which tasks are safe to run in parallel. 
\todo{elaborate on the fact that task annotations can be nested here?}
Task annotations are translated by
the DARMA front end through standard \CC\ constructs (e.g., lambdas,
  reference counted pointers) and \gls{template
metaprogramming} to expose and understand the parallelism inherent in the code.  
The \gls{translation layer} requires \CC11 standard features with a small subset of
\CC14 required for advanced features (details provided in
    Chapter~\ref{chap:translation_layer}), 
  however the \gls{front end} \gls{API} does \emph{not}
require knowledge of \CC14 to use. Furthermore, the \gls{back end} is a
simple set of abstract \CC\ classes whose functionality must be implemented
according to the specfication in Chapter~\ref{chap:back_end}.
%providing the runtime the flexibility to optimize performance and
%exploit additional parallelism when possible.   


In DARMA, \gls{spmd} is the dominant parallelism and the notion of a \gls{rank}
is maintained within the \gls{API}.   
This provides application developers a convenience mechanism for creating
the initial problem decomposition and distribution.  Immediately after launch, deferred tasks
are free to be migrated by the runtime, if it will result in
better performance. 
Within a \gls{rank}, DARMA provides \gls{sequential semantics}, 
meaning that application developers can reason about the code as
though it were being deployed sequentially within the rank.   
\todo[inline]{add comments about initial mapping and over-decomposition}


The declarative programming style of DARMA is provided in part by \gls{coordination
semantics}:  rather than explicitly move data between ranks via direct communication
(i.e.,  \inlinecode{send/recv}), processes \emph{coordinate} by putting/getting
data associated with a unique \inlinecode{key} in a
\gls{key-value store} or \gls{tuple space}.  
\Gls{coordination semantics} promote out-of-order message arrival, deferred execution, and task migration
since the app declares or describes the data it needs/produces rather than enforcing an explicit delivery mechanism.

Although not yet supported in version 0.2 of the specification, several
important features will play a role in the DARMA programming model:
\begin{compactdesc}
\item{\bf Expressive Underlying Abstract Machine Model:}
Notions of \glspl{execution space} and \glspl{memory space} will be introduced
formally in later
versions of the specification.  These abstractions (or similar ones) appear in other runtime
solutions~\cite{kokkos, others} \todo{add relevant citations here} to
address deficiencies in the abstract machine model used by runtimes that
support \gls{spmd} parallelism (i.e., uniform compute elements, flat memory
    spaces).  Using such abstractions
1) facilitates performance portable application development across 
a variety of execution spaces, and 2)
  provides finer-grained control and additional flexibility in the
  communication of policies regarding data locality and data movement. 
\todo[inline]{add discussion of data staging between memory spaces}
\item{\bf Collectives:}
Collectives will be supported by DARMA in version 0.3 of the specification.
\todo[inline]{add details}
\end{compactdesc}
