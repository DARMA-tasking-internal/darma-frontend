%!TEX root = sandReportSpec.tex

\section{Programming Model}
\label{sec:programming_model}
The focus of the application-facing frontend API is the programming model.
The programming model must provide application developers a mechanism for expressing both a \emph{correct} and a \emph{performant algorithm}.
Of highest priority is the ability for developers to write a correct algorithm.
Above all else, an application must produce correct answers.
The programming model actually demands more than just performance. 
It demands performance portability - the ability to map a single code onto high-performance execution across multiple platforms.
Performance portability demands code transformations - either at runtime or compile-time - to change execution from, e.g. a GPU-optimal execution to a CPU-optimal execution.

Correctness and performance portability must both balance the level of abstraction.
Algorithms written at too low a level are highly error-prone and not portable.
In particular, they may over-express correctness constraints.
A compiler/runtime must obey the correctness requirements (e.g. instruction order) prescribed by an application code.
Low-level code may be unnecessarily prevent the compiler or runtime from optimizing program execution.
For example, code hard-wired for a particular loop structure and data layout may perform very well on one system,
but that may prevent the compiler from tiling optimizations of the loops on other systems. 
What is the correct way and level of abstraction for programmers to 1) express constraints (correctness) while 2) still enabling the compiler/runtime to transform execution (performance portability) while also 3) allowing developers to direct (hand-optimize) execution when the compiler/runtime misses optimizations apparent to a human developer. 

As such, to enable the easy (or as easy as possible) expression of algorithms, DARMA uses sequential semantics.
When necessary, DARMA also enables the semantics of communicating sequential processes (CSPs).
All developers know how to write correct sequential codes and all MPI programmers know how to write correct CSP code.
In this way, the developers can express correctness constraints through a familiar and intuitive programming model.
The application-level API of DARMA is therefore designed to capture the minimum number of constraints required for correct execution expressed in sequential code.

Code transformations are already ubiquitous at the compiler-level.
Compilers will add, delete, swap, or reorder instructions to avoid unnecessary operations, improve data locality, or improve pipelining.
DARMA aims for a similar goal, but at runtime.
Many transformations of program execution that benefit performance will be unknowable until the program actually runs.
These dynamic optimizations occurring at runtime are much more expensive than compile-time optimizations.
Clearly, dynamically transforming an application at the level of individual instructions is not feasible.
As such, \emph{tasks} are the basis of DARMA dynamic transformations in the same way that instructions are the basis of compiler static transformations.
The issue of granularity is therefore of paramount importance in task-based model.
How should the extra flexibility of fine-grained tasks be balanced against amortizing the cost of runtime analysis?
For the time-being, the choice of a correct task granularity seems beyond the abilities of current compilers.
As such, choosing task granularity is the responsibility of the application developer.
The application-level API therefore expresses both correctness and granularity.

The translation layer bridges the programming model and the actual program execution.
The translation layer itself does not perform any transformations of the program execution from sequential order.
Rather, the translation layer interprets the sequential semantics in the application and creates events in an ``intermediate representation'' suitable for the backend runtime.
The backend API is therefore intended to communicate the algorithm at the right level of abstraction.
The program representation created by backend API calls should enable a runtime to make intelligent dynamic decisions about task order and task locality or possibly even task deletion and task replication when appropriate.

Strictly speaking, the backend API calls only generate a stream of deferred tasks (tasks with corresponding data inputs/outputs) that describes the inherent data flow.
However, the information passed from translation layer to backend is sufficient to (and intended to) support a CDAG (computational directed acyclic graph) representation of the application.
In a DAG representation, tasks are vertices (V) in a graph (G) with directed edges (E).
An edge from vertex $v_1$ to vertex $v_2$ indicates a precedence constraint.
Instead of directly defining task-task precedence constraints, DARMA generally describes task-data precedence constraints.
There are two types of vertices - tasks (T) and data (D) that compose the complete set of vertices (V).
Edges never directly connect two tasks and instead edges are only ever described between a task vertex, $t$, and a data vertex, $d$
indicating that (depending on direction of the edge) data is either consumed or produced by a task.
The \emph{task-DAG} indicating task-task precedence constraints can always be obtained form the data-flow task graph,
making the data-flow graph more general and therefore more useful for enabling runtime code transformations.
Although beyond the scope of this specification document, the interested reader will find numerous works discussing heuristics and order-preserving convex transformations of task graphs that demonstrate the utility of a coarse-grained CDAG for enabling dynamic runtime optimization of an algorithm.
We reiterate, though, that the CDAG is only a concept guiding the design of the backend API and not strictly part of the specificaion.

A final concern, not addressed in the current specification, is the issue of programmer-directed optimization.
While an abstract algorithm may make more information available to the compiler or runtime for performance-tuning transformations,
compilers and runtime schedulers may not always understand the global nature of the problem.
As such,  they may not make peformance-improving optimizations that are apparent to an application developer.
It will be critically important as runtimes develop to enable developers to steer the runtime towards optimizations when compilers or schedulers fail.

DARMA is a mixed \gls{imperative}/\gls{declarative} \gls{programming model}.
As much as possible, sequential imperative semantics are used to produce intuitive, maintainable code.
However, the ``procedural imperative'' function calls and code blocks do not execute immediately.
Rather than explicitly perform all work in program order and block on data requests,
DARMA provides \CC-embedded task annotations that allow work to be deferred and performed asynchronously.
The ability to defer work and advance ahead is what gives the backend runtime the ability to make performance-improving transformations.
Deferred execution makes DARMA also \emph{declarative}, leaving the exact control-flow up to the runtime.

Task parallelism is primarily achieved through permissions/access qualifiers
on data that enable that enable a runtime to reason about which tasks can run in parallel and which tasks are strictly ordered.
Task annotations are translated by the DARMA front end through \CC\ constructs (e.g., lambdas, reference counted pointers, template
metaprogramming) to expose and understand the parallelism inherent in the code.  
The \gls{translation layer} requires \CC11 standard features with a small subset of
\CC14 required for advanced features (details provided in Chapter~\ref{chap:translation_layer}), 
however the \gls{front end} \gls{API} does \emph{not} require knowledge of \CC14 to use. 
Furthermore, the \gls{back end} is a simple set of abstract \CC\ classes whose functionality must be implemented
according to the specfication in Chapter~\ref{chap:back_end}.
%providing the runtime the flexibility to optimize performance and
%exploit additional parallelism when possible.   


Most applications written in DARMA will likely have \gls{spmd} as the dominant parallelism.
To simplify the implementation of SPMD-structured codes, the notion of a \gls{rank} is maintained within the \gls{API}.   
This provides application developers a convenience mechanism for creating
the initial problem decomposition and distribution.  
Immediately after launch, deferred tasks are free to be migrated by the runtime, if it will result in better performance. 
Within a \gls{rank}, DARMA provides \gls{sequential semantics},  meaning that application developers can reason about the code as
though it were being deployed sequentially within the rank.   
Thus DARMA emphasizes sequential semantics, but supports a CSP model.


The ``communication'' in DARMA's CSP model is actually provided by \gls{coordination semantics}:  
rather than explicitly move data between ranks via direct communication
(i.e.,  \inlinecode{send/recv}), processes \emph{coordinate} by putting/getting data associated with a unique \inlinecode{key} in a
\gls{key-value store} or \gls{tuple space}.  
\Gls{coordination semantics} promote out-of-order message arrival, deferred execution, and task migration
since the app declares or describes the data it needs/produces rather than enforcing an explicit delivery mechanism.
The key to performance in the DARMA CSP model are exploiting zero-copy mechanisms and tuple caching that enable a key-value store programming model
to produce execution equivalent to an MPI send/recv code. 

Although not yet supported in version 0.3 of the specification, several
important features will play a role in the DARMA programming model:
\begin{compactdesc}
\item{\bf Expressive Underlying Abstract Machine Model:}
Notions of \glspl{execution space} and \glspl{memory space} will be introduced formally in later versions of the specification.  
These abstractions (or similar ones) appear in other runtime solutions~\cite{kokkos, others} \todo{add relevant citations here} to address deficiencies in the abstract machine model used by runtimes that support \gls{spmd} parallelism 
(i.e., uniform compute elements, flat memory spaces).  Using such abstractions
1) facilitates performance portable application development across 
a variety of execution spaces, and 2)
  provides finer-grained control and additional flexibility in the
  communication of policies regarding data locality and data movement. 
\item {\bf Data Staging:}
The memory and execution space concepts introduced above will enable 1) performance portable tasks that can run in multiple environments through a single code and 2) 
user-directed placement hints to tell the runtime where tasks should run
\item {\bf Collectives:}
Some collectives will be supported by DARMA in version 0.3 of the specification, including all-reduce, reduce-scatter, and barrier collectives.
Collectives will be data-centric rather rank-centric, as done in MPI.
\end{compactdesc}
