%!TEX root = sandReportSpec.tex

\todo[inline]{Note the use of the type keyword in the glossary entries. We can use this to create subglossaries according to topic.  Right now, everything is just going to a single main glossary. We should think about whether or not we want subglossaries, and if so, how we would like to structure these.}
\todo[inline]{Throughout glossaries and text, make sure usage of concurrency and
  parallelism are consistent and correct.}
\todo[inline]{Figure out how to make inlinecode and gls play nice together in latex,
  then update text accordingly}
\todo[inline]{Make a pass to add text to glossary where appropriate}

\chapter{Introduction}
\label{chap:introduction}
As we look ahead to next generation platforms and exascale computing,  hardware 
will be characterized by dynamic behavior, increased
heterogeneity, and a marked increase in overall system \gls{concurrency}~\cite{doe_arch, dav_exascale}. 
Consequently, there is a profuse and diverse body of research within the
computer science community aimed at mitigating challenges associated with
the shifts in \gls{HPC} system architectures.  Many of these research efforts focus on targeting
specific components within the \gls{HPC} runtime software stack (see
Figure~\ref{fig:basicHPCStack} for a notional illustration of the stack).
Examples of some such targeted efforts incude:
\begin{compactenum}
\item Abstraction layers at the top of the stack that provide compile-time generation 
of performance portable code across a variety of memory and execution
abstractions~\cite{Kokkos, RAJA}. 
\item Operating system-level resource management and aribtration services~\cite{Qthreads, Hobbs, Kitten}. 
\item Data and I/O abstractions~\cite{Kelpie, Nessie, Adios, DataSpaces, LLNLandLANLDW}.  
\end{compactenum}

Furthermore, there is a marked increase in runtime systems
research that spans a broader swath of the \gls{HPC} runtime stack.  
Alternative programming  and \glspl{execution model}, including \gls{AMT}
models are rising in popularity.  
%The \gls{AMT} model breaks from the dominant \gls{CSP} or \gls{spmd} model in use over the
%last 20 years.
An \gls{AMT} model aims to exploit all available \gls{task parallelism} and
\gls{pipeline parallelism}, rather just rely on basic \gls{data parallelism}
for \gls{concurrency}. The term asynchronous encompasses the idea that 
1) processes (threads) can diverge to different tasks, rather than execute 
the same tasks in the same order; and 2) concurrency is maximized (the 
  minimal amount of synchronization is performed) by 
leveraging multiple forms of parallelism. The term many-task encompasses 
the idea that the application is decomposed into many 
\gls{migratable} units of work, to enable the overlap of communication and 
computation as well as asynchronous load balancing strategies.

\begin{figure}
\centering
%\includegraphics{}
\caption{}
\label{fig:basicHPCStack}
\end{figure}

The \gls{AMT} \gls{runtime system} research community is active, with many efforts being
explored~\cite{OCR,STAPL,Legion,StencilHPX,Charm++,Uintah, Loci}.
Each effort represents a different design point within the design space of AMT
models. For example, some of the runtimes provide high-level abstractions to
facilitate creation and management of parallelism by application developers,
while others target users at a lower level of the runtime system stack.
Irrespective of the swath of the runtime software stack that they span, it should be
noted that, in large part, \gls{AMT} research efforts are complementary to the 
more narrowly focused research efforts, such as those enumerated above.
%While such targeted research efforts often originate within projects that
%assume a \gls{CSP}/\gls{MPI} based execution model, 
There is a tremendous potential to explore the integration of
the research stemming from these different communities.
 However,  this exploration is currently hampered by the combinatorial
 complexity posed by the lack of standards, common terminology, and 
resusable components within the \gls{AMT} community.

\section{Scope}\label{sec:scope}
In this document we provide the specification for DARMA,
a co-design research vehicle for \gls{AMT} programming models and runtime
systems.  This specification serves to facilitate three primary activities: 
\begin{compactdesc}
\item[Gathering and communicating application requirements:]
Experimenting with the \gls{API} provides an agile method for application
developers to reason about the \gls{API} (e.g., does it allow them to intuitively
    express their algorithms?) and the execution requirements they would like
the runtime system to support.
The specification provides a mechanism for tracking the
provenance of design decisions and requirements as they evolve throughout the co-design
process. Chapter~\ref{chap:requirements} provides a list of the application
requirements gathered,  and
Chapter~\ref{chap:evolution} tracks the evolution of
the specification, highlighting which requirements
motivated changes to the specification.
\item[Exploring AMT design space tradeoffs:]
DARMA aims to be as agnostic as possible with regards to data, memory, and execution
models in order to faciliate exploration of the AMT design space. The specification is
starting with a programming model design that seeks to clearly separate policy
from mechanism. \todo{elaborate on what we mean by this} The set of policies specified by DARMA do not uniquely map to any
existing AMT programming model -- rather this specification captures features and policies
supported across a variety of existing runtimes.
\gls{AMT} design space exploration is currently made difficult by the lack of
common terminology and reusable components within the \gls{AMT} community.  It
is our hope that the formal specification process will encourage multiple
implementations, each leveraging
existing different \gls{AMT} runtime systems or components thereof.  
%This excercise will be incredibly fruitful in helping the AMT community assess
%similarities and differences across the various proposed approaches. 
This excercise would make it easier to start identifying 
software components that are interchangeable between various runtimes, which is a first
step towards generating best practices within the \gls{AMT} runtime community.
It should be noted that some of the existing \gls{AMT} runtimes 
may have design decisions that preclude their direct use in an implemntation of the
DARMA specification. The co-design feedback from such runtime devlopers 
will be incredibly fruitful from a best practices perspective. Consider 
the possible outcomes from such an exercise: 1) either the DARMA 
programming model specification will evolve to account for and support new
features, 2) the existing \gls{AMT} runtime may evolve to account for the DARMA
requirements, or 3) both 1 and 2 occur.  Regardless of outcome (1-3), the process will contribute to the
broader community discussion surrounding design tradeoffs and best practices. 

\item[Communication between AMT community and other CS research components]
There are a number of abstractions and capabilities being developed that are
agnostic to whether they are deployed within an \gls{AMT} model or a \gls{CSP} model.
The DARMA co-design effort will seek to actively engage members from diverse
research efforts, so that abstractions and policies from within these communities can be
incorporated into the specification as appropriate.
 This will encourage cross-fertilization of ideas and efforts between these two
 communities as implementations of the specification are developed.
\end{compactdesc}

The DARMA code base comprises three main layers: the \gls{front end} \gls{API},
\gls{translation layer}, and
\gls{back end} \gls{API}.  The \gls{front end} is the user-level \gls{API}.  It
is implemented as an \gls{EDSL} in \CC,  inheriting the generic language
constructs of \CC, and adding \gls{semantics} that facilitate distributed
parallel programming.
The \gls{translation layer} is a library that heavily leverages \CC\ \gls{template metaprogramming} 
to map the user's code onto interactions with a \gls{back end} \gls{API}.  
The DARMA \gls{translation layer} requires \CC\ features supported by the \CC11
standard with a small subset of \CC14 required for advanced features. 
DARMA will work with the following compilers: \inlinecode{gcc >= 4.8, clang >= 3.5}.
The \gls{back end} \gls{API} is a set of abstract classes, that runtime system
developers must implement in accordance with the specification requirements in
order to be compliant with the specification.

The \gls{API} will evolve as its design and specification is comprehensively vetted over
the course of 2016 with teams whose expertise spans the software stack,
including application
developers, \gls{AMT} runtime community, as well as feedback from
research groups focused on specific components of the \gls{HPC} software stack.
This document captures version 0.2 of the specification, in which we 
specify several key features of the DARMA programming model. The features
described in 0.2 are not comprehensive, meaning they do not capture all of the
application requirements that are currently driving DARMA's design philosopy. 
However, we are formalizing the specification very early in the
design process, layering in features in 0.1 version increments throughout 2016.

\section{Programming Model}
\label{sec:programming_model}
In DARMA \gls{spmd} is the dominant parallelism and the notion of a \gls{rank}
is maintained within the \gls{API}.   
  The DARMA \gls{API} simplifies the introduction
  of asynchronous task parallelism via the use of standard \CC\ constructs, such
  as \gls{reference counted pointers} and \glspl{lambda}.

  
The \gls{API} is being \gls{co-design}ed with application developers
and computer scientists whose knowledge spans the entire runtime software stack.
Current applications affecting the design include portions of the \gls{Trilinos} software stack
and \gls{ASC} \gls{ATDM} efforts at Sandia National Laboratories. 
 
% NOTE: this is morphed text from open mp/ocr specs.  is this needed?
%This specification covers user-directed parallelization, where the user
%has annotated that portions of the work could be performed asynchronously.
%The DARMA runtime will schedule all asynchronous work in a \gls{deferred} fashion,
%checking that all \gls{input dependencies} are met prior to performing work.  
%The user is responsible for ensuring that the application using the DARMA
%\gls{API} constructs executes correctly.



%!TEX root = sandReportSpec.tex


\section{Execution Models}
\label{sec:exec_model}
The main focus of \gls{DARMA} is the \gls{programming model} and corresponding
\gls{translation layer} that maps a program expressed via a combination of
\gls{CSP} semantics, \gls{coordination semantics}, and additional \CC{}-embedded task annotations into a generic data-flow based
description of an algorithm based on deferred tasks.

\gls{DARMA}-compliant \gls{back end} \gls{runtime systems} are required to enable an efficient
\gls{spmd} launch of their program, similar to an MPI launch.
This is based off application developer feedback, which has indicated that 
two of the most critical challenges for scientific applications with massive data parallelism in a task-based model
include initial problem decomposition and distribution.
\gls{DARMA}'s efficient \gls{spmd} runtime-based launch requirement 
will be modified if solutions are developed to support massive \gls{spmd} launches
through compiler-based transformations.

Other than this requirement, \gls{DARMA} prescribes very little
about execution.
For example, \gls{DARMA} prescribes nothing about the scheduling of tasks nor the implementation 
of the data structures (e.g., \gls{key-value store}, \gls{tuple space})
  required to support \gls{coordination semantics}.
A \gls{back end} \gls{runtime system} scheduler is therefore free to use, for example,
either depth-first or breadth-first priorities in deferred tasks (as captured
in a \gls{CDAG}).
Similarly, a scheduler may use \glspl{thread pool} with work queues to manage
tasks or it may use a \gls{fork-join} model that creates new threads for each task.
In this way, \gls{DARMA} codes are \gls{execution model}-agnostic, only
requiring that a \gls{back end} \gls{runtime system} preserve the
\glspl{data-flow dependency} expressed in the application and derived by the
\gls{translation layer}.

The main focus of DARMA is the programming model and corresponding translation layer that maps CSP semantics onto a data-flow description of the algorithm based on deferred tasks.
DARMA prescribes nothing about the scheduling of tasks or implementation of key-value store semantics supporting process coordination (see below).
A backend scheduler is therefore free to use, e.g. either depth-first or breadth-first priorities in executing tasks on the CDAG.
Similarly, a scheduler may use thread pools with work queues to run tasks or it may use a fork-join model that creates new threads for each task.
DARMA codes are therefore execution model-agnostic, only requiring that a backend runtime preserve the data-flow constraints expressed in the application and derived by the translation layer.

DARMA furthermore prescribes nothing about the internals of each task.
DARMA is fully compatible with parallel elastic tasks - tasks with flexible fine-grained parallelism, usually data parallelism.
For example, depending on dynamic conditions, more or fewer threads may be requested for a GPU kernel.
Although the DARMA application-level API currently only allows expressing task granularity and task data-flow,
we expect the API to also express task elasticity in future versions.




\section{Memory Model}
\label{sec:mem_model}
The memory model for \gls{DARMA} encompasses how variables are accessed %(e.g., pointer, iterator, accessor) 
  and when updates become visible to parallel threads (concurrency).  
Within a \gls{DARMA} execution stream, memory is local or private, and the
standard \CC{} memory model applies. 
To share memory between execution streams, \gls{DARMA} uses a flat global memory space in
which data is identified by unique \gls{tuple} identifiers, e.g. a \gls{key-value
  store} or \gls{tuple space} abstraction.  
Any object published into the \gls{tuple space} can be read/written by any thread/process. 

In \gls{DARMA} a data \gls{handle} is conceptually a \gls{reference counted pointer} into the
\gls{tuple space}.  
Data \glspl{handle} are used to manage the
complexities associated with \gls{task parallelism} and inter-\gls{rank} communication.  
When data needs to be made accessible off-\codelink{rank}, the application developer 
\codelink{publish}es the \gls{handle}.  Each \gls{handle} has a globally unique handle ID
(e.g., a \codelink{key} into the \gls{tuple space}).  

When publishing, the user must specify an \gls{access group} for that data.  
Declaring an \gls{access group} informs the \gls{runtime system} that other
\codelink{rank}s currently needs or will need the data,  
allowing garbage collection and \gls{anti-dependency} resolution.
In most cases, the \gls{access group} will be declared as the number of readers (1, in the case of simple point-to-point send).
Once all read \glspl{handle} are released (go out of scope in \CC{} terms),
\gls{garbage collection} or \gls{anti-dependency} resolution can occur.

In addition to facilitating coordination between \codelink{rank}s, \gls{handle} data structures 
support \gls{sequential semantics} (see Chapter~\ref{chap:translation_layer} for details).
Within a \codelink{task}, \glspl{handle} identifiers are resolved to a
specific local address and the standard \CC{} memory model applies.

\todo{what is confusing?}
%JEREMY: While the following is illustrative, some of the phrasing was a little
%confusing
%This is similar to UPC which provides a global address space, but data is always accessed by pointer address rather than tuple. 
%This further contrasts with MPI 2-sided in which data are accessed by address, but there is no global memory space.  
%Data within in an MPI process is ?private? and can only be exchanged via messages.  
%DARMA , like UPC, further contrasts with PGAS (partitioned global address spaces) in that the global memory space is ?flat? and is not explicitly partitioned across processes.
%The tuple space vs address space is only relevant for creating and managing coarse-grained tasks. 

Also critical is concurrency in the \gls{memory model} and when/how updates data are made visible to parallel threads.  
Again, within tasks, the \CC{} \gls{memory model} applies. 
At the task level (coarse-grained), \gls{DARMA} ensures atomicity of all tasks. 
The \gls{DARMA} scheduler enforces the \CC{} sequential consistency model at
the level of tasks in the same way that \CC{} ensures sequential consistency at the level of instructions. 
\gls{DARMA} understands read/write usages of tasks and ensures that writes are always visible to subsequent reads - and reads always complete before subsequent writes.  
The use of \glspl{handle} enables this to happen automatically within a
\codelink{rank} 
(shared memory), but requires the application developer to use coordination
for objects exchanged between \codelink{rank} to ensure data consistency.






\section{Compatible Data Models}
\label{sec:data_model}


DARMA provides a very flexible data model, leaving the application developer
free to use whatever data structures they like. 
DARMA provides serialization mechanisms for bitwise copiable data structures.
The user must provide serialization information for non-bitwise copiable data
structures. 
\todo[inline]{add details here}






\section{Document organization}
\label{sec:organization}
This docuemnt is organized as follows.  In Chapter~\ref{chap:front_end} we
introduce the \gls{front end} \gls{API}.  In
Chapter~\ref{chap:translation_layer} we
provide a description of the \gls{translation layer}, and in
Chapter~\ref{chap:back_end} we provide the specifics regarding what must be
supproted by each of the \gls{back end} abstract classes in order to implement
the DARMA specification. In Chapter~\ref{chap:requirements} we include a list
of application requirements driving the specification (along with a list of the
    applications contributing to the requirements to date).
We conclude this document with
Chapter~\ref{chap:evolution}, which includes a brief history of changes between
previous versions of the specificaiton, along with a list of the planned changes 
in upcoming versions.

