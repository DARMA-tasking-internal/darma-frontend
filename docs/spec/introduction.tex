%!TEX root = sandReportSpec.tex
\chapter{Introduction}
\label{chap:introduction}
As we look ahead to next generation platforms and exascale computing, hardware 
will be characterized by dynamic behavior, increased
heterogeneity, decreased reliability, deep memory hierarchies, and a marked increase in system
\gls{concurrency} (both on-node and system-wide)~\cite{doe_arch, dav_exascale}. 
These architectural shifts are posing significant programming challenges for
application developers as they seek solutions for: effective management of hybrid parallelism 
at an unprecedented scale,  efficient load-balancing and work-stealing
strategies that mitigate both application and system load imbalance, and effective management and
staging of data across deep memory hierarchies.  To further complicate matters,
application codes must be made performance portable across a variety of planned
system architectures and be made resilient to the increased number of anticipated faults.  

\gls{AMT} \glspl{programming model} and \glspl{runtime system} 
are a significant shift away from the current \gls{CSP} programming model that
show promise to mitigate these challenges associated with the changes in \gls{HPC} system architectures.  
\gls{AMT} models strive to exploit all available \gls{task parallelism} and
\gls{pipeline parallelism}, rather than rely solely on \gls{data parallelism}
for \gls{concurrency}. The term {\em \gls{asynchronous}} encompasses the idea that 
1) processes (threads) can diverge to different tasks, rather than execute 
the same tasks in the same order; and 2) \gls{concurrency} is maximized (the 
  minimal amount of synchronization is performed) by 
leveraging multiple forms of parallelism. The term {\em many-task} encompasses 
the idea that the application is decomposed into many 
\gls{migratable} units of work, to enable the overlap of communication and 
computation as well as asynchronous load balancing strategies.
A key design goal of \gls{AMT} models is to enable performance-based
optimizations of code dynamically at runtime.
We note that performance-based code transformations are ubiquitous at the compiler-level.
Compilers will add, delete, swap, or reorder instructions to avoid unnecessary operations, improve data locality, or improve pipelining.
Furthermore, there are a number of compile-time optimization tools being
developed~\cite{Kokkos,RAJA}, that provide the ability to map a
single code kernel onto high-performance execution across diverse compute platforms.
However, many optimizations that benefit performance are unknowable until the program
actually runs -- as these decisions may be based on current system performance,
or the data needs of the application itself. Such dynamic runtime optimizations are much more expensive 
than compile-time optimizations, thus the use of \emph{tasks} as a basis
for dynamic runtime transformations.

The \gls{AMT} community is currently very active (e.g.,
~\cite{OCR,STAPL,Legion,Realm,StencilHPX,Charm++,Uintah,Loci,PARSEC,DaGuE,Cilk}),
representing a range of different design points within the
design space of \gls{AMT} models. While the technologies show significant
potential to address challenges, the community has not yet identified best
practices and existing systems still represent a variety of different \gls{programming model},
\gls{execution model}, \gls{memory model}, and \gls{data model} design choices.  

\begin{compactdesc}
\item[Programming model:]
From a \gls{programming model} perspective, \gls{AMT} models all have some notion of decomposing applications into small, \gls{migratable} units of work. 
\Gls{task parallelism} can be expressed in a \gls{fork-join} fashion, with users managing control-flow explicitly themselves. 
  In other \gls{AMT} \glspl{programming model}, the user expresses a step-by-step algorithm and, under some simplifying assumptions, the runtime derives the synchronizations required.
  This often takes the form of read/write data access annotations under the
  assumption of \gls{sequential semantics} enabling runtime dependency analysis.
  Models leveraging runtime analysis are best suited for coarse-grained \gls{task parallelism}, as \gls{runtime system} overheads must be amortized.
\item[Execution model:]
  Execution models broadly cover how the algorithm and corresponding
  correctness and performance constraints specified in the \gls{programming
  model} are translated to actual execution.
  For example, \gls{AMT} runtimes implement a variety of \glspl{execution
  model}, including \gls{event-based}, \gls{fork-join} (either \gls{fully
  strict} or \gls{terminally strict}), \gls{actor model}, or ubiquitous
  \gls{CSP} model.  More subtle details include whether a constant number of
  threads are always executing (e.g. pure MPI codes), new tasks are allocated
  to a thread pool, or if new threads are allocated (forked) for new tasks.
  These details will also affect the synchronizations required in an execution
  model. For example, fully strict \gls{fork-join} models will generally not
  require barriers between sibling tasks. In contrast, \gls{CSP} models will require barriers to synchronize parallel workers.
\item[Memory model:]
  \gls{HPC} \glspl{memory model} will have several properties including distributed or shared and coherent or incoherent. 
  \Glspl{distributed memory
  model} include message-passing models like MPI. \gls{PGAS} models have distinct address regions,
  but are ``shared'' in the sense that any memory location can be accessed across the system by specifying both a pointer address AND process ID. 
  In \gls{DARMA}, computational tasks by default can only operate on their local data. When remote data
     is required, it is communicated between the remote and local tasks.
    Across the spectrum of memory models, memory locations are usually
    accessed via address (put/get or send/recv), but \gls{key-value store}
    (\gls{tuple space}) models identify data regions by key identifiers (coordination).
    In coordination, parallel workers never directly communicate, instead `coordinating'' indirectly via a \gls{key-value store} or \gls{tuple space}.  
\item[Data model:]
  In order for data-flow \gls{AMT} models to make effective data management
  decisions (e.g., \gls{slicing} the data and making copies to increase
  parallelism), they must have some knowledge of the structure of the data.
  One option for providing structural information regarding data is to impose a
  \gls{data model}.  Another option is to require application developers to define
  \gls{serialization}, \gls{slicing}, and \glspl{interference test} for their data blocks. 
\end{compactdesc}

\section{Scope}\label{sec:scope}
Although the \gls{AMT} model community is quite active, the lack of
standards impedes adoption of these technologies by the application community.
Although it is premature to standardize, there is sufficient  
breadth and depth in the \gls{AMT} research community to begin developing community best
practices.  Towards this end, this document provides the specification for \gls{DARMA},
a research vehicle for \gls{AMT} \gls{programming model} \gls{co-design}.  
\gls{DARMA} aims to serve four primary purposes:
\begin{compactdesc}
\item[Insulate applications from runtime system and hardware idiosyncrasies:]
  \gls{DARMA}'s design includes separate application-facing \gls{front end} and
  \gls{runtime system}-facing \gls{back end} \glspl{API}. This separation of concerns 
  enables an application team to explore the impact of \gls{runtime system}
  design space decisions. For example, application developers can build their code using different
  \gls{DARMA}-compliant \gls{back end} implementations, without
  having to deal with the combinatorial complexity of
  implementing their application in many different \gls{front end} \glspl{API}. 
  It should be noted that \gls{DARMA}'s \gls{front end} \gls{API} is not
  fixed -- it will evolve based on \gls{co-design} feedback from both application
  and \gls{runtime system} developers. 
\item[Improve AMT runtime programmability by co-designing a front end  API directly with application developers:]
  Recent work~\cite{L2Sand2015} highlighted gaps with respect to productivity
  in some existing \gls{AMT} \glspl{runtime system}, in particular noting requirements gaps and 
  deficiencies in existing \glspl{API}. Co-designing \gls{DARMA}'s \gls{front end} \gls{API}
  directly with application developers provides a mechanism for capturing
  different application's \gls{runtime system} requirements-- giving them a voice in the design of an
  asynchronous tasking \gls{API}.  Experimenting with the \gls{API} provides an agile method for application
  developers to reason about the \gls{API} 
  %(e.g., does it allow them to intuitively express their algorithms?) 
  and better articulate their \gls{runtime system} execution requirements.
\item[Synthesize application co-design activities into meaningful requirements
  for runtimes:]
  The specification provides a mechanism for tracking the
  provenance of design decisions and requirements as they evolve throughout the
  \gls{co-design} process. Chapter~\ref{chap:requirements} provides a list of the application
  requirements gathered,  and Chapter~\ref{chap:evolution} tracks the evolution of
  the specification, highlighting which requirements motivated changes to the specification.
  \Gls{runtime system} software stack developers benefit 
  from 1) \gls{DARMA}’s application-informed requirements, and 2) access to code
  kernels and proxy applications developed via the \gls{front end}
  \gls{co-design} process.
\item[Facilitate AMT design space characterization, accelerating the
development of AMT best practices:]
  In the discussion above we summarize a range of high-level design decisions for
  \gls{AMT} programming, execution, memory, and data models. \gls{DARMA}'s
  separation of \gls{front end} and \gls{back end} \glspl{API} seeks to
  facilitate this design space characterization and exploration.  There 
  is a notable tension between the design of 1) a \gls{front end} \gls{API} that is expressive, simple, 
  and easy to incorporate within existing application code bases, and 2) a
  \gls{back end} \gls{API} that is simple enough 
  to support multiple \gls{DARMA}-compliant implementations that leverage existing \gls{runtime
  system} technologies. 
  Consequently, \gls{DARMA} \gls{API}s (both \gls{front end} and \gls{back end}) are
  intended to evolve based on iterative feedback from 
  application, \gls{programming model}, and \gls{runtime system} teams. 
\end{compactdesc}


The rest of this chapter provides a high-level description of 
\gls{DARMA}'s structural design  along with a brief summary of \gls{DARMA}'s programming,
memory, data, and (compatible) execution models. We note that throughout the
\gls{co-design} process, decisions are first and foremost, 
made to best support application requirements.  Furthermore, we target a
\gls{back end} \gls{API} specification that is general enough to support 
\gls{AMT} \gls{runtime system}
design space exploration, via build out of \gls{DARMA}-compliant
back ends using existing \gls{AMT} \gls{runtime system}
technologies.
Lastly, we note that the features detailed in Chapters~\ref{chap:front_end} and
~\ref{chap:back_end}  are not entirely comprehensive -- meaning they do not yet capture all of the
application requirements driving \gls{DARMA} \gls{co-design}.  This is because 
we are formalizing the specification process from the inception of \gls{DARMA}, layering-in features incrementally to
provide the community opportunity for input, and active engagement in the
\gls{co-design} process.  Suggested enhancements and changes 
to the \gls{DARMA} specification are welcome and can be made via a \gls{DEP} (see
Appendix~\ref{chap:DEP} for details on this process and a \gls{DEP} template). 

\section{High-level Design}
\gls{DARMA} is a translation layer between an 
application-facing \gls{front end} \gls{API} and a \gls{runtime system} facing \gls{back end} 
\gls{API}. 
\gls{DARMA}'s \gls{front end} \gls{API} 
is an \gls{EDSL} in \CC,  inheriting the generic
language constructs of \CC and adding \gls{semantics} that facilitate
distributed, deferred, asynchronous, parallel programming. Though the \gls{EDSL} uses
\CC{} constructs unfamiliar to many programmers to implement these semantics, 
it is nonetheless fully embedded in the \CC{} language and
requires a widely supported subset of \CC{}14 functionality \compilerReqs.
The \gls{front end} \gls{API} is the center of \gls{programming model}
\gls{co-design} activities, which seek to involve a wide variety of both
application and \gls{runtime system} developers. 


\gls{DARMA}'s \gls{translation layer} leverages \CC{} \gls{template
metaprogramming} to map the user's \gls{front end} \gls{API} calls onto the \gls{back end} runtime \gls{API},
bridging the \gls{programming model} and actual program execution.
We note however that the \gls{DARMA} \gls{translation layer} itself does not
perform any performance optimizations -- these are left entirely to the \gls{back end} \gls{runtime system} implementations.
Rather, the translation layer converts the application code specified with
\gls{DARMA}'s \gls{front end} \gls{API} into
an ``intermediate representation'' that enables a \gls{runtime system} to make
intelligent, dynamic decisions (e.g., 
about task order and task locality or possibly even task deletion and task
replication when appropriate).

The \gls{back end} \gls{API} is a set of abstract classes and function
signatures that \gls{runtime system} developers must implement in accordance with the
specification requirements in order to interface with application code written
to the \gls{DARMA} front end. 
Strictly speaking, the \gls{back end} \gls{API} calls only generate a stream of
\emph{deferred tasks} (tasks with corresponding data inputs/outputs) that
implicitly capture the program's data flow.
The information passed through the \gls{translation layer} to
the \gls{back end} is sufficient
to (and intended to) support a \gls{CDAG} representation of the application.
In a \gls{task-DAG} representation, tasks are vertices $V$ in a graph $G$ with directed edges $E$.
An edge from vertex $v_1$ to vertex $v_2$ indicates a precedence constraint.
A \gls{CDAG} representation describes task-data precedence constraints, rather
than just task-task precedence constraints.  In a \gls{CDAG} there are two
types of vertices - tasks $T$ and data $D$ that compose the complete set of vertices $V$.
Edges never directly connect two tasks and instead edges are only ever described between a task vertex, $t$, and a data vertex, $d$
indicating that (depending on direction of the edge) data is either consumed or produced by a task.
The \gls{task-DAG} indicating task-task precedence constraints can always be
obtained form the \gls{CDAG}, which captures the data-flow task graph.
The \gls{CDAG} is thus more general, capturing additional information to
enable runtime code transformations.\footnote{
Although beyond the scope of this specification document, the interested reader
will find numerous works discussing heuristics and order-preserving convex
transformations of task graphs that demonstrate the utility of a coarse-grained
\gls{CDAG} for enabling dynamic runtime optimization of an
algorithm~\cite{missingCitations}.}
We reiterate that the \gls{CDAG} is only a concept guiding the design
of the \gls{back end} \gls{API} and not strictly part of the \gls{DARMA}
specification.


Finally, we highlight that a \gls{DARMA} executable application must link to a \gls{runtime
  system} that implements the abstract \gls{back end} runtime \gls{API}.
It is intended that these implementations will be external, drawing upon
existing AMT technologies.  However,  a reference implementation will be provided
in the \gls{DARMA} code distribution.

The terms ``rank'', ``task'', and `process'' are loaded terms with many definitions across the literature. 
Here we give special attention to define rigorous and limited definitions for special terms.
We use ``process'' in the usual UNIX sense. Other terms are:
\begin{compactdesc}
\item [Task]: The work unit instantiated directly by the application developer. 
Tasks are also the smallest granularity of \emph{migratable} work unit. 
In the current spec, tasks cannot migrate after beginning execution.
Tasks are guaranteed to make forward progress, but are interruptible.
\item [Execution stream]: An execution stream will consist of a sequence of many tasks, and, like tasks, is guaranteed to make forward progress.
All execution streams are tasks, but execution streams specifically have no parent task and are the root of an independent task-DAG.
Each execution stream is guaranteed to have a unique stack and, any point time, will have a local context of variables.
A physical process (in the UNIX sense) can be running many parallel execution streams.
Allowing multiple execution streams per physical process is the basis for overdecomposition.
Since several execution streams can exist in the same process address space, this introduces a strict requirement of no global variables.
An execution stream is the DARMA generalization of a thread, except that extra privatization of variables is necessary since no assumption of shared memory between independent streams can be made (even if streams happen to be executing in the same process).
Just as processes must perform special operations to exchange data between them (message-passing, mmap), independent execution streams must perform special key-value store operations to exchange data between them.
Execution streams are always assigned a unique identifier by the runtime.
\item [Operation]: Used synonymously with work unit. This is a unit of execution that is guaranteed to be non-interruptible. 
An operation is not equivalent to a task since tasks are interruptible.  
Operations are the smallest, schedulable units of work.  
A task consists of a sequence of operations.
While tasks are explicitly instantiated by the app developer, operations (individual portions of task) can be implicitly instantiated by the runtime.
Tasks can yield at the beginning/end of its component operations, allowing the runtime to schedule new work units for execution.
\item [Rank]: A unique integer ID for an execution stream. This matches the MPI notion of rank as an integer identifying a process within an MPI communicator.
The term rank will often be used in the specification as a synonym for execution stream (more precisely, a metonymy for execution stream).
Generally speaking, $N$ parallel execution streams are created in an SPMD launch (more in Section \ref{}).
The runtime then assigns unique identifiers (rank IDs) $0$ through $N-1$ to each stream.
Referring to ``rank 0'' will therefore function as shorthand for ``the execution stream that has been assigned rank ID 0 by the runtime in an SPMD launch.''
Similarly, referring generically to a ``rank'' is shorthand for `an execution stream created by an SPMD launch with a particular rank ID.''
\end{compactdesc} 


%!TEX root = sandReportSpec.tex

\section{Programming Model}
\label{sec:programming_model}
The focus of the application-facing frontend API is the programming model.
The programming model must provide application developers a mechanism for expressing both a \emph{correct} and a \emph{performant algorithm}.
Of highest priority is the ability for developers to write a correct algorithm.
Above all else, an application must produce correct answers.
The programming model actually demands more than just performance. 
It demands performance portability - the ability to map a single code onto high-performance execution across multiple platforms.
Performance portability demands code transformations - either at runtime or compile-time - to change execution from, e.g. a GPU-optimal execution to a CPU-optimal execution.

Correctness and performance portability must both balance the level of abstraction.
Algorithms written at too low a level are highly error-prone and not portable.
In particular, they may over-express correctness constraints.
A compiler/runtime must obey the correctness requirements (e.g. instruction order) prescribed by an application code.
Low-level code may be unnecessarily prevent the compiler or runtime from optimizing program execution.
For example, code hard-wired for a particular loop structure and data layout may perform very well on one system,
but that may prevent the compiler from tiling optimizations of the loops on other systems. 
What is the correct way and level of abstraction for programmers to 1) express constraints (correctness) while 2) still enabling the compiler/runtime to transform execution (performance portability) while also 3) allowing developers to direct (hand-optimize) execution when the compiler/runtime misses optimizations apparent to a human developer. 

As such, to enable the easy (or as easy as possible) expression of algorithms, DARMA uses sequential semantics.
When necessary, DARMA also enables the semantics of communicating sequential processes (CSPs).
All developers know how to write correct sequential codes and all MPI programmers know how to write correct CSP code.
In this way, the developers can express correctness constraints through a familiar and intuitive programming model.
The application-level API of DARMA is therefore designed to capture the minimum number of constraints required for correct execution expressed in sequential code.

Code transformations are already ubiquitous at the compiler-level.
Compilers will add, delete, swap, or reorder instructions to avoid unnecessary operations, improve data locality, or improve pipelining.
DARMA aims for a similar goal, but at runtime.
Many transformations of program execution that benefit performance will be unknowable until the program actually runs.
These dynamic optimizations occurring at runtime are much more expensive than compile-time optimizations.
Clearly, dynamically transforming an application at the level of individual instructions is not feasible.
As such, \emph{tasks} are the basis of DARMA dynamic transformations in the same way that instructions are the basis of compiler static transformations.
The issue of granularity is therefore of paramount importance in task-based model.
How should the extra flexibility of fine-grained tasks be balanced against amortizing the cost of runtime analysis?
For the time-being, the choice of a correct task granularity seems beyond the abilities of current compilers.
As such, choosing task granularity is the responsibility of the application developer.
The application-level API therefore expresses both correctness and granularity.

The translation layer bridges the programming model and the actual program execution.
The translation layer itself does not perform any transformations of the program execution from sequential order.
Rather, the translation layer interprets the sequential semantics in the application and creates events in an ``intermediate representation'' suitable for the backend runtime.
The backend API is therefore intended to communicate the algorithm at the right level of abstraction.
The program representation created by backend API calls should enable a runtime to make intelligent dynamic decisions about task order and task locality or possibly even task deletion and task replication when appropriate.

Strictly speaking, the backend API calls only generate a stream of deferred tasks (tasks with corresponding data inputs/outputs) that describes the inherent data flow.
However, the information passed from translation layer to backend is sufficient to (and intended to) support a CDAG (computational directed acyclic graph) representation of the application.
In a DAG representation, tasks are vertices (V) in a graph (G) with directed edges (E).
An edge from vertex $v_1$ to vertex $v_2$ indicates a precedence constraint.
Instead of directly defining task-task precedence constraints, DARMA generally describes task-data precedence constraints.
There are two types of vertices - tasks (T) and data (D) that compose the complete set of vertices (V).
Edges never directly connect two tasks and instead edges are only ever described between a task vertex, $t$, and a data vertex, $d$
indicating that (depending on direction of the edge) data is either consumed or produced by a task.
The \emph{task-DAG} indicating task-task precedence constraints can always be obtained form the data-flow task graph,
making the data-flow graph more general and therefore more useful for enabling runtime code transformations.
Although beyond the scope of this specification document, the interested reader will find numerous works discussing heuristics and order-preserving convex transformations of task graphs that demonstrate the utility of a coarse-grained CDAG for enabling dynamic runtime optimization of an algorithm.
We reiterate, though, that the CDAG is only a concept guiding the design of the backend API and not strictly part of the specificaion.

A final concern, not addressed in the current specification, is the issue of programmer-directed optimization.
While an abstract algorithm may make more information available to the compiler or runtime for performance-tuning transformations,
compilers and runtime schedulers may not always understand the global nature of the problem.
As such,  they may not make peformance-improving optimizations that are apparent to an application developer.
It will be critically important as runtimes develop to enable developers to steer the runtime towards optimizations when compilers or schedulers fail.

DARMA is a mixed \gls{imperative}/\gls{declarative} \gls{programming model}.
As much as possible, sequential imperative semantics are used to produce intuitive, maintainable code.
However, the ``procedural imperative'' function calls and code blocks do not execute immediately.
Rather than explicitly perform all work in program order and block on data requests,
DARMA provides \CC-embedded task annotations that allow work to be deferred and performed asynchronously.
The ability to defer work and advance ahead is what gives the backend runtime the ability to make performance-improving transformations.
Deferred execution makes DARMA also \emph{declarative}, leaving the exact control-flow up to the runtime.

Task parallelism is primarily achieved through permissions/access qualifiers
on data that enable that enable a runtime to reason about which tasks can run in parallel and which tasks are strictly ordered.
Task annotations are translated by the DARMA front end through \CC\ constructs (e.g., lambdas, reference counted pointers, template
metaprogramming) to expose and understand the parallelism inherent in the code.  
The \gls{translation layer} requires \CC11 standard features with a small subset of
\CC14 required for advanced features (details provided in Chapter~\ref{chap:translation_layer}), 
however the \gls{front end} \gls{API} does \emph{not} require knowledge of \CC14 to use. 
Furthermore, the \gls{back end} is a simple set of abstract \CC\ classes whose functionality must be implemented
according to the specfication in Chapter~\ref{chap:back_end}.
%providing the runtime the flexibility to optimize performance and
%exploit additional parallelism when possible.   


Most applications written in DARMA will likely have \gls{spmd} as the dominant parallelism.
To simplify the implementation of SPMD-structured codes, the notion of a \gls{rank} is maintained within the \gls{API}.   
This provides application developers a convenience mechanism for creating
the initial problem decomposition and distribution.  
Immediately after launch, deferred tasks are free to be migrated by the runtime, if it will result in better performance. 
Within a \gls{rank}, DARMA provides \gls{sequential semantics},  meaning that application developers can reason about the code as
though it were being deployed sequentially within the rank.   
Thus DARMA emphasizes sequential semantics, but supports a CSP model.


The ``communication'' in DARMA's CSP model is actually provided by \gls{coordination semantics}:  
rather than explicitly move data between ranks via direct communication
(i.e.,  \inlinecode{send/recv}), processes \emph{coordinate} by putting/getting data associated with a unique \inlinecode{key} in a
\gls{key-value store} or \gls{tuple space}.  
\Gls{coordination semantics} promote out-of-order message arrival, deferred execution, and task migration
since the app declares or describes the data it needs/produces rather than enforcing an explicit delivery mechanism.
The key to performance in the DARMA CSP model are exploiting zero-copy mechanisms and tuple caching that enable a key-value store programming model
to produce execution equivalent to an MPI send/recv code. 

Although not yet supported in version 0.3 of the specification, several
important features will play a role in the DARMA programming model:
\begin{compactdesc}
\item{\bf Expressive Underlying Abstract Machine Model:}
Notions of \glspl{execution space} and \glspl{memory space} will be introduced formally in later versions of the specification.  
These abstractions (or similar ones) appear in other runtime solutions~\cite{kokkos, others} \todo{add relevant citations here} to address deficiencies in the abstract machine model used by runtimes that support \gls{spmd} parallelism 
(i.e., uniform compute elements, flat memory spaces).  Using such abstractions
1) facilitates performance portable application development across 
a variety of execution spaces, and 2)
  provides finer-grained control and additional flexibility in the
  communication of policies regarding data locality and data movement. 
\item {\bf Data Staging:}
The memory and execution space concepts introduced above will enable 1) performance portable tasks that can run in multiple environments through a single code and 2) 
user-directed placement hints to tell the runtime where tasks should run
\item {\bf Collectives:}
Some collectives will be supported by DARMA in version 0.3 of the specification, including all-reduce, reduce-scatter, and barrier collectives.
Collectives will be data-centric rather rank-centric, as done in MPI.
\end{compactdesc}

%!TEX root = sandReportSpec.tex


\section{Execution Models}
\label{sec:exec_model}
The main focus of \gls{DARMA} is the \gls{programming model} and corresponding
\gls{translation layer} that maps a program expressed via a combination of
\gls{CSP} semantics, \gls{coordination semantics}, and additional \CC{}-embedded task annotations into a generic data-flow based
description of an algorithm based on deferred tasks.

\gls{DARMA}-compliant \gls{back end} \gls{runtime systems} are required to enable an efficient
\gls{spmd} launch of their program, similar to an MPI launch.
This is based off application developer feedback, which has indicated that 
two of the most critical challenges for scientific applications with massive data parallelism in a task-based model
include initial problem decomposition and distribution.
\gls{DARMA}'s efficient \gls{spmd} runtime-based launch requirement 
will be modified if solutions are developed to support massive \gls{spmd} launches
through compiler-based transformations.

Other than this requirement, \gls{DARMA} prescribes very little
about execution.
For example, \gls{DARMA} prescribes nothing about the scheduling of tasks nor the implementation 
of the data structures (e.g., \gls{key-value store}, \gls{tuple space})
  required to support \gls{coordination semantics}.
A \gls{back end} \gls{runtime system} scheduler is therefore free to use, for example,
either depth-first or breadth-first priorities in deferred tasks (as captured
in a \gls{CDAG}).
Similarly, a scheduler may use \glspl{thread pool} with work queues to manage
tasks or it may use a \gls{fork-join} model that creates new threads for each task.
In this way, \gls{DARMA} codes are \gls{execution model}-agnostic, only
requiring that a \gls{back end} \gls{runtime system} preserve the
\glspl{data-flow dependency} expressed in the application and derived by the
\gls{translation layer}.

Furthermore, \gls{DARMA} prescribes nothing about the internals of each task.
\gls{DARMA} is fully compatible with parallel elastic tasks - tasks with
flexible fine-grained parallelism, usually \gls{data parallelism}.
For example, depending on dynamic conditions, more or fewer threads may be
requested for a \gls{CUDA} kernel.
Although the \gls{DARMA} \gls{front end} \gls{API} currently only allows
expressing task granularity and task data-flow in the \specVersion\ version of
the specification, task elasticity will be specified in future version of the
specification.




\section{Memory Model}
\label{sec:mem_model}
The memory model for \gls{DARMA} encompasses how variables are accessed %(e.g., pointer, iterator, accessor) 
  and when updates become visible to parallel threads (concurrency).  
Within a \gls{DARMA} execution stream, memory is local or private, and the
standard \CC{} memory model applies. 
To share data between execution streams, \gls{DARMA} uses a flat global memory space in
which data is identified by unique \gls{tuple} identifiers, i.e. a \gls{key-value
  store} in which keys exist in a \gls{tuple space}.
Any object published into the \gls{tuple space} can be read/written by any thread/process. 

In \gls{DARMA} a data \gls{handle} is conceptually a \gls{reference counted pointer} into the
\gls{tuple space}.  
Data \glspl{handle} are used to manage the
complexities associated with \gls{task parallelism} and inter-\gls{rank} communication.  
When data needs to be made accessible off-\codelink{rank}, the application developer 
\codelink{publish}es the \gls{handle}.  Each \gls{handle} has a globally unique handle ID
(e.g., a \codelink{key} into the \gls{tuple space}).  
Before a \codelink{task} can begin, \glspl{handle} identifiers are resolved by the runtime to a
specific local address. Within the task, the standard \CC{} memory model applies.

When publishing, the user must specify an \gls{access group} for that data.  
Declaring an \gls{access group} informs the \gls{runtime system} that other
\codelink{rank}s currently needs or will need the data,  
allowing the runtime to manage garbage collection and \gls{anti-dependency} resolution.
In most cases, the \gls{access group} will be declared as the number of readers (1, in the case of simple point-to-point send).
Once all read \glspl{handle} are released (go out of scope in \CC{} terms),
\gls{garbage collection} or \gls{anti-dependency} resolution can occur.


In addition to facilitating coordination between \codelink{rank}s, \gls{handle} data structures 
support \gls{sequential semantics} (see Chapter~\ref{chap:translation_layer} for details).
Here concurrency is critical to the \gls{memory model} and when/how updates data are made visible to parallel threads.  
Again, within tasks, the \CC{} \gls{memory model} applies. 
At the task level (coarse-grained), \gls{DARMA} ensures atomicity of all tasks. 
The \gls{DARMA} scheduler enforces the \CC{} sequential consistency model at
the level of tasks in the same way that \CC{} ensures sequential consistency at the level of instructions. 
\gls{DARMA} understands read/write usages of tasks and ensures that writes are always visible to subsequent reads - and reads always complete before subsequent writes.  
The use of \glspl{handle} enables this to happen automatically within an execution stream.





%!TEX root = sandReportSpec.tex

\section{Data Model}
\label{sec:data_model}
\gls{DARMA} only implements a \gls{data model} through its \gls{serialization} interface.
The notion of data structure, data layout, and data type only exist in the
application and \gls{translation layer} (see
    Chapter~\ref{chap:translation_layer}).
Thus, a \gls{runtime system} implementing the \gls{DARMA} specification is only
aware of \gls{tuple} or \codelink{key} identifiers for a coarse-grained data block of a given size.
To actually migrate data, a \gls{back end} \gls{runtime system} invokes
\gls{serialization} hooks implemented by the application.
In future versions, an \gls{API} similar to the \gls{serialization} interface will support the definition of data subsets and data slices.
Again, the \gls{back end} \gls{runtime system} will only understand data and task dependencies,
  requiring the type-aware application and \glspl{translation layer} to define
  the details of subsetting and slicing operations.
This leaves the application developer free to use arbitrary data structures,
     but puts more responsibility on the application developer to articulate
     the structure of the data.






\section{Document organization}
\label{sec:organization}
This document is organized as follows.  In Chapter~\ref{chap:front_end} we
introduce the \gls{front end} \gls{API}.  In
Chapter~\ref{chap:translation_layer} we
provide a description of the \gls{translation layer}, and in
Chapter~\ref{chap:back_end} we provide the specifics regarding what must be
supported by each of the \gls{back end} abstract classes in order to implement
the DARMA specification. In Chapter~\ref{chap:requirements} we include a list
of application requirements driving the specification (along with a list of the
    applications contributing to the requirements to date).
We conclude this document with
Chapter~\ref{chap:evolution}, which includes a brief history of changes between
previous versions of the specification, along with a list of the planned changes 
in upcoming versions.
Appendix~\ref{chap:examples} provides a suite of examples that illustrate the
 \gls{front end} \gls{API} features. Appendix~\ref{chap:vasp} provides
additional technical details regarding %\gls{vasps}.
Appendix~\ref{chap:DEP} provides information regarding how the broader
community can shape the \gls{DARMA} specification and includes a template for 
a \gls{DEP}. 

