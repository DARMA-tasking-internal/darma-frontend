\todo[inline]{Note the use of the type keyword in the glossary entries. We can use this to create subglossaries according to topic.  Right now, everything is just going to a single main glossary. We should think about whether or not we want subglossaries, and if so, how we would like to structure these.}
\todo[inline]{Throughout glossaries and text, make sure usage of concurrency and
  parallelism are consistent and correct.}
\todo[inline]{Figure out how to make inlinecode and gls play nice together in latex,
  then update text accordingly}
\todo[inline]{Make a pass to add text to glossary where appropriate}

\chapter{Introduction}
\label{chap:introduction}
As we look ahead to next generation platforms and exascale computing,  hardware 
will be characterized by dynamic behavior, increased
heterogeneity, and a marked increase in overall system \gls{concurrency}~\cite{doe_arch, dav_exascale}. 
Consequently, there is a profuse and diverse body of research within the
computer science community aimed at mitigating challenges associated with
the shifts in \gls{HPC} system architectures.  Many of these research efforts focus on targeting
specific components within the \gls{HPC} runtime software stack (see
Figure~\ref{fig:basicHPCStack} for a notional illustration of the stack).
Examples of some such targeted efforts incude:
\begin{compactenum}
\item Abstraction layers at the top of the stack that provide compile-time generation 
of performance portable code across a variety of memory and execution
abstractions~\cite{Kokkos, RAJA}. 
\item Operating system-level resource management and aribtration services~\cite{Qthreads, Hobbs, Kitten}. 
\item Data and I/O abstractions~\cite{Kelpie, Nessie, Adios, DataSpaces, LLNLandLANLDW}.  
\end{compactenum}

Furthermore, there is a marked increase in runtime systems
research that spans a broader swath of the \gls{HPC} runtime stack.  
Alternative programming  and \glspl{execution model}, including \gls{AMT}
models are rising in popularity.  
%The \gls{AMT} model breaks from the dominant \gls{CSP} or \gls{spmd} model in use over the
%last 20 years.
An \gls{AMT} model aims to exploit all available \gls{task parallelism} and
\gls{pipeline parallelism}, rather just rely on basic \gls{data parallelism}
for \gls{concurrency}. The term asynchronous encompasses the idea that 
1) processes (threads) can diverge to different tasks, rather than execute 
the same tasks in the same order; and 2) concurrency is maximized (the 
  minimal amount of synchronization is performed) by 
leveraging multiple forms of parallelism. The term many-task encompasses 
the idea that the application is decomposed into many 
\gls{migratable} units of work, to enable the overlap of communication and 
computation as well as asynchronous load balancing strategies.

\begin{figure}
\centering
%\includegraphics{}
\caption{}
\label{fig:basicHPCStack}
\end{figure}

The \gls{AMT} \gls{runtime system} research community is active, with many efforts being
explored~\cite{OCR,STAPL,Legion,StencilHPX,Charm++,Uintah, Loci}.
Each effort represents a different design point within the design space of AMT
models. For example, some of the runtimes provide high-level abstractions to
facilitate creation and management of parallelism by application developers,
while others target users at a lower level of the runtime system stack.
Irrespective of the swath of the runtime software stack that they span, it should be
noted that, in large part, \gls{AMT} research efforts are complementary to the 
more narrowly focused research efforts, such as those enumerated above.
%While such targeted research efforts often originate within projects that
%assume a \gls{CSP}/\gls{MPI} based execution model, 
There is a tremendous potential to explore the integration of
the research stemming from these different communities.
 However,  this exploration is currently hampered by the combinatorial
 complexity posed by the lack of standards, common terminology, and 
resusable components within the \gls{AMT} community.

\section{Scope}\label{sec:scope}
In this document we provide the specification for DARMA,
a co-design research vehicle for \gls{AMT} programming models and runtime
systems.  This specification serves to facilitate three primary activities: 
\begin{compactdesc}
\item[Gathering and communicating application requirements:]
Experimenting with the \gls{API} provides an agile method for application
developers to reason about the \gls{API} (e.g., does it allow them to intuitively
    express their algorithms?) and the execution requirements they would like
the runtime system to support.
The specification provides a mechanism for tracking the
provenance of design decisions and requirements as they evolve throughout the co-design
process. Chapter~\ref{chap:requirements} provides a list of the application
requirements gathered,  and
Chapter~\ref{chap:evolution} tracks the evolution of
the specification, highlighting which requirements
motivated changes to the specification.
\item[Exploring AMT design space tradeoffs:]
DARMA aims to be as agnostic as possible with regards to data, memory, and execution
models, to faciliate exploration of the AMT design space. The specification 
gives programming model design that seeks to clearly separate policy from
mechanism.
\todo{elaborate on what we mean by this} The set of policies specified by DARMA do not uniquely map to any existing AMT programming model -- rather this specification captures features and policies
supported across a variety of existing runtimes.
\gls{AMT} design space exploration is currently made difficult by the lack of
common terminology and reusable components within the \gls{AMT} community.  It
is our hope that the formal specification process will encourage multiple
implementations, each leveraging
existing different \gls{AMT} runtime systems or components thereof.  
%This excercise will be incredibly fruitful in helping the AMT community assess
%similarities and differences across the various proposed approaches. 
This excercise would make it easier to start identifying 
software components that are interchangeable between various runtimes, which is a first
step towards generating best practices within the \gls{AMT} runtime community.
It should be noted that some of the existing \gls{AMT} runtimes 
may have design decisions that preclude their direct use in an implemntation of the
DARMA specification. The co-design feedback from such runtime devlopers 
will be incredibly fruitful from a best practices perspective. (Consider 
the possible outcomes from such an exercise: 1) either the DARMA 
programming model specification will evolve to account for and support new
features, 2) the existing \gls{AMT} runtime may evolve to account for the DARMA
requirements, or 3) both 1) and 2) occur.  Regardless of which of these
outcomes occurs, the process could be documented and contribute to the
broader community discussion surrounding design tradeoffs and best practices.) 

\item[Communication between AMT community and other CS research components]
There are a number of abstractions and capabilities being developed that are
agnostic to whether they are deployed within an \gls{AMT} model or a \gls{CSP} model.
The DARMA co-design effort will seek to actively engage members from diverse
research efforts, so that abstractions and policies from within these communities can be
incorporated into the specification as appropriate.
 This will encourage cross-fertilization of ideas and efforts between these two
 communities as implementations of the specification are developed.
\end{compactdesc}

The DARMA code base comprises three main layers: the \gls{front end} \gls{API},
\gls{translation layer}, and
\gls{back end} \gls{API}.  The \gls{front end} is the user-level \gls{API}.  It
is implemented as an \gls{EDSL} in \CC,  inheriting the generic language
constructs of \CC, and adding \gls{semantics} that facilitate distributed
parallel programming.
The \gls{translation layer} is a library that heavily leverages \CC\ \gls{template metaprogramming} 
to map the user's code onto interactions with a \gls{back end} \gls{API}.  
The DARMA \gls{translation layer} requires \CC\ features supported by the \CC14
standard, and will work with the following compilers: \inlinecode{gcc >= 4.8, clang >= 3.5}.
The \gls{back end} \gls{API} is a set of abstract classes, that runtime system
developers must implement in accordance with the specification requirements in
order to be compliant with the specification.

The \gls{API} will evolve quite a bit as its design and specification is comprehensively vetted over
the course of 2016 with teams whose expertise spans the software stack,
including application
developers, \gls{AMT} runtime community, as well as feedback from
research groups focused on specific components of the \gls{HPC} software stack.
This document captures version 0.2 of the specification, in which we 
specify several key features of the DARMA programming model. The features
described in 0.2 are not comprehensive, meaning they do not capture all of the
application requirements that are currently driving DARMA's design philosopy. 
However, we are formalizing the specification very early in the
design process, layering in features in 0.1 version increments throughout 2016.

%!TEX root = sandReportSpec.tex

\section{Programming Model}
\label{sec:programming_model}
The focus of the application-facing frontend API is the programming model.
The programming model must provide application developers a mechanism for expressing both a \emph{correct} and a \emph{performant algorithm}.
Of highest priority is the ability for developers to write a correct algorithm.
Above all else, an application must produce correct answers.
The programming model actually demands more than just performance. 
It demands performance portability - the ability to map a single code onto high-performance execution across multiple platforms.
Performance portability demands code transformations - either at runtime or compile-time - to change execution from, e.g. a GPU-optimal execution to a CPU-optimal execution.

Correctness and performance portability must both balance the level of abstraction.
Algorithms written at too low a level are highly error-prone and not portable.
In particular, they may over-express correctness constraints.
A compiler/runtime must obey the correctness requirements (e.g. instruction order) prescribed by an application code.
Low-level code may be unnecessarily prevent the compiler or runtime from optimizing program execution.
For example, code hard-wired for a particular loop structure and data layout may perform very well on one system,
but that may prevent the compiler from tiling optimizations of the loops on other systems. 
What is the correct way and level of abstraction for programmers to 1) express constraints (correctness) while 2) still enabling the compiler/runtime to transform execution (performance portability) while also 3) allowing developers to direct (hand-optimize) execution when the compiler/runtime misses optimizations apparent to a human developer. 

As such, to enable the easy (or as easy as possible) expression of algorithms, DARMA uses sequential semantics.
When necessary, DARMA also enables the semantics of communicating sequential processes (CSPs).
All developers know how to write correct sequential codes and all MPI programmers know how to write correct CSP code.
In this way, the developers can express correctness constraints through a familiar and intuitive programming model.
The application-level API of DARMA is therefore designed to capture the minimum number of constraints required for correct execution expressed in sequential code.

Code transformations are already ubiquitous at the compiler-level.
Compilers will add, delete, swap, or reorder instructions to avoid unnecessary operations, improve data locality, or improve pipelining.
DARMA aims for a similar goal, but at runtime.
Many transformations of program execution that benefit performance will be unknowable until the program actually runs.
These dynamic optimizations occurring at runtime are much more expensive than compile-time optimizations.
Clearly, dynamically transforming an application at the level of individual instructions is not feasible.
As such, \emph{tasks} are the basis of DARMA dynamic transformations in the same way that instructions are the basis of compiler static transformations.
The issue of granularity is therefore of paramount importance in task-based model.
How should the extra flexibility of fine-grained tasks be balanced against amortizing the cost of runtime analysis?
For the time-being, the choice of a correct task granularity seems beyond the abilities of current compilers.
As such, choosing task granularity is the responsibility of the application developer.
The application-level API therefore expresses both correctness and granularity.

The translation layer bridges the programming model and the actual program execution.
The translation layer itself does not perform any transformations of the program execution from sequential order.
Rather, the translation layer interprets the sequential semantics in the application and creates events in an ``intermediate representation'' suitable for the backend runtime.
The backend API is therefore intended to communicate the algorithm at the right level of abstraction.
The program representation created by backend API calls should enable a runtime to make intelligent dynamic decisions about task order and task locality or possibly even task deletion and task replication when appropriate.

Strictly speaking, the backend API calls only generate a stream of deferred tasks (tasks with corresponding data inputs/outputs) that describes the inherent data flow.
However, the information passed from translation layer to backend is sufficient to (and intended to) support a CDAG (computational directed acyclic graph) representation of the application.
In a DAG representation, tasks are vertices (V) in a graph (G) with directed edges (E).
An edge from vertex $v_1$ to vertex $v_2$ indicates a precedence constraint.
Instead of directly defining task-task precedence constraints, DARMA generally describes task-data precedence constraints.
There are two types of vertices - tasks (T) and data (D) that compose the complete set of vertices (V).
Edges never directly connect two tasks and instead edges are only ever described between a task vertex, $t$, and a data vertex, $d$
indicating that (depending on direction of the edge) data is either consumed or produced by a task.
The \emph{task-DAG} indicating task-task precedence constraints can always be obtained form the data-flow task graph,
making the data-flow graph more general and therefore more useful for enabling runtime code transformations.
Although beyond the scope of this specification document, the interested reader will find numerous works discussing heuristics and order-preserving convex transformations of task graphs that demonstrate the utility of a coarse-grained CDAG for enabling dynamic runtime optimization of an algorithm.
We reiterate, though, that the CDAG is only a concept guiding the design of the backend API and not strictly part of the specificaion.

A final concern, not addressed in the current specification, is the issue of programmer-directed optimization.
While an abstract algorithm may make more information available to the compiler or runtime for performance-tuning transformations,
compilers and runtime schedulers may not always understand the global nature of the problem.
As such,  they may not make peformance-improving optimizations that are apparent to an application developer.
It will be critically important as runtimes develop to enable developers to steer the runtime towards optimizations when compilers or schedulers fail.

DARMA is a mixed \gls{imperative}/\gls{declarative} \gls{programming model}.
As much as possible, sequential imperative semantics are used to produce intuitive, maintainable code.
However, the ``procedural imperative'' function calls and code blocks do not execute immediately.
Rather than explicitly perform all work in program order and block on data requests,
DARMA provides \CC-embedded task annotations that allow work to be deferred and performed asynchronously.
The ability to defer work and advance ahead is what gives the backend runtime the ability to make performance-improving transformations.
Deferred execution makes DARMA also \emph{declarative}, leaving the exact control-flow up to the runtime.

Task parallelism is primarily achieved through permissions/access qualifiers
on data that enable that enable a runtime to reason about which tasks can run in parallel and which tasks are strictly ordered.
Task annotations are translated by the DARMA front end through \CC\ constructs (e.g., lambdas, reference counted pointers, template
metaprogramming) to expose and understand the parallelism inherent in the code.  
The \gls{translation layer} requires \CC11 standard features with a small subset of
\CC14 required for advanced features (details provided in Chapter~\ref{chap:translation_layer}), 
however the \gls{front end} \gls{API} does \emph{not} require knowledge of \CC14 to use. 
Furthermore, the \gls{back end} is a simple set of abstract \CC\ classes whose functionality must be implemented
according to the specfication in Chapter~\ref{chap:back_end}.
%providing the runtime the flexibility to optimize performance and
%exploit additional parallelism when possible.   


Most applications written in DARMA will likely have \gls{spmd} as the dominant parallelism.
To simplify the implementation of SPMD-structured codes, the notion of a \gls{rank} is maintained within the \gls{API}.   
This provides application developers a convenience mechanism for creating
the initial problem decomposition and distribution.  
Immediately after launch, deferred tasks are free to be migrated by the runtime, if it will result in better performance. 
Within a \gls{rank}, DARMA provides \gls{sequential semantics},  meaning that application developers can reason about the code as
though it were being deployed sequentially within the rank.   
Thus DARMA emphasizes sequential semantics, but supports a CSP model.


The ``communication'' in DARMA's CSP model is actually provided by \gls{coordination semantics}:  
rather than explicitly move data between ranks via direct communication
(i.e.,  \inlinecode{send/recv}), processes \emph{coordinate} by putting/getting data associated with a unique \inlinecode{key} in a
\gls{key-value store} or \gls{tuple space}.  
\Gls{coordination semantics} promote out-of-order message arrival, deferred execution, and task migration
since the app declares or describes the data it needs/produces rather than enforcing an explicit delivery mechanism.
The key to performance in the DARMA CSP model are exploiting zero-copy mechanisms and tuple caching that enable a key-value store programming model
to produce execution equivalent to an MPI send/recv code. 

Although not yet supported in version 0.3 of the specification, several
important features will play a role in the DARMA programming model:
\begin{compactdesc}
\item{\bf Expressive Underlying Abstract Machine Model:}
Notions of \glspl{execution space} and \glspl{memory space} will be introduced formally in later versions of the specification.  
These abstractions (or similar ones) appear in other runtime solutions~\cite{kokkos, others} \todo{add relevant citations here} to address deficiencies in the abstract machine model used by runtimes that support \gls{spmd} parallelism 
(i.e., uniform compute elements, flat memory spaces).  Using such abstractions
1) facilitates performance portable application development across 
a variety of execution spaces, and 2)
  provides finer-grained control and additional flexibility in the
  communication of policies regarding data locality and data movement. 
\item {\bf Data Staging:}
The memory and execution space concepts introduced above will enable 1) performance portable tasks that can run in multiple environments through a single code and 2) 
user-directed placement hints to tell the runtime where tasks should run
\item {\bf Collectives:}
Some collectives will be supported by DARMA in version 0.3 of the specification, including all-reduce, reduce-scatter, and barrier collectives.
Collectives will be data-centric rather rank-centric, as done in MPI.
\end{compactdesc}

%!TEX root = sandReportSpec.tex


\section{Execution Models}
\label{sec:exec_model}
The main focus of \gls{DARMA} is the \gls{programming model} and corresponding
\gls{translation layer} that maps a program expressed via a combination of
\gls{CSP} semantics, \gls{coordination semantics}, and additional \CC{}-embedded task annotations into a generic data-flow based
description of an algorithm based on deferred tasks.

\gls{DARMA}-compliant \gls{back end} \gls{runtime systems} are required to enable an efficient
\gls{spmd} launch of their program, similar to an MPI launch.
This is based off application developer feedback, which has indicated that 
two of the most critical challenges for scientific applications with massive data parallelism in a task-based model
include initial problem decomposition and distribution.
\gls{DARMA}'s efficient \gls{spmd} runtime-based launch requirement 
will be modified if solutions are developed to support massive \gls{spmd} launches
through compiler-based transformations.

Other than this requirement, \gls{DARMA} prescribes very little
about execution.
For example, \gls{DARMA} prescribes nothing about the scheduling of tasks nor the implementation 
of the data structures (e.g., \gls{key-value store}, \gls{tuple space})
  required to support \gls{coordination semantics}.
A \gls{back end} \gls{runtime system} scheduler is therefore free to use, for example,
either depth-first or breadth-first priorities in deferred tasks (as captured
in a \gls{CDAG}).
Similarly, a scheduler may use \glspl{thread pool} with work queues to manage
tasks or it may use a \gls{fork-join} model that creates new threads for each task.
In this way, \gls{DARMA} codes are \gls{execution model}-agnostic, only
requiring that a \gls{back end} \gls{runtime system} preserve the
\glspl{data-flow dependency} expressed in the application and derived by the
\gls{translation layer}.

Furthermore, \gls{DARMA} prescribes nothing about the internals of each task.
\gls{DARMA} is fully compatible with parallel elastic tasks - tasks with
flexible fine-grained parallelism, usually \gls{data parallelism}.
For example, depending on dynamic conditions, more or fewer threads may be
requested for a \gls{CUDA} kernel.
Although the \gls{DARMA} \gls{front end} \gls{API} currently only allows
expressing task granularity and task data-flow in the \specVersion\ version of
the specification, task elasticity will be specified in future version of the
specification.





\section{Memory Model}
\label{sec:mem_model}
The memory model for \gls{DARMA} encompasses how variables are accessed %(e.g., pointer, iterator, accessor) 
  and when updates become visible to parallel threads (concurrency).  
Within a \gls{DARMA} execution stream, memory is local or private, and the
standard \CC{} memory model applies. 
To share data between execution streams, \gls{DARMA} uses a flat global memory space in
which data is identified by unique \gls{tuple} identifiers, i.e. a \gls{key-value
  store} in which keys exist in a \gls{tuple space}.
Any object published into the \gls{tuple space} can be read/written by any thread/process. 

In \gls{DARMA} a data \gls{handle} is conceptually a \gls{reference counted pointer} into the
\gls{tuple space}.  
Data \glspl{handle} are used to manage the
complexities associated with \gls{task parallelism} and inter-\gls{rank} communication.  
When data needs to be made accessible off-\codelink{rank}, the application developer 
\codelink{publish}es the \gls{handle}.  Each \gls{handle} has a globally unique handle ID
(e.g., a \codelink{key} into the \gls{tuple space}).  
Before a \codelink{task} can begin, \glspl{handle} identifiers are resolved by the runtime to a
specific local address. Within the task, the standard \CC{} memory model applies.

When publishing, the user must specify an \gls{access group} for that data.  
Declaring an \gls{access group} informs the \gls{runtime system} that other
\codelink{rank}s currently needs or will need the data,  
allowing the runtime to manage garbage collection and \gls{anti-dependency} resolution.
In most cases, the \gls{access group} will be declared as the number of readers (1, in the case of simple point-to-point send).
Once all read \glspl{handle} are released (go out of scope in \CC{} terms),
\gls{garbage collection} or \gls{anti-dependency} resolution can occur.


In addition to facilitating coordination between \codelink{rank}s, \gls{handle} data structures 
support \gls{sequential semantics} (see Chapter~\ref{chap:translation_layer} for details).
Here concurrency is critical to the \gls{memory model} and when/how updates data are made visible to parallel threads.  
Again, within tasks, the \CC{} \gls{memory model} applies. 
At the task level (coarse-grained), \gls{DARMA} ensures atomicity of all tasks. 
The \gls{DARMA} scheduler enforces the \CC{} sequential consistency model at
the level of tasks in the same way that \CC{} ensures sequential consistency at the level of instructions. 
\gls{DARMA} understands read/write usages of tasks and ensures that writes are always visible to subsequent reads - and reads always complete before subsequent writes.  
The use of \glspl{handle} enables this to happen automatically within an execution stream.






%!TEX root = sandReportSpec.tex

\section{Data Model}
\label{sec:data_model}
\gls{DARMA} only implements a \gls{data model} through its \gls{serialization} interface.
The notion of data structure, data layout, and data type only exist in the
application and \gls{translation layer} (see
    Chapter~\ref{chap:translation_layer}).
Thus, a \gls{runtime system} implementing the \gls{DARMA} specification is only
aware of \gls{tuple} or \codelink{key} identifiers for a coarse-grained data block of a given size.
To actually migrate data, a \gls{back end} \gls{runtime system} invokes
\gls{serialization} hooks implemented by the application.
In future versions, an \gls{API} similar to the \gls{serialization} interface will support the definition of data subsets and data slices.
Again, the \gls{back end} \gls{runtime system} will only understand data and task dependencies,
  requiring the type-aware application and \glspl{translation layer} to define
  the details of subsetting and slicing operations.
This leaves the application developer free to use arbitrary data structures,
     but puts more responsibility on the application developer to articulate
     the structure of the data.






\section{Document organization}
\label{sec:organization}
This docuemnt is organized as follows.  In Chapter~\ref{chap:front_end} we
introduce the \gls{front end} \gls{API}.  In
Chapter~\ref{chap:translation_layer} we
provide a description of the \gls{translation layer}, and in
Chapter~\ref{chap:back_end} we provide the specifics regarding what must be
supproted by each of the \gls{back end} abstract classes in order to implement
the DARMA specification. In Chapter~\ref{chap:requirements} we include a list
of application requirements driving the specification (along with a list of the
    applications contributing to the requirements to date).
We conclude this document with
Chapter~\ref{chap:evolution}, which includes a brief history of changes between
previous versions of the specificaiton, along with a list of the planned changes 
in upcoming versions.

