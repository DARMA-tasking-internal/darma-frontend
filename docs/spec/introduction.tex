%!TEX root = sandReportSpec.tex
\chapter{Introduction}
\label{chap:introduction}
As we look ahead to next generation platforms and exascale computing, hardware 
will be characterized by dynamic behavior, increased
heterogeneity, decreased reliability, deep memory hierarchies, and a marked increase in system
\gls{concurrency} (both on-node and system-wide)~\cite{doe_arch, dav_exascale}. 
These architectural shifts are posing significant programming challenges for
application developers as they seek solutions for: effective management of hybrid parallelism 
at an unprecedented scale,  efficient load-balancing and work-stealing
strategies that mitigate both application and system load imbalance, and effective management and
staging of data across deep memory hierarchies.  To further complicate matters,
application codes must be made performance portable across a variety of planned
system architectures and be made resilient to the increased number of anticipated faults.  

\gls{AMT} \glspl{programming model} and \glspl{runtime system} 
are a significant shift away from the current \gls{CSP} programming model that
show promise to mitigate these challenges associated with the changes in \gls{HPC} system architectures.  
\gls{AMT} models strive to exploit all available \gls{task parallelism} and
\gls{pipeline parallelism}, rather than rely solely on \gls{data parallelism}
for \gls{concurrency}. The term {\em \gls{asynchronous}} encompasses the idea that 
1) processes (threads) can diverge to different tasks, rather than execute 
the same tasks in the same order; and 2) \gls{concurrency} is maximized (the 
  minimal amount of synchronization is performed) by 
leveraging multiple forms of parallelism. The term {\em many-task} encompasses 
the idea that the application is decomposed into many 
\gls{migratable} units of work, to enable the overlap of communication and 
computation as well as asynchronous load balancing strategies.
A key design goal of \gls{AMT} models is to enable performance-based
optimizations of code dynamically at runtime.
We note that performance-based code transformations are ubiquitous at the compiler-level.
Compilers will add, delete, swap, or reorder instructions to avoid unnecessary operations, improve data locality, or improve pipelining.
Furthermore, there are a number of compile-time optimization tools being
developed~\cite{Kokkos,RAJA}, that provide the ability to map a
single code kernel onto high-performance execution across diverse compute platforms.
However, many optimizations that benefit performance are unknowable until the program
actually runs -- as these decisions may be based on current system performance,
or the data needs of the application itself. Such dynamic runtime optimizations are much more expensive 
than compile-time optimizations, thus the use of \emph{tasks} as a basis
for dynamic runtime transformations.

The \gls{AMT} community is currently very active (e.g.,
~\cite{OCR,STAPL,Legion,Realm,StencilHPX,Charm++,Uintah,Loci,PARSEC,DaGuE,Cilk}),
representing a range of different design points within the
design space of \gls{AMT} models. While the technologies show significant
potential to address challenges, the community has not yet identified best
practices and existing systems still represent a variety of different \gls{programming model},
\gls{execution model}, \gls{memory model}, and \gls{data model} design choices.  

\begin{compactdesc}
\item[Programming model:]
From a \gls{programming model} perspective, \gls{AMT} models all have some notion of decomposing applications into small, \gls{migratable} units of work. 
\Gls{task parallelism} can be expressed in a \gls{fork-join} fashion, with users managing control-flow explicitly themselves. 
  In other \gls{AMT} \glspl{programming model}, the user expresses a step-by-step algorithm and, under some simplifying assumptions, the runtime derives the synchronizations required.
  This often takes the form of read/write data access annotations under the
  assumption of \gls{sequential semantics} enabling runtime dependency analysis.
  Models leveraging runtime analysis are best suited for coarse-grained \gls{task parallelism}, as \gls{runtime system} overheads must be amortized.
\item[Execution model:]
  Execution models broadly cover how the algorithm and corresponding
  correctness and performance constraints specified in the \gls{programming
  model} are translated to actual execution.
  For example, \gls{AMT} runtimes implement a variety of \glspl{execution
  model}, including \gls{event-based}, \gls{fork-join} (either \gls{fully
  strict} or \gls{terminally strict}), \gls{actor model}, or ubiquitous
  \gls{CSP} model.  More subtle details include whether a constant number of
  threads are always executing (e.g. pure MPI codes), new tasks are allocated
  to a thread pool, or if new threads are allocated (forked) for new tasks.
  These details will also affect the synchronizations required in an execution
  model. For example, fully strict \gls{fork-join} models will generally not
  require barriers between sibling tasks. In contrast, \gls{CSP} models will require barriers to synchronize parallel workers.
\item[Memory model:]
  \gls{HPC} \glspl{memory model} will have several properties including distributed or shared and coherent or incoherent. 
  \Glspl{distributed memory
  model} include message-passing models like MPI. \gls{PGAS} models have distinct address regions,
  but are ``shared'' in the sense that any memory location can be accessed across the system by specifying both a pointer address AND process ID. 
  In \gls{DARMA}, computational tasks by default can only operate on their local data. When remote data
     is required, it is communicated between the remote and local tasks.
    Across the spectrum of memory models, memory locations are usually
    accessed via address (put/get or send/recv), but \gls{key-value store}
    (\gls{tuple space}) models identify data regions by key identifiers (coordination).
    In coordination, parallel workers never directly communicate, instead `coordinating'' indirectly via a \gls{key-value store} or \gls{tuple space}.  
\item[Data model:]
  In order for data-flow \gls{AMT} models to make effective data management
  decisions (e.g., \gls{slicing} the data and making copies to increase
  parallelism), they must have some knowledge of the structure of the data.
  One option for providing structural information regarding data is to impose a
  \gls{data model}.  Another option is to require application developers to define
  \gls{serialization}, \gls{slicing}, and \glspl{interference test} for their data blocks. 
\end{compactdesc}

\section{Scope}\label{sec:scope}
Although the \gls{AMT} model community is quite active, the lack of
standards impedes adoption of these technologies by the application community.
Although it is premature to standardize, there is sufficient  
breadth and depth in the \gls{AMT} research community to begin developing community best
practices.  Towards this end, this document provides the specification for \gls{DARMA},
a research vehicle for \gls{AMT} \gls{programming model} \gls{co-design}.  
\gls{DARMA} aims to serve four primary purposes:
\begin{compactdesc}
\item[Insulate applications from runtime system and hardware idiosyncrasies:]
  \gls{DARMA}'s design includes separate application-facing \gls{front end} and
  \gls{runtime system}-facing \gls{back end} \glspl{API}. This separation of concerns 
  enables an application team to explore the impact of \gls{runtime system}
  design space decisions. For example, application developers can build their code using different
  \gls{DARMA}-compliant \gls{back end} implementations, without
  having to deal with the combinatorial complexity of
  implementing their application in many different \gls{front end} \glspl{API}. 
  It should be noted that \gls{DARMA}'s \gls{front end} \gls{API} is not
  fixed -- it will evolve based on \gls{co-design} feedback from both application
  and \gls{runtime system} developers. 
\item[Improve AMT runtime programmability by co-designing a front end  API directly with application developers:]
  Recent work~\cite{L2Sand2015} highlighted gaps with respect to productivity
  in some existing \gls{AMT} \glspl{runtime system}, in particular noting requirements gaps and 
  deficiencies in existing \glspl{API}. Co-designing \gls{DARMA}'s \gls{front end} \gls{API}
  directly with application developers provides a mechanism for capturing
  different application's \gls{runtime system} requirements-- giving them a voice in the design of an
  asynchronous tasking \gls{API}.  Experimenting with the \gls{API} provides an agile method for application
  developers to reason about the \gls{API} 
  %(e.g., does it allow them to intuitively express their algorithms?) 
  and better articulate their \gls{runtime system} execution requirements.
\item[Synthesize application co-design activities into meaningful requirements
  for runtimes:]
  The specification provides a mechanism for tracking the
  provenance of design decisions and requirements as they evolve throughout the
  \gls{co-design} process. Chapter~\ref{chap:requirements} provides a list of the application
  requirements gathered,  and Chapter~\ref{chap:evolution} tracks the evolution of
  the specification, highlighting which requirements motivated changes to the specification.
  \Gls{runtime system} software stack developers benefit 
  from 1) \gls{DARMA}’s application-informed requirements, and 2) access to code
  kernels and proxy applications developed via the \gls{front end}
  \gls{co-design} process.
\item[Facilitate AMT design space characterization, accelerating the
development of AMT best practices:]
  In the discussion above we summarize a range of high-level design decisions for
  \gls{AMT} programming, execution, memory, and data models. \gls{DARMA}'s
  separation of \gls{front end} and \gls{back end} \glspl{API} seeks to
  facilitate this design space characterization and exploration.  There 
  is a notable tension between the design of 1) a \gls{front end} \gls{API} that is expressive, simple, 
  and easy to incorporate within existing application code bases, and 2) a
  \gls{back end} \gls{API} that is simple enough 
  to support multiple \gls{DARMA}-compliant implementations that leverage existing \gls{runtime
  system} technologies. 
  Consequently, \gls{DARMA} \gls{API}s (both \gls{front end} and \gls{back end}) are
  intended to evolve based on iterative feedback from 
  application, \gls{programming model}, and \gls{runtime system} teams. 
\end{compactdesc}


The rest of this chapter provides a high-level description of 
\gls{DARMA}'s structural design  along with a brief summary of \gls{DARMA}'s programming,
memory, data, and (compatible) execution models. We note that throughout the
\gls{co-design} process, decisions are first and foremost, 
made to best support application requirements.  Furthermore, we target a
\gls{back end} \gls{API} specification that is general enough to support 
\gls{AMT} \gls{runtime system}
design space exploration, via build out of \gls{DARMA}-compliant
back ends using existing \gls{AMT} \gls{runtime system}
technologies.
Lastly, we note that the features detailed in Chapters~\ref{chap:front_end} and
~\ref{chap:back_end}  are not entirely comprehensive -- meaning they do not yet capture all of the
application requirements driving \gls{DARMA} \gls{co-design}.  This is because 
we are formalizing the specification process from the inception of \gls{DARMA}, layering-in features incrementally to
provide the community opportunity for input, and active engagement in the
\gls{co-design} process.  Suggested enhancements and changes 
to the \gls{DARMA} specification are welcome and can be made via a \gls{DEP} (see
Appendix~\ref{chap:DEP} for details on this process and a \gls{DEP} template). 

\section{High-level Design}
\gls{DARMA} is a translation layer between an 
application-facing \gls{front end} \gls{API} and a \gls{runtime system} facing \gls{back end} 
\gls{API}. 
\gls{DARMA}'s \gls{front end} \gls{API} 
is an \gls{EDSL} in \CC,  inheriting the generic
language constructs of \CC and adding \gls{semantics} that facilitate
distributed, deferred, asynchronous, parallel programming. Though the \gls{EDSL} uses
\CC{} constructs unfamiliar to many programmers to implement these semantics, 
it is nonetheless fully embedded in the \CC{} language and
requires a widely supported subset of \CC{}14 functionality \compilerReqs.
The \gls{front end} \gls{API} is the center of \gls{programming model}
\gls{co-design} activities, which seek to involve a wide variety of both
application and \gls{runtime system} developers. 


\gls{DARMA}'s \gls{translation layer} leverages \CC{} \gls{template
metaprogramming} to map the user's \gls{front end} \gls{API} calls onto the \gls{back end} runtime \gls{API},
bridging the \gls{programming model} and actual program execution.
We note however that the \gls{DARMA} \gls{translation layer} itself does not
perform any performance optimizations -- these are left entirely to the \gls{back end} \gls{runtime system} implementations.
Rather, the translation layer converts the application code specified with
\gls{DARMA}'s \gls{front end} \gls{API} into
an ``intermediate representation'' that enables a \gls{runtime system} to make
intelligent, dynamic decisions (e.g., 
about task order and task locality or possibly even task deletion and task
replication when appropriate).

The \gls{back end} \gls{API} is a set of abstract classes and function
signatures that \gls{runtime system} developers must implement in accordance with the
specification requirements in order to interface with application code written
to the \gls{DARMA} front end. 
Strictly speaking, the \gls{back end} \gls{API} calls only generate a stream of
\emph{deferred tasks} (tasks with corresponding data inputs/outputs) that
implicitly capture the program's data flow.
The information passed through the \gls{translation layer} to
the \gls{back end} is sufficient
to (and intended to) support a \gls{CDAG} representation of the application.
In a \gls{task-DAG} representation, tasks are vertices $V$ in a graph $G$ with directed edges $E$.
An edge from vertex $v_1$ to vertex $v_2$ indicates a precedence constraint.
A \gls{CDAG} representation describes task-data precedence constraints, rather
than just task-task precedence constraints.  In a \gls{CDAG} there are two
types of vertices - tasks $T$ and data $D$ that compose the complete set of vertices $V$.
Edges never directly connect two tasks and instead edges are only ever described between a task vertex, $t$, and a data vertex, $d$
indicating that (depending on direction of the edge) data is either consumed or produced by a task.
The \gls{task-DAG} indicating task-task precedence constraints can always be
obtained from the \gls{CDAG}, which captures the data-flow task graph.
The \gls{CDAG} is thus more general, capturing additional information to
enable runtime code transformations.\footnote{
Although beyond the scope of this specification document, the interested reader
will find numerous works discussing heuristics and order-preserving convex
transformations of task graphs that demonstrate the utility of a coarse-grained
\gls{CDAG} for enabling dynamic runtime optimization of an
algorithm~\cite{missingCitations}.}
We reiterate that the \gls{CDAG} is only a concept guiding the design
of the \gls{back end} \gls{API} and not strictly part of the \gls{DARMA}
specification.


Finally, we highlight that a \gls{DARMA} executable application must link to a \gls{runtime
  system} that implements the abstract \gls{back end} runtime \gls{API}.
It is intended that these implementations will be external, drawing upon
existing AMT technologies.  However,  a reference implementation will be provided
in the \gls{DARMA} code distribution.

The terms ``rank'', ``task'', and `process'' are loaded terms with many definitions across the literature. 
Here we give special attention to define rigorous and limited definitions for special terms.
We use ``process'' in the usual UNIX sense. Other terms are:
\begin{compactdesc}
\item [Task]: The work unit instantiated directly by the application developer. 
Tasks are also the smallest granularity of \emph{migratable} work unit. 
In the current spec, tasks cannot migrate after beginning execution.
Tasks are guaranteed to make forward progress, but are interruptible.
\item [Execution stream]: An execution stream will consist of a sequence of many tasks, and, like tasks, is guaranteed to make forward progress.
All execution streams are tasks, but execution streams specifically have no parent task and are the root of an independent task-DAG.
Each execution stream is guaranteed to have a unique stack and, any point time, will have a local context of variables.
A physical process (in the UNIX sense) can be running many parallel execution streams.
Allowing multiple execution streams per physical process is the basis for overdecomposition.
Since several execution streams can exist in the same process address space, this introduces a strict requirement of no global variables.
An execution stream is the DARMA generalization of a thread, except that extra privatization of variables is necessary since no assumption of shared memory between independent streams can be made (even if streams happen to be executing in the same process).
Just as processes must perform special operations to exchange data between them (message-passing, mmap), independent execution streams must perform special key-value store operations to exchange data between them.
Execution streams are always assigned a unique identifier by the runtime.
\item [Operation]: Used synonymously with work unit. This is a unit of execution that is guaranteed to be non-interruptible. 
An operation is not equivalent to a task since tasks are interruptible.  
Operations are the smallest, schedulable units of work.  
A task consists of a sequence of operations.
While tasks are explicitly instantiated by the app developer, operations (individual portions of task) can be implicitly instantiated by the runtime.
Tasks can yield at the beginning/end of its component operations, allowing the runtime to schedule new work units for execution.
\item [Rank]: A unique integer ID for an execution stream. This matches the MPI notion of rank as an integer identifying a process within an MPI communicator.
The term rank will often be used in the specification as a synonym for execution stream (more precisely, a metonymy for execution stream).
Generally speaking, $N$ parallel execution streams are created in an SPMD launch (more in Section \ref{}).
The runtime then assigns unique identifiers (rank IDs) $0$ through $N-1$ to each stream.
Referring to ``rank 0'' will therefore function as shorthand for ``the execution stream that has been assigned rank ID 0 by the runtime in an SPMD launch.''
Similarly, referring generically to a ``rank'' is shorthand for `an execution stream created by an SPMD launch with a particular rank ID.''
\end{compactdesc} 


\section{Programming Model}
\label{sec:programming_model}
In DARMA \gls{spmd} is the dominant parallelism and the notion of a \gls{rank}
is maintained within the \gls{API}.   
  The DARMA \gls{API} simplifies the introduction
  of asynchronous task parallelism via the use of standard \CC\ constructs, such
  as \gls{reference counted pointers} and \glspl{lambda}.

  
The \gls{API} is being \gls{co-design}ed with application developers
and computer scientists whose knowledge spans the entire runtime software stack.
Current applications affecting the design include portions of the \gls{Trilinos} software stack
and \gls{ASC} \gls{ATDM} efforts at Sandia National Laboratories. 
 
% NOTE: this is morphed text from open mp/ocr specs.  is this needed?
%This specification covers user-directed parallelization, where the user
%has annotated that portions of the work could be performed asynchronously.
%The DARMA runtime will schedule all asynchronous work in a \gls{deferred} fashion,
%checking that all \gls{input dependencies} are met prior to performing work.  
%The user is responsible for ensuring that the application using the DARMA
%\gls{API} constructs executes correctly.



%!TEX root = sandReportSpec.tex


\section{Execution Models}
\label{sec:exec_model}
The main focus of \gls{DARMA} is the \gls{programming model} and corresponding
\gls{translation layer} that maps a program expressed via a combination of
\gls{CSP} semantics, \gls{coordination semantics}, and additional \CC{}-embedded task annotations into a generic data-flow based
description of an algorithm based on deferred tasks.

\gls{DARMA}-compliant \gls{back end} \gls{runtime systems} are required to enable an efficient
\gls{spmd} launch of their program, similar to an MPI launch.
This is based off application developer feedback, which has indicated that 
two of the most critical challenges for scientific applications with massive data parallelism in a task-based model
include initial problem decomposition and distribution.
\gls{DARMA}'s efficient \gls{spmd} runtime-based launch requirement 
will be modified if solutions are developed to support massive \gls{spmd} launches
through compiler-based transformations.

Other than this requirement, \gls{DARMA} prescribes very little
about execution.
For example, \gls{DARMA} prescribes nothing about the scheduling of tasks nor the implementation 
of the data structures (e.g., \gls{key-value store}, \gls{tuple space})
  required to support \gls{coordination semantics}.
A \gls{back end} \gls{runtime system} scheduler is therefore free to use, for example,
either depth-first or breadth-first priorities in deferred tasks (as captured
in a \gls{CDAG}).
Similarly, a scheduler may use \glspl{thread pool} with work queues to manage
tasks or it may use a \gls{fork-join} model that creates new threads for each task.
In this way, \gls{DARMA} codes are \gls{execution model}-agnostic, only
requiring that a \gls{back end} \gls{runtime system} preserve the
\glspl{data-flow dependency} expressed in the application and derived by the
\gls{translation layer}.

The main focus of DARMA is the programming model and corresponding translation layer that maps CSP semantics onto a data-flow description of the algorithm based on deferred tasks.
DARMA prescribes nothing about the scheduling of tasks or implementation of key-value store semantics supporting process coordination (see below).
A backend scheduler is therefore free to use, e.g. either depth-first or breadth-first priorities in executing tasks on the CDAG.
Similarly, a scheduler may use thread pools with work queues to run tasks or it may use a fork-join model that creates new threads for each task.
DARMA codes are therefore execution model-agnostic, only requiring that a backend runtime preserve the data-flow constraints expressed in the application and derived by the translation layer.

DARMA furthermore prescribes nothing about the internals of each task.
DARMA is fully compatible with parallel elastic tasks - tasks with flexible fine-grained parallelism, usually data parallelism.
For example, depending on dynamic conditions, more or fewer threads may be requested for a GPU kernel.
Although the DARMA application-level API currently only allows expressing task granularity and task data-flow,
we expect the API to also express task elasticity in future versions.



\section{Memory Model}
\label{sec:mem_model}
The memory model for \gls{DARMA} encompasses how variables are accessed %(e.g., pointer, iterator, accessor) 
  and when updates become visible to parallel threads (concurrency).  
Within a \gls{DARMA} execution stream, memory is local or private, and the
standard \CC{} memory model applies. 
To share memory between execution streams, \gls{DARMA} uses a flat global memory space in
which data is identified by unique \gls{tuple} identifiers, e.g. a \gls{key-value
  store} or \gls{tuple space} abstraction.  
Any object published into the \gls{tuple space} can be read/written by any thread/process. 

In \gls{DARMA} a data \gls{handle} is conceptually a \gls{reference counted pointer} into the
\gls{tuple space}.  
Data \glspl{handle} are used to manage the
complexities associated with \gls{task parallelism} and inter-\gls{rank} communication.  
When data needs to be made accessible off-\codelink{rank}, the application developer 
\codelink{publish}es the \gls{handle}.  Each \gls{handle} has a globally unique handle ID
(e.g., a \codelink{key} into the \gls{tuple space}).  

When publishing, the user must specify an \gls{access group} for that data.  
Declaring an \gls{access group} informs the \gls{runtime system} that other
\codelink{rank}s currently needs or will need the data,  
allowing garbage collection and \gls{anti-dependency} resolution.
In most cases, the \gls{access group} will be declared as the number of readers (1, in the case of simple point-to-point send).
Once all read \glspl{handle} are released (go out of scope in \CC{} terms),
\gls{garbage collection} or \gls{anti-dependency} resolution can occur.

In addition to facilitating coordination between \codelink{rank}s, \gls{handle} data structures 
support \gls{sequential semantics} (see Chapter~\ref{chap:translation_layer} for details).
Within a \codelink{task}, \glspl{handle} identifiers are resolved to a
specific local address and the standard \CC{} memory model applies.

\todo{what is confusing?}
%JEREMY: While the following is illustrative, some of the phrasing was a little
%confusing
%This is similar to UPC which provides a global address space, but data is always accessed by pointer address rather than tuple. 
%This further contrasts with MPI 2-sided in which data are accessed by address, but there is no global memory space.  
%Data within in an MPI process is ?private? and can only be exchanged via messages.  
%DARMA , like UPC, further contrasts with PGAS (partitioned global address spaces) in that the global memory space is ?flat? and is not explicitly partitioned across processes.
%The tuple space vs address space is only relevant for creating and managing coarse-grained tasks. 

Also critical is concurrency in the \gls{memory model} and when/how updates data are made visible to parallel threads.  
Again, within tasks, the \CC{} \gls{memory model} applies. 
At the task level (coarse-grained), \gls{DARMA} ensures atomicity of all tasks. 
The \gls{DARMA} scheduler enforces the \CC{} sequential consistency model at
the level of tasks in the same way that \CC{} ensures sequential consistency at the level of instructions. 
\gls{DARMA} understands read/write usages of tasks and ensures that writes are always visible to subsequent reads - and reads always complete before subsequent writes.  
The use of \glspl{handle} enables this to happen automatically within a
\codelink{rank} 
(shared memory), but requires the application developer to use coordination
for objects exchanged between \codelink{rank} to ensure data consistency.





\section{Compatible Data Models}
\label{sec:data_model}


DARMA provides a very flexible data model, leaving the application developer
free to use whatever data structures they like. 
DARMA provides serialization mechanisms for bitwise copiable data structures.
The user must provide serialization information for non-bitwise copiable data
structures. 
\todo[inline]{add details here}






\section{Document organization}
\label{sec:organization}
This document is organized as follows.  In Chapter~\ref{chap:front_end} we
introduce the \gls{front end} \gls{API}.  In
Chapter~\ref{chap:translation_layer} we
provide a description of the \gls{translation layer}, and in
Chapter~\ref{chap:back_end} we provide the specifics regarding what must be
supported by each of the \gls{back end} abstract classes in order to implement
the DARMA specification. In Chapter~\ref{chap:requirements} we include a list
of application requirements driving the specification (along with a list of the
    applications contributing to the requirements to date).
We conclude this document with
Chapter~\ref{chap:evolution}, which includes a brief history of changes between
previous versions of the specification, along with a list of the planned changes 
in upcoming versions.
Appendix~\ref{chap:examples} provides a suite of examples that illustrate the
 \gls{front end} \gls{API} features. Appendix~\ref{chap:vasp} provides
additional technical details regarding %\gls{vasps}.
\todo[inline]{Add glossary entry for vasps and uncomment above.}
Appendix~\ref{chap:DEP} provides information regarding how the broader
community can shape the \gls{DARMA} specification and includes a template for 
a \gls{DEP}. 

