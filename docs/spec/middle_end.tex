\chapter{Translation Layer}
\label{chap:middle_end}

A key design principle of DARMA is the ability to explore the design space of
backend AMT runtime implementations without requiring changes in the application
code.  Since the frontend is EDSL-like (embedded domain specific language-like)
and most backends with which we want to interface use traditional C or C++
constructs, a layer is needed that translates EDSL-based application code into
C++ constructs that the backends can easily implement and interact with.  Given
that DARMA is strictly embedded in C++, this layer makes heavy use of newer C++
motifs and features from C++11 and C++14, such as template metaprogramming,
perfect forwarding, constant expressions (\inlinecode{constexpr}), and lambda
capture.  Many of the additions to C++ in recent years have centered around
making it easier for the user to express compile-time optimizations and
transformations that the compiler can make to reduce runtime overhead.  As such,
much of the translation DARMA does between the frontend EDSL and the backend API
happens at compile time, and should result in minimal runtime overhead with most
modern compilers.

Even though neither the frontend application programmers nor the backend
implementation developers need to understand the implementation details of this
layer, it is useful to document several of the idioms and ``tricks'' used in the
translation layer for those wishing to have a thorough understanding of all
DARMA layers, and particularly for those who wish to contribute to the expansion
and adaptation of the EDSL that is the frontend.  Thus, a few of the basic
techniques and concepts used by the translation layer are documented below.

\section{Lambda Capture for Automatic Dependency Detection and Versioning}

The most pivotal trick to document here is (semi-)automatic dependency detection
through the \CC11 lambda mechanism and the copy {\it capture-default} (that is,
\inlinecode{[=]}).  The \CC\ standard specifies that if the copy
{\it capture-default} is given, any variables that are ODR-used\footnote{``one
definition rule''-used.  See
\url{http://en.cppreference.com/w/cpp/language/definition\#ODR-use} for
details.} 
As this source states, ``Informally, an object is odr-used if its address is
taken, or a reference is bound to it, and a function is odr-used if a function
call to it is made or its address is taken.'' inside of the lambda's scope but
defined outside of it\footnote{technically, defined in the lambdas
``reaching scope,'' which is also formally defined} are copied by value into
that lambda's scope.  Furthermore, if a lambda with a copy {\it capture-default}
is moved\footnote{e.g., with \inlinecode{std::move()}; lambdas have a deleted
copy constructor}, the move constructors of the inner scope copies will be
invoked (or, if no user-defined move constructor is given but a user-defined
copy constructor is, the copy constructor is invoked).  We can leverage this
fact along with a thread-safe (and thread-specific) global context object to
associate captured \inlinecode{AccessHandle<T>} objects with the capturing
\inlinecode{create_work(...)} invocation as dependencies.  Furthermore, if we
make the relevant members of \inlinecode{AccessHandle<T>} \inlinecode{mutable},
we can modify the handle that has been {\it copied from} to increment the
version, so that later tasks will depend on the completion of earlier
\inlinecode{create_work} calls that capture the same handle.  Thus, the capture
mechanism can be used for both dependency detection and sequential semantics.
\todo[inline]{add proof of concept example}

\section{Hierarchical Versioning}

Hierarchical versioning of handles in the translation layer is critical to the
implementation of sequential semantics in DARMA. \todo[inline]{finish this}

\section{Keyword Arguments}

The tricks used to emulate keyword arguments in \CC\ are well-known and have been
exploited elsewhere\footnote{e.g., the Boost::Parameter library} to similar
effect.  The addition of perfect forwarding and constant expression semantics to
\CC11 and \CC14 allow this to be done with rigorously zero runtime overhead ---
all transformations used to interpret keyword arguments as traditional,
positional arguments can occur at compile time.
