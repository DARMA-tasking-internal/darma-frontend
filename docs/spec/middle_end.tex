%!TEX root = sandReportSpec.tex

\chapter{Translation Layer}
\label{chap:translation_layer}

A key design principle of \gls{DARMA} is the ability to explore the design space of
\gls{back end} \gls{AMT} \gls{runtime system} implementations without requiring changes in the application
code.  Since the \gls{front end} \gls{API} is essentially an \gls{EDSL}
and most \gls{back end} \glspl{runtime system} with which we want to interface
use traditional C or \CC{}
constructs, a layer is needed that translates \gls{EDSL}-based application code into
\CC{} constructs that the \gls{back end} \glspl{runtime system} can easily implement and interact with.  Given
that \gls{DARMA} is strictly embedded in \CC{}, this layer makes heavy use of
newer \CC{} motifs and features from \CC{}11 and \CC{}14, such as \gls{template
metaprogramming}, \gls{perfect forwarding}, constant expressions
(\inlinecode{constexpr}), and \gls{lambda} \gls{capture}.  Many of the
additions to \CC{} in recent years have centered around
making it easier for the user to express compile-time optimizations and
transformations that the compiler can make to reduce runtime overhead.  As such,
much of the translation \gls{DARMA} does between the \gls{front end} \gls{EDSL}
and the \gls{back end} \gls{runtime system} \gls{API}
happens at compile time, and should result in minimal runtime overhead with most
modern compilers.

\section{Separation of Responsibilities Across Layers}
\gls{DARMA} separates responsibilities across the three different layers:
application, translation, and \gls{back end} \gls{runtime system}.
The list below describes the most important quantities and concepts that are
required for writing and running \gls{DARMA} applications.
Each layer will either read, write, or never use each quantity.  
Some of these quantities are parts of the specification while certain other quantities 
are introduced for illustrating concepts, and are not strictly part of the specification.
Some quantities are repeated from previous sections.

\begin{compactitem}
\item AccessHandle: a variable (templated on data type) that is used in the application as arguments to 
  \glspl{task} and for reading/writing values in a data block.
  Each \gls{task} has its own unique copy of AccessHandle. AccessHandles are
  never shared across \glspl{task}.
\item Data Type: the type of a variable, e.g., \inlinecode{int}, \inlinecode{vector<double>}.
\item Data Layout: the layout or internal structure of a data type, usually telling whether a type is contiguous in memory and whether a type holds only data (e.g., double) or has lookup pointers.
\item Data Size: the total size a data block occupies in memory (number of bytes).
\item Task Dependencies: a relationship between a \gls{task} and data
  indicating the \gls{task} depends on the data being available before the
  \gls{task} can run.
\item Task Precedence Constraints: a relationship between \glspl{task} indicating
  that an ordering constraint exists between \glspl{task}.
\item Access Permissions: access permissions (read, read-write, etc.) for an
  AccessHandle within a \gls{task}.
\item Address: a pointer through which data is accessed.
The pointer provides no information on the size or type of memory being accessed.
It merely provides a means of accessing data at a particular memory location.
\end{compactitem}

The following items are not strictly part of the specification, but are useful
for having a rigorous vocabulary to explain and understand the \gls{translation
layer}.  These are quantities that are \emph{likely} to be used in a \gls{back
end} \gls{runtime system} implementation, but are \emph{not required}.
As will be discussed in the \gls{back end} section, all of these quantities (if
used by a \gls{back end} \gls{runtime system}),
exist in an abstract class \inlinecode{Instance} that the \gls{translation
layer} interacts with.
\begin{compactitem}
  \item Handle ID: the generalization of a variable in \CC{}. A globally unique ID identifying a block of data that represents the ``same'' quantity across time.  
This corresponds to, e.g., values mesh that are updated iteratively.  
This is NOT synonymous with an actual physical location.
\item Generation: an ID that distinguishes logically distinct generations of the same Handle ID. 
Updating the values in a Data Handle changes the data and therefore progresses the generation. 
Two data blocks with the same Handle ID that contain different logical times (usually different iterations) will be different generations.
\item Data Blocks: the actual physical memory allocations where data lives. 
Data blocks comprise not just an address, but potentially size and location information such as whether memory is DRAM, HBM, GPU, or remote.
\item Logical ID: a tuple of Generation and Handle ID. 
Two Data Handles with the same logical ID must access exactly the same values, but potentially different physical locations, and thus are logically the same. 
All objects with the same Logical ID are required to have the same Handle ID.
Thus, Logical ID equivalence is a stronger condition than Handle ID equivalence.
\item Physical ID: a tuple of Address, Generation, and Handle ID (although not required to be implemented as a tuple). 
Two Data Handles with the same Physical ID not only access exactly the same values, but must also access exactly the same memory location. 
Two logically distinct blocks that happen to access the same memory location at different times do NOT share a Physical ID. 
All data \glspl{handle} with the same Physical ID must have the same Logical ID and therefore the same Handle ID. 
Physical ID equivalence is then a stronger condition than Logical ID or Handle ID.
\end{compactitem}

The way in which quantities are used in each layer of the software stack can have several possibilities:
\begin{compactitem}
\item Modifies: The layer both reads and manipulates the given quantity.
\item Creates: A subset of Modifies. The layer creates the initial version of something, but is not allowed to modify the quantity thereafter.
\item Reads: The layer reads and understand a quantity, but is not allowed to manipulate it.
\item Opaque Modify: The caller layer understands operations that need to be performed that will modify a struct, but the implementation details are hidden by an interface. 
For example, a caller can pass a forward-declared struct by pointer to a function. 
The function (callee) can modify integer members within the struct. 
Even though the caller initiates the modification, the internal details of the struct are opaque to the caller and are only known to the callee (function).
\item Opaque Pass: A caller provides values in a struct to be read by a callee function, but the values are opaque to the caller function. 
Similar to Opaque Modify, but the callee function cannot modify the struct.
\item DNE: The concept does not exist - it is neither manipulated nor read by a layer.
\end{compactitem}
We now summarize where quantities are created, modified, and read in Table \ref{tbl:conceptsInLayers}.

\begin{table}
  \centering
\begin{tabular}{l l l l}
\hline
Quantity & App & Translation Layer & Backend \\
\hline
Data Type & Creates & Reads & DNE \\
Data Layout & Modifies & Reads & DNE \\
Data Size & Modifies & Reads & Reads \\
Task Dependencies & Modifies & Reads & Reads \\
Task Order Constraints & DNE & Creates & Modifies \\
Address & Reads & DNE & Modifies \\
Access Permissions & Modifies & Reads & Reads \\
Handle ID & DNE & Opaque Create & Creates \\
Generation & DNE & Opaque Modify & Creates \\
Logical ID & DNE & Opaque Pass & Creates \\
Physical ID & DNE & Opaque Pass & Modifies 
\end{tabular}
\caption{Which concepts are modified by a given layer, which exist opaquely, and finally which concepts do not exist (DNE).}
\label{tbl:conceptsInLayers}
\end{table}

A critical part of the \gls{DARMA} design avoids potential interference between layers that can read/modify the same quantity. 
Operations occurring in the \gls{back end} must have guarantees that the application
and \gls{translation layer} are not creating conflicts.
We must therefore define a life cycle for each \gls{task}:
\begin{compactitem}
\item Precursor: Dependencies are modified during the execution of a precursor
  \gls{task}.
\item Initialization: The \gls{task} is created and initialized before being
  passed to a scheduler or \gls{task} queue.
\item Waiting:  The \gls{task} has been created is waiting in a queue to be scheduled or run.
\item Running: The \gls{task} has been popped from the queue and is actively running.
\item Deletion: The \gls{task} has finished running and is releasing its resources.
\end{compactitem}

\begin{table}
  \centering
\begin{tabular}{l l l l l l}
\hline
Quantity & Precursor & Initializing & Waiting & Running & Deletion \\
\hline
Data Type & App Creates & TL Reads & TL Reads & App Reads & TL Reads \\
Data Size & App Creates & TL,BE Reads & BE Reads & App Modifies & TL,BE Reads \\ 
Access Permissions & App Modifies & TL,BE Reads & BE Reads & n/a & TL,BE Reads \\
Task Dependencies & App Creates & TL,BE Reads & BE Reads & n/a & TL,BE Reads \\
Task Order Constraints & n/a & TL Creates & BE Modifies & n/a & BE Reads \\
Address & n/a & BE Modifies & BE Modifies & App Reads & BE Reads 
\end{tabular}
\caption{Usage of different quantities throughout a single \gls{task}'s life cycle}
\label{tbl:taskLifeCycle}
\end{table}

Critically, Table \ref{tbl:taskLifeCycle} ensures that each layer has no conflicts. 
No two layers simultaneously have modify permissions during the life cycle of a
\gls{task}, nor can one layer read simultaneously as another layer modifies. 
Two layers can simultaneously read, although even that rarely happens.

\section{Important C++ Concepts}
Even though neither the \gls{front end} \gls{API} application programmers nor the \gls{back end}
implementation developers need to understand the implementation details of the
\gls{translation layer}, it is useful to document several of the idioms and ``tricks'' used in the
\gls{translation layer} for those wishing to have a thorough understanding of all
\gls{DARMA} layers, and particularly for those who wish to contribute to the expansion
and adaptation of the \gls{EDSL} that is the \gls{front end} \gls{API}.  Thus, a few of the basic
techniques and concepts used by the translation layer are documented below.

\subsection{Lambda Capture for Automatic Dependency Detection and Versioning}
The most pivotal trick to document here is (semi-)automatic dependency detection
through the \CC11 \gls{lambda} mechanism and the copy {\it capture-default} (that is,
\inlinecode{[=]}).  The \CC\ standard specifies that if the copy
{\it capture-default} is given, any variables that are ODR-used\footnote{``one
definition rule''-used.  See
\url{http://en.cppreference.com/w/cpp/language/definition\#ODR-use} for
details.
As this source states, ``Informally, an object is odr-used if its address is
taken, or a reference is bound to it, and a function is odr-used if a function
call to it is made or its address is taken.''}
inside of the \gls{lambda}'s scope but
defined outside of it\footnote{technically, defined in the \gls{lambda}'s
``reaching scope,'' which is also formally defined} are copied by value into
that \gls{lambda}'s scope.  Furthermore, if a \gls{lambda} with a copy {\it capture-default}
is moved\footnote{e.g., with \inlinecode{std::move()}; \glspl{lambda} have a deleted
copy constructor}, the move constructors of the inner scope copies will be
invoked (or, if no user-defined move constructor is given but a user-defined
copy constructor is, the copy constructor is invoked).  We can leverage this
fact along with a thread-safe (and thread-specific) global context object to
associate captured \inlinecode{AccessHandle<T>} objects with the capturing
\inlinecode{create_work(...)} invocation as dependencies.  Furthermore, if we
make the relevant members of \inlinecode{AccessHandle<T>} \inlinecode{mutable},
we can modify the \gls{handle} that has been {\it copied from} to increment the
version, so that later tasks will depend on the completion of earlier
\inlinecode{create_work} calls that \gls{capture} the same \gls{handle}.  Thus, the
\gls{capture} mechanism can be used for both dependency detection and \gls{sequential
semantics}.

\subsection{Keyword Arguments}
\label{sec:kwargs}
The tricks used to emulate \glspl{keyword argument} in \CC\ are well-known and have been
exploited elsewhere\footnote{e.g., the Boost::Parameter library} to similar
effect.  The addition of perfect forwarding and constant expression semantics to
\CC11 and \CC14 allow this to be done with rigorously zero runtime overhead ---
all transformations used to interpret \glspl{keyword argument} as traditional,
positional arguments can occur at compile time.
