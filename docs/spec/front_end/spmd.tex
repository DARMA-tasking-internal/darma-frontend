%!TEX root = ../sandReportSpec.tex

\section{SPMD support}
\label{sec:spmd}
\lstMakeShortInline[style=CppCodeInlineStyle]{\|}
% DSH: Removed this TODO.  It's good enough for now.
%\todo[inline]{this section needs to be revised.  Didn't Jeremy already take a
%crack at it?  His changes don't appear to be here\ldots} 
%\gls{spmd} being the
%dominant model of parallelism in DARMA, \todo{are we comfortable saying this like this?  Is it even really true?} the spin-up of the
%runtime in each \gls{spmd} unit is kept analogous to MPI. 
Most applications written in or ported to DARMA will likely have \gls{spmd} as
the dominant form of parallelism.
To simplify the implementation of SPMD-structured codes, the notion of 
a \gls{rank} is maintained within the \gls{API}.   
This should make it a little easier for existing MPI based application codes to transition 
to DARMA. The initialization and termination of the runtime in each unit 
is via the calls |darma_init()| and |darma_finalize()|. 
The total number of \gls{spmd} units are queried with the call |darma_spmd_size()|,
and the rank is queried with |darma_spmd_rank()|. A typical user 
written main program will look as follows:
\begin{CppCode}
int main(int argc, char**argv){

	darma_init(argc, argv);
	size_t n_ranks = darma_spmd_size();
	size_t me = darma_spmd_rank();

	//lots of code

	darma_finalize();
}
\end{CppCode}

The rank is a very useful concept to orchestrate dependencies in a \gls{spmd}
model since data pertaining to a rank can be associated with a key that utilize
the rank for uniqueness. Similarly, the unique key used by a different rank
from which data needs to be fetched can be constructed from the knowledge of
the rank creating the data.  The example below illustrates this concept, where
the rank is integral to the key associated with data originating on that rank
\begin{CppCode}
size_t me = darma_spmd_rank();
auto data_handle = initial_access<double>("data_key", me);
\end{CppCode}  
Note that in DARMA, SPMD ranks are actually just a special kind of task that
happens to have a name containing the rank, and can be treated as such. 
However, the similarity to traditional, MPI-style SPMD upon launch will simplify
the learning and porting process significantly.


\lstDeleteShortInline{\|}