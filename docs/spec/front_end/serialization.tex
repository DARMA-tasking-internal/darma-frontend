%!TEX root = ../sandReportSpec.tex

\section{Serialization and Layout Description}
\label{sec:serialization}

\lstMakeShortInline[style=CppCodeInlineStyle]{\|}

The DARMA front end programming model provides an extremely flexible and
extensible interface for describing serialization and/or layout of C++ types. 
In spite of this flexibility, the vast majority of use cases only require the
understanding and use of one or two very basic abstractions.  However, the DARMA
serialization interface provides a wide variety of features to handle complex
and corner cases, as well as features to tune and optimize performance-critical
cases.  The following section describes the DARMA serialization interface,
beginning with abstractions that handle the vast majority of use cases and
expanding to progressively more niche features later in the section.

Note that this section is entitled ``Serialization and Layout Description''
rather than just ``Serialization'' because the interface provides ways to
specify movement of data in ways that aren't traditionally considered
serialization, such as describing a type as a series of \gls{RDMA} pointers with
associated sizes.  More details to follow.

\subsection{Basic Intrusive Interface}
\label{ssec:serbasic}

The most basic and straightforward way to specify serialization of a user type
in DARMA, and the method that should be used in the vast majority of cases
(with, perhaps, one simple extension discussed below), is providing a publicly
accessible\footnote{later versions of the spec may allow private
implementations with a \inlinecode{friend} specification} |serialize| method
in the user class.  The |serialize| method provided for this purpose should be
non-|const| and should take a single argument, which in the simplest case will
be an lvalue reference to a
|darma_runtime::serialization::SimplePackUnpackArchive| object.  For instance,
consider the following (somewhat contrived) user-defined class:
%
\begin{CppCodeNumb}
class MyClass {
  private:
    double a_, b_;
    std::string label_;
    double prod_sqrt_;
  public:
    static constexpr const char unlabeled_string[] = "<unlabeled>";
    MyClass(int a, int b)
      : MyClass(unlabeled_string, a, b)
    { }
    MyClass(std::string const& label, int a, int b)
      : a_(a), b_(b), label_(label),
        prod_sqrt_(a_ == b_ ? a_ : std::sqrt(a_*b_))
    { }
};
\end{CppCodeNumb}
%
The simplest way to allow DARMA to interact with |MyClass| is to provide the
a |serialize| method in the class definition:
\begin{CppCodeNumb}
using darma_runtime::serialization::SimplePackUnpackArchive;
class MyClass {
  public:
    /* ... */
    void serialize(SimplePackUnpackArchive& ar) {
      ar | a_ | b_ | label_ | prod_sqrt_;
    }
};
\end{CppCodeNumb}
As you can see, the type |SimplePackUnpackArchive| has an overload for
\inlinecode{operator|()} and takes a serializable type and returns itself (more
on what constitues a ``serializable type'' later).

\subsection{\texttt{SimplePackUnpackArchive}}

DARMA encapsulates advanced serialization behaviors in the archive
\gls{concept}.  The only archive type fully implemented in the current
specification is |SimplePackUnpackArchive|, which performs serialization in the
most basic and traditional way.  On the sender side, DARMA performs
two serialization passes: one in sizing mode and one in packing mode.  The receiver
only requires one pass: unpacking.  These modes can be queried using the archive
object methods |is_sizing()|, |is_packing()|, and |is_unpacking()|.  (Exactly
one of these will return true at any given time; also, all archive types
implement these methods).

\subsection{Generic Archive Serialization}

As DARMA develops further and as more performance considerations are addressed,
the DARMA team and our collaborators plan to provide other archive types which
take a more advanced serialization strategies, such as enabling \gls{RDMA}
access to pieces of a type.  In order to write code that can take advantage of
these features when they become available, the vast majority of user types can
simply provide a templated |serialize| method:
\begin{CppCodeNumb}
class MyClass {
  public:
    /* ... */
    template <typename Archive>
    void serialize(Archive& ar) {
      ar | a_ | b_ | label_ | prod_sqrt_;
    }
};
\end{CppCodeNumb}

\subsection{Different Behaviors in Different Modes}

Consider again the |MyClass| example above.  Since |prod_sqrt_| can be
recomputed on the fly, it may be desirable to avoid including it in the data to
be moved and instead just recompute it on the receiving side.  To do this, however,
we need the |serialize| method to perform different actions in unpacking mode
than in the other modes.  The |is_unpacking()| method makes this easy:
%
\begin{CppCodeNumb}
class MyClass {
  public:
    /* ... */
    template <typename Archive>
    void serialize(Archive& ar) {
      ar | a_ | b_ | label_;
      if(ar.is_unpacking())
        prod_sqrt_ = a_ == b_ ? a_ : std::sqrt(a_*b_);
    }
};
\end{CppCodeNumb}
%
Notice also that the |label_| field of |MyClass| has the same static value if a
label is ungiven every time.  If |MyClass| often does not have a |label_|, it
may be advantagous to pack a boolean indicating whether the label exists,
followed by the label itself only if the label is given.  We can do this using
the same approach:
\begin{CppCodeNumb}
class MyClass {
  public:
    /* ... */
    template <typename Archive>
    void serialize(Archive& ar) {
      ar | a_ | b_;
      bool has_label;
      if(!ar.is_unpacking()) {
        has_label = label_ != unlabeled_string;
        ar | has_label;
        if(has_label) ar | label_;
      }
      else { // ar.is_unpacking()
        ar | has_label;
        if(has_label) ar | label_;
        else label_ = unlabeled_string;    
        // From before:
        prod_sqrt_ = a_ == b_ ? a_ : std::sqrt(a_*b_);
      }
    }
};
\end{CppCodeNumb}

\subsection{Seperate Methods for Seperate Modes}
The performance-concious reader may have noticed that the consecutive
serialization of a large number of |MyClass| objects may involve a significant
number of unnecessary extra branch instructions (i.e., |if|
statements),\footnote{in many cases, an advanced optimizer or branch predictor
{\it may} be able to coallesce some of these, but this isn't the general case}
which can cause branch prediction misses and thus significantly reduce
performance.  For this and other reasons, DARMA allows the user to specify
seperate |pack|, |unpack|, and |compute_size| methods as needed.  Each takes an
archive object as an argument, and the |pack| and |compute_size| methods must
be |const|.  The first example that recomputes |prod_sqrt_| could then be
rewritten as:
\begin{CppCodeNumb}
class MyClass {
  public:
    /* ... */
    template <typename Archive>
    void serialize(Archive& ar) {
      ar | a_ | b_ | label_;
    }
    template <typename Archive>
    void unpack(Archive& ar) {
      ar | a_ | b_ | label_;
      prod_sqrt_ = a_ == b_ ? a_ : std::sqrt(a_*b_);
    }
};
\end{CppCodeNumb}
The more specialized |pack|, |unpack|, and |compute_size| methods always have
higher priority than |serialize|, so in this case DARMA will invoke |unpack|
during the unpacking pass while still calling |serialize| in the sizing and
packing passes.

The {\it very} performance-concious reader may have further noticed that since
the \inlinecode{operator|()} implementation must function in all three modes,
there must be further branch instructions inside its implementation.  Thus, in a
performance-critical context, the user may want to provide all three methods and
use an operator that is specific to the phase in question.\footnote{This could
also be accomplished (and may be in the future) by passing in different types
for the sizing, packing, and unpacking phases, which is yet another reason to
use the templated versions of these methods instead.}  This can be done using
|operator<<()| for packing, |operator>>()| for unpacking, and |operator%| for
sizing.  The final |serialize| with both the |prod_sqrt_| and |label_|
optimizations could then be rewritten as:
%
\begin{CppCodeNumb}
class MyClass {
  public:
    /* ... */
    template <typename Archive>
    void compute_size(Archive& ar) const {
      ar % a_ % b_;
      if(label_ == unlabeled_string) ar % false;
      else ar % true % label_;
    }
    template <typename Archive>
    void pack(Archive& ar) const {
      ar << a_ << b_;
      if(label_ == unlabeled_string) ar << false;
      else ar << true << label_;
    }
    template <typename Archive>
    void unpack(Archive& ar) {
      ar >> a_ >> b_;
      bool has_label;
      ar >> has_label;
      if(has_label) ar >> label_;
      else label_ = unlabeled_string;
      prod_sqrt_ = a_ == b_ ? a : std::sqrt(a_*b_);
    }
};
\end{CppCodeNumb}
%
Besides being significantly more verbose, keep in mind that these optimizations
have important implications for maintainability of your code.  If another member
variable |c_| were added to |MyClass|, the serialization implementation in the
final example would have to be modified in {\it three} places, whereas the
earlier examples, while potentially less performant, only have to be updated in
one place.  Also, failure to ensure that the order of member variable
serialization is identical in multiple places can lead to hard-to-detect bugs. 
Thus, we recommend using the single |serialize| method except in
performance-critical, inner-loop-like code.

\subsection{Serializing Pointers and Ranges}

DARMA also provides a convenient way to serialize iterables of serializable
objects using |darma_runtime::serialization::range|.  As a simple example:
\begin{CppCodeNumb}
template <typename T>
class MyData {
  private:
    T* data_;
    size_t n_items;
  public:
    MyData(T const* copy_from, size_t n)
      : n_items(n) {
      data_ = new T[n];
      std::copy(copy_from, copy_from+n, data_);
    }
    ~MyData() { delete[] data_; }
};
\end{CppCodeNumb}
If we restrict ourselves to only making |MyData<T>| instances that hold
serializable types |T|, we can write the |serialize| method for this class as
\begin{CppCodeNumb}
using darma_runtime::serialization::range;
template <typename T>
class MyData {
  public:
    /* ... */
    template <typename Archive>
    void serialize(Archive& ar) {
      ar | range(data_, data_ + n_items);
    }
};
\end{CppCodeNumb}

\subsection{Non-intrusive Interface}

Classes for which the user cannot define an intrusive |serialize| method (or
any of the other intrusive methods) for one reason or another can still be
made serializable by defining a specialization (partial or full) of the class
|darma_runtime::serialization::Serializer<T>| for the type in
question.\footnote{Generic implementations requiring partial specialization
with an enable_if clause should use
\inlinecode{darma_runtime::serialization::detail::Serializer_enabled_if<T,
Enable>}.  Consult source code for more details.}  Like the intrusive interface,
these classes can define a |serialize| method; individual |compute_size|,
|pack|, and |unpack| methods; or some combination of these, with the specific
versions having higher priority.  All of these methods must be |const| (the
Serializer object itself isn't allowed to have state anyway; it's just a
convenient mechanism for grouping functions for a class nonintrusively).  Their
signatures are a bit different from the intrusive analogues.  Consider a
slightly different version of |MyClass| from above, the public
interface of which is specified as:
\begin{CppCodeNumb}
class YourClass {
  public:
    double get_a() const;
    void set_a(double val);
    double get_b() const;
    void set_b(double val);
    std::string const& get_label() const;
    void set_label(std::string const& val);
    double get_product_sqrt() const;
};
\end{CppCodeNumb}
Assuming |YourClass| is default constructible, way to specify a serialization
for |YourClass| non-intrusively is:
\begin{CppCodeNumb}
namespace darma_runtime { namespace serialization {
template <>
struct Serializer<YourClass> {
  template <typename Archive>
  void serialize(YourClass& yc, Archive& ar) const {
    if(!ar.is_unpacking()) {
      double a = yc.get_a();
      double b = yc.get_b();
      std::string label = yc.get_label();
      ar | a | b | label;
    }
    else {
      double a, b;
      std::string label;
      ar | a | b | label;
      yc.set_a(a);
      yc.set_b(b);
      yc.set_label(label);
    }
  }
};
}} // end namespace darma_runtime::serialization
\end{CppCodeNumb}
As before, we can split this into |serialize| and |unpack| methods.  However,
the |unpack| method requires a slightly different signature:
\begin{CppCodeNumb}
template <typename Archive>
darma_runtime::serialization::Serializer<YourClass>::unpack(
  void* allocated, Archive& ar
) const;
\end{CppCodeNumb}
Rather than being a reference to an instance of the class itself, the first
argument to the non-intrusive |unpack| method is a pointer to the beginning of a
chunk of memory of size |sizeof(YourClass)| allocated by the backend, but not
constructed.  This allows for the unpacking of non-default-constructible
classes.  The |unpack| method must construct the object at that memory
location using the C++ placement new.  The syntax of placement new might be a
little strange if you've never seen it before, but once you see it, its use is
pretty straightforward.  The non-intrusive |Serializer| for |YourClass| can then
be written as:
\begin{CppCodeNumb}
namespace darma_runtime { namespace serialization {
template <>
struct Serializer<YourClass> {
  template <typename Archive>
  void serialize(YourClass& yc, Archive& ar) const {
    assert(!ar.is_unpacking()) // just in case
    double a = yc.get_a();
    double b = yc.get_b();
    std::string label = yc.get_label();
    ar | a | b | label;
  }
  template <typename Archive>
  void unpack(void* allocated, Archive& ar) const {
    double a, b;
    std::string label;
    ar | a | b | label;
    // Since YourClass is default-constructible, the placement new
    // that we want looks like this:
    YourClass* yc = new (allocated) YourClass();
    yc->set_a(a);
    yc->set_b(b);
    yc->set_label(label);
  }
};
}} // end namespace darma_runtime::serialization
\end{CppCodeNumb}
The nonintrusive interface versions of |pack| and |compute_size| have similar
signatures to that of |serialize|, except they take a |const| lvalue reference
as their first argument:
\begin{CppCodeNumb}
template <typename Archive>
darma_runtime::serialization::Serializer<YourClass>::compute_size(
  YourClass const& val, Archive& ar
) const;
template <typename Archive>
darma_runtime::serialization::Serializer<YourClass>::pack(
  YourClass const& val, Archive& ar
) const;
\end{CppCodeNumb}

The non-intrusive serialization interface has higher priority than the intrusive
one, but in general the user should not define both or mixed intrusive and
nonintrusive serializations

\subsection{Definition of ``Serializable''}

Having introduced the archive concept, the intrusive interface, and the
nonintrusive interface, we're finally ready to formally define ``serializable''
as DARMA sees it.  DARMA views the serializability of a given type as a property
associated with that type and a given archive type --- a type T can be described
as ``serializable with archive type A''.  We've been sloppy about this up to
this point because it's usually clear from context which archive type we're
referring to (or if we're referring to a generic archive type given as a
template parameter).  This allows for the development of archive types that, for
instance, only handle performance sensative types but in a highly performant
manner.  

\subsection{Implementations for Builtin and Standard Library Types}

The DARMA translation layer has default |Serializer| implementations for many
builtin and standard library types.  Currently this includes anything that meets
the standard container \gls{concept} (for which the value types are also
serializable), POD types, |std::pair| of serializable types, and
compile-time sized arrays of serializable types.  Many of the implementations in
the current backend are with respect to a generic archive type, but since
|SimplePackUnpackArchive| is the only archive type currenly fully implemented,
the code is only tested with this archive thus far.

You can define a serialization for any bitwise copyable, POD type simply by
specializing |darma_runtime::serialization::serialize_as_pod<T>| to inherit from
|std::true_type|:
\begin{CppCodeNumb}
class MyPlainOldData {
  int i, j, k;
  double x, y, z; 
};
namespace darma_runtime { namespace serialization {
template <>
struct serialize_as_pod<MyPlainOldData> : std::true_type { };
}} // end namespace darma_runtime::serialization
\end{CppCodeNumb}

\subsection{Polymorphism}

Deserialization into polymorphic base class pointers is currently not supported
by the DARMA serialization interface.  If a type is to be used in a context that
requires DARMA to deserialize it (most importantly, as the wrapped by an
|AccessHandle<T>| or the type of an argument passed to a
functor-style |create_work| that could be migrated), the concrete type must be
known at compile time.  Support for this may be forthcoming, but it is not a
high priority.  There are a number of other patterns in the programming
literature that can be used to mimic run-time polymorphism, and we suggest the
user consider these if necessary.

\subsection{Serialization Pitfalls}

\begin{itemize}
  \item Because DARMA detects the various intrusive and nonintrusive
  |serialize|, |pack|, |unpack|, etc. methods, |const|-incorrectness or an
  otherwise incorrect signature can cause these methods to go undetected and
  lead to unexpected behavior.  For instance,
\begin{CppCodeNumb}
class MyClass {
  public:
    /* ... */
    template <typename Archive>
    void serialize(Archive& ar) {
      ar | a_ | b_ | label_;
    }
    template <typename Archive>
    void unpack(Archive ar) { // $\no$ missing lvalue reference in parameter!
      ar | a_ | b_ | label_;
      prod_sqrt_ = a_ == b_ ? a_ : std::sqrt(a_*b_);
    }
};
\end{CppCodeNumb}
would fail to ever define |prod_sqrt_| because the |unpack| method would not be
detected and DARMA would fall back on |serialize| for the unpacking process. 
Most of the common mistakes we anticipate are checked with |static_assert|s, but
it is impossible to check all possible mistakes.  Care should be taken when this
issue could arise.  Note that if only |serialize| had been defined incorrectly,
for instance:
\begin{CppCodeNumb}
class MyClass {
  public:
    /* ... */
    template <typename Archive>
    void serialize(Archive& ar) const { // $\no$ should not be const!
      ar | a_ | b_ | label_ | prod_sqrt_;
    }
};
\end{CppCodeNumb}
then the code that uses |MyClass| would simply fail to compile, since |MyClass|
isn't serializable with any archive types.
\end{itemize}

\lstDeleteShortInline{\|}
