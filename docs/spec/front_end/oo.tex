
\lstMakeShortInline[style=CppCodeInlineStyle]{\|}

\section{Object-Oriented Programming:  Classes with Deferred Methods in DARMA}

The abstractions presented thus far are very useful for small-
and medium-scale code development, and could certainly be used for
large-scale development.  Most modern C++ codes, however, make heavy use of
object-oriented (OO) programming to facilitate a scalable development process. 
OO programming has a number of well-documented benefits, including, for
instance, better encapsulation and code reuse than many other programming
paradigms.  Though the features presented thus far can interact with basically
any ``traditional'' C++ class, the standard use of OO in C++ does not provide
the same support for asynchronous, declarative programming that the DARMA
programming model encourages.  Analogous to the difference between a
``traditional'' C++ function and the DARMA deferred functor interface (see
\S~\ref{sec:functor}), the DARMA programming model provides an OO interface in
which method invocations can be treated as deferred work creation and member
variables are encapsulated in \ahandleT objects with deferred access to
data.

\subsection{A Demonstrative Example}
\label{subsec:oodemon}

Consider the following (albeit contrived) class with fields and methods that pay
homage to the three stooges and the Simpsons, respectively:
\begin{CppCodeNumb}
class MyClassTraditional {
  private:
    int larry = 0;
    
  public:
    double curly = 0.0;
    string moe;

    // writes larry, reads curly
    void marge() {
      larry = floor(sqrt(curly));
    }

    // takes a double argument and modifies curly
    void homer(double a) {
      curly = (a/2.0)*(a/3.0)
        * (curly == 0.0 ? 1.0 : curly);
    }

    // takes a string argument, reads curly, modifies moe
    void lisa(string const& new_moe) {
      if(curly > 3.14) moe = "hello-" + moe;
      else moe = new_moe;
    }

    // reads larry, modifies moe, calls lisa
    void bart() {
      if(larry < 0) lisa("bart-was-here");
      else {
        string old_moe = moe;
        for(int i = 0; i < larry; ++i) moe += old_moe;
      }
    }

};
\end{CppCodeNumb}
Even though it is obviously contrived, |MyClassTraditional| shares many
characteristics with a lot of typical use cases of OO programming in C++.  It
has some public and private data members and some methods that act on some or
all of these data members.  These methods read and write specific data members
of the class, and some even call other methods of the class.  Unfortunately, the
C++ programming model currently provides no way to express {\it which} of these
fields are read and written by a given method.  In fact, the C++ syntax only
provides one piece of information in this regard:  a method can be marked
|const| if it doesn't modify {\it any} of the members or call any non-|const|
methods.\footnote{The \inlinecode{mutable} keyword, of course, creates
exceptions to this rule, and \inlinecode{volatile} complicates things further,
but we're ignoring those here for simplicity}  Otherwise, the method is assumed
to modify {\it all} of the member variables in the class.  (The |const| method
case isn't much more helpful: the method is considered to read {\it all} of the
member variables).  

For the purposes of exposing the maximum asynchrony to a modern AMT runtime,
this programming model is woefully inadequate.  A much more helpful syntax might
look something like
\begin{CppCodeNumb}
// Not real C++, but it would be nice if it were
void marge() [[const(curly), nonconst(larry)]] {
  larry = floor(sqrt(curly));
}
\end{CppCodeNumb}
Short of writing C++ compiler extensions, we can't make this syntax available in
DARMA.  Through some clever template metaprogramming tricks (see
Chapter~\ref{chap:translation_layer}), though, we can expose an interface that 
achieves the same effect and has a syntax that is almost as intuitive. To whet
your appetite, here's an implementation of the |marge| method, from the example
above, as part of an analogous class called |MyClassDARMA|:
\begin{CppCodeNumb}
// ...preceded by some other stuff to be explained later...
template<>
struct MyClassDARMA_method<marge>
  : darma_method<MyClassDARMA,
      reads_<curly>,
      modifies_<larry>
    >
{
  using darma_method::darma_method;
  void operator()() {
    larry.set_value(floor(sqrt(curly.get_value())));
  }
};
\end{CppCodeNumb}
Note that a few details were omitted above for brevity, such as namespaces and a
few preprocessor macros (all explained below), but assuming the right
declarations and `using` statements precede this snippet, the above code is
actually what the definition of the |marge| method looks like when written in
the DARMA OO interface.  More importantly, it works like you would expect it to
from the given syntax.

\subsection{Preprocessor Macros for Defining ``Name Tags''}
\label{subsec:oomacros}

Like much of the DARMA interface, the DARMA OO interface only requires very
sparing use of preprocessor macros.\footnote{As DARMA develops, we may provide
more ``convenience'' preprocessor macros, but our intention is that the user
can use the vast majority of the functionality with only minimal use of
preprocessor macros, if they want.  Additionally, we try to only hide code with
macros that can be understood with just a moderate level of C++ expertise, such
that the intermediate and advanced user can understand the purpose and function
of the generated code with a little bit of work.  The macros in the OO
interface are of this variety, and are explained in rough terms in
Chapter~\ref{chap:translation_layer}.}  Users of the DARMA OO interface need
only interact with two preprocessor macros --- |DARMA_OO_DEFINE_TAG()| and
|DARMA_OO_DECLARE_CLASS()|.  While more details are given in
Chapter~\ref{chap:translation_layer}, the basic function of these macros is
pretty straightforward:  they allow subsequent code to use a given
name\footnote{``symbol'' in typical C++ lingo} as a method or data member, in
the former case, and as a class in the latter case.  Their use is very
simpmle:
\begin{CppCodeNumb}
// Allow DARMA to use `larry` as a field name or a method name in *any*
// DARMA class defined subsequently
DARMA_OO_DEFINE_TAG(larry);

// Declare all of the symbols necessary to define a class named `MyClassDARMA`
DARMA_OO_DECLARE_CLASS(MyClassDARMA);
\end{CppCodeNumb}

\subsubsection{\texttt{DARMA\_OO\_DEFINE\_TAG}}

The first thing to note about |DARMA_OO_DEFINE_TAG| is that just use of the
macro is sufficient to define a name that can be used in multiple (potentially
unrelated) classes.  Once defined, a name can be used as a field or a method in
any class, and can have a different type or signature in each of
them.\footnote{A tag can't be used as a method and as a field in the same class
(just as you can't have a method and a field with the same name in a regular
C++ class), but it is perfectly acceptable to use a tag as a data member
(field) in one class and as a method in another)} Thus, in a large project, it
may be useful to define all of the tags that the code will use (or at least all
of the tags that will be used in a given namespace) in one place to avoid
accidental naming clashes.  Also, like any class name or variable name in a
large C++ project, it is important to properly protect the symbols declared this
way in an appropriate namespace, particularly when they are used in a header file.

The only publicly visible symbol this macro defines is the one given as the
macro argument.  All other symbols are for internal use and prefixed with
|_darma__| (recall that all symbols prefixed with |_darma__| are reserved for
internal use by DARMA).

\subsubsection{\texttt{DARMA\_OO\_DECLARE\_CLASS}}

The easiest way to conceptualize the effect of |DARMA_OO_DECLARE_CLASS| is to
think of it as a (albeit required) forward declaration of the class given as the
macro argument.  Thus, for instance, this macro must be given in the same
namespace that the class is defined and must come before that definition. 
Unlike |DARMA_OO_DEFINE_TAG|, the declaration by |DARMA_OO_DECLARE_CLASS| cannot
be reused in multiple contexts (again, just like an ordinary forward
declaration of a class).

In addition to forward declaring the class |Name| (supposing that |Name| is the
argument to the macro), this macro defines the class |Name_constructors| and the
template |Name_method<>|, both of which are discussed below.

\subsection{Defining a Class}

Equipped with the preprocessor macros from \S~\ref{subsec:oomacros}, we are now
ready to define the structure of the class |MyClassDARMA|, a revised version of
|MyClassTraditional| from \S~\ref{subsec:oodemon}.  The tags can be defined in
the same file as the class if necessary, or in a header file for common use in a
larger project.  Either way, the following lines of code need to somehow precede
the definition of the class |MyClassDARMA|\footnote{Though they do not
necessarily need to precede its {\it declaration}, via the macro
\inlinecode{DARMA_OO_DECLARE_CLASS}}:
\begin{CppCodeNumb}
// Alphabetizing is a good idea, but (of course) not necessary
DARMA_OO_DEFINE_TAG(bart);
DARMA_OO_DEFINE_TAG(curly);
DARMA_OO_DEFINE_TAG(homer);
DARMA_OO_DEFINE_TAG(larry);
DARMA_OO_DEFINE_TAG(lisa);
DARMA_OO_DEFINE_TAG(marge);
DARMA_OO_DEFINE_TAG(moe);
\end{CppCodeNumb}
Now we'll give the complete definition of the class |MyClassDARMA| and elaborate
on the details below.  Brace yourself\ldots
\begin{CppCodeNumb}
// Forward declaration:
DARMA_OO_DECLARE_CLASS(MyClassDARMA);

// Class definition:
struct MyClassDARMA
  : darma_class<MyClassDARMA,
      private_fields<
        int, larry
      >,
      public_fields<
        double, curly,
        string, moe
      >,
      public_methods<
        marge,
        homer,
        lisa,
        bart
      >
    >
{ using darma_class::darma_class; } $\label{ln:ootraditional}$
\end{CppCodeNumb}
If you're not used to heavy duty C++ template metaprogramming (or even if you
are), the above code can be very intimidating to parse as C++, but don't panic:
you don't really have to understand how it works to know what it means.  Even
though it is proper C++, if it's helpful you can think of it as a new syntax
for expressing classes.

Once you get past the excessive angle brackets, the snippet reads pretty
straightforwardly:  it defines a class with a private |int| field named |larry|,
two public field (|double curly| and |string moe|), and four methods: |marge|,
|homer|, |lisa|, and |bart|.  These methods' signatures and dependencies will be
defined later, using the template |MyClassDARMA_method<>| (generated by the
|DARMA_OO_DECLARE_CLASS| macro).

Notice that, syntactically speaking, the only
contents of the ``traditional'' C++ class (between the curly braces) is 
|using darma_class::darma_class;|.  (This tells C++ to inherit constructors from
the |darma_class| template, but you don't need to think of it this way.  Actual
user-defined constructors for classes in the DARMA OO interface are discussed
below.)  Note that having these contents of the ``traditional'' part of the C++
class (line \ref{ln:ootraditional}) and nothing more is a hard requirement of
the DARMA OO syntax.  Nothing else should go between the curly braces.

The class templates |private_fields|, |public_fields|, and |public_methods|
(along with most of the symbols in this section) are defined in the namespace
|darma_runtime::oo|.  They can be given in any order and any number of times. 
Multiple lists (for instance, multiple |public_fields|), given to the
same |darma_class| template will be merged (though, of course, the same name
still should not be given for more than one field or method).

An instance of |MyClassDARMA| will hold an \ahandleT data member for each of the
|public_fields| and |private_fields| given in the class's definition.  Just like
a regular class, public members are accessible via the dot operator on an
instance of the class:

\begin{CppCodeNumb}
MyClassDARMA my_instance;
my_instance.curly = initial_access<double>("some", "key", "here");
\end{CppCodeNumb}
These members can be passed to work created using the functor interface like any
other variable.  However, if they are to be used with the lambda interface, more
care needs to be taken to avoid capturing the entire |MyClassDARMA| instance
(and thus, all of its field)
\begin{CppCodeNumb}
MyClassDARMA my_instance;
/* ...set up my_instance... */

// $\no$ all members of my_instance captured
// (works, but program is oversynchronized)
create_work([=]{
  my_instance.curly.set_value(3.14)
});

// $\yes$ captures only my_instance.curly
// note the reference in auto& and enclosing scope to release the reference
{
  auto& my_curly = my_instance.curly;
  create_work([=]{
    my_curly.set_value(3.14);
  });
}

// $\yes$ alternative syntax; doesn't require enclosing scope
create_work([=,my_curly=my_instance.curly]{
  my_curly.set_value(3.14);
});
\end{CppCodeNumb}

\subsection{Methods}

As previously alluded to, methods can of the class |MyClassDARMA| are defined by
specializing the template |MyClassDARMA_method<>|.  As previewed in
\S~\ref{subsec:oodemon}, the |marge| method for |MyClassDARMA| can be written as
\begin{CppCodeNumb}
template<>
struct MyClassDARMA_method<marge>
  : darma_method<MyClassDARMA,
      reads_<curly>,
      modifies_<larry>
    >
{
  using darma_method::darma_method;
  void operator()() {
    larry.set_value(floor(sqrt(curly.get_value())));
  }
};
\end{CppCodeNumb}
We can now explain the details of this snippet more thoroughly.  Methods of
DARMA classes specify their requirements by including a template specifying the
access they need in the list of template arguments to the |darma_method|
template that they inherit from.  As you would expect, the
|reads_<>|\footnote{note the trailing underscore.  This is to distinguish it
from the reads() decorator used with arguments to \inlinecode{create_work}. 
(Even though these are in different namespaces, we thought it would be too
confusing to have two different things with the same name like this). The other
templates also have a trailing underscore for consistency.} template requests read-only permissions\footnote{Both immediate and scheduling permissions, to be more precise} on the given member, and the |modifies_<>| template requests read-write permissions.  The argument to these templates must
be one of the fields listed in the |public_fields| or |private_fields| lists for
the corresponding class (and thus, the class definition must precede any method
definitions).  DARMA also provides the |reads_value_| and |modifies_value_|
templates that automatically dereference the corresponding \ahandleT, similar to
taking a reference formal parameter to the DARMA functor interface (see
\S~\ref{sec:functor}).  Thus, the |marge| method could have been written more
compactly as
\begin{CppCodeNumb}
template<>
struct MyClassDARMA_method<marge>
  : darma_method<MyClassDARMA,
      reads_value_<curly>,
      modifies_value_<larry> $\label{ln:modlarry}$
    >
{
  using darma_method::darma_method;
  void operator()() {
    larry = floor(sqrt(curly));
  }
};
\end{CppCodeNumb}

While this syntax is a bit more verbose than the ``traditional'' C++ method
syntax, the extra verbosity allows for substantially increased flexibility in
terms of asynchrony.  It also provides for a substantial amount of compile-time
checking.  If, for instance, the snippet above omitted line \ref{ln:modlarry}
which indicates that |marge| modifies |larry|, a normal, standards-compliant C++
compiler would refuse to compile the code because |marge| does not have access
to the member |larry|.  Similarly, if |marge| had requested read-only access to
|larry| instead of |modifies_value_|, a normal C++ compiler would complain about
an attempt to modify a |const| value (or, in the case of |read_| instead of
|modify_|, it would complain about trying to call |set_value| on a
|ReadAccessHandle<T>|).

\subsection{Methods Calling Other Methods}

Just like traditional C++ methods, methods of DARMA classes can call other
methods in the same class.  However, the callee's required permissions must be a
(potentially trivial) subset of the permissions required by the caller.  If not,
a compile-time error will result.  Furthermore, the caller must have scheduling
permissions on all of the fields required by the callee (see
\S~\ref{sec:handlerules} for details of scheduling and immediate permissions). 
Since the |reads_value_| and |modifies_value_| templates don't expose an
\ahandleT to a method (and thus, implicitly, do not request scheduling
permissions), permissions requests with those templates do not allow a method to
call another method requiring the given member.  Use |reads_| and |modifies_| instead for this purpose.

By way of example, consider the |bart| and |lisa| methods in
|MyClassTraditional| from \S~\ref{subsec:oodemon}.  Their implementation in
|MyClassDARMA| might look like:
\begin{CppCodeNumb}
template<>
struct MyClassDARMA_method<lisa>
  : darma_method<MyClassDARMA,
      reads_value_<curly>,
      modifies_value_<moe>
    >
{
  using darma_method::darma_method;
  void operator()(string const& new_moe) {
    if(curly > 3.14) moe = "hello-" + moe;
    else moe = new_moe;
  }
};

template<>
struct MyClassDARMA_method<bart>
  : darma_method<MyClassDARMA,
      modifies_<moe>,
      reads_value_<larry>,
      reads_<curly> // via lisa() invocation
    >
{
  using darma_method::darma_method;
  void operator()() {
    if(larry < 0) lisa("bart-was-here");
    else {
      string old_moe = moe.get_value();
      for(int i = 0; i < larry; ++i) 
        moe.set_value(moe.get_value() + old_moe);
    }
  }
};
\end{CppCodeNumb}

\subsection{Controlling Invocation with \texttt{immediate} and
\texttt{deferred}}

By default, calling a method on an instance of a DARMA class enqueues a deferred
invocation of that method --- analogous to calling |create_work<Functor>(...)|
instead of |Functor()(...)|.  When a DARMA method requests the appropriate
permissions to do so (e.g., member access on the callee must be a subset of
member access on the caller, etc.), however, it can also make an immediate call
to another method in the same class.  To do so, simply prefix the method
invocation with |immediate::|.  For instance, if benchmarking determined that
it was unnecessary to schedule |lisa| as a separate task in the above example,
|bart| could have been implemented instead as
\begin{CppCodeNumb}
template<>
struct MyClassDARMA_method<bart>
  : darma_method<MyClassDARMA,
      modifies_value_<moe>,
      reads_value_<larry>,
      reads_value_<curly>
    >
{
  using darma_method::darma_method;
  void operator()() {
    if(larry < 0) this->immediate::lisa("bart-was-here");
    else {
      string old_moe = moe;
      for(int i = 0; i < larry; ++i) moe += old_moe;
    }
  }
};
\end{CppCodeNumb}
(Note that the |this->| is unnecessary in the above example, but was included
for clarity.  |immediate::lisa(...)| would work just fine, but
some may think of this as the syntax for static method invocation, which this
is not a case of).  Notice that the definition of |bart| no longer requires
scheduling permissions on |curly| and |moe|, since we are making an
|immediate::| call to |lisa|, so we can use |modifies_value_| and |reads_value_|
instead of |modifies_| and |reads_|.

\subsection{Method Arguments}

Methods of DARMA classes handle arguments the same way that the DARMA functor
interface does.  For instance, you can pass an |AccessHandle<int>| to a method
that takes an |int&| as a formal parameter and DARMA will infer a request of
modify permissions\footnote{technically, modify-leaf permissions} on that
argument; or you can also pass an |AccessHandle<int>| to a method that takes a
formal parameter of |ReadAccessHandle<long>| and DARMA will infer read
permissions and perform the proper conversions.  For more information, see
\S~\ref{sec:functor}.

\subsection{User-defined Constructors}

User-defined constructors of DARMA classes follow an analogous syntactic pattern
to that of DARMA methods, with some notable differences.  For a class named
|MyClassDARMA|, user-defined constructors are implemented in the overloads
of the call operator of a class named |MyClassDARMA_constructors| that inherits
from a specialization of |darma_constructors<>| on |MyClassDARMA|.  For
instance, if we wanted to provide a constructor to |MyClassDARMA| that
initializes the \ahandleT objects to have keys with variable names appended, we
could do something like:
\begin{CppCodeNumb}
struct MyClassDARMA_constructors
  : darma_constructors<MyClassDARMA>
{
  void operator()(darma_runtime::types::key_t& key_part) {
    larry = initial_access<int>(key_part, "larry");
    curly = initial_access<double>(key_part, "curly");
    moe = initial_access<string>(key_part, "moe");
  }
};
\end{CppCodeNumb}
It is important that the implemenation of |MyClassDARMA_constructors|
not contain {\it any} data members --- only methods.  The constructor will
execute the code in the corresponding overload of the call operator.  Multiple
overloads (with different arguments) of the call operator can be given, just as
multiple constructors can be given for a traditional C++ class.

One more caveat for DARMA class constructors:  if you wish to given a
user-defined copy or move constructor, you have wrap the class type in the
|darma_class_instance<>| template; the rest of the constructor signature remains
the same:
\begin{CppCodeNumb}
struct MyClassDARMA_constructors
  : darma_constructors<MyClassDARMA>
{
  /* ... */
  // copy constructor:
  void operator()(darma_class_instance<MyClassDARMA> const& other) {
    // ... copy constructor stuff here ...
  }
  // move constructor:
  void operator()(darma_class_instance<MyClassDARMA>&& other) {
    // ... move constructor stuff here ...
  }
};
\end{CppCodeNumb}

\lstDeleteShortInline{\|}

 








