
%%%%%%%%%%%%%

%\clearpage

\subsection{\texttt{darma\_spmd\_rank}}

\paragraph{Summary}\mbox{}\\
Returns the rank of the calling process within the DARMA environment.

\paragraph{Syntax}\mbox{}\\
\begin{CppCode}
int darma_runtime::darma_spmd_rank();
\end{CppCode}

\paragraph{Positional Arguments} \mbox{}\\
None. 

\paragraph{Output}\mbox{}\\
The rank value.

\paragraph{Details}\mbox{}\\
This function returns the rank of the calling process 
in the DARMA environment. If the size of the environment is $N$, 
then the returned rank, $r$, is always $0 <= r <= N-1$.
\todo{anything missing? details?}

\paragraph{Code Snippet}\mbox{}\\ 
\begin{figure}[!h]
\begin{CppCodeNumb}
#include <darma.h>
int darma_main(int argc, char** argv)
{
  using namespace darma_runtime;
  darma_init(argc, argv);

  // get my rank
  const int myRank = darma_spmd_rank();
  // get size 
  const int size = darma_spmd_size();

  std::cout << "Rank " << myRank << "/" << size << std::endl;

  darma_finalize();
  return 0;
}
\end{CppCodeNumb}
\label{fig:fe_api_ranksize}
\caption{Basic usage of \texttt{darma\_spmd\_rank}.}
\end{figure}


