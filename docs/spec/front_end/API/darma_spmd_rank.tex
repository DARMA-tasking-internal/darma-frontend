
%%%%%%%%%%%%%

%\clearpage

\subsection{\texttt{darma\_spmd\_rank}}

\paragraph{Summary}\mbox{}\\
\codetarget{darma_spmd_rank} returns the \gls{rank} index associated with the \gls{opaque context}
\todo{David: Originally you had the term opaque rank context here, but everywhere else
opaque context.  Please confirm this edit is ok} from which this
function was invoked.

\paragraph{Syntax}\mbox{}\\
\begin{CppCode}
/* unspecified */ darma_runtime::darma_spmd_rank();
\end{CppCode}

\paragraph{Positional Arguments} \mbox{}\\
None. 

\paragraph{Output}\mbox{}\\
An object of unspecified type that may be treated as a \inlinecode{std::size_t}
which is less than the value returned by \codelink{darma_spmd_size}.

\paragraph{Details}\mbox{}\\
This function returns the rank index of the calling DARMA \gls{rank}.  (Recall that a
rank is the \gls{opaque context} between the invocation of \codelink{darma_init} and
a corresponding invocation of \codelink{darma_finalize}).  If the value returned
by \codelink{darma_spmd_size} is convertible to a \inlinecode{std::size_t} with the
value $N$, then the value returned by this function will be convertible to a
\inlinecode{std::size_t} with the value $r$, which will always satisfy $0 <=
r < N$.  Furthermore, the type of the value returned by this function will
always be directly comparable to the type returned by \codelink{darma_spmd_size}
and to $0$ such that this previous condition is met.  The value returned is also
equality comparable with $0$, the value returned will be true for equality
comparison with $0$ on exactly one rank.  The value returned by this function
will be unique on every \gls{rank} (in the equality sense), and will be the same
across multiple invocations of the function within a given \gls{rank}.  The value
returned will also be the same at any asynchronous work invocation depth within
a rank's \gls{opaque context}, regardless of whether that work gets stolen or
migrated.

\paragraph{Code Snippet}\mbox{}\\ 
\begin{figure}[!h]
\begin{CppCodeNumb}
#include <darma.h>
int darma_main(int argc, char** argv)
{
  using namespace darma_runtime;
  darma_init(argc, argv);

  // get my rank
  const size_t myRank = darma_spmd_rank();
  // get size 
  const size_t size = darma_spmd_size();

  std::cout << "Rank " << myRank << "/" << size << std::endl;

  darma_finalize();
  return 0;
}
\end{CppCodeNumb}
\label{fig:fe_api_ranksize}
\caption{Basic usage of \protect\codelink{darma_spmd_rank}.}
\end{figure}


