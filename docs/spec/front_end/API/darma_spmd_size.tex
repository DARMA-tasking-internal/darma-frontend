\subsection{\texttt{darma\_spmd\_size}}


\paragraph{Summary}\mbox{}\\
Returns the number of ranks in the DARMA environment.

\paragraph{Syntax}\mbox{}\\
\begin{CppCode}
/* unspecified */ darma_runtime::darma_spmd_size();
\end{CppCode}

\paragraph{Positional Arguments} \mbox{}\\
None. 

\paragraph{Return} \mbox{}\\
An object of unspecified type that may be treated as a \texttt{std::size\_t}
giving the number of ranks in the DARMA environment.

\paragraph{Details} \mbox{}\\
This function gives the number of ranks DARMA is executing the program with. 
Specifically, it is the number of times the backend has invoked
\texttt{darma\_main} anywhere in the system for this particular run of the
program (and thus, it is also the number of times the backend expects the user
to invoke \texttt{darma\_init}).

\paragraph{Code Snippet} \mbox{}\\
\begin{figure}[!h]
\begin{CppCodeNumb}
#include <darma.h>
int darma_main(int argc, char** argv)
{
  using namespace darma_runtime;
  darma_init(argc, argv);

  const size_t size = darma_spmd_size();
  // ...

  darma_finalize();
  return 0;
}
\end{CppCodeNumb}
\label{fig:fe_api_ranksize}
\caption{Basic usage of \texttt{darma\_spmd\_size}.}
\end{figure}

\paragraph{Restrictions and Pitfalls}\mbox{} \\
\begin{itemize}
  \item The value returned by this function will always return \texttt{true}
  for greater-than comparison with 0, and will always be convertible to a
  \texttt{std::size\_t} with a value greater than 0.
  \item The return type is unspecified to allow future expansion to generalized
  ranks.  For instance, future versions of the spec may allow the user to
  request the rank as an \texttt{\{x, y, z\}} tuple of indices in a structured
  lattice.
\end{itemize}

