\subsection{\texttt{initial\_access}}

\paragraph{Summary}\mbox{}\\ 
Handle to data that does not yet exist in the system 
but needs to be created.

\paragraph{Syntax}\mbox{}\\ 
\begin{CppCode}
AccessHandle<T> darma_runtime::initial_access<T>(arg1, arg2, ...);
\end{CppCode}

\paragraph{Positional Arguments}\mbox{}\\ 
arg1, arg2, ...: arbitrary tuple of values defining the key of the data.

\paragraph{Output}\mbox{}\\ 
The handle.  

\paragraph{Details}\mbox{}\\ 
This construct creates a handle to data that does not yet 
exist but needs to be created. It takes as input an arbitrary 
tuple of values. Note that this key has to be unique. 
One cannot define two handles with same the same key, even if 
it is created by different ranks. 
When data is initialized, given the SPMD environment provided by DARMA, 
each rank needs to create a unique label for the data. 
One basic way to ensure this is the case is to always use the rank 
as one component of the key. 
\todo{anything missing? details?}

\paragraph{Code Snippet}\mbox{}\\
\begin{figure}[!h]
\begin{CppCodeNumb}
  auto my_handle1 = initial_access<double>("data_key_1", myRank);
  auto my_handle2 = initial_access<int>("data_key_2", myRank, "_online");
}
\end{CppCodeNumb}
\label{fig:fe_api_initialaccess}
\caption{Basic usage of \texttt{initial\_access}.}
\end{figure}




%%%%%%%%%%%%%

