\subsection{\texttt{initial\_access}}

\paragraph{Summary}\mbox{}\\ 
\codetarget{initial_access} creates a handle to data that does not yet exist in the system 
but needs to be created.

\paragraph{Syntax}\mbox{}\\ 
\begin{CppCode}
AccessHandle<T> darma_runtime::initial_access<T>(arg1, arg2, ...);
\end{CppCode}

\paragraph{Positional Arguments}\mbox{}\\ 
arg1, arg2, ...: arbitrary tuple of values defining the key of the data.

\paragraph{Return}\mbox{}\\ 
An object of unspecified type that may be treated as an |AccessHandle<T>|
with the key given by the arguments.

\paragraph{Details}\mbox{}\\ 
This construct creates a handle to data that does not yet 
exist but needs to be created.  The handle is created with {\it Modify}
scheduling permissions and {\it None} immediate permissions.  The function takes
as input an arbitrary tuple of values.
Note that this key has to be unique (see Section \ref{}).
\todo{fix section ref}
One cannot define two handles with same the same key, even if they are created by different ranks.
One basic way to ensure this is the case is to always use the rank 
as one component of the key. 

\paragraph{Code Snippet}\mbox{}\\
\begin{figure}[!h]
\begin{CppCodeNumb}
  auto my_handle1 = initial_access<double>("data_key_1", myRank);
  auto my_handle2 = initial_access<int>("data_key_2", myRank, "_online");
\end{CppCodeNumb}
\label{fig:fe_api_initialaccess}
\caption{Basic usage of \lstinline|initial_access|.}
\end{figure}

\paragraph{Restrictions and Pitfalls}\mbox{}\\ 
\begin{itemize}
  \item Because the actual type returned by |initial_access<T>| is
  unspecified, you should generally use |auto| instead of naming type on 
  the left hand side of the assignment (this is generally a good idea in modern
  C++). In other words,
  \begin{CppCode}
	// $\yes$ good, preferred
	auto my_handle1 = initial_access<double>("good"); 

	// $\no$ still compiles, but not preferred (may miss out
	//  on some future optimizations and compile-time checks)
	AccessHandle<double> my_handle1 = initial_access<double>("bad"); 
  \end{CppCode}
\end{itemize}


%%%%%%%%%%%%%

