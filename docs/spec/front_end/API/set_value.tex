\subsection{\texttt{set\_value}}

\paragraph{Summary} \mbox{}\\
\codetarget{set_value} sets the value of the data pointed to by a
\gls{handle}.

\paragraph{Syntax} \mbox{}\\
\begin{CppCode}
template <typename U>
void AccessHandle<T>::set_value(U&& value)
\end{CppCode}

\paragraph{Positional Arguments}\mbox{}\\
\begin{itemize}
  \item value (type convertible to |T|): The new value for the data.
\end{itemize}

\paragraph{Details}\mbox{}\\
This invokes |T::operator=(U&&)| (|T|'s assignment operator to
a universal reference to |U|) with the argument |value|.  If the type |T| has no
assignment operator for this type, calling \codelink{set_value} will be a compile-time
error.  If you need to invoke an in-place constructor instead, use \codelink{emplace_value}.

\paragraph{Code Snippet}\mbox{}\\
\begin{figure}[!h]
\begin{CppCodeNumb}
auto h = initial_access<double>("key_1");
create_work([=]{
  h.set_value(55.343);
});
\end{CppCodeNumb}
\label{fig:fe_api_set_value}
\caption{Basic usage of \protect\codelink{set_value}.}
\end{figure}

\paragraph{Restrictions and Pitfalls}\mbox{}\\
\begin{itemize}
  \item The specification of the method is likely to change in the future to
  be analogous to the behavior of, e.g., |std::vector<T>::push_back()| (as it
  relates to |std::vector<T>::emplace_back()|).  If this could be a
  problem for |T|, you should probably use \codelink{emplace_value} for now.
\end{itemize}




