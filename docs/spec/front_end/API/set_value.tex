\subsection{\texttt{set\_value}}

\paragraph{Summary} \mbox{}\\
Set the value of the data pointed to by the handle.

\paragraph{Syntax} \mbox{}\\
\begin{CppCode}
void AccessHandle<T>::set_value(value)
\end{CppCode}

\paragraph{Positional Arguments}\mbox{}\\
\begin{itemize}
  \item value (type convertible to |T|): The new value for the data.
\end{itemize}

\paragraph{Details}\mbox{}\\
\todo{Is this still true?  If so, remove ``0.2 only'', otherwise remove item completely.}
{\color{red}[0.2 only]} In the 0.2 spec, this just directly forwards to
|emplace_value()| so that the relevant copy constructor of {\tt T} is
called.  Thus, all comments and examples for |emplace_value()| apply
here.  See \S~\ref{ssec:api_fe_emplace_value} for more.

\paragraph{Code Snippet}\mbox{}\\
\begin{figure}[!h]
\begin{CppCodeNumb}
auto h = initial_access<double>("key_1");
create_work([=]{
  h.set_value(55.343);
});
\end{CppCodeNumb}
\label{fig:fe_api_set_value}
\caption{Basic usage of \lstinline|set_value|.}
\end{figure}

\paragraph{Restrictions and Pitfalls}\mbox{}\\
\begin{itemize}
  \item The specification of the method is likely to change in the future to
  analogous to the behavior, e.g., |std::vector<T>::push_back()| (as it
  relates to |std::vector<T>::emplace_back()|).  If this could be a
  problem for |T|, you should probably use |emplace_value()| for now.
\end{itemize}




