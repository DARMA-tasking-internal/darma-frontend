\subsection{\texttt{get\_value}}

\paragraph{Summary} \mbox{}\
Access the data pointed to by the handle in a read-only manner.

\paragraph{Syntax}\mbox{}\\
\begin{CppCode}
const T& AccessHandle<T>::get_value();
\end{CppCode}

\paragraph{Positional Arguments}\mbox{}\\ 
None.

\paragraph{Return}\mbox{}\\ 
A const reference to the data associated with the handle.  

\paragraph{Details}\mbox{}\\ 
Calling \texttt{get\_value()} on a handle requires {\it Read} or {\it
Modify} immediate permissions (see \S~\ref{sec:handlerules}).


\paragraph{Code Snippet}\mbox{}\\ 
\begin{figure}[!h]
\begin{CppCodeNumb}
AccessHandle<double> my_handle = initial_access<double>("key_1", myRank);
create_work([=]{
  my_handle.set_value(3.14);
});
create_work(reads(my_handle), [=]{
  double myValue = my_handle.get_value();
});
\end{CppCodeNumb}
\label{fig:fe_api_get_value}
\caption{Basic usage of \texttt{get\_value}.}
\end{figure}


\paragraph{Restrictions and Pitfalls}\mbox{}\\ 
\begin{itemize}
  \item Do not hold the reference returned by this method across an asyncronous
  operation on the source handle.  For example, the following results in
  undefined behavior:
  \begin{CppCode}
	// $\yes$ good, preferred
	auto h = initial_access<int>("my_key"); 
	create_work([=]{ h.set_value(5); });
	
	create_work([=]{ 
	  auto const& v = h.get_value();
	  create_work([=]{ h.set_value(10); });
	  cout << v << endl; // $\no$ undefined behavior!!
	});
  \end{CppCode}
\end{itemize}