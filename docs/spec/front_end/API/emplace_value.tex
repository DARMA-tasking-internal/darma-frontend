\subsection{\texttt{emplace\_value}}
\label{ssec:api_fe_emplace_value}

\paragraph{Summary}\mbox{}\\ 
Construct an object of the type pointed to by an \texttt{AccessHandle<T>} object
(that is, \texttt{T}) in place by forwarding the arguments to the constructor
for \texttt{T}.

\paragraph{Syntax}\mbox{}\\ 
\begin{CppCode}
// functional:
some_handle.emplace_value(arg1, arg2, ...);

// Formal:
void AccessHandle<T>::emplace_value(Args&&... args);
\end{CppCode}

\paragraph{Positional Arguments}\mbox{}\\ 
\begin{itemize}
  \item \texttt{args...} (deduced types):  Arguments to forward to the
  constructor of \texttt{T}.
\end{itemize}

\paragraph{Details}\mbox{}\\ 

\texttt{AccessHandle<T>::emplace\_value(...)} mimics the syntax for in-place
construction in standard library containers.  See, for instance,
\texttt{std::vector<T>::emplace\_back(...)}.  If in-place construction is
unnecessary or undesired, \texttt{set\_value()} can be used instead.  
Note that
calling \texttt{emplace\_value()} on a handle requires {\it Modify} immediate
permissions (see \S~\ref{sec:handlerules}).

\paragraph{Code Snippet}\mbox{}\\ 
\begin{figure}[!h]
\begin{CppCodeNumb}
struct LoudMouth {
  LoudMouth(int i, double j) { cout << "Ctor: " << i << ", " << j << endl; }
};
auto h = initial_access<LoudMouth>("key");
create_work([=]{
  h.emplace_value(42, 3.14); // prints "Ctor: 42, 3.14" 
});
\end{CppCodeNumb}
\label{fig:fe_api_initialaccess}
\caption{Basic usage of \texttt{emplace\_value}.}
\end{figure}

\paragraph{Restrictions and Pitfalls}\mbox{}\\ 
\begin{itemize}
  \item {\color{red}[0.2 only]} Because the 0.2 spec lacks a means of specifying
  the constructor of \texttt{T} to be called by default, the first usage of
  every modifiable handle (i.e., the first {\texttt{create\_work}} capturing the
  handle) after being setup with \texttt{initial\_access<T>} should call
  \texttt{emplace\_value()} or \texttt{set\_value()} before performing any
  operations on the pointed-to object.  In other words, the pointed-to object
  contains uninitialized memory until the user invokes 
  \texttt{emplace\_value(...)} or \texttt{set\_value()} to construct the object.
  Failure to do so leads to undefined behavior (just as, for instance,
  \texttt{int x; cout << x;} leads to undefined behavior in C). 
\end{itemize}




