\subsection{\texttt{n\_readers}}

\paragraph{Summary}\mbox{}\\ 

Keyword argument to \texttt{AccessHandle<T>::publish()}.  See
\S~\ref{ssec:api_fe_publish} for details.

In namespace \texttt{darma\_runtime::keyword\_arguments\_for\_publication}.

%\paragraph{Syntax}\mbox{}\\ 
%These two are equivalent.
%\begin{CppCode}
%n_readers = ...; 
%n_readers(...); 
%\end{CppCode}
%See \S~\ref{ssec:api_fe_publish} for details on \texttt{publish}.
%
%
%\paragraph{Positional Argument}\mbox{}\\ 
%The number of times the data will be read.
%See \S~\ref{ssec:api_fe_publish} for details on \texttt{publish}.
%
%
%\paragraph{Details}\mbox{}\\ 
%Specifies the number of times a certain piece of 
%data will be read. This keyword has to be used within 
%the \texttt{publish} method of an access handle.
%More specifically, this refers to the number of \texttt{read\_access} 
%handles that will be defined to read this data.\todo{anything missing? details?}
%
%
%\paragraph{Code Snippet}\mbox{}\\ 
%\begin{figure}[!h]
%\begin{CppCodeNumb}
%auto my_handle = initial_access<double>("data", myRank);
%create_work([=]
%{
%  my_handle.emplace_value(0.5 + (double) myRank);
%  my_handle.publish(n_readers=1);
%});
%\end{CppCodeNumb}
%\label{fig:fe_api_n_readers}
%\caption{Basic usage of \texttt{n\_readers}.}
%\end{figure}




