\subsection{\texttt{n\_readers}}


\paragraph{Summary}\mbox{}\\ 

Keyword argument to |AccessHandle<T>::publish()|.  See
\S~\ref{ssec:api_fe_publish} for details.

In namespace |darma_runtime::keyword_arguments_for_publication|.

%\paragraph{Syntax}\mbox{}\\ 
%These two are equivalent.
%\begin{CppCode}
%n_readers = ...; 
%n_readers(...); 
%\end{CppCode}
%See \S~\ref{ssec:api_fe_publish} for details on |publish|.
%
%
%\paragraph{Positional Argument}\mbox{}\\ 
%The number of times the data will be read.
%See \S~\ref{ssec:api_fe_publish} for details on |publish|.
%
%
%\paragraph{Details}\mbox{}\\ 
%Specifies the number of times a certain piece of 
%data will be read. This keyword has to be used within 
%the |publish| method of an access handle.
%More specifically, this refers to the number of |read_access| 
%handles that will be defined to read this data.\todo{anything missing? details?}
%
%
%\paragraph{Code Snippet}\mbox{}\\ 
%\begin{figure}[!h]
%\begin{CppCodeNumb}
%auto my_handle = initial_access<double>("data", myRank);
%create_work([=]
%{
%  my_handle.emplace_value(0.5 + (double) myRank);
%  my_handle.publish(n_readers=1);
%});
%\end{CppCodeNumb}
%\label{fig:fe_api_n_readers}
%\caption{Basic usage of |n_readers|.}
%\end{figure}




