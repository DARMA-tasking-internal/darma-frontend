%!TEX root = ../sandReportSpec.tex
\subsection{\texttt{read\_access}}
\label{ssec:api_fe_read_access}

\paragraph{Summary}\mbox{}\\
Creates a handle with \text{read-only} access to data that has been or will be
published elsewhere in the system.

\paragraph{Syntax}\mbox{}\\
\begin{CppCode}
/* unspecified, convertible to AccessHandle<T> */
darma_runtime::read_access<T>(KeyParts..., version=KeyExpression);
\end{CppCode}

\paragraph{Positional Arguments}\mbox{}\\
\begin{itemize}
  \item |KeyParts...|: tuple of values identifying the key of the data to
  be read.  This is also called the ``name key'' of the handle.
\end{itemize}

\paragraph{Keyword Arguments}\mbox{}\\
\begin{itemize}
  \item |version=KeyExpression| or |version(KeyParts...) (optional) A
  |KeyExpression| corresponding to the key given as the |version| when
  |publish()| was invoked on a handle with the same name key.  See
  \S~\ref{ssec:api_fe_publish} for more details.
\end{itemize}


\paragraph{Return}\mbox{}\\
An object of unspecified type that may be treated as an |AccessHandle<T>|
with the key given by the arguments.

\paragraph{Details}\mbox{}\\
This function creates a handle to data that already exists and 
needs to be accessed with read-only privileges. 
It takes as input the tuple of values uniquely 
identifying the data that needs to be read.  \todo{finish this, talking about
how corresponding publish has to happen at some point Read/None permissions }

In general, it is used to access data that is migratable, i.e. data 
that potentially is stored off node.
\todo{anything missing? details?}


\paragraph{Code Snippet}\mbox{}\\
\begin{figure}[!h]
\begin{CppCodeNumb}
/* on one rank: */
auto my_handle1 = initial_access<double>("key_1");
create_work([=]{
  my_handle1.emplace_value(5.3);
});
my_handle1.publish(n_readers=1, version="final");

//...

/* on another rank: */
auto readHandle = read_access<double>("key_1", version="final");
create_work([=]{
  std::cout << readHandle.get_value() << std::endl;
});
\end{CppCodeNumb}
\label{fig:fe_api_readaccess}
\caption{Basic usage of \texttt{read\_access}.}
\end{figure}

\todo{pitfalls: talk about how unspecified type may provide compile-time
checking that e.g. set\_value() wasn't called in the future}


