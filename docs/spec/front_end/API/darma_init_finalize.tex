%\clearpage

\subsection{\texttt{darma\_init}}
\label{subsec:darma_init}

\paragraph{Summary}\mbox{}\\
\codetarget{darma_init} initializes the \gls{DARMA} execution environment for a \gls{rank}.

\paragraph{Syntax}\mbox{}\\
\begin{CppCode}
void darma_runtime::darma_init(int& argc, char**& argv);
\end{CppCode}

\paragraph{Positional Arguments}\mbox{}\\
\begin{itemize}
\item argc: command line arguments count.
\item argv: array arguments.
\end{itemize}
The input parameters are the command line argument count 
and array arguments provided to main.  
Note that the backend will process and remove 
any \gls{DARMA} \gls{back end}-specific arguments from these, leaving any
application-specific arguments untouched.


\paragraph{Details}\mbox{}\\
Must be called exactly once per \gls{rank} (``exactly once'' may change in later
spec versions) before any other \gls{DARMA} function is called.
Together with \codelink{darma_finalize} (see \S~\ref{ssec:api_fe_finalize}),
this creates an \gls{execution stream} that defines a \gls{DARMA} \gls{rank}.

\paragraph{Code Snippet}\mbox{}\\ 
See code for \codelink{darma_init} in Figure~\ref{fig:fe_api_init}.

%\clearpage

\subsection{\texttt{darma\_finalize}}
\label{ssec:api_fe_finalize}

\paragraph{Summary}\mbox{}\\
\codetarget{darma_finalize} the \gls{DARMA} execution environment for a \gls{rank}.

\paragraph{Syntax}\mbox{}\\ 
\begin{CppCode}
void darma_runtime::darma_finalize();
\end{CppCode}

\paragraph{Positional Arguments}\mbox{} \\
None. 

\paragraph{Details}\mbox{} 
Called to signify the end of the \gls{execution stream} that defines a
\gls{DARMA}
\gls{rank}.  At
least by the time this function {\em returns}, the backend guarantees that all work
(tasks) created between the corresponding \codelink{darma_init} call and this
invocation, as well as all of the decendents of that work, must be
completed.  No user-level \gls{DARMA} operations are allowed after this call, though
the implicit invocation of the destructors of \codelink{AccessHandle<T>} objects
(at, e.g., the final closing brace of \codelink{darma_main}) is allowed.  Must
be called exactly once for each call of \codelink{darma_init} (which, in turn
must be called exactly once per \gls{rank} in the current version of the spec).

\paragraph{Code Snippet}\mbox{} \\
\begin{figure}[!h]
\begin{CppCodeNumb}
#include <darma.h>
int darma_main(int argc, char** argv)
{
  using namespace darma_runtime;

  darma_init(argc, argv);
  std::cout << "DARMA initialized" << std::endl;

  // code goes here

  std::cout << "Finalizing DARMA..." << std::endl;
  darma_finalize();
  return 0;
}
\end{CppCodeNumb}
\caption{Basic usage of \protect\codelink{darma_init} and \protect\codelink{darma_finalize} 
to initialize and finalize environment.}
\label{fig:fe_api_init}
\end{figure}

\paragraph{Restrictions and Pitfalls}\mbox{} \\
\begin{itemize}
  \item \codelink{darma_finalize} should be called at the outermost task depth on
    a \gls{rank}.  In other words, it should {\it never} be called from within a
    \codelink{create_work} or other asynchronous context.
    %\item \codelink{darma_finalize} should be called in the same immediately
  %enclosing scope (IES) in which \codelink{darma_init} was invoked.  
\end{itemize}




