%\clearpage

\subsection{\texttt{darma\_init}}

\paragraph{Summary}\mbox{}\\
Initializes the DARMA execution environment.

\paragraph{Syntax}\mbox{}\\
\begin{CppCode}
void darma_runtime::darma_init(int& argc, char**& argv);
\end{CppCode}

\paragraph{Positional Arguments}\mbox{}\\
\begin{itemize}
\item argc: command line arguments count.
\item argv: array arguments.
\end{itemize}
The input parameters are the command line argument count 
and array arguments provided to main.  
Note that the back-end will process and remove 
any DARMA back-end-specific arguments from these.

\paragraph{Output}\mbox{}\\
None.


\paragraph{Details}\mbox{}\\
Must be called only once before any other DARMA function is called.
It needs to be combined with \texttt{darma\_finalize()} (see below).
\todo{anything missing? details?}

\paragraph{Code Snippet}\mbox{}\\ 
See code for \texttt{darma\_finalize} in Figure~\ref{fig:fe_api_init}.


%\clearpage

\subsection{\texttt{darma\_finalize}}
\label{ssec:api_fe_finalize}

\paragraph{Summary}\mbox{}\\
Finalizes the DARMA execution environment.

\paragraph{Syntax}\mbox{}\\ 
\begin{CppCode}
void darma_runtime::darma_finalize();
\end{CppCode}

\paragraph{Positional Arguments}\mbox{} \\
None. 

\paragraph{Output}\mbox{} \\
None. 

\paragraph{Details}\mbox{} 
Terminates the execution environment, cleaning up all DARMA processes.
Needs to be called {\it after} \texttt{darma\_init}.\todo{anything missing? details?}


\paragraph{Code Snippet}\mbox{} \\
\begin{figure}[!h]
\begin{CppCodeNumb}
#include <darma.h>
int darma_main(int argc, char** argv)
{
  using namespace darma_runtime;

  std::cout << "Initializing darma" << std::endl;
  darma_init(argc, argv);

  // code goes here

  std::cout << "Finalizing darma" << std::endl;
  darma_finalize();
  return 0;
}
\end{CppCodeNumb}
\caption{Basic usage of \texttt{darma\_init} and \texttt{darma\_finalize} 
to initialize and finalize environment.}
\label{fig:fe_api_init}
\end{figure}






