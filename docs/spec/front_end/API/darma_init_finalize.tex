%\clearpage

\subsection{\texttt{darma\_init}}

\paragraph{Summary}\mbox{}\\
Initializes the DARMA execution environment for a rank.

\paragraph{Syntax}\mbox{}\\
\begin{CppCode}
void darma_runtime::darma_init(int& argc, char**& argv);
\end{CppCode}

\paragraph{Positional Arguments}\mbox{}\\
\begin{itemize}
\item argc: command line arguments count.
\item argv: array arguments.
\end{itemize}
The input parameters are the command line argument count 
and array arguments provided to main.  
Note that the backend will process and remove 
any DARMA backend-specific arguments from these, leaving any
application-specific arguments untouched.


\paragraph{Details}\mbox{}\\
Must be called exactly once per rank (``exactly once'' may change in later
spec versions) before any other DARMA function is called.
Together with |darma\_finalize()| (see \S~\ref{ssec:api_fe_finalize}),
this creates an opaque context that defines a DARMA rank.

\paragraph{Code Snippet}\mbox{}\\ 
See code for |darma\_finalize| in Figure~\ref{fig:fe_api_init}.

%\clearpage

\subsection{\texttt{darma\_finalize}}
\label{ssec:api_fe_finalize}

\paragraph{Summary}\mbox{}\\
Finalizes the DARMA execution environment for a rank.

\paragraph{Syntax}\mbox{}\\ 
\begin{CppCode}
void darma_runtime::darma_finalize();
\end{CppCode}

\paragraph{Positional Arguments}\mbox{} \\
None. 

\paragraph{Details}\mbox{} 
Called to signify the end of the opaque context that defines a DARMA rank.  At
least by the time this function returns, the backend guarentees that all work
(tasks) created between the corresponding |darma\_init| call and this
invocation, as well as and all of the decendents of that work, must be
completed.  No user-level DARMA operations are allowed after this call, though
the implicit invocation of the destructors of |AccessHandle<T>| objects
(at, e.g., the final closing brace of |darma\_main()|) is allowed.  Must
be called exactly once for each call of |darma\_init| (which, in turn
must be called exactly once in the current version of the spec).

\paragraph{Code Snippet}\mbox{} \\
\begin{figure}[!h]
\begin{CppCodeNumb}
#include <darma.h>
int darma_main(int argc, char** argv)
{
  using namespace darma_runtime;

  darma_init(argc, argv);
  std::cout << "DARMA initialized" << std::endl;

  // code goes here

  std::cout << "Finalizing DARMA..." << std::endl;
  darma_finalize();
  return 0;
}
\end{CppCodeNumb}
\caption{Basic usage of \texttt{darma\_init} and \texttt{darma\_finalize} 
to initialize and finalize environment.}
\label{fig:fe_api_init}
\end{figure}

\paragraph{Restrictions and Pitfalls}\mbox{} \\
\begin{itemize}
  \item |darma\_finalize| should be called at the outermost task depth on
  a rank.  In other words, it should {\it never} be called from within a
  |create\_work| or other asynchronous context.
  %\item |darma\_finalize| should be called in the same immediately
  %enclosing scope (IES) in which |darma\_init| was invoked.  
\end{itemize}




