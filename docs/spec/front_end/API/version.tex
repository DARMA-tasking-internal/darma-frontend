\subsection{\texttt{version}}
\label{ssec:api_fe_version}

\paragraph{Summary}\mbox{}\\ 
\codetarget{version} is a \gls{keyword argument} to \codelink{publish} and
\codelink{read_access<T>}.  

In namespace |darma_runtime::keyword_arguments_for_publication|.

%\paragraph{Syntax}\mbox{}\\ 
%These two are equivalent.
%\begin{CppCode}
%version = ...; 
%version(arg1, arg2, ...);
%\end{CppCode}
%
%\paragraph{Positional Arguments}\mbox{}\\ 
%A tuple of values.
%
%\paragraph{Details}\mbox{}\\ 
%Specifies explicitly a version to associated to some data 
%at the time of the publishing operation.  
%It is used when the same data is overwritten multiple times 
%and one needs to keep track of the various versions.
%
%Version can be an arbitrary tuple of values.
%\todo{anything missing? details?}
%
%\paragraph{Code Snippet}\mbox{}\\ 
%\begin{figure}[!h]
%\begin{CppCodeNumb}
%//...
%auto my_handle = initial_access<double>("data", myRank);
%create_work([=]{
%  my_handle.emplace_value(0.5 + (double) myRank);
%  my_handle.publish(n_readers=1,version=(0,"basic"));
%});
%// ...
%// reset value and update version
%create_work([=]{
%  my_handle.set_value(2.5 + (double) myRank);
%  my_handle.publish(n_readers=1,version=1);
%});
%\end{CppCodeNumb}
%\label{fig:fe_api_version}
%\caption{Basic usage of |version|.}
%\end{figure}



