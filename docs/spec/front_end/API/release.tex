\subsection{\texttt{=0} or \texttt{release}}

\paragraph{Summary} \mbox{}\\
Releases the handle.

\paragraph{Syntax} \mbox{}\\
These two are equivalent.
\begin{CppCode}
void some_handle = 0;    
void some_handle.release()
\end{CppCode}


\paragraph{Positional Arguments} \mbox{}\\
None.

\paragraph{Output} \mbox{}\\
None.

\paragraph{Details} \mbox{}\\
This method can be called to release a handle to some data.
\todo{anything missing? details?}


\paragraph{Code Snippet} \mbox{}\\
\begin{figure}[!h]
\begin{CppCodeNumb}
// ...
AccessHandle<double> my_handle1 = initial_access<double>("key_1", myRank);
create_work([=]{
	my_handle1.get_reference() = 242.343;
});
my_handle1.release();
// ...
\end{CppCodeNumb}
\label{fig:fe_api_release}
\caption{Basic usage of \texttt{=0} or \texttt{release}.}
\end{figure}





