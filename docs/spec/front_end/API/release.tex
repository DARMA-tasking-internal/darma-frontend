\subsection{\texttt{=0} or \texttt{release}}

\paragraph{Summary} \mbox{}\\
Releases the reference to the data held by the handle.

\paragraph{Syntax} \mbox{}\\
These two are equivalent.
\begin{CppCode}
// Functional:
some_handle = 0;    
some_handle.release()

// Formal
void AccessHandle<T>::operator=(std::nullptr_t);
void AccessHandle<T>::release();
\end{CppCode}


\paragraph{Positional Arguments} \mbox{}\\
None.

\paragraph{Return} \mbox{}\\
None.

\paragraph{Details} \mbox{}\\
Release the reference to the underlying data held by a given handle.  Note that
this effectively only decrements the reference count; the data itself will not
be deleted unless there are no other existing handles referring to it.  Releasing
at the earliest possible time can help avoid some deadlock situations,
particularly with published data, and potentially increase concurrency.

\paragraph{Code Snippet} \mbox{}\\
\begin{figure}[!h]
\begin{CppCodeNumb}
// ...
AccessHandle<double> my_handle1 = initial_access<double>("key_1", myRank);
create_work([=]{
	my_handle1.get_reference() = 242.343;
});
my_handle1.release();
// ...
\end{CppCodeNumb}
\label{fig:fe_api_release}
\caption{Basic usage of \lstinline|=0| or \lstinline|release|.}
\end{figure}


