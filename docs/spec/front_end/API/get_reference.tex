\subsection{\texttt{get\_reference}}

\paragraph{Summary}\mbox{}\\
Get a non-constant reference to the data pointed to by the handle.

\paragraph{Syntax}\mbox{}\\
\begin{CppCode}
T& AccessHandle<T>::get_reference()
\end{CppCode}

\paragraph{Positional Arguments}\mbox{}\\
None.

\paragraph{Return}\mbox{}\\
A non-constant reference to the data.  

\paragraph{Details}\mbox{}\\
This method requires {\it Modify} immediate permissions.  See
\S~\ref{sec:handlerules} for more information on immediate permissions.

\paragraph{Code Snippet}\mbox{}\\
\begin{figure}[!h]
\begin{CppCodeNumb}
AccessHandle<double> my_handle1 = read_access<double>("key_1", myRank);
create_work([=]{
  my_handle1.get_reference() = 242.343;
});
\end{CppCodeNumb}
\label{fig:fe_api_getreference}
\caption{Basic usage of \lstinline|get_reference|.}
\end{figure}


\paragraph{Restrictions and Pitfalls}\mbox{}\\ 
\begin{itemize}
  \item Do not hold the reference returned by this method across an asyncronous
  operation on the source handle.  For example, the following results in
  undefined behavior:
  \begin{CppCode}
	auto h = initial_access<int>("my_key"); 
	create_work([=]{ h.set_value(5); });
	
	create_work([=]{ 
	  auto& v = h.get_reference();
	  create_work([=]{ h.set_value(10); });
	  cout << v << endl; // $\no$ undefined behavior!!
	});
  \end{CppCode}
  See recommendations in \S~\ref{ssec:api_fe_get_value} for more.
\end{itemize}
