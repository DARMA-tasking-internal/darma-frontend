\subsection{\texttt{operator->}}

\paragraph{Summary}\mbox{}\\
\codetarget{operator->} is a dereference operator to directly access the object 
pointed to by the \gls{handle}.

\paragraph{Syntax}\mbox{}\\
\begin{CppCode}
T* AccessHandle<T>::operator->();
\end{CppCode}

\paragraph{Input Parameters}\mbox{}\\
None.

\paragraph{Return}\mbox{}\\
Returns a reference to the data pointed to by the \gls{handle}.

\paragraph{Details}\mbox{}\\
Just like \codelink{set_value} and \codelink{get_reference}, this operator
requires {\it Modify} \gls{immediate permissions} to invoke safely.  Unlike 
\codelink{set_value} and \codelink{get_reference}, however, the
\codelink{operator->} can
also be invoked on \glspl{handle} that only have {\it Read} \gls{immediate
permissions}.  In
that case, it is up to the user to ensure that only |const| methods are
called on the resulting object.  In other words,
|AccessHandle<T>::operator->()| lets you ``shoot yourself in the foot.''
If more safety is desired, use the more  verbose forms with
\codelink{set_value} and \codelink{get_reference}.


\paragraph{Code Snippet}\mbox{}\\
\begin{figure}[!h]
\begin{CppCodeNumb}
//...
typedef std::vector<double> vec;
AccessHandle<vec> my_handle2 = initial_access<vec>("key_2", myRank);

create_work([=]{
  my_handle2.emplace_value(0.0);
  my_handle2->resize(4);
  double * vecPtr = my_handle2->data();    
});
\end{CppCodeNumb}
\label{fig:fe_api_arrow}
\caption{Basic usage of \protect\codelink{:operator->}.}
\end{figure}

