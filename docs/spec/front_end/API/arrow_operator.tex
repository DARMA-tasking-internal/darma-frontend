\subsection{\texttt{operator->}}

\paragraph{Summary}\mbox{}\\
A dereference operator to directly access the object 
pointed to by the handle.

\paragraph{Syntax}\mbox{}\\
\begin{CppCode}
T* AccessHandle<T>::operator->();
\end{CppCode}

\paragraph{Input Parameters}\mbox{}\\
None.

\paragraph{Return}\mbox{}\\
Returns a reference to the data pointed to by the handle.

\paragraph{Details}\mbox{}\\
Just like \texttt{set\_value} and \texttt{get\_reference}, this operator
requires {\it Modify} immediate permissions to invoke safely.  Unlike 
\texttt{set\_value} and \texttt{get\_reference}, however, the arrow operator can
also be invoked on handles that only have {\it Read} immediate permissions.  In
that case, it is up to the user to ensure that only \texttt{const} methods are
called on the resulting object.  In other words,
\texttt{AccessHandle<T>::operator->()} lets you ``shoot yourself in the foot''.
If more safety is desired, use the more  verbose forms with
\texttt{set\_value()} and \texttt{get\_reference()}.


\paragraph{Code Snippet}\mbox{}\\
\begin{figure}[!h]
\begin{CppCodeNumb}
//...
typedef std::vector<double> vec;
AccessHandle<vec> my_handle2 = initial_access<vec>("key_2", myRank);

create_work([=]{
  my_handle2.emplace_value(0.0);
  my_handle2->resize(4);
  double * vecPtr = my_handle2->data();    
});
\end{CppCodeNumb}
\label{fig:fe_api_arrow}
\caption{Basic usage of \texttt{AccessHandle<T>::operator->()}.}
\end{figure}

