\subsection{\texttt{get\_key}}

\paragraph{Summary}\mbox{}\\ 
Get the key identifying the data pointed to by the handle.

\paragraph{Syntax}\mbox{}\\ 
\begin{CppCode}
darma_runtime::types::key_t const& AccessHandle<T>::get_key();
\end{CppCode}

\paragraph{Positional Arguments}\mbox{}\\ 
None.

\paragraph{Return}\mbox{}\\ 
The key identifying the data.

\paragraph{Details}\mbox{}\\ 
This method can be called at any time after the handle is created.  It does not
require any permissions, scheduling or immediate.

\paragraph{Code Snippet}\mbox{}\\ 
\begin{figure}[!h]
\begin{CppCodeNumb}
//...
auto myRank = darma_spmd_rank();
AccessHandle<double> my_handle1 = read_access<double>("key_1", myRank);
auto myK = my_handle1.get_key();

create_work([=]{
	my_handle1.get_reference() = 242.343;
	auto myK = my_handle1.get_key();
	assert(myRank == myK.get_key().component<1>().as<int>());
});
\end{CppCodeNumb}
\label{fig:fe_api_get_key}
\caption{Basic usage of \lstinline|get_key|.}
\end{figure}




