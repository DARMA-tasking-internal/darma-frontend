\subsection{\texttt{get\_key}}

\paragraph{Summary}\mbox{}\\ 
Get a copy of the key identifying the data pointed to by the handle.

\paragraph{Syntax}\mbox{}\\ 
\todo{what is the key  type here?}
\begin{CppCode}
key_type & some_handle.get_key()
\end{CppCode}

\paragraph{Positional Arguments}\mbox{}\\ 
None.

\paragraph{Output}\mbox{}\\ 
A copy of the key identifying the data.

\paragraph{Details}\mbox{}\\ 
This method can be called at any time after the handle is created.
\todo{anything missing? details?}


\paragraph{Code Snippet}\mbox{}\\ 
\begin{figure}[!h]
\begin{CppCodeNumb}
//...
AccessHandle<double> my_handle1 = read_access<double>("key_1", myRank);
auto myK = my_handle1.get_key();

create_work([=]{
	my_handle1.get_reference() = 242.343;
	auto myK = my_handle1.get_key();
});
\end{CppCodeNumb}
\label{fig:fe_api_get_key}
\caption{Basic usage of \texttt{get\_key}.}
\end{figure}




