%\clearpage
\subsection{\texttt{create\_work}}
\label{ssec:api_fe_cw}

\paragraph{Summary}\mbox{}\\
\codetarget{create_work} instantiates \gls{deferred work} to be executed by the \gls{runtime
system}.

\todo{Jeremy/David: Need to update all pargraphs to inlclude relevant syntax,
argument changes, pitfalls, examples for functor interface}

\paragraph{Syntax}\mbox{}\\
\begin{CppCode}
// Functionally:
create_work([=]{
  // Code expressing deferred work goes here
});
// or:
create_work(
  ConstraintExpressions..., 
  [=]{
    // Code expressing deferred work goes here
  }
);

// Formally:
/* unspecified */ create_work(Arguments..., LambdaExpression);
\end{CppCode}


\paragraph{Positional Arguments}\mbox{}\\
\begin{compactitem}
\item \inlinecode{LambdaExpression} A \CC{}11 \gls{lambda} expression with a copy
    default-\gls{capture} (i.e., |[=]|) and taking no arguments.  More details
  below.
\item \inlinecode{ConstraintExpressions...} (optional) If given, these
  arguments can be used to express modifications in the default \gls{capture} behavior
  of \codelink{AccessHandle<T>} objects captured by the \inlinecode{LambdaExpression}
  given as the final argument.  In the \specVersion-specification, the only valid permission
  modification expression is the return value of the \codelink{reads()} modifier
  (see \S~\ref{ssec:api_fe_reads}), which indicates that only read operations
  are performed on a given \gls{handle} or \glspl{handle} within the
  \inlinecode{LambdaExpression} that follows.
\end{compactitem}


\paragraph{Return}\mbox{}\\
Currently \inlinecode{void} in the \specVersion-specification, but may be an object of unspecified
type in future implementations.

\paragraph{Details}\mbox{}\\
This function expresses work to be executed by the \gls{runtime system}.  Any
\codelink{AccessHandle} variables used in the \inlinecode{LambdaExpression} will be
captured and made available inside the capturing context as if they were used in
sequence with previous capture operations on the same \gls{handle}.  Depending on the
scheduling permissions available to the \inlinecode{AccessHandle<T>} at the time of
\codelink{create_work} invocation and on the \inlinecode{ConstraintExpressions...}
given as arguments, this function call expresses either a {\it read-only
capture} or a {\it modify capture} operation on a given \gls{handle} (see
\S~\ref{sec:handlerules}).  If a \gls{handle} \inlinecode{h} has {\it Read} scheduling
permissions when it is captured or if the explicit constraint expression
\inlinecode{reads(h)} is given in the \inlinecode{ConstraintExpressions...} arguments,
\codelink{create_work} functions as a {\it read-only capture} operation on that
\gls{handle}.
Otherwise, it functions as a {\it modify capture}.  

Additional general discussion on use of \codelink{create_work} can be found in
\S~\ref{sec:deferred}.


\paragraph{Code Snippet}\mbox{}\\
\begin{figure}[!h]
\begin{CppCodeNumb}
create_work([=]{
  std::cout << " Hello world! " << std::endl;
});
\end{CppCodeNumb}
\label{fig:fe_api_cw}
  \caption{Basic usage of \protect\codelink{create_work}.}
\end{figure}



\paragraph{Restrictions and Pitfalls}\mbox{} \\
Most of the general restrictions and pitfalls related to \codelink{create_work}
are discussed in \S~\ref{sec:deferred}.  Some more technical restrictions are
given here.
\begin{compactitem}
  \item Because of the way in which \codelink{create_work} is implemented, placement of
    multiple \codelink{create_work} operations on the same line of code will not compile. 
  For instance:
  \begin{CppCode}
// $\no$ does not compile, gives cryptic error message
create_work([=]{}); create_work([=]{}); 
  \end{CppCode}
  This is particularly easy to accidentally do when defining preprocessor
  macros:
  \begin{CppCode}
// $\no$ does not compile, gives even more cryptic error message
#define foo(...) __VA_ARGS__
foo(
  create_work([=]{}); 
  create_work([=]{}); 
)
  \end{CppCode}
  Note that this is not a problem when using nested \codelink{create_work} calls:
  \begin{CppCode}
// $\yes$ not a problem
create_work([=]{ create_work([=]{}); }); // works fine
  \end{CppCode}
  Other than the obvious solution of putting the \codelink{create_work} invocations on
  multiple lines, this issue can be worked around by putting any of the later
  \codelink{create_work} calls within their own scopes:
  \begin{CppCode}
// $\yes$ works fine
create_work([=]{}); { create_work([=]{}); }
// $\yes$ also fine
foo(
  create_work([=]{}); 
  { create_work([=]{}); }
  { create_work([=]{}); }
)
  \end{CppCode}
\end{compactitem}

