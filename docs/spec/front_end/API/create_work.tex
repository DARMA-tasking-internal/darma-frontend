%\clearpage
\subsection{\texttt{create\_work}}
\label{ssec:api_fe_cw}

\paragraph{Summary}\mbox{}\\
Communicates work to the runtime system.

\paragraph{Syntax}\mbox{}\\
\begin{CppCode}
create_work([=]
{ 
	// code goes here
});
\end{CppCode}

\paragraph{Input Parameters}\mbox{}\\
Input arguments are optional. If no input is provided, the 
\texttt{create\_work} has privileges on the handles used inside 
as determined by the semantics of the code. 
On the contrary, using the keyword \texttt{reads}, it can take as 
argument a list of handles that are captured with read-only permissions. 
See \S~\ref{ssec:api_fe_reads} for more information on how to do this. 
This constraints the permissions to be read-only 
on these handles passed to the function.
\todo{anything missing? details?}


\paragraph{Output}\mbox{}\\
None.

\paragraph{Details}\mbox{}\\
This function communicates work to the runtime system. 
Specifically, this is needed to construct deferred work. 
Note: the block of code enclosed will be executed only when all the 
proper permissions on the data it uses will be available.
\todo{anything missing? details?}


\paragraph{Code Snippet}\mbox{}\\
\begin{figure}[!h]
\begin{CppCodeNumb}
#include <darma.h>
int darma_main(int argc, char** argv)
{
  using namespace darma_runtime;
  darma_init(argc, argv);
  const int myRank = darma_spmd_rank();
  const int size = darma_spmd_size();

  create_work([=]
  {
  	std::cout << " Hello world! " << std::endl;
  });

  darma_finalize();
  return 0;
}
\end{CppCodeNumb}
\label{fig:fe_api_cw}
\caption{Basic usage of \texttt{create\_work}.}
\end{figure}


