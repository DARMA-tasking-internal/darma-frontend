%\clearpage
\subsection{\texttt{create\_work}}
\label{ssec:api_fe_cw}

\paragraph{Summary}\mbox{}\\
Communicates work to the runtime system.

\paragraph{Syntax}\mbox{}\\
\begin{CppCode}
// Functionally:
create_work([=]{
  // deferred work goes here
});
// or:
create_work(
  ConstraintExpressions..., 
  [=]{
	// deferred work goes here
  }
);

// Formally:
/* unspecified */ create_work(Arguments..., LambdaExpression);
\end{CppCode}

\paragraph{Positional Arguments}\mbox{}\\
\begin{itemize}
  \item \texttt{LambdaExpression} \todo{write this}
  \item \texttt{ConstraintExpressions...} (optional) If given, these
  arguments can be used to express modifications in the default capture behavior
  of \texttt{AccessHandle<T>} objects captured by the \texttt{LambdaExpression}
  given as the final argument.  In the 0.2 spec, the only valid permission
  modification expression is the return value of the \texttt{reads()} modifier
  (see \S~\ref{ssec:api_fe_reads}), which indicates that only read operations
  are performed on a given handle or handles within the
  \texttt{LambdaExpression} that follows.
\end{itemize}

%Input arguments are optional. If no input is provided, the 
%\texttt{create\_work} has privileges on the handles used inside 
%as determined by the semantics of the code. 
%On the contrary, using the keyword \texttt{reads}, it can take as 
%argument a list of handles that are captured with read-only permissions. 
%See \S~\ref{ssec:api_fe_reads} for more information on how to do this. 
%This constraints the permissions to be read-only 
%on these handles passed to the function.
%\todo{anything missing? details?}


\paragraph{Return}\mbox{}\\
Currently \texttt{void} in the 0.2 spec, but may be an object of unspecified
type in future implementations.

\paragraph{Details}\mbox{}\\
This function communicates work to the runtime system. 
Specifically, this is needed to construct deferred work. 
Note: the block of code enclosed will be executed only when all the 
proper permissions on the data it uses will be available.
\todo{anything missing? details?}


\paragraph{Code Snippet}\mbox{}\\
\begin{figure}[!h]
\begin{CppCodeNumb}
#include <darma.h>
int darma_main(int argc, char** argv)
{
  using namespace darma_runtime;
  darma_init(argc, argv);
  const int myRank = darma_spmd_rank();
  const int size = darma_spmd_size();

  create_work([=]
  {
  	std::cout << " Hello world! " << std::endl;
  });

  darma_finalize();
  return 0;
}
\end{CppCodeNumb}
\label{fig:fe_api_cw}
\caption{Basic usage of \texttt{create\_work}.}
\end{figure}


