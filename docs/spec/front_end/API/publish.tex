\subsection{\texttt{publish}}
\label{ssec:api_fe_publish}

\paragraph{Summary} \mbox{}\\
Publish the data pointed to by a given handle so that it can be retrieved on 
other DARMA ranks.

\paragraph{Syntax} \mbox{}\\
\begin{CppCode}
void
AccessHandle<T>::publish(n_readers=..., version=...)
\end{CppCode}

\paragraph{Positional Arguments} \mbox{}\\
None.

\paragraph{Keyword Arguments} \mbox{}\\
\begin{compactitem}
\item |n_readers=size_t| (optional): informs the runtime how many times
\codelink{read_access} will be called in order to access this data.
If omitted, it defaults to 1.
\item |version=KeyExpression| (or |version(KeyExpression...)|,
see \S~\ref{sec:keyword} for multiple-right-hand-side keyword argument usage)
(optional):
informs the runtime what version to associate with the data being published. 
The value can be an arbitrary |KeyExpression|.
If omitted, the version defaults to an empty key (i.e., a key tuple with zero 
components).  Omitting this keyword implicitly indicates to the runtime system
that the handle (or any handle with the same name key) will not be published
again in the remaining lifetime of the program.
\end{compactitem}

\paragraph{Details} \mbox{}\\

Publish the data associated with a given handle |h| such that it can be
retrieved |n_readers| times anywhere via a |read_access<T>| invocation
that gives the same name key as |h| and the same |version| key as
the one given to the keyword argument to |publish()|.
A |publish()| is a {\it read-only capture} operation (see
\S~\ref{sec:handlerules}).

%\paragraph{Code Snippet} \mbox{}\\
%\begin{figure}[!h]
\begin{CppCodeNumb}
auto me = darma_spmd_rank();
assert(darma_spmd_size() >= 2);
if(me == 0) {
  auto my_handle = initial_access<double>("key_1");
  create_work([=]{
    my_handle.emplace_value(5.3);
  });
  my_handle.publish(n_readers=1, version="only");
}
else if(me == 1) {
  auto my_handle = read_access<double>("key_1", version="only");
  create_work([=]{
    cout << my_handle.get_value() << endl; // prints "5.3"
  });
}
\end{CppCodeNumb}
%\label{fig:fe_api_publish}
%\caption{Basic usage of \lstinline|publish()|.}
%\end{figure}

\paragraph{Restrictions and Pitfalls}\mbox{}\\
\begin{compactitem}
  \item |publish()| puts more burden on the programmer to avoid race conditions and deadlock, 
  which are automatically avoided when relying entirely on sequential semantics.
  This is similar to deadlock situations with sends and receives in MPI in which communicating processes block on a receive before sending to each other.  
  For instance, the following snippet deadlocks:
  %\begin{center}
  \begin{CppCode}
// This code deadlocks!
auto me = darma_spmd_rank();
assert(darma_spmd_size() >= 2);
if(me == 0) {
  auto h1 = initial_access<int>("key", 0);
  auto h2 = read_access<int>("key", 1);
  create_work([=]{ 
    h1.set_value(42);
    h1.publish();
    cout << h2.get_value() << endl;
  }); 
}
else if (me == 1) {
  auto h3 = initial_access<int>("key", 1);
  auto h4 = read_access<int>("key", 0);
  create_work([=]{ 
    h3.set_value(73);
    h3.publish();
    cout << h4.get_value() << endl; 
  }); 
}
// Deadlock! (eventually, at the latest when darma_finalize() is
// called): neither of the above create_work()s can ever run
  \end{CppCode}
  This snippet deadlocks because a dependency loop has been created between two
  |publish()|/|read_access()| pairs. While the deadlock is
  relatively obvious here, it can be much more difficult to decipher in a more
  complex code, especially if, for instance, |h1| and |h2| are
  arguments to a function, or if the parts of the keys used to construct the
  handles are variables with values dependent on some previous computation.
  %%%%%%%%%%%%%%%%%%%%%%%%%%%%%%%%%%%%%%%%%%%%%%%%%%%%%%%%%%%%%%%%%%%%%%%%%%%%%
  \item It is particularly easy to create deadlock scenarios by publishing a
  handle and fetching it within the same rank.  For this reason, we recommend
  extreme caution when this scenario could arise, and in general we suggest that
  the user should avoid doing so if at all possible.
  %%%%%%%%%%%%%%%%%%%%%%%%%%%%%%%%%%%%%%%%%%%%%%%%%%%%%%%%%%%%%%%%%%%%%%%%%%%%%
  \item Since |publish()| is a {\it read-only capture} operation, it must
  have scheduling privileges of {\it Read} or {\it Modify}; calling publish on a
  handle with other scheduling permissions is a runtime error.  Also, as with
  all {\it read-only capture} operations, calling |publish()| on a
  handle with {\it Modify} immediate permissions results in a handle with {\it
  Read} immediate permissions in the continuing context.  See
  \S~\ref{sec:handlerules} for more details.  For example, the following code
  results in a runtime error at the marked line:
%\begin{minipage}[t]{0.95\linewidth}%
%  \centering
  \begin{CppCode}
  auto h = initial_access<int>("key");
  create_work([=]{
    h.set_value(5);
    h.publish();
    h.set_value(10); // $\no$ h does not have Modify immediate permissions
  });
  \end{CppCode}
%\end{minipage}
  \item It is an error to call |publish()| on a handle with the same key and version more than once.
  %%%%%%%%%%%%%%%%%%%%%%%%%%%%%%%%%%%%%%%%%%%%%%%%%%%%%%%%%%%%%%%%%%%%%%%%%%%%%
  \item If |publish()| is called on a given handle without the
  |version| keyword argument, it is an error to call |publish()|
  again on that handle or any other handle with the same name key for the
  remaining lifetime of the program.  Note that because of the default behavior
  of the |version| keyword argument, giving an explicit version that is
  the empty key (e.g., |h.publish(version())| or
  |h.publish(version=make_key())|) will lead to this same behavior.
  %%%%%%%%%%%%%%%%%%%%%%%%%%%%%%%%%%%%%%%%%%%%%%%%%%%%%%%%%%%%%%%%%%%%%%%%%%%%%
\end{compactitem}



