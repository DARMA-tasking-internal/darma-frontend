\section{Deferred Work Creation}
\label{sec:deferred}
In DARMA, like other AMT runtime systems, the user creates blocks of work that
are executed when the proper permissions on the data they use are available. 
Deferred work is communicated to the runtime system via the
\inlinecode{create_work()} function, which utilizes the C++ lambda mechanism to
yield the following syntax:
% DSH merged [=]{ into one line for a couple of reasons; chief reason is we may
% want to macro this at some point so that it can conditionally be compiled into
% [=] __cuda_inline__ {  or something like that for GPU use.
\begin{CppCode}
  create_work([=]{
    // work to do
  });
\end{CppCode}

While this syntax leverages \CC11 lambdas, the user does not need to understand
\CC11 standard features to use \inlinecode{create_work()} (this
complexity is managed by DARMA's translation layer, as summarized in
Chapter~\ref{chap:translation_layer}). All the work specified within a
\inlinecode{create_work} is queued for deferred execution, and will be run when
all of its \glspl{dependency} and \glspl{anti-dependency} are met.  
These \glspl{dependency} are determined from the use of 
\inlinecode{AccessHandle<T>} objects in the deferred context. 
\inlinecode{AccessHandle<T>} objects are described in more detail below, in
Section \ref{sec:handles}.  In a basic sense, handles help maintain
\gls{sequential semantics} in the programming model.  For instance, the
following code should print ``first: 42, second: 84'':
\begin{CppCode}
AccessHandle<int> my_handle = initial_access("some_data_key");
create_work([=]{
  my_handle.set_value(42);
});
create_work([=]{
  cout << "first: " << my_handle.get_value();
});
create_work([=]{
  my_handle.set_value(my_handle.get_value()*2);
});
create_work([=]{
  cout << ", second: " << my_handle.get_value();
});
\end{CppCode}
This intuitive way of coding is called \gls{sequential semantics}, 
and its use is pivotal to the DARMA programming model.

% DSH took this out for now
% which manage the complexity necessary (within DARMA's
%\gls{translation layer}) to 1) preserve \gls{sequential semantics} within a \gls{rank}, while performing work asynchronously, and 2) \inlinecode{publish} data visibly outside a \gls{rank}.
%\Glspl{dependency} and \glspl{anti-dependency} for a 
%\inlinecode{create_work} are inferred by the runtime via
%\gls{introspection} (again, within the \gls{translation layer}). 
%The following simple example illustrates how \inlinecode{handles} and
%\inlinecode{create_work} can be used to gether in a simple ``hello world''
%example:
%\begin{CppCode}
%#include <darma.h>
%using namespace darma_runtime;
%using namespace darma_runtime::keyword_arguments_for_publication;
%
%int main(int argc, char** argv) {
%
%  darma_init(argc, argv);
%
%  size_t me = darma_spmd_rank();
%  size_t n_ranks = darma_spmd_size();
%
%  auto greetingMessage = initial_access<std::string>("myName", me);
%
%  create_work([=]
%  {
%    greetingMessage.set_value("hello world!");
%  });
%
%  create_work([=]
%  {
%    std::cout << "DARMA rank " << me << " says " 
%              << greetingMessage.get_value() << std::endl;
%  });
%
%  darma_finalize();
%
%}
%\end{CppCode}
%In this example, 1) the DARMA environment is initialized, 2) each rank creates
%% a task to store a greeting message into a string, and 3) each rank then creates a
%task to print the message and its rank to standard output.
%Note that, in this example, \inlinecode{publish} is not required as the
%dependency \inlinecode{greeting_message} is createed for each rank individually
%(i.e., none of the dependencies must be visible outside of its rank).
%In order to guarantee \gls{sequential semantics} (i.e., to make sure the value
%of \inlinecode{greetingMessage} is set before it is printed),
%\inlinecode{greetingMessage} must be a \inlinecode{handle}.


