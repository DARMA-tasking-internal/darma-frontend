%\lstMakeShortInline[style=CppCodeInlineStyle]{|}
\section{Deferred Work Creation}
\label{sec:deferred}
In DARMA, like other AMT runtime systems, the user creates blocks of work and defines the preconditions for the task to begin executing.
Rather than require app developers to explicitly define vertices and edges in a task-DAG or use explicit fork-join constructs,
task preconditions are implicit in either the sequential order of tasks or the data-flow inherent in the key-value store coordination for publish/fetch pairs (more below).
Deferred work is instantiated (but not necessarily executed) via the \inlinecode{create_work()} function. 
For inline tasks (as compared to functor-based tasks, more below), this utilizes the C++ lambda mechanism to yield the following syntax:

\begin{CppCode}
//parent task
create_work([=]{
  // deferred work in captured context
});
//continuing task in parent task
\end{CppCode}

In DARMA, \gls{deferred work} is one generic term we use for work
performed by code inside the capturing \gls{lambda}.
This does not necessarily imply that the \gls{continuing context} (after the \codelink{create_work}) will be
executed before the \gls{captured work} (note that ``\gls{captured work}'' and
``\gls{captured context}'' are two other generic terms we use interchangeably
with \gls{deferred work}).  
Deferred work does not \emph{need} to be deferred.
If a task's preconditions are all satisfied (data is available with correct permissions),
the \gls{runtime system} may execute it immediately.  
In fact, the runtime need not execute either the parent continuing context or the child deferred context,
\todo[inline]{Jeremy: How did you want to finish this sentence?}
% DSH merged [=]{ into one line for a couple of reasons; chief reason is we may
% want to macro this at some point so that it can conditionally be compiled into
% [=] __cuda_inline__ {  or something like that for GPU use.

While this syntax leverages \CC11 lambdas, the user does not need to understand \CC11 standard features to use \inlinecode{create_work()} (this
complexity is managed by DARMA's translation layer, as summarized in Chapter~\ref{chap:translation_layer}). 
All the work specified within a \inlinecode{create_work} is queued for deferred execution. 
The task does not need to execute immediately and may be executed by the backend runtime any time after all its preconditions are satisfied. 
Preconditions are either dependencies (waiting for data to be produced) or anti-dependencies (waiting for data to be released so it can be overwritten).   
Preconditions for a task are never given explicitly, but are instead derived implicitly based on sequential usage of \inlinecode{AccessHandle} objects, discussed in detail below.
For example, to follow sequential semantic the following code should print ``first: 42, second: 84'':
\begin{CppCode}
auto my_handle = initial_access("some_data_key");
create_work([=]{
  my_handle.set_value(42);
});
create_work([=]{
  cout << "first: " << my_handle.get_value();
});
create_work([=]{
  my_handle.set_value(my_handle.get_value()*2);
});
create_work([=]{
  cout << ", second: " << my_handle.get_value();
});
\end{CppCode}
The code produces results equivalent to a C++ code in which \inlinecode{create_work} is removed and \inlinecode{AccessHandle} is just replaced with the underlying type.
These sequential semantics are pivotal to the DARMA programming model.

Relying solely on sequential semantics is the ``ideal'' way of coding in terms of limiting programmer burden, avoiding deadlock, and enabling runtime optimizations.
The details of this are beyond the scope of the current work.
For certain cases, particularly those involving massive SPDM parallelism across a distributed memory machine,
it may be more scalable and natural to code in a CSP-like framework involving parallel execution streams.
Rather than coding as if only a single execution stream, the programmer codes must be aware of multiple parallel streams.
Rather than exchanging data through send/recv pairs, however, DARMA uses coordination to communicate data between parallel execution streams.
Two execution streams never explicitly exchange data. Instead they put and get data from a key-value store.
Coordinating (rather than communicating) abstracts physical data locations to better support task migration.
Additionally, it removes message-ordering requirements to better support asynchronous data transfers.
While the key-value store appears to be a centralized, global data store that copies data in/out,
the key-value store can be implemented as a distributed hash table (DHT) that supports zero-copy transfers.
Thus both sequential semantics and coordination semantics follow the same principle in DARMA.
An intuitive programming model that simplifies reasoning about algorithms is transformed to a parallel, scalable execution by the translation layer and backend runtime.

In the example here, variables are not passed down from a parent task to child tasks.
Instead, one execution stream produces a value and publishes it to a key-value store.
Another execution stream reads the value by fetching it from a key-value store.
The processes coordinate with publish/fetch pairs similar to send/recv pairs in the CSP model of MPI.

\begin{minipage}{0.45\textwidth}
Execution Stream 0:
\begin{CppCode}
Some publish/fetch code
\end{CppCode}
\end{minipage}
\begin{minipage}{0.45\textwidth}
Execution Stream 1:
\begin{CppCode}
Some publish/fetch code
\end{CppCode}
\end{minipage}
\todo{finish this example}

Instead of defining task preconditions implicitly via sequential order,
task preconditions are specified more explicitly by requiring that a particular block of data be fetched from the key-value store.
More on SPMD programs and parallel execution streams is given in Section \ref{}.
