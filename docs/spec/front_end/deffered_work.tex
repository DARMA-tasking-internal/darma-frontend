%\lstMakeShortInline[style=CppCodeInlineStyle]{|}
\section{Deferred Work Creation}
\label{sec:deferred}
In \gls{DARMA}, like other \gls{AMT} \glspl{runtime system}, 
the application developer creates blocks of work (a \gls{task})and defines the
\glspl{precondition} for the \gls{task} to begin executing.
Rather than require application developers to explicitly define vertices and
edges in a \gls{task-DAG} or use explicit \gls{fork-join} constructs,
in \gls{DARMA}, \gls{task} \glspl{precondition} are implicit in either the sequential 
order of \glspl{task} or the data-flow inherent in the \gls{key-value store}
coordination (more below).
\Gls{deferred work} is instantiated (but not necessarily executed) via the \codelink{create_work} function. 
For inline \glspl{task} (as compared to functor-based \glspl{task}, more
below), this utilizes the \CC{} \gls{lambda} \gls{capture} mechanism to yield the following syntax:

\begin{CppCode}
//outer task
create_work([=]{
  // <-- deferred work in captured context
});
//continuing context in outer task
\end{CppCode}

In \gls{DARMA}, \gls{deferred work} and \gls{task} are generic terms we use for work
performed by code inside the capturing \gls{lambda}.
This does not necessarily imply that the \gls{continuing context} (after the \codelink{create_work}) will be
executed before the \gls{captured work} (note that ``\gls{captured work}'' and
``\gls{captured context}'' are two other generic terms we use interchangeably
with \gls{deferred work}).  
We highlight here that \gls{deferred work} does not \emph{need} to be deferred.
A more precise term may therefore be \emph{deferrable} work, but we use deferred to match previous literature.
If a \gls{task}'s \glspl{precondition} are all satisfied (data is available with correct permissions),
the \gls{runtime system} may execute it immediately.  
In fact, the runtime may execute the outer continuing context, the inner deferred context, or another context entirely if there are pending tasks.

% DSH merged [=]{ into one line for a couple of reasons; chief reason is we may
% want to macro this at some point so that it can conditionally be compiled into
% [=] __cuda_inline__ {  or something like that for GPU use.

While this syntax leverages \CC11 \glspl{lambda}, the user does not need to
understand \CC11 standard features to use \codelink{create_work} (this
complexity is managed by \gls{DARMA}'s \gls{translation layer}, as summarized in Chapter~\ref{chap:translation_layer}). 
All the work specified within a \codelink{create_work} is queued for
\gls{deferred execution}.  The \gls{task} does not need to execute immediately 
and may be executed by the \gls{back end} \gls{runtime system} any time after
all of its \glspl{precondition} are satisfied. 
\Glspl{precondition} are either \gls{dependency} (waiting for data to be produced) or
\gls{anti-dependency} (waiting for data to be released so it can be overwritten).   
\Glspl{precondition} for a \gls{task} are never given explicitly, but are instead derived 
implicitly based on sequential usage of \codelink{AccessHandle} objects, discussed in detail below.
For example, to satisfy \gls{sequential semantics} the following code should print ``first: 42, second: 84'':
\begin{CppCode}
auto my_handle = initial_access("some_data_key");
create_work([=]{
  my_handle.set_value(42);
});
create_work([=]{
  cout << "first: " << my_handle.get_value();
});
create_work([=]{
  my_handle.set_value(my_handle.get_value()*2);
});
create_work([=]{
  cout << ", second: " << my_handle.get_value();
});
\end{CppCode}
The code produces results equivalent to a \CC{} code in which \codelink{create_work} 
is removed and \codelink{AccessHandle} is just replaced with the underlying type.
These \gls{sequential semantics} are pivotal to the \gls{DARMA}
\gls{programming model}.

\Gls{sequential semantics} provide a simple and intuitive way of coding
asynchronous work, by limiting programmer burden, avoiding deadlock,
and enabling runtime optimizations.
%The details of this are beyond the scope of the current work.
However, in cases with massive \gls{spmd} parallelism across a distributed memory machine,
it may be more scalable and natural to code in a \gls{CSP}-like framework
involving parallel \glspl{execution stream}.
Rather than coding as if only operating within a single \gls{execution stream}, 
the programmer must be aware of multiple parallel \glspl{execution stream}.

\gls{DARMA} uses coordination data between parallel \glspl{execution stream},
rather than exchanging data through \inlinecode{send/recv} pairs. 
Two \glspl{execution stream} never explicitly exchange data. Instead they
\codelink{publish} and \gls{fetch} from a \gls{key-value store}.
Coordinating (rather than communicating) abstracts physical data locations to
better support \gls{task} migration.
Additionally, it removes message-ordering requirements to better support
\gls{asynchronous} data transfers.
While the \gls{key-value store} appears to be a centralized, global data store that copies data in/out,
the \gls{key-value store} can be implemented as a \gls{DHT} that supports
\gls{zero-copy} transfers.
Thus both \gls{sequential semantics} and \gls{coordination semantics} follow
the same principle in \gls{DARMA}: intuitive \gls{programming model}
concepts that simplify reasoning about algorithms are 
transformed to a parallel, scalable execution by the \gls{translation layer}
and \gls{back end} \gls{runtime system}.

In the example here, variables are not passed down from a \gls{parent task} to
\glspl{child task}.
Instead, one \gls{execution stream} produces a value and publishes it to a
\gls{key-value store}.
Another \gls{execution stream} reads the value by \gls{fetch}ing it from a
\gls{key-value store}.
The processes coordinate with \inlinecode{publish/fetch} pairs similar to
\inlinecode{send/recv} pairs in
the \gls{CSP} model of MPI.

\begin{minipage}{0.45\textwidth}
Execution Stream 0:
\begin{CppCode}
auto sender = 
   initial_access<int>("counter");
sender.set_value(42);
sender.publish(n_readers=1);
\end{CppCode}
\end{minipage}
\begin{minipage}{0.45\textwidth}
Execution Stream 1:
\begin{CppCode}
auto recver = 
   read_access<int>("counter");
\end{CppCode}
\end{minipage}

Instead of defining \gls{task} \glspl{precondition} implicitly via sequential order,
\gls{task} \glspl{precondition} are specified more explicitly by requiring that
a particular block of data be \gls{fetch}ed from the \gls{key-value store}.
More on \gls{spmd} programs and parallel \glspl{execution stream} are given in
Section~\ref{sec:spmd}.
