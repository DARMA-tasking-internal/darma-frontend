%!TEX root = ../sandReportSpec.tex

\section{Creating Deferred Work using Functors}
\label{sec:functor}

\lstMakeShortInline[style=CppCodeInlineStyle]{\|}

Thus far, the only method we've introduced for creating deferred work is using
C++ lambdas.  For instance,
\begin{CppCodeNumb}
auto h = initial_access<int>("my_key");
create_work([=]{ h.set_value(42); });
create_work(reads(h), [=]{ 
  cout << h.get_value() << endl; // prints "42"
});
\end{CppCodeNumb}
While this is a useful shorthand that makes it easy to get simple programs up
and running quickly, DARMA also provides a far more powerful and flexible
mechanism for describing and creating deferred work:  functors.  While functors
are significantly more verbose than the in-line lambda syntax, they are also
much more feature rich and allow DARMA to perform some additional optimizations
that aren't available to lambdas because of the limitations inherent to the C++
language itself.  The same piece of code from above can be written with
functors:
\begin{CppCodeNumb}
struct SetTo42 {
  void operator()(AccessHandle<int> h) const {
    h.set_value(42);
  }
};

struct PrintIntValue {
  void operator()(int v) const {
    cout << v << endl;
  }
};

int darma_main(...) {
  /* ... */
  auto my_handle = initial_access<int>("my_key");
  create_work<SetTo42>(my_handle);
  create_work<PrintIntValue>(my_handle);
  /* ... */
}
\end{CppCodeNumb}
Even though this code snippet is substantially more verbose than the lambda
version, it provides some useful advantages.  Most noticeably, the functors
|SetTo42| and |PrintIntValue| are reusable, just like normal functions.  They
can be implemented in different files or even different translation units for
code cleanliness and modularization.  

There are some more subtle differences
too, though.  Notice that in |PrintIntValue|, |AccessHandle::get_value()| never
needs to be called.  As long as the type to which the |AccessHandle| refers is
convertible to the formal parameter given in the functor call operator, DARMA
will call |get_value()| automatically.  Also, since the formal parameter is a
value (as opposed to a reference), DARMA can deduce at compile time that this
|PrintIntValue| makes a read-only usage of its argument.  (This would work the
same way if the formal parameter had been |int const&|, a const lvalue
reference).  Even more subtly, the fact that the formal parameter for
|PrintIntValue| isn't an |AccessHandle| communicates to DARMA at compile time
that |PrintIntValue| won't schedule any tasks that depend on |my_handle| inside
of |PrintIntValue| (we call this a {\it leaf} task with respect to |my_handle|),
which is useful information that the backend can utilize to make informed
scheduling decisions.  To accomplish the same effect for a modify usage, we can
give a formal parameter that is a non-|const| lvalue (e.g., |int&|).  The
|SetTo42| functor could then be rewritten:
\begin{CppCodeNumb}
struct SetTo42 {
  void operator()(int& val) const {
    val = 42;
  }
};
\end{CppCodeNumb}
As you can see, the functor code starts to look very much like regular C++ code.

The lambda interface can still be mixed with the functor interface.  For
instance,
\begin{CppCodeNumb}
struct Compute42 {
  void operator()(AccessHandle<int> h) const {
    create_work([=]{
      h.set_value(21);
    });
    create_work([=]{
      h.set_value( h.get_value() * 2 );
    });
  }
};
\end{CppCodeNumb}
As you can see, if we want to be able to schedule more deferred uses of a
handle, we have to take an |AccessHandle<T>| as a formal
parameter.\footnote{Equivalently, an lvalue reference or a \inlinecode{const}
lvalue reference to an \inlinecode{AccessHandle} can be given. 
\inlinecode{AccessHandle} objects ignore \inlinecode{const}, and the copy
overhead is negligible (though giving a reference parameter will be {\it
slightly} more efficient)}  If we want to pass on a read-only |AccessHandle|, we
can do so by giving |ReadAccessHandle<T>|\footnote{Note that
\inlinecode{ReadAccessHandle<T>} is identical to \inlinecode{AccessHandle<T>}
in every way except that it is known to be read-only at compile time, whereas
\inlinecode{AccessHandle<T>} has unknown compile-time permissions.  All
compile-time qualified \inlinecode{AccessHandle<T>} variant are castable to
\inlinecode{AccessHandle<T>} (and to any other \inlinecode{AccessHandle<T>}
variant with greater compile-time permissions)} and the formal parameter type:
\begin{CppCodeNumb}
struct PrintIntReadHandle {
  void operator()(ReadAccessHandle<int> h) const {
    create_work([=]{
      std::ofstream f("42.txt");
      f << h.get_value() << endl;
    });
    create_work([=]{
      cout << h.get_value() << endl;
    });
  }
};
\end{CppCodeNumb}
The nested tasks will request read permissions on |h|, just as if they had been
created with |create_work(reads(h), ...)|.  Moreover, any attempts to call
|h.set_value()| inside of |PrintIntReadHandle| will result in a {\it
compile-time} error (unlike in the pure lambda case), since DARMA knows at
compile time that the |AccessHandle<T>| is read-only.  

\subsection{Mixing Deferred and Immediate Arguments}

In DARMA, deferred functor invocations can also take normal, value arguments. 
These can, of course, be mixed (i.e., an invocation can take some
|AccessHandle| arguments and some value arguments in the same call).  However,
since deferred execution can be tricky, there are some special rules involved with value arguments to |create_work|.
\begin{itemize}
  \item When the formal parameter to the functor is a non-const lvalue
  reference, deferred invocation of the functor can only be made
  with an |AccessHandle| as that argument.  For instance,
\begin{CppCodeNumb}
int i;
auto j = initial_access<int>("mykey");;
create_work<SetTo42S>(i); // $\no$ compile-error!
create_work<SetTo42>(j); // $\yes$
\end{CppCodeNumb}
  \item When the formal parameter is a |const| lvalue reference, the deferred
  invocation must also take |AccessHandle<T>| as argument.  
  References cannot be made to regular C++ variables for deferred execution since, by deferring,
  the referred to variable may no longer exist.
  An implicit copy would be required that isn't apparent from the syntax to make the variable permanent.  
  To do this, you need to either use a value formal parameter or explicitly use
  |darma_runtime::darma_copy|:
\begin{CppCodeNumb}
struct PrintIntRef {
  void operator()(const int& val) const { cout << val << endl; }
};
int darma_main(...) {
  /* ... */
  int i = 42;
  auto j = read_access<int>("mykey");
  create_work([=]{ j.set_value(42); });
  create_work<PrintIntRef>(i); // $\no$ compile error, implicit copy of i $\label{ln:badprintintref}$
  create_work<PrintIntRef>(42); // $\yes$ 42 is an rvalue (prvalue)
  create_work<PrintIntRef>(j); // $\yes$ j is an AccessHandle<int>
  create_work<PrintIntRef>(darma_copy(i)); // $\yes$ explicit darma_copy used
  create_work<PrintIntRef>(std::move(i)); // $\yes$ i is an rvalue (xvalue)
  /* ... */
}
\end{CppCodeNumb}
  This isn't a matter of constness, but one of reference-ness; even if |i| had
  been declared as |const int i = 42;|, line \ref{ln:badprintintref} would still
  be a compile-time error, since the reference may have expired by the time the
  deferred work actually runs and a copy needs to be made.
\item Normal arguments by default can not be implicitly converted to |AccessHandle| formal
  parameters (whereas the reverse is allowed, and even encouraged when appropriate).
\end{itemize}


\subsection{Functor Interface Pitfalls}

\begin{itemize}
  \item Rvalue references (|T&&|) can't be given as formal parameters to DARMA
  functors, and doing so may lead to unexpected and/or undefined compile-time
  and/or runtime behavior.  This is related to how DARMA detects attributes of
  the formal parameters for functors, but it's also redundant --- you can just
  use a regular value parameter.  Value parameters can always be moved out of,
  and because of the way deferred execution works, rvalues must be moved into
  storage until the actual invocation occurs anyway, so there is no savings
  from using an rvalue reference over a value parameter.
  \item The DARMA functor interface doesn't currently support deferred
  invocation of functors with templated call operators (the detection idiom we
  use woudn't work here, and besides, we wouldn't even know if the deduced type
  is supposed to be an AccessHandle or not!).  Limited support for this may be
  available in the future.  Note that the functor class itself can still be
  templated, as long as the functor's template parameters are given explicitly
  at the invocation site:
\begin{CppCodeNumb}
struct CantBeUsed {
  template <typename T>
  operator()(T val) const;
};
template <typename T>
struct ThisIsFine {
  operator()(T val) const;
};
int darma_main(...) {
  /* ... */
  auto h = initial_access<int>("hello");
  create_work<ThisIsFine<int>>(h);
  /* ... */
}
\end{CppCodeNumb}
  \item Functors don't have state in DARMA.  (Even if they did, there is no
  access to the functor instance at the call site, so that state wouldn't be useful).
\end{itemize}

\lstDeleteShortInline{\|}