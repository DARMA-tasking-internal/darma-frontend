%!TEX root = ../sandReportSpec.tex

\section{Creating Deferred Work using Functors}
\label{sec:functor}

\lstMakeShortInline[style=CppCodeInlineStyle]{\|}

Thus far, the only method we've introduced for creating deferred work is using
C++ lambdas.  For instance,
\begin{CppCodeNumb}
auto h = initial_access<int>("my_key");
create_work([=]{ h.set_value(42); });
create_work(reads(h), [=]{ 
  cout << h.get_value() << endl; // prints "42"
});
\end{CppCodeNumb}
While this is a useful shorthand that makes it easy to get simple programs up
and running quickly, DARMA also provides a far more powerful and flexible
mechanism for describing and creating deferred work:  functors.  While functors
are significantly more verbose than the in-line lambda syntax, they are also
much more feature rich and allow DARMA to perform some additional optimizations
that aren't available to lambdas because of the limitations inherent to the C++
language itself.  The same piece of code from above can be written with
functors:
\begin{CppCodeNumb}
struct SetTo42 {
  void operator()(AccessHandle<int> h) const {
    h.set_value(42);
  }
};

struct PrintIntValue {
  void operator()(int v) const {
    cout << v << endl;
  }
};

int main() {
  /* ... */
  auto my_handle = initial_access<int>("my_key");
  create_work<SetTo42>(my_handle);
  create_work<PrintIntValue>(my_handle);
  /* ... */
}
\end{CppCodeNumb}
Even though this code snippet is substantially more verbose than the lambda
version, it provides some useful advantages.  Most noticably, the functors
|SetTo42| and |PrintIntValue| are reusable, just like normal functions.  They
can be implemented in different files or even different translation units for
code cleanliness and modularization.  There are some more subtle differences
too, though.  Notice that in |PrintIntValue|, |AccessHandle::get_value()| never
needs to be called.  As long as the type to which the |AccessHandle| refers is
convertible to the formal parameter given in the functor call operator, DARMA
will call |get_value()| automatically.  Also, since the formal parameter is a
value (as opposed to a reference), DARMA can deduce at compile time that this
|PrintIntValue| makes a read-only usage of its argument.  (This would work the
same way if the formal parameter had been |int const&|, a const lvalue
reference).  Even more subtly, the fact that the formal parameter for
|PrintIntValue| isn't an |AccessHandle| communicates to DARMA at compile time
that |PrintIntValue| won't schedule any tasks that depend on |my_handle| inside
of |PrintIntValue| (we call this a {\it leaf} task with respect to |my_handle|),
which is useful information that the backend can utilize to make informed
scheduling decisions.  To accomplish the same effect for a modify usage, we can
give a formal parameter that is a non-|const| lvalue (e.g., |int&|).  The
|SetTo42| functor could then be rewritten:
\begin{CppCodeNumb}
struct SetTo42 {
  void operator()(int& val) const {
    val = 42;
  }
};
\end{CppCodeNumb}
As you can see, the functor code starts to look very much like regular C++ code.

The lambda interface can still be mixed with the functor interface.  For
instance,
\begin{CppCodeNumb}
struct SetTo42Differently {
  void operator()(AccessHandle<int> h) const {
    create_work([=]{
      h.set_value(21);
    });
    create_work([=]{
      h.set_value( h.get_value() * 2 );
    });
  }
};
\end{CppCodeNumb}
As you can see, if we want to be able to schedule more deferred uses of a
handle, we have to take an |AccessHandle<T>| as a formal
parameter.\footnote{Equivalently, an lvalue reference or a \inlinecode{const}
lvalue reference to an \inlinecode{AccessHandle} can be given. 
\inlinecode{AccessHandle} objects ignore \inlinecode{const}, and the copy
overhead is negligable (though giving a reference parameter will be {\it
slightly} more efficient)}  If we want to pass on a read-only |AccessHandle|, we
can do so by giving |ReadAccessHandle<T>|\footnote{Note that
\inlinecode{ReadAccessHandle<T>} is identical to \inlinecode{AccessHandle<T>}
in every way except that it is known to be read-only at compile time, whereas
\inlinecode{AccessHandle<T>} has unknown compile-time permissions.  All
compile-time qualified \inlinecode{AccessHandle<T>} variant are castable to
\inlinecode{AccessHandle<T>} (and to any other \inlinecode{AccessHandle<T>}
variant with greater compile-time permissions)} and the formal parameter type:
\begin{CppCodeNumb}
struct PrintIntValueDifferently {
  void operator()(ReadAccessHandle<int> h) const {
    create_work([=]{
      std::ofstream f("42.txt");
      f << h.get_value() << endl;
    });
    create_work([=]{
      cout << h.get_value() << endl;
    });
  }
};
\end{CppCodeNumb}
The nested tasks will request read permissions on |h|, just as if they had been
created with |create_work(reads(h), ...)|.  Moreover, any attempts to call
|h.set_value()| inside of |PrintIntValueDifferently| will result in a {\it
compile-time} error (unlike in the pure lambda case), since DARMA knows at
compile time that the |AccessHandle<T>| is read-only.  

\subsection{Mixing Deferred and Immediate Arguments}


\subsection{Functor Interface Pitfalls}

While powerful, the functor interface 

\lstDeleteShortInline{\|}