%!TEX root = ../sandReportSpec.tex
%%%%%%%%%%%%%%%%%%%%%%%%%%%%%%%%%%%%%%%%%%%%%%%%%%%%%%%%%
%%%%%%%%%%%%%%%%%%%%%%%%%%%%%%%%%%%%%%%%%%%%%%%%%%%%%%%%%
%%%% 									API
%%%%%%%%%%%%%%%%%%%%%%%%%%%%%%%%%%%%%%%%%%%%%%%%%%%%%%%%%
%%%%%%%%%%%%%%%%%%%%%%%%%%%%%%%%%%%%%%%%%%%%%%%%%%%%%%%%%

\lstMakeShortInline[style=CppCodeInlineStyle]{\|}

\section{API: Creating and Managing Work}
In this section we provide details regarding the DARMA-\specVersion\ \gls{API}.
\clearpage

%%%%%%%%%%%%%

%\clearpage

\subsection{\texttt{darma\_main}}
\todo{David: flesh out darma\_main}

\paragraph{Summary}\mbox{}\\
\codetarget{darma_main}

\paragraph{Syntax}\mbox{}\\
\begin{CppCode}
\end{CppCode}

\paragraph{Positional Arguments} \mbox{}\\

\paragraph{Output}\mbox{}\\

\paragraph{Details}\mbox{}\\

\paragraph{Code Snippet}\mbox{}\\ 
\begin{figure}[!h]
\begin{CppCodeNumb}
#include <darma.h>
int darma_main(int argc, char** argv)
{
  return 0;
}
\end{CppCodeNumb}
\label{fig:fe_api_main}
  \caption{Basic usage of \protect\codelink{darma_main}.}
\end{figure}



\clearpage
%\clearpage

\subsection{\texttt{darma\_init}}
\label{subsec:darma_init}

\paragraph{Summary}\mbox{}\\
\codetarget{darma_init} initializes the \gls{DARMA} execution environment for a \gls{rank}.

\paragraph{Syntax}\mbox{}\\
\begin{CppCode}
void darma_runtime::darma_init(int& argc, char**& argv);
\end{CppCode}

\paragraph{Positional Arguments}\mbox{}\\
\begin{itemize}
\item argc: command line arguments count.
\item argv: array arguments.
\end{itemize}
The input parameters are the command line argument count 
and array arguments provided to main.  
Note that the backend will process and remove 
any \gls{DARMA} \gls{back end}-specific arguments from these, leaving any
application-specific arguments untouched.


\paragraph{Details}\mbox{}\\
Must be called exactly once per \gls{rank} (``exactly once'' may change in later
spec versions) before any other \gls{DARMA} function is called.
Together with \codelink{darma_finalize} (see \S~\ref{ssec:api_fe_finalize}),
this creates an \gls{execution stream} that defines a \gls{DARMA} \gls{rank}.

\paragraph{Code Snippet}\mbox{}\\ 
See code for \codelink{darma_init} in Figure~\ref{fig:fe_api_init}.

%\clearpage

\subsection{\texttt{darma\_finalize}}
\label{ssec:api_fe_finalize}

\paragraph{Summary}\mbox{}\\
\codetarget{darma_finalize} the \gls{DARMA} execution environment for a \gls{rank}.

\paragraph{Syntax}\mbox{}\\ 
\begin{CppCode}
void darma_runtime::darma_finalize();
\end{CppCode}

\paragraph{Positional Arguments}\mbox{} \\
None. 

\paragraph{Details}\mbox{} 
Called to signify the end of the \gls{execution stream} that defines a
\gls{DARMA}
\gls{rank}.  At
least by the time this function {\em returns}, the backend guarantees that all work
(tasks) created between the corresponding \codelink{darma_init} call and this
invocation, as well as all of the decendents of that work, must be
completed.  No user-level \gls{DARMA} operations are allowed after this call, though
the implicit invocation of the destructors of \codelink{AccessHandle} objects
(at, e.g., the final closing brace of \codelink{darma_main}) is allowed.  Must
be called exactly once for each call of \codelink{darma_init} (which, in turn
must be called exactly once per \gls{rank} in the current version of the spec).

\paragraph{Code Snippet}\mbox{} \\
\begin{figure}[!h]
\begin{CppCodeNumb}
#include <darma.h>
int darma_main(int argc, char** argv)
{
  using namespace darma_runtime;

  darma_init(argc, argv);
  std::cout << "DARMA initialized" << std::endl;

  // code goes here

  std::cout << "Finalizing DARMA..." << std::endl;
  darma_finalize();
  return 0;
}
\end{CppCodeNumb}
\caption{Basic usage of \protect\codelink{darma_init} and \protect\codelink{darma_finalize} 
to initialize and finalize environment.}
\label{fig:fe_api_init}
\end{figure}

\paragraph{Restrictions and Pitfalls}\mbox{} \\
\begin{itemize}
  \item \codelink{darma_finalize} should be called at the outermost task depth on
    a \gls{rank}.  In other words, it should {\it never} be called from within a
    \codelink{create_work} or other asynchronous context.
    %\item \codelink{darma_finalize} should be called in the same immediately
  %enclosing scope (IES) in which \codelink{darma_init} was invoked.  
\end{itemize}





\clearpage
\subsection{\texttt{darma\_spmd\_size}}


\paragraph{Summary}\mbox{}\\
Returns the number of ranks in the DARMA environment.

\paragraph{Syntax}\mbox{}\\
\begin{CppCode}
/* unspecified */ darma_runtime::darma_spmd_size();
\end{CppCode}

\paragraph{Positional Arguments} \mbox{}\\
None. 

\paragraph{Return} \mbox{}\\
An object of unspecified type that may be treated as a \texttt{std::size\_t}
giving the number of ranks in the DARMA environment.

\paragraph{Details} \mbox{}\\
This function gives the number of ranks DARMA is executing the program with. 
Specifically, it is the number of times the backend has invoked
\texttt{darma\_main} anywhere in the system for this particular run of the
program (and thus, it is also the number of times the backend expects the user
to invoke \texttt{darma\_init}).

\paragraph{Code Snippet} \mbox{}\\
\begin{figure}[!h]
\begin{CppCodeNumb}
#include <darma.h>
int darma_main(int argc, char** argv)
{
  using namespace darma_runtime;
  darma_init(argc, argv);

  const size_t size = darma_spmd_size();
  // ...

  darma_finalize();
  return 0;
}
\end{CppCodeNumb}
\label{fig:fe_api_ranksize}
\caption{Basic usage of \texttt{darma\_spmd\_size}.}
\end{figure}

\paragraph{Restrictions and Pitfalls}\mbox{} \\
\begin{itemize}
  \item The value returned by this function will always return \texttt{true}
  for greater-than comparison with 0, and will always be convertible to a
  \texttt{std::size\_t} with a value greater than 0.
  \item The return type is unspecified to allow future expansion to generalized
  ranks.  For instance, future versions of the spec may allow the user to
  request the rank as an \texttt{\{x, y, z\}} tuple of indices in a structured
  lattice.
\end{itemize}


\clearpage

%%%%%%%%%%%%%

%\clearpage

\subsection{\texttt{darma\_spmd\_rank}}

\paragraph{Summary}\mbox{}\\
\codetarget{darma_spmd_rank} returns the \gls{rank} index associated with the
\gls{execution stream} from which this function was invoked.

\paragraph{Syntax}\mbox{}\\
\begin{CppCode}
/* unspecified */ darma::darma_spmd_rank();
\end{CppCode}

\paragraph{Positional Arguments} \mbox{}\\
None. 

\paragraph{Output}\mbox{}\\
An object of unspecified type that may be treated as a \inlinecode{std::size_t}
which is less than the value returned by \codelink{darma_spmd_size}.

\paragraph{Details}\mbox{}\\
This function returns the \gls{rank} index of the calling \gls{DARMA} \gls{execution
stream}.  If the value returned
by \codelink{darma_spmd_size} is convertible to a \inlinecode{std::size_t} with the
value $N$, then the value returned by this function will be convertible to a
\inlinecode{std::size_t} with the value $r$, which will always satisfy $0 <=
r < N$.  Furthermore, the type of the value returned by this function will
always be directly comparable to the type returned by \codelink{darma_spmd_size}
and to $0$ such that this previous condition is met.  The value returned is also
equality comparable with $0$, the value returned will be true for equality
comparison with $0$ on exactly one rank.  The value returned by this function
will be unique on every \gls{rank} (in the equality sense), and will be the same
across multiple invocations of the function within a given \gls{rank}.  The value
returned will also be the same at any asynchronous work invocation depth within
a \gls{rank}'s \gls{execution stream}, regardless of whether that work gets stolen or
migrated.

\paragraph{Code Snippet}\mbox{}\\ 
\begin{figure}[!h]
\begin{CppCodeNumb}
#include <darma.h>
int darma_main(int argc, char** argv)
{
  using namespace darma;
  darma_init(argc, argv);

  // get my rank
  const size_t myRank = darma_spmd_rank();
  // get size 
  const size_t size = darma_spmd_size();

  std::cout << "Rank " << myRank << "/" << size << std::endl;

  darma_finalize();
  return 0;
}
\end{CppCodeNumb}
\label{fig:fe_api_ranksize}
\caption{Basic usage of \protect\codelink{darma_spmd_rank}.}
\end{figure}



\clearpage
%\clearpage
\subsection{\texttt{create\_work}}
\label{ssec:api_fe_cw}

\paragraph{Summary}\mbox{}\\
\codetarget{create_work} instantiates \gls{deferred work} to be executed by the \gls{runtime
system}.

\paragraph{Syntax}\mbox{}\\
\begin{CppCode}
// Functionally:
create_work([=]{
  // Code expressing deferred work goes here
});
// or:
create_work(
  ConstraintExpressions..., 
  [=]{
    // Code expressing deferred work goes here
  }
);
// or:
create_work<FunctorType>(ArgumentsToFunctor...);

// Formally:
/* unspecified */ create_work(Arguments..., LambdaExpression);
/* unspecified */ create_work<FunctorWrapper>(Arguments...);
\end{CppCode}


\paragraph{Positional Arguments}\mbox{}\\
\begin{compactitem}
\item \inlinecode{LambdaExpression} A \CC{}11 \gls{lambda} expression with a copy
    default-\gls{capture} (i.e., |[=]|) and taking no arguments.  More details
  below.
\item \inlinecode{ConstraintExpressions...} (optional) If given, these
  arguments can be used to express modifications in the default \gls{capture} behavior
  of \codelink{AccessHandle<T>} objects captured by the \inlinecode{LambdaExpression}
  given as the final argument.  In the \specVersion-specification, the only valid permission
  modification expression is the return value of the \codelink{reads()} modifier
  (see \S~\ref{ssec:api_fe_reads}), which indicates that only read operations
  are performed on a given \gls{handle} or \glspl{handle} within the
  \inlinecode{LambdaExpression} that follows.
\item \inlinecode{ArgumentsToFunctor...} In the deferred functor invocation
  version, these arguments are pattern-matched with the formal parameters of the
  functor, causing the deferred invocation to invoke the call operator of
  \inlinecode{FunctorType} with arguments derived from these as described in
  \S~\ref{sec:functor}.  Constraint expressions may also be used in the
  corresponding positional argument spots for a given \ahandleT argument.
\end{compactitem}


\paragraph{Return}\mbox{}\\
Currently \inlinecode{void} in the \specVersion-specification, but may be an object of unspecified
type in future implementations.

\paragraph{Details}\mbox{}\\
This function expresses work to be executed by the \gls{runtime system}.  Any
\codelink{AccessHandle} variables used in the \inlinecode{LambdaExpression} or
given in \inlinecode{ArgumentsToFunctor...} will be captured and made available
inside the capturing context or \inlinecode{FunctorType} call operator as if
they were used in sequence with previous capture operations or deferred functor
invocations with the same \gls{handle}.  Depending on the scheduling permissions
available to the \inlinecode{AccessHandle<T>} at the time of
\codelink{create_work} invocation and on the \inlinecode{ConstraintExpressions...}
given as arguments, this function call expresses either a {\it read-only
capture} or a {\it modify capture} operation on a given \gls{handle} (see
\S~\ref{sec:handlerules}).  If a \gls{handle} \inlinecode{h} has {\it Read} scheduling
permissions when it is captured or if the explicit constraint expression
\inlinecode{reads(h)} is given in the \inlinecode{ConstraintExpressions...} arguments,
\codelink{create_work} functions as a {\it read-only capture} operation on that
\gls{handle}.
Otherwise, it functions as a {\it modify capture}.  Formal parameters to the
\inlinecode{FunctorType} call operator can also affect the type of capture
operation that is performed, as discussed in \S~\ref{sec:functor}.

Additional general discussion on use of \codelink{create_work} can be found in
\S~\ref{sec:deferred}.


\paragraph{Code Snippet}\mbox{}\\
\begin{figure}[!h]
\begin{CppCodeNumb}
create_work([=]{
  std::cout << " Hello world! " << std::endl;
});
\end{CppCodeNumb}
\label{fig:fe_api_cw}
  \caption{Basic usage of \protect\codelink{create_work}.}
\end{figure}



\paragraph{Restrictions and Pitfalls}\mbox{} \\
Most of the general restrictions and pitfalls related to \codelink{create_work}
are discussed in \S~\ref{sec:deferred}.  Some more technical restrictions are
given here.
\begin{compactitem}
  \item Because of the way in which \codelink{create_work} is implemented, placement of
    multiple \codelink{create_work} operations on the same line of code will not compile. 
  For instance:
  \begin{CppCode}
// $\no$ does not compile, gives cryptic error message
create_work([=]{}); create_work([=]{}); 
  \end{CppCode}
  This is particularly easy to accidentally do when defining preprocessor
  macros:
  \begin{CppCode}
// $\no$ does not compile, gives even more cryptic error message
#define foo(...) __VA_ARGS__
foo(
  create_work([=]{}); 
  create_work([=]{}); 
)
  \end{CppCode}
  Note that this is not a problem when using nested \codelink{create_work} calls:
  \begin{CppCode}
// $\yes$ not a problem
create_work([=]{ create_work([=]{}); }); // works fine
  \end{CppCode}
  Other than the obvious solution of putting the \codelink{create_work} invocations on
  multiple lines, this issue can be worked around by putting any of the later
  \codelink{create_work} calls within their own scopes:
  \begin{CppCode}
// $\yes$ works fine
create_work([=]{}); { create_work([=]{}); }
// $\yes$ also fine
foo(
  create_work([=]{}); 
  { create_work([=]{}); }
  { create_work([=]{}); }
)
  \end{CppCode}
\end{compactitem}


\clearpage

%\todo[inline]{Jeremy/David: darmacopy is missing}

\section{API: Data Access Handles}

In this section, we discuss the handle types in the DARMA \specVersion\ spec.
%\todo[inline]{Jeremy/David: AccessHandle and ReadAccessHandle are missing}
\subsection{\texttt{initial\_access}}

\paragraph{Summary}\mbox{}\\ 
Create a handle to data that does not yet exist in the system 
but needs to be created.

\paragraph{Syntax}\mbox{}\\ 
\begin{CppCode}
AccessHandle<T> darma_runtime::initial_access<T>(arg1, arg2, ...);
\end{CppCode}

\paragraph{Positional Arguments}\mbox{}\\ 
arg1, arg2, ...: arbitrary tuple of values defining the key of the data.

\paragraph{Return}\mbox{}\\ 
An object of unspecified type that may be treated as an |AccessHandle<T>|
with the key given by the arguments.

\paragraph{Details}\mbox{}\\ 
This construct creates a handle to data that does not yet 
exist but needs to be created.  The handle is created with {\it Modify}
scheduling permissions and {\it None} immediate permissions.  The function takes
as input an arbitrary tuple of values.
Note that this key has to be unique (see Section \ref{}).
One cannot define two handles with same the same key, even if they are created by different ranks.
One basic way to ensure this is the case is to always use the rank 
as one component of the key. 

\paragraph{Code Snippet}\mbox{}\\
\begin{figure}[!h]
\begin{CppCodeNumb}
  auto my_handle1 = initial_access<double>("data_key_1", myRank);
  auto my_handle2 = initial_access<int>("data_key_2", myRank, "_online");
\end{CppCodeNumb}
\label{fig:fe_api_initialaccess}
\caption{Basic usage of \lstinline|initial_access|.}
\end{figure}

\paragraph{Restrictions and Pitfalls}\mbox{}\\ 
\begin{itemize}
  \item Because the actual type returned by |initial_access<T>| is
  unspecified, you should generally use |auto| instead of naming type on 
  the left hand side of the assignment (this is generally a good idea in modern
  C++). In other words,
  \begin{CppCode}
	// $\yes$ good, preferred
	auto my_handle1 = initial_access<double>("good"); 

	// $\no$ still compiles, but not preferred (may miss out
	//  on some future optimizations and compile-time checks)
	AccessHandle<double> my_handle1 = initial_access<double>("bad"); 
  \end{CppCode}
\end{itemize}


%%%%%%%%%%%%%


\clearpage
%!TEX root = ../sandReportSpec.tex
\subsection{\texttt{read\_access}}
\label{ssec:api_fe_read_access}

\paragraph{Summary}\mbox{}\\
\codetarget{read_access<T>} creates a \gls{handle} with \text{read-only} access to data that has been or will be
published elsewhere in the system.


\paragraph{Syntax}\mbox{}\\
\begin{CppCode}
/* unspecified, convertible to AccessHandle<T> */
darma::read_access<T>(KeyParts..., version=KeyExpression);
\end{CppCode}


\paragraph{Positional Arguments}\mbox{}\\
\begin{itemize}
  \item |KeyParts...|: tuple of values identifying the key of the data to
  be read.  This is also called the ``name key'' of the handle.
\end{itemize}

\paragraph{Keyword Arguments}\mbox{}\\
\begin{itemize}
  \item |version=KeyExpression| (or |version(KeyExpression...)|,
see \S~\ref{sec:keyword} for multiple-right-hand-side keyword argument usage):
the version used to publish the data to be accessed.
The value can be an arbitrary |KeyExpression|.
\end{itemize}


\paragraph{Return}\mbox{}\\
An object of unspecified type that may be treated as an \codelink{AccessHandle<T>}
with the key given by the arguments.

\paragraph{Details}\mbox{}\\
This function creates a \gls{handle} to data that already exists and 
needs to be accessed with read-only privileges. 
It takes as input the \gls{tuple} of values uniquely 
identifying the data that needs to be read.
Immediately following this function, the \gls{handle} will have Read
\gls{scheduling permissions} and None \gls{immediate permissions}.
The \codelink{key}-\codelink{version} requested must eventually match that of a
\codelink{key}-\codelink{version} that was \codelink{publish}ed.

In general, \codetarget{read_access} data is migratable and potentially stored off-node.

\paragraph{Code Snippet}\mbox{}\\
\begin{CppCodeNumb}
/* on one rank: */
auto my_handle1 = initial_access<double>("key_1");
create_work([=]{
  my_handle1.emplace_value(5.3);
});
my_handle1.publish(n_readers=1, version="final");

//...

/* potentially on another rank: */
auto readHandle = read_access<double>("key_1", version="final");
create_work([=]{
  std::cout << readHandle.get_value() << std::endl;
});
\end{CppCodeNumb}

\paragraph{Restrictions and Pitfalls}\mbox{}\\ 
\begin{itemize}
  \item Because the actual type returned by |read_access<T>| is
  unspecified, you should generally use |auto| instead of naming the type on 
  the left hand side of the assignment (this is generally a good idea in modern
  \CC{}). In other words,
  \begin{CppCode}
	// $\yes$ good, preferred
	auto my_handle1 = read_access<double>("good", version=17); 

	// $\no$ still compiles, but not preferred (may miss out
	//  on some future optimizations and compile-time checks)
	AccessHandle<double> my_handle1 = read_access<double>("bad", version="oops"); 
  \end{CppCode}
  For more, see \S~\ref{ssec:ahtraits}.
\end{itemize}

\clearpage

\section{API: \texttt{AccessHandle} methods}
\codetarget{AccessHandle}, \codetarget{AccessHandle<T>} The lightweight wrappers for variables that enables deferred execution of tasks involving variables wrapped with \inlinecode{AccessHandle}. 
In this section, we describe the methods that can be called 
on \texttt{AccessHandle<T>} objects (i.e., the objects returned by the 
\texttt{initial\_access<T>} and \texttt{read\_access<T>} functions).


\subsection{\texttt{emplace\_value}}
\label{ssec:api_fe_emplace_value}

\paragraph{Summary}\mbox{}\\ 
Construct an object of the type pointed to by an |AccessHandle<T>| object
(that is, |T|) in place by forwarding the arguments to the constructor
for |T|.

\paragraph{Syntax}\mbox{}\\ 
\begin{CppCode}
// functional:
some_handle.emplace_value(arg1, arg2, ...);

// Formal:
void AccessHandle<T>::emplace_value(Args&&... args);
\end{CppCode}

\paragraph{Positional Arguments}\mbox{}\\ 
\begin{itemize}
  \item |args...| (deduced types):  Arguments to forward to the
  constructor of |T|.
\end{itemize}

\paragraph{Details}\mbox{}\\ 

|AccessHandle<T>::emplace_value(...)| mimics the syntax for in-place
construction in standard library containers.  See, for instance,
|std::vector<T>::emplace_back(...)|.  If in-place construction is
unnecessary or undesired, |set_value()| can be used instead.  
Note that
calling |emplace_value()| on a handle requires {\it Modify} immediate
permissions (see \S~\ref{sec:handlerules}).

\paragraph{Code Snippet}\mbox{}\\ 
\begin{figure}[!h]
\begin{CppCodeNumb}
struct LoudMouth {
  LoudMouth(int i, double j) { cout << "Ctor: " << i << ", " << j << endl; }
};
auto h = initial_access<LoudMouth>("key");
create_work([=]{
  h.emplace_value(42, 3.14); // prints "Ctor: 42, 3.14" 
});
\end{CppCodeNumb}
\label{fig:fe_api_initialaccess}
\caption{Basic usage of \lstinline|emplace_value|.}
\end{figure}

\paragraph{Restrictions and Pitfalls}\mbox{}\\ 
\todo{Jeremy/David: Is this still an issue?  If so, remove ``0.2 only'', otherwise remove item completely}
\begin{itemize}
  \item {\color{red}[0.2 only]} Because the 0.2 spec lacks a means of specifying
  the constructor of |T| to be called by default, the first usage of
  every modifiable handle (i.e., the first {|create_work|} capturing the
  handle) after being setup with |initial_access<T>| should call
  |emplace_value()| or |set_value()| before performing any
  operations on the pointed-to object.  In other words, the pointed-to object
  contains uninitialized memory until the user invokes 
  |emplace_value(...)| or |set_value()| to construct the object.
  Failure to do so leads to undefined behavior (just as, for instance,
  |int x; cout << x;| leads to undefined behavior in C). 
\end{itemize}





\clearpage
\subsection{\texttt{publish}}
\label{ssec:api_fe_publish}

\paragraph{Summary} \mbox{}\\
Publish the data pointed to by a given handle so that it can be retrieved on 
other DARMA ranks.

\paragraph{Syntax} \mbox{}\\
\begin{CppCode}
void
AccessHandle<T>::publish(n_readers=..., version=...)
\end{CppCode}

\paragraph{Positional Arguments} \mbox{}\\
None.

\paragraph{Keyword Arguments} \mbox{}\\
\begin{compactitem}
\item \texttt{n\_readers=size\_t} (optional): informs the runtime how many . If
omitted, it defaults to 1.
\item \texttt{version=KeyExpression} (or \texttt{version(KeyExpression\ldots)},
see \S~\ref{sec:keyword} for multiple-right-hand-side keyword argument usage)
(optional):
informs the runtime what version to associate with the data being published. 
The value can be an arbitrary \texttt{KeyExpression}.
If omitted, the version defaults to an empty key (i.e., a key tuple with zero 
components).  Omitting this keyword implicitly indicates to the runtime system
that the handle (or any handle with the same name key) will not be published
again in the remaining lifetime of the program.
\end{compactitem}

\paragraph{Details} \mbox{}\\

Publish the data associated with a given handle \texttt{h} such that it can be
retrieved \texttt{n\_readers} times anywhere via a \texttt{read\_access<T>} invocation
that gives the same name key as \texttt{h} and the same \texttt{version} key as
the one given to the keyword argument to \texttt{publish()}.
A \texttt{publish()} is a {\it read-only capture} operation (see
\S~\ref{sec:handlerules}).

\paragraph{Code Snippet} \mbox{}\\
\begin{figure}[!h]
\begin{CppCodeNumb}
auto me = darma_spmd_rank();
assert(darma_spmd_size() >= 2);
if(me == 0) {
  auto my_handle = initial_access<double>("key_1");
  create_work([=]{
    my_handle.emplace_value(5.3);
  });
  my_handle.publish(n_readers=1, version="only");
}
else if(me == 1) {
  auto my_handle = read_access<double>("key_1", version="only");
  create_work([=]{
    cout << my_handle.get_value() << endl; // prints "5.3"
  });
}
\end{CppCodeNumb}
\label{fig:fe_api_publish}
\caption{Basic usage of \texttt{publish()}.}
\end{figure}

\paragraph{Restrictions and Pitfalls}\mbox{}\\
\begin{compactitem}
  \item \texttt{publish()} is one of the ways DARMA lets you ``shoot yourself in
  the foot.''  While it is very difficult to create dependency loops, race
  conditions, and deadlock using handles within a given rank and without
  \texttt{publish()} operations, it is quite easy to do so with
  \texttt{publish()}/\texttt{read\_access()} pairs (just like, for instance,
  with blocking sends and receives in MPI).  For instance, the following snippet
  deadlocks:
  %\begin{center}
  \begin{CppCode}
// This code deadlocks!
auto me = darma_spmd_rank();
assert(darma_spmd_size() >= 2);
if(me == 0) {
  auto h1 = initial_access<int>("key", 0);
  auto h2 = read_access<int>("key", 1);
  create_work([=]{ 
    h1.set_value(42);
    h1.publish();
    cout << h2.get_value() << endl;
  }); 
}
else if (me == 1) {
  auto h3 = initial_access<int>("key", 1);
  auto h4 = initial_access<int>("key", 0);
  create_work([=]{ 
    h3.set_value(73);
    h3.publish();
    cout << h4.get_value() << endl; 
  }); 
}
// Deadlock! (eventually, at the latest when darma_finalize() is
// called): neither of the above create_work()s can ever run
  \end{CppCode}
  This snippet deadlocks because a dependency loop has been created between two
  \texttt{publish()}/\texttt{read\_access()} pairs. While the deadlock is
  relatively obvious here, it can be much more difficult to decypher in a more
  complex code, especially if, for instance, \texttt{h1} and \texttt{h2} are
  arguments to a function, or if the parts of the keys used to construct the
  handles are variables with values dependent on some previous computation.
  %%%%%%%%%%%%%%%%%%%%%%%%%%%%%%%%%%%%%%%%%%%%%%%%%%%%%%%%%%%%%%%%%%%%%%%%%%%%%
  \item It is particularly easy to create deadlock scenarios by publishing a
  handle and fetching it within the same rank.  For this reason, we recommend
  extreme caution with this scenario could arise, and in general we suggest that
  the user should avoid doing so if at all possible.
  %%%%%%%%%%%%%%%%%%%%%%%%%%%%%%%%%%%%%%%%%%%%%%%%%%%%%%%%%%%%%%%%%%%%%%%%%%%%%
  \item Since \texttt{publish()} is a {\it read-only capture} operation, it must
  have scheduling privileges of {\it Read} or {\it Modify}; calling publish on a
  handle with other scheduling permissions is a runtime error.  Also, as with
  all {\it read-only capture} operations, calling \texttt{publish()} on a
  handle with {\it Modify} immediate permissions results in a handle with {\it
  Read} immediate permissions in the continuing context.  See
  \S~\ref{sec:handlerules} for more details.  For example, the following code
  results in a runtime error at the marked line:
%\begin{minipage}[t]{0.95\linewidth}%
%  \centering
  \begin{CppCode}
  auto h = initial_access<int>("key");
  create_work([=]{
    h.set_value(5);
    h.publish();
    h.set_value(10); // $\no$ h has Read immediate permissions
  });
  \end{CppCode}
%\end{minipage}
  \item It is an error to call \texttt{publish()} on a handle with a given
  name key (or any other handle with that same name key) more than once with
  a given version key.
  %%%%%%%%%%%%%%%%%%%%%%%%%%%%%%%%%%%%%%%%%%%%%%%%%%%%%%%%%%%%%%%%%%%%%%%%%%%%%
  \item If \texttt{publish()} is called on a given handle without the
  \texttt{version} keyword argument, it is an error to call \texttt{publish()}
  again on that handle or any other handle with the same name key for the
  remaining lifetime of the program.  Note that because of the default behavior
  of the \texttt{version} keyword argument, giving an explicit version that is
  the empty key (e.g., \texttt{h.publish(version())} or
  \texttt{h.publish(version=make\_key())}) will lead to this same behavior.
  %%%%%%%%%%%%%%%%%%%%%%%%%%%%%%%%%%%%%%%%%%%%%%%%%%%%%%%%%%%%%%%%%%%%%%%%%%%%%
  \item {\color{red} [0.2 only]} Because the 0.2 spec does not include a
  serialization interface, only handles with bitwise-copiable types may be
  published, though some backends may support more functionality.
\end{compactitem}




\clearpage
\subsection{\texttt{get\_value}}

\paragraph{Summary} \mbox{}\
Access the data pointed to by the handle in a read-only manner.

\paragraph{Syntax}\mbox{}\\
\begin{CppCode}
const T& AccessHandle<T>::get_value();
\end{CppCode}

\paragraph{Positional Arguments}\mbox{}\\ 
None.

\paragraph{Return}\mbox{}\\ 
A const reference to the data associated with the handle.  

\paragraph{Details}\mbox{}\\ 
Calling \texttt{get\_value()} on a handle requires {\it Read} or {\it
Modify} immediate permissions (see \S~\ref{sec:handlerules}).


\paragraph{Code Snippet}\mbox{}\\ 
\begin{figure}[!h]
\begin{CppCodeNumb}
AccessHandle<double> my_handle = initial_access<double>("key_1", myRank);
create_work([=]{
  my_handle.set_value(3.14);
});
create_work(reads(my_handle), [=]{
  double myValue = my_handle.get_value();
});
\end{CppCodeNumb}
\label{fig:fe_api_get_value}
\caption{Basic usage of \texttt{get\_value}.}
\end{figure}


\paragraph{Restrictions and Pitfalls}\mbox{}\\ 
\begin{itemize}
  \item Do not hold the reference returned by this method across an asyncronous
  operation on the source handle.  For example, the following results in
  undefined behavior:
  \begin{CppCode}
	// $\yes$ good, preferred
	auto h = initial_access<int>("my_key"); 
	create_work([=]{ h.set_value(5); });
	
	create_work([=]{ 
	  auto const& v = h.get_value();
	  create_work([=]{ h.set_value(10); });
	  cout << v << endl; // $\no$ undefined behavior!!
	});
  \end{CppCode}
\end{itemize}
\clearpage
\subsection{\texttt{set\_value}}

\paragraph{Summary} \mbox{}\\
Set the value of the data pointed to by the handle.

\paragraph{Syntax} \mbox{}\\
\begin{CppCode}
template <typename U>
void AccessHandle<T>::set_value(U&& value)
\end{CppCode}

\paragraph{Positional Arguments}\mbox{}\\
\begin{itemize}
  \item value (type convertible to |T|): The new value for the data.
\end{itemize}

\paragraph{Details}\mbox{}\\
This invokes |T::operator=(U&&)| (|T|'s assignment operator to
a universal reference to |U|) with the argument |value|.  If the type |T| has no
assignment operator for this type, calling |set_value| will be a compile-time
error.  If you need to invoke an in-place constructor instead, use |emplace_value|.

\paragraph{Code Snippet}\mbox{}\\
\begin{figure}[!h]
\begin{CppCodeNumb}
auto h = initial_access<double>("key_1");
create_work([=]{
  h.set_value(55.343);
});
\end{CppCodeNumb}
\label{fig:fe_api_set_value}
\caption{Basic usage of \lstinline|set_value|.}
\end{figure}

\paragraph{Restrictions and Pitfalls}\mbox{}\\
\begin{itemize}
  \item The specification of the method is likely to change in the future to
  analogous to the behavior, e.g., |std::vector<T>::push_back()| (as it
  relates to |std::vector<T>::emplace_back()|).  If this could be a
  problem for |T|, you should probably use |emplace_value()| for now.
\end{itemize}





\clearpage
\subsection{\texttt{get\_reference}}

\paragraph{Summary}\mbox{}\\
Get a non-constant reference to the data pointed to by the handle.

\paragraph{Syntax}\mbox{}\\
\begin{CppCode}
value_type & some_handle.get_reference()
\end{CppCode}

\paragraph{Positional Arguments}\mbox{}\\
None.

\paragraph{Output}\mbox{}\\
A non-constant reference to the data.  

\paragraph{Details}\mbox{}\\
This method can be called within a context with write privileges 
on the target handle.\todo{anything missing? details?}


\paragraph{Code Snippet}\mbox{}\\
\begin{figure}[!h]
\begin{CppCodeNumb}
AccessHandle<double> my_handle1 = read_access<double>("key_1", myRank);
create_work([=]{
	my_handle1.get_reference() = 242.343;
});
\end{CppCodeNumb}
\label{fig:fe_api_getreference}
\caption{Basic usage of \texttt{get\_reference}.}
\end{figure}




\clearpage
\subsection{\texttt{->}}

\paragraph{Summary}\mbox{}\\
A dereference operator to directly access the object 
pointed to by the handle.


\paragraph{Syntax}\mbox{}\\
\todo{how to write syntax here??}
\begin{CppCode}
some_handle->
\end{CppCode}

\paragraph{Input Parameters}\mbox{}\\
None.

\paragraph{Output Parameters}\mbox{}\\
The return type of the method called on the object.

\paragraph{Details}\mbox{}\\
This method can be called within a deferred work to directly 
access the object pointed to by the handle.\todo{anything missing? details?}


\paragraph{Code Snippet}\mbox{}\\
\begin{figure}[!h]
\begin{CppCodeNumb}
//...
typedef std::vector<double> vec;
AccessHandle<vec> my_handle2 = initial_access<vec>("key_2", myRank);

create_work([=]{
	my_handle2.emplace_value(0.0);
  my_handle2->resize(4);
  double * vecPtr = my_handle2->data();    
});
\end{CppCodeNumb}
\label{fig:fe_api_arrow}
\caption{Basic usage of arrow operator \texttt{->}.}
\end{figure}


\clearpage
\subsection{\texttt{get\_key}}

\paragraph{Summary}\mbox{}\\ 
\codetarget{get_key} gets the \codelink{key} identifying the data pointed to by
the \gls{handle}.

\paragraph{Syntax}\mbox{}\\ 
\begin{CppCode}
darma_runtime::types::key_t const& AccessHandle<T>::get_key();
\end{CppCode}

\paragraph{Positional Arguments}\mbox{}\\ 
None.

\paragraph{Return}\mbox{}\\ 
The \codelink{key} identifying the data.

\paragraph{Details}\mbox{}\\ 
This method can be called at any time after the \gls{handle} is created.  It does not
require any \gls{scheduling permissions} nor \gls{immediate permissions}.

\paragraph{Code Snippet}\mbox{}\\ 
\begin{figure}[!h]
\begin{CppCodeNumb}
//...
auto myRank = darma_spmd_rank();
AccessHandle<double> my_handle1 = read_access<double>("key_1", myRank);
auto myK = my_handle1.get_key();

create_work([=]{
	my_handle1.get_reference() = 242.343;
	auto myK = my_handle1.get_key();
	assert(myRank == myK.get_key().component<1>().as<int>());
});
\end{CppCodeNumb}
\label{fig:fe_api_get_key}
\caption{Basic usage of \protect\codelink{get_key}.}
\end{figure}





\clearpage
\subsection{\texttt{=0} or \texttt{release}}

\paragraph{Summary} \mbox{}\\
Releases the reference to the data held by the handle.

\paragraph{Syntax} \mbox{}\\
These two are equivalent.
\begin{CppCode}
// Functional:
some_handle = 0;    
some_handle.release()

// Formal
void AccessHandle<T>::operator=(std::nullptr_t);
void AccessHandle<T>::release();
\end{CppCode}


\paragraph{Positional Arguments} \mbox{}\\
None.

\paragraph{Return} \mbox{}\\
None.

\paragraph{Details} \mbox{}\\
Release the reference to the underlying data held by a given handle.  Note that
this effectively only decrements the reference count; the data itself will not
be deleted unless there are no other existing handles referring to it.

\paragraph{Code Snippet} \mbox{}\\
\begin{figure}[!h]
\begin{CppCodeNumb}
// ...
AccessHandle<double> my_handle1 = initial_access<double>("key_1", myRank);
create_work([=]{
	my_handle1.get_reference() = 242.343;
});
my_handle1.release();
// ...
\end{CppCodeNumb}
\label{fig:fe_api_release}
\caption{Basic usage of \texttt{=0} or \texttt{release}.}
\end{figure}






\clearpage

\section{API: Keywords}

In this section, we describe the keywords of the \specVersion\ API. 

\subsection{\texttt{reads}}
\label{ssec:api_fe_reads}

\paragraph{Summary} \mbox{}\\
Creates an argument to a |create_work| to constrain
privileges of a set of handles to be read-only within that context.

\paragraph{Syntax} \mbox{}\\
\begin{CppCode}
create_work(reads(handles...), [=]{
  // code 
});
\end{CppCode}

\paragraph{Positional Arguments} \mbox{}\\
\begin{itemize}
  \item |handles...|: list of |AccessHandle<T>| objects to
  constrain to read-only privileges.
\end{itemize}


\paragraph{Details} \mbox{}\\
Used as argument to a |create_work| to constrain
privileges of a list of handles to be read-only within that context.
It can contain a single handle or a list of handles.  See also
\S~\ref{ssec:api_fe_cw} for more information on |create_work|.


\paragraph{Code Snippet} \mbox{}\\
\begin{figure}[!h]
\begin{CppCodeNumb}
// ...
auto my_handle = initial_access<double>("data", myRank);
create_work([=]{
  my_handle.emplace_value(0.55);
});
create_work(reads(my_handle), [=]{
  std::cout << " " << my_handle.get_value() << std::endl;
  my_handle.set_value(3.14); // $\no$ runtime error
});
// ... 
\end{CppCodeNumb}
\label{fig:fe_api_reads}
\caption{Basic usage of \lstinline|reads|.}
\end{figure}


\paragraph{Restrictions and Pitfalls}\mbox{}\\
\begin{itemize}
  \item This can only be called as argument to |create_work|.  Use in
  other contexts will lead to compile-time errors, run-time errors, or
  undefined behavior.
\end{itemize}

\clearpage
\subsection{\texttt{n\_readers}}

\paragraph{Summary}\mbox{}\\ 
Specifies the number of times a certain piece of 
data will be read after being published. This keyword has to be used within 
the \texttt{publish} method of an access handle.


\paragraph{Syntax}\mbox{}\\ 
These two are equivalent.
\begin{CppCode}
n_readers = ...; 
n_readers(...); 
\end{CppCode}
See \S~\ref{ssec:api_fe_publish} for details on \texttt{publish}.


\paragraph{Positional Argument}\mbox{}\\ 
The number of times the data will be read.
See \S~\ref{ssec:api_fe_publish} for details on \texttt{publish}.


\paragraph{Details}\mbox{}\\ 
Specifies the number of times a certain piece of 
data will be read. This keyword has to be used within 
the \texttt{publish} method of an access handle.
More specifically, this refers to the number of \texttt{read\_access} 
handles that will be defined to read this data.\todo{anything missing? details?}


\paragraph{Code Snippet}\mbox{}\\ 
\begin{figure}[!h]
\begin{CppCodeNumb}
auto my_handle = initial_access<double>("data", myRank);
create_work([=]
{
  my_handle.emplace_value(0.5 + (double) myRank);
  my_handle.publish(n_readers=1);
});
\end{CppCodeNumb}
\label{fig:fe_api_n_readers}
\caption{Basic usage of \texttt{n\_readers}.}
\end{figure}





\clearpage
\subsection{\texttt{version}}
\label{ssec:api_fe_version}

\paragraph{Summary}\mbox{}\\ 
Keyword argument to \texttt{AccessHandle<T>::publish()} and
\texttt{read\_access<T>()}.  See \S~\ref{ssec:api_fe_publish} for details on
\texttt{AccessHandle<T>::publish()} and \S~\ref{ssec:api_fe_read_access} for
details on \texttt{read\_access<T>()}.

In namespace \texttt{darma\_runtime::keyword\_arguments\_for\_publication}.

%\paragraph{Syntax}\mbox{}\\ 
%These two are equivalent.
%\begin{CppCode}
%version = ...; 
%version(arg1, arg2, ...);
%\end{CppCode}
%
%\paragraph{Positional Arguments}\mbox{}\\ 
%A tuple of values.
%
%\paragraph{Details}\mbox{}\\ 
%Specifies explicitly a version to associated to some data 
%at the time of the publishing operation.  
%It is used when the same data is overwritten multiple times 
%and one needs to keep track of the various versions.
%
%Version can be an arbitrary tuple of values.
%\todo{anything missing? details?}
%
%\paragraph{Code Snippet}\mbox{}\\ 
%\begin{figure}[!h]
%\begin{CppCodeNumb}
%//...
%auto my_handle = initial_access<double>("data", myRank);
%create_work([=]{
%  my_handle.emplace_value(0.5 + (double) myRank);
%  my_handle.publish(n_readers=1,version=(0,"basic"));
%});
%// ...
%// reset value and update version
%create_work([=]{
%  my_handle.set_value(2.5 + (double) myRank);
%  my_handle.publish(n_readers=1,version=1);
%});
%\end{CppCodeNumb}
%\label{fig:fe_api_version}
%\caption{Basic usage of \texttt{version}.}
%\end{figure}





\lstDeleteShortInline{\|}
