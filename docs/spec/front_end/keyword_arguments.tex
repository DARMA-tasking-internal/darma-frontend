\section{Keyword arguments}
\label{sec:keyword}
%\lstMakeShortInline[style=CppCodeInlineStyle]{\|}
Similar to higher-level languages like Python, the DARMA \CC{} interface allows the user
to specify arguments to many of the \gls{API} functions and constructs using
either \glspl{positional argument}
or \glspl{keyword argument}. In addition, many optional arguments may {\em only} be specified using
\glspl{keyword argument}. The syntax for specifying a \gls{keyword argument} is identical to that
of Python: \inlinecode{keyword=value}.  For instance, if there is 
a function \inlinecode{some_function} in the DARMA \gls{API} that accepts 
positional or \glspl{keyword argument} \inlinecode{arg_a}, 
\inlinecode{count}, and \inlinecode{flag}, that function can be invoked 
equivalently in any of the following ways:
\begin{CppCode}
/* some_function signature:
 *  void some_function(std::string arg_a, int count, bool flag);
 */
// All of the following are equivalent:
some_function("hello", 42, true);
some_function(arg_a="hello", count=42, flag=true);
some_function(count=42, flag=true, arg_a="hello");
some_function("hello", flag=true, count=42);
\end{CppCode}
Note that \glspl{positional argument} may {\em not} be specified after the
first \gls{keyword argument},
and an argument cannot be specified more than once, even as a positional and
\gls{keyword argument}.  Both of these lead to compile-time errors. Omitting a required argument is 
also a compile-time error, as is giving an argument of the incorrect type: 
\begin{CppCode}
// Error: arg_a specified more than once
some_function("hello", 42, true, arg_a="whoops!");

// Error: missing required argument flag
some_function("hello", count=42);

// Error: cannot convert bool to std::string
some_function(arg_a=false, flag=true, count=42);
some_function(false, 42, true);

// Error: positional argument given after first keyword argument
some_function(arg_a="hello", 42, flag=true);
\end{CppCode}
The enabling of Python-like keyword arguments introduces no runtime overhead.
For those interested in C++ details, keyword arguments are accomplished using
\inlinecode{constexpr} class instances with overloaded assignment operator,
with arguments passed to the callable using perfect forwarding.
More implementation details are given in Section \ref{sec:kwargs}.  

%\todo{Keywords and namespaces - I don't think we ever resolved this}
%The illusion of Python-like \glspl{keyword argument} is accomplished 
%using \inlinecode{constexpr} instances
%of a class with the assignment operator overloaded.  The arguments are passed to the 
%callable using \gls{perfect forwarding}, and thus the overhead from the
%\gls{keyword argument}
%trick is entirely at compile-time.  (More implementation details are given in
%Section \ref{sec:kwargs}).  These instances are defined in very descriptive namespaces,
%and frequently used keywords are aliased into more general namespaces. 
%For instance, the keywords for \inlinecode{some_function}, above, would likely be 
%defined in a namespace named using the following convention:
%\inlinecode{darma_runtime::keyword_arguments_for_some_function}.
%If \inlinecode{some_function} was a 
%widely used construct, its more important keywords would also be defined 
%in the namespace \inlinecode{darma_runtime::keyword_arguments}. 
%If \inlinecode{some_function} belonged to a broader category of 
%constructs, certain of its keywords may also be aliased into a 
%categorical namespace as well.  For instance, if \inlinecode{some_function} 
%implemented some memory management functionality,
%its keywords may be aliased into, for example, \inlinecode{darma_runtime::keyword_arguments_for_memory_management}.
%All of this is done so that the user can minimize verbosity without introducing
%naming conflicts.  In certain contexts, it may be expedient for the user to simply
%put \inlinecode{using darma_runtime::keyword_arguments} at the beginning of the calling context. 
%In other, more complicated contexts, importing all DARMA keywords into the 
%local context may lead to naming conflicts with local variables. 
%You don't have to guess at which namespaces
%a keyword is provided in or aliased into; every time a \gls{keyword argument} or a callable
%accepting that argument is introduced in the documentation, the namespace to which 
%it belongs and all namespaces it is aliased into are given.

