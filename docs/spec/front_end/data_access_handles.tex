%!TEX root = ../sandReportSpec.tex
%\lstMakeShortInline[style=CppCodeInlineStyle]{\|}
\section{Data Access Handles}
\label{sec:handles}

Change this - dependency managment and coordination/communication between
ranks via the key value store
When publishing, the user must specify an \gls{access group}
for that data.  Declaring an access group informs the runtime who currently needs or will need the data,  
allowing garbage collection and \gls{anti-dependency} resolution.
In most cases, the access group will be declared as the number of global read handles.
For performance optimizations in future versions, an explicit list of ranks (similar to MPI) can be given.
Once all read handles are released (go out of scope in C++ terms), garbage collection or anti-dependency resolution can occur.
In dynamic cases in which the number of readers is not known,
the access group is declared through an (asynchronous) phase barrier.

Data \glspl{handle},
\codetarget{AccessHandle<T>}, 
  of different access types to differentiate the various transactions. 
At the most basic level user code can declare 
\begin{compactenum}
  \item a \gls{handle} to data that does not yet exist in the system
but needs to be created, or
\item a \gls{handle} to data that already exists and needs to
be read, or 
\item a \gls{handle} to data that it wants to overwrite or modify. 
  (Note that this type of handle does not exist in the \specVersion-spec.)
\end{compactenum}

Type 1 is denoted as \codelink{initial_access} in DARMA, 
which informs the \gls{runtime system} that the data with the 
specified \codelink{key} does not yet exist, and the user intends to 
create this data and (potentially) \codelink{publish} it.
Hence, an \codelink{initial_access} data \gls{handle} is usually 
followed by a memory allocation, a value assignment and 
finally a \codelink{publish} operation, as illustrated below: 

\begin{CppCode}
auto float_handle = initial_access<float>("float_key");
create_work([=]{
  float_handle.set_value(3.14);
});
\end{CppCode}
The second type of \gls{handle}, requesting read-only access 
to a piece of data, is via \codelink{read_access}. 

%\todo[inline]{This needs to be revised or changed, since
%``wait\_and\_get\_value()'' (or whatever we end up calling it) is an
%advanced feature (and not part of the 0.2 spec).}
%\begin{CppCode}
%auto float_handle = read_access<float>("another_float_key");
%{
%  float val = float_handle.wait_and_get_value();
%  std::cout << "Value read with key another_float_key is " << val;
%}
%\end{CppCode}
%or, alternatively using deferred execution
\begin{CppCode}
auto float_handle = read_access<float>("another_float_key");
create_work([=]{
  float val = float_handle.get_value();
  std::cout << "Value read with key another_float_key is " << val;  
});
\end{CppCode}
%The runtime guarantees that |get_value| will wait until the value
% exists and is ready locally.
You can only call \codelink{get_value} in a context where you have read access
to the data associated with the handle, either by capturing into a deferred
context or by explicitly waiting on the value (though waiting on the value is
not supported in the \specVersion\ spec).
This might involve moving data if the \inlinecode{float} is on a remote node.
In general, any calls to \codelink{get_value} should occur 
within a scoped code block to avoid dangling references to stale physical memory locations.
Even when it becomes possible to wait on the data associated with a handle (in
later versions of the spec), best practice is to access data via deferred tasks
inside a \codelink{create_work} block when possible.

The third handle type, which informs the runtime that the data 
with the specified key will be read and overwritten or modified, 
is with the \codelink{read_write} handle, illustrated below. 
%Again, we show an
%example done within the same task and an example using a deferred task.
%\begin{CppCode}
%auto float_handle = read_write<float>("yet_another_float_key");
%{
%  std::cout << "Value read with key yet_another_float_key is " 
%          << float_handle.get_value();
%  float_handle.set_value(3.14*2.0);
%}
%\end{CppCode}
\begin{CppCode}
auto float_handle = read_write<float>("yet_another_float_key");
create_work([=] {
  std::cout << "Value read with key yet_another_float_key is " 
          << float_handle.get_value();
  float_handle.set_value(3.14*2.0);
});
\end{CppCode}
This sort of access is not available in the \specVersion\ version of the spec, and in
later versions this sort of usage will be conceptualized as ``ownership
transfer'' of a data block, since exactly one |AccessHandle| with
modification privileges on a block of data can exist at any given time.
%As with \codelink{read_access}, value get/set should at least occur within a
% scoped code block, and will likely perform better if placed in a deferred task block.

Tasks should only ever request the privileges they need. 
Over-requesting privileges will limit the amount of available parallelism in the code.
It may occur that the task is using a read-write handle, but needs to create a
task that only requires read access.

\begin{CppCode}
auto float_handle = read_write<float>("yet_another_float_key");
create_work(reads(float_handle), [=] {
  std::cout << "Value read with key yet_another_float_key is " 
          << float_handle.get_value() << std::endl;
})
create_work(reads(float_handle), [=] {
  float val = float_handle.get_value();
  if (val > 0) std::out << "Value is positive" << std::end;
})
//read-write work down here
\end{CppCode}
In this case, subtasks are created that only need read access. 
Without the \codelink{reads} qualifier, these tasks would conflict since they
would by default request read-write privileges.

\subsection{Publish}
\label{subsec:publish}
By default, unless explicitly published, data handles are visible only to tasks
within the same scope (that is, tasks that have a copy of the actual
\codelink{AccessHandle<T>} object, created as discussed in
Section \ref{sec:handles}.
For data to be globally visible in the global memory space (\gls{key-value
store}),
the application developer must explicitly \codelink{publish} data.  Unpublished 
data will be reclaimed once the last handle to it goes out of scope,
freeing the memory and resolving any anti-dependencies analogous to the
destructor invocation in C++ when a class goes out of scope.  Unpublished data
can leverage the sequential semantics of the application for garbage
collection.  Published data, however, is globally visible to all workers and
requires more ``permanence.''  In order to reclaim the data (garbage collect or
resolve anti-dependence), publish data must know its access group.
When all read handles within an access group have been deleted or released
\emph{globally}, the published data can be reclaimed.
The easiest way to declare an access group (and currently supported method) is
to simply give the total number of additional read \codelink{AccessHandle<T>}
objects that will be created referring to it.
In future versions, hints will be supported about which specific tasks will need
to read data.
The publish/fetch mechanism replaces an analogous \inlinecode{MPI_Send/Recv} or
even potentially an \inlinecode{MPI_Bcast}.
In MPI, these function calls force an \inlinecode{MPI_Send} or
\inlinecode{MPI_Wait} to block until the runtime guarantees that the data has
been delivered.
An access group in DARMA provides a similar guarantee.
Until all readers in an access group have received or released their data, DARMA
cannot reclaim (garbage collect, clear anti-dependencies).

\begin{CppCode}
auto float_handle = initial_access<float>("float_key");
create_work([=]{
  float_handle.set_value(3.14);
});
float_handle.publish(n_readers=1);
\end{CppCode}
The \codelink{n_readers} specification in the publish call is a keyword
argument (see Section \ref{sec:keyword}) that informs the runtime that the data
(associated with \inlinecode{float_key}) will only ever be read once, and hence
can be safely garbage collected soon after.  This code is essentially a direct
replacement of a send/recv.

Publish operations are treated as asynchronous read operations --- that is,
\inlinecode{h.publish(/*...*/)} is treated like
\inlinecode{create_work(reads(h),[=]\{/*...*/\}); }. 
This means that the same precautions should be taken as with asynchronous reads. 
In particular, even if the handle was in a modifiable state before the 
\codelink{publish} it is no
longer valid to call \inlinecode{h.set_value()} after the \codelink{publish}, since the
asynchronous read done by the publish may or may not have occurred yet.  In this
scenario, one should use \inlinecode{create_work([=]\{ h.set_value(/*...*/); \});}
instead.



\subsubsection{Publication Versions}

If a handle is going to be published multiple times (or, more specifically, if
the key with which the handle was created is going to be published multiple
times), it needs to be published with a different version each time.  Versions
are just like keys --- an arbitrary tuple of values (see Section
\ref{subsec:keys}).  For instance:

\begin{CppCode}
auto float_handle = initial_access<float>("float_key");
auto int_handle = initial_access<int>("int_key");
/*...*/
int_handle.publish(n_readers=3, version=77);
// Use version() instead of version= for multi-part version keys
float_handle.publish(n_readers=1, version("alpha",42));

/* Elsewhere... (e.g., on a different rank) */
auto my_int = read_access<int>("int_key", version=77);
auto my_flt = read_access<float>("float_key", version("alpha",42));

\end{CppCode}

% DSH took this out, since it's supposed to be in 0.3
%\subsection{Subscribe}
%\label{subsec:subscribe}

\subsection{Keys}
\label{subsec:keys}
In the examples in this Section, the |key| to the
\codelink{AccessHandle<T>} has always been a single string. 
A \codetarget{key} in DARMA 
can be an arbitrary \gls{tuple} of values.  This 
makes it very easy for the application developer to create an expressive
and descriptive \codelink{key} for each piece of data.  Glspl{tuple} can comprise
different bit-wise copiable data types.  The example at the end of
Section~\ref{sec:spmd} illustrates the use of the \gls{rank} within the handle
\codelink{key}.  The following example shows the use of an
aribitrary \gls{tuple} as a \codelink{key}:
\begin{CppCode}
  int neighbor_id
  double other_identifier;

  // some code that sets neighborID and other_identifier
  
  auto float_handle = initial_access<float>("float_key", 
                                            neighbor_id, 
                                            other_identifier);
\end{CppCode}


\subsection{Handle Usage Rules}
\label{sec:handlerules}
Handles are assigned states, and these states change 
based on the operations applied to them. In other words, 
handles' states transition. However, not all states 
are allowed at all times. The ``permissions'' on 
what it is allowed changes based on the context. 
Permissions fall under two main categories: 
\begin{itemize}
\item[a] {\it Scheduling}: permissions on a handle 
within a \codelink{create_work} (more generally within a deferred work).
\item[b] {\it Immediate}: permissions that apply immediately.
\end{itemize}

In the \specVersion\ spec, as described above, there are two main types 
of handle supported:
\begin{itemize}
\item \codelink{initial_access<T>}: when a handle of 
this type is first created, it is assigned 
``Modify/None'' permissions.
%
\item \codelink{read_access<T>}: when a handle of 
this type is first created, it is assigned 
``Read/None'' permissions.
\end{itemize}



\begin{table}[!t]
\begin{center}
{\small
\begin{tabular}{cc|cc|cc|cc}
 \hline
 \multicolumn{2}{c|}{|}|
  & \multicolumn{2}{c|}{\codelink{get_value()}} 
 & \multicolumn{2}{c|}
 {
   \specialcell{ \codelink{emplace_value()} \\ 
                  \codelink{set_value()}\\
                  \codelink{get_reference()}\\
                  } 
 } 
  & \multicolumn{2}{c}{\codelink{release()}} \\
 \hline
 \specialcell{Scheduling\\ permissions} 
 & \specialcell{Immediate\\ permissions}  
 & { {\footnotesize Allowed? } } \hspace{-0.cm} & { {\footnotesize Continuing as}}
 & { {\footnotesize Allowed? } } \hspace{-0.cm} & { {\footnotesize Continuing as}}
 & { {\footnotesize Allowed? } } \hspace{-0cm} & { {\footnotesize Continuing as }}\\
 \hline
 None & None
 & No & -
 & No & -
 & Yes${}^*$ & {\em None/None} \\
 %
 Read & None
 & No & -
 & No & -
 & Yes & {\em None/None} \\
 %
 Read & Read
 & Yes & {\em Read/Read}
 & No & -
 & Yes & {\em None/None}   \\
 %
 Modify & None
 & No & -
 & No & -
 & Yes & {\em None/None}   \\
 %
 Modify & Read
 & Yes & {\em Modify/Read}  
 & No & -
 & Yes & {\em None/None}   \\
 %
 Modify & Modify
 & Yes & {\em Modify/Modify}  
 & Yes & {\em Modify/Modify}  
 & Yes & {\em None/None}   \\
\hline
\end{tabular}
}
\caption{Operations on the various states. 
Transitions marked with an asterisk (*) effectively
represent no-ops and could generate warnings.}
\label{tab:immsimp}
\end{center}
\end{table}
%
\begin{table}[!t]
\begin{center}
{\small
\begin{tabular}{cc|ccc|ccc}
 \hline
 \multicolumn{2}{c|}{} 
 & \multicolumn{3}{c|}{\em{read-only capture}} 
 & \multicolumn{3}{c}{\em{modify capture}}  \\
 \hline
 \specialcell{Scheduling\\ permissions} 
 & \specialcell{Immediate\\ permissions}  
 & {\footnotesize Allowed? } & {\footnotesize Captured } 
 & {\footnotesize Continuing as} 
 & {\footnotesize Allowed? } & {\footnotesize Captured } 
 & {\footnotesize Continuing as} \\
 \hline
 None & None & No & - & - & No & - & - \\
 %
 Read & None 
 & Yes 
 & {\em Read/Read}
 & {\em Read/None}
 & No
 & -
 & - \\
 %
 Read & Read
 & Yes 
 & {\em Read/Read}
 & {\em Read/Read}
 & No
 & -
 & - \\
 %
 Modify & None
 & Yes 
 & {\em Read/Read}
 & {\em Modify/None}
 & Yes
 & {\em Modify/Modify} 
 & {\em Modify/None} \\
 %
 Modify & Read
 & Yes 
 & {\em Read/Read}
 & {\em Modify/Read}
 & Yes
 & {\em Modify/Modify} 
 & {\em Modify/None} \\
 %
 Modify & Modify
 & Yes 
 & {\em Read/Read} 
 & {\em Modify/Read} 
 & Yes
 & {\em Modify/Modify}
 & {\em Modify/None} \\
\end{tabular}
}
\caption{Deferred (capturing) operations on the various states.}
\label{tab:capsimp}
\end{center}
\end{table}

Example (1): 
\hspace{-0.75cm}
\begin{minipage}[t]{0.45\linewidth}%
\centering
WRONG
\begin{vaspPseudo}
initial_access<int> a
//a is in Modify/None
a.set_value(1) $\no$ 
a.get_value()  $\no$
\end{vaspPseudo}
\end{minipage}
\hspace{0.55cm}
\begin{minipage}[t]{0.45\linewidth}
\centering
CORRECT
\begin{vaspPseudo}
initial_access<int> a
//a is in Modify/None
create_work([=]{ //modify capture
    a.emplace_value(1)  $\yes$
    a.set_value(1)      $\yes$
    a.get_reference()=1 $\yes$
});
\end{vaspPseudo}
\end{minipage}



Example (2): 
\hspace{-0.75cm}
\begin{minipage}[t]{0.45\linewidth}%
\centering
WRONG
\begin{vaspPseudo}
read_access<int> b
//b is in Read/None
b.get_value()   $\no$
b.set_value(1)  $\no$
create_work([=]{ //capture
  b.set_value(1) $\no$
});
\end{vaspPseudo}
\end{minipage}
\hspace{0.55cm}
\begin{minipage}[t]{0.45\linewidth}
\centering
CORRECT
\begin{vaspPseudo}
read_access<int> b
//b is in Read/None
create_work([=]{ // capture
  b.get_value()  $\yes$
});
\end{vaspPseudo}
\end{minipage}

