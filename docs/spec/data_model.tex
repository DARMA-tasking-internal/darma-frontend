%!TEX root = sandReportSpec.tex

\section{Data Model}
\label{sec:data_model}
\gls{DARMA} only implements a \gls{data model} through its \gls{serialization} interface.
The notion of data structure, data layout, and data type only exist in the
application and \gls{translation layer} (see Section \ref{TL}).
Thus, a \gls{runtime system} implementing the \gls{DARMA} specification is only
aware of \gls{tuple} or \codelink{key} identifiers for a coarse-grained data block of a given size.
To actually migrate data, a \gls{back end} \gls{runtime system} invokes
\gls{serialization} hooks implemented by the application.
In future versions, an \gls{API} similar to the \gls{serialization} interface will support the definition of data subsets and data slices.
Again, the backend runtime will only understand data and task dependencies,
  requiring the type-aware application and \glspl{translation layer} to define
  the details of subsetting and slicing operations.
This leaves the application developer free to use arbitrary data structures,
     but puts more responsibility on the application developer to articulate
     the structure of the data.


