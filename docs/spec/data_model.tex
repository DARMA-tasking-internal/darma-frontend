%!TEX root = sandReportSpec.tex

\section{Compatible Data Models}
\label{sec:data_model}
DARMA only implements a data model through its serialization interface.
The notion of data structure, data layout, and data type only exist in the application and translation layer (see Section \ref{TL}).
The backend runtime in DARMA is only aware of tuple or key identifiers for a coarse-grained data block of a given size.
To actually migrate data, the backend invokes serialization hooks implemented by the application.
In future versions, an API similar to the serialization interface will support the definition of data subsets and data slices.
Again, the backend runtime will only understand data and task dependencies, requiring the type-aware application and translation layers to define the details of serialization and slicing operations.
This leaves the application developer free to use arbitrary data structures, but puts more responsibility on the application developer.


